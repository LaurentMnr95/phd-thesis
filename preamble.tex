\pdfoutput=1
\newcommand{\mtbxx}[1]{\textcolor{black}{#1}}
\newcommand{\br}[1]{\textcolor{black}{#1}}

\newcommand{\ccc}[1]{\textcolor{black}{#1}}
\newcommand{\cc}[1]{\textcolor{black}{#1}}
\newcommand{\brtwo}[1]{\textcolor{black}{#1}}

\newcommand{\carola}[1]{\textcolor{black}{#1}}
\newcommand{\olivier}[1]{\textcolor{black}{#1}}






%\newcommand{\todo}[1]{\textcolor{red}{#1}}
\newcommand{\removeme}[1]{\textcolor{black}{#1}}
\newcommand{\TODODimo}[1]{{\color{red} TODO Dimo: #1}}

% \newtheorem{coro}{Corollary}[section]
% \newtheorem{lemme}{Lemma}[section]
% \newtheorem{defn}{Definition}[section]
% \newtheorem{prop}{Proposition}[section]
% \newtheorem{thm}{Theorem}[section]
% \newtheorem{property}{Property}[section]
% \newtheorem{rmq}{Remark}
% \newtheorem{corol}{Corollary}
% \newtheorem{proposition}{Proposition}[section]
% \newtheorem{prv}{Proof}[section]
\newtheorem{coro}{Corollary}
\newtheorem{prop}{Proposition}
\newtheorem{thm}{Theorem}
\newtheorem{property}{Property}
\newtheorem{corol}{Corollary}
\newtheorem{assump}{Assumption}
\newtheorem{corollary}{Corollary}
\newtheorem{lemma}{Lemma}
\newtheorem{definition}{Definition}
\newtheorem{example}{Example}
\newtheorem{rmq}{Remark}
\newtheorem*{counterexample*}{Counterexample}
\newtheorem*{consequence*}{Consequence}

\newtheorem*{assump*}{Assumption}

\newtheorem*{example*}{Example}
\newtheorem*{prop*}{Proposition}
\newtheorem*{thm*}{Theorem}
\newtheorem*{prv*}{Proof}

% Notations

%\newcommand{\GOT}{\textsc{MS}\xspace}
\newcommand{\MOT}{\textsc{EOT}\xspace}
\newcommand{\MOTe}{\textsc{EOT}^{\bm{\varepsilon}}\xspace}
% \newcommand{\DOT}{\textsc{MOT}\xspace}
\newcommand{\KL}{\textsc{KL}\xspace}
\newcommand{\TV}{\textsc{TV}\xspace}
\newcommand{\ent}{\textsc{H}\xspace}
\newcommand{\wass}{\textsc{W}\xspace}
\newcommand{\riskemp}{\widehat{\mathcal{R}}}

\newcommand{\riskadv}{\mathcal{R}_{\epsilon}}
\newcommand{\valuerand}{\mathcal{V}^{rand}}
\newcommand{\valuedet}{\mathcal{V}^{det}}
\newcommand{\dualvalue}{\mathcal{D}}

\DeclareMathOperator*{\supp}{supp}   
\DeclareMathOperator*{\sign}{sign}   
\DeclareMathOperator*{\essinf}{essinf}   

\newcommand{\QQ}{\mathbb{Q}}
\newcommand{\PP}{\mathbb{P}}
\newcommand{\EE}{\mathbb{E}}
\newcommand{\RR}{\mathbb{R}}

\newcommand{\XX}{\mathcal{X}}
\newcommand{\YY}{\mathcal{Y}}
\newcommand{\ZZ}{\mathcal{Z}}


\newcommand{\risk}{\mathcal{R}}
\newcommand{\loss}{L}

\newcommand{\probmap}{h}




\newcommand{\bookboxx}[1]{\small
\par\medskip\noindent
\framebox[0.99\textwidth]{
\begin{minipage}{0.97\dimexpr\textwidth-\parindent\relax} {#1} \end{minipage} } \par\medskip }
\newcommand{\forceindent}{\leavevmode{\parindent=1em\indent}}

\DeclareMathOperator*{\argminB}{argmin}   
\DeclareMathOperator*{\esssup}{ess\text{  }sup}  
\DeclareMathOperator*{\argmaxB}{argmax} 
\DeclareMathOperator*{\Vol}{Vol}


\newcommand{\lints}{\textsc{LinTS}\xspace}
\newcommand{\expfour}{\textsc{Exp}$4$\xspace}
\newcommand{\expthree}{\textsc{Exp}$3$\xspace}
\newcommand{\oful}{\textsc{OFUL}\xspace}
\newcommand{\linucb}{\textsc{LinUCB}\xspace}
\newcommand{\ucb}{\textsc{UCB}\xspace}
\newcommand{\epsgreedy}{\textsc{$\varepsilon$-greedy}\xspace}


\DeclareRobustCommand{\eg}{e.g.,\@\xspace}
\DeclareRobustCommand{\ie}{i.e.,\@\xspace}
\DeclareRobustCommand{\aka}{a.k.a.\@\xspace}
\DeclareRobustCommand{\wrt}{w.r.t.\@\xspace}
\DeclareRobustCommand{\wp}{w.p.\@\xspace}
\DeclareRobustCommand{\st}{s.t.\@\xspace}
\newcommand{\wt}[1]{\widetilde{#1}}
\newcommand{\wh}[1]{\widehat{#1}}
\newcommand{\wb}[1]{\overline{#1}}


\newcommand{\PCadvRisk}{\text{PC-Risk}_{\alpha}}
\newcommand{\EoT}{\text{EoT}}


\newcommand{\numsamples}{n}

% TODO: numlabels, inputspace, labelspace, 