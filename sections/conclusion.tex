\chapter{Conclusion}
\minitoc
\section{Summary of the thesis}

In this thesis, we studied the problem of  classication in presernce of adversaries from different point of views for theoretical and practical finalities. We have tried to analyze the problem using both a high level
and a more precise analysis. We summarize our findings as follows.

\begin{enumerate}
    \item We provide a better understanding of the adversarial problem studying the nature of equilibria in this game. We proved the existence of mixed Nash equilibria for very general assumptions. We hope this research directions will lead to principled results that can be used in practice for better defending against adversarial examples.
    \item We studied and closed the problem of calibration in the adversarial binary-classification setting providing necessary and sufficient conditions. We paved a way to prove consistency results, and hope being able to conclude on consistency of shifted odd losses. It remains to find necessary and sufficient conditions for consistency.
    \item We derived a principled way based on dynamical system to build $1$-Lipschitz layers. Interestingly, we recovered some existing methods from the literature, but we were also able to build new interesting layers, namely the Convex Potential Layers. We hope this work would lead to study other possible dynamical systems and provide new provably robust neural networks.
\end{enumerate}
\section{Open Questions}




\subsection{Understanding Randomization in Adversarial Classification}

\begin{itemize}
    \item Statistical Bounds for Adversarial Robustness in the Case of Randomized Classifiers
    \item Designing an Algorithm for computing Nash Equilibria in the General Case
\end{itemize}


\subsection{Loss Calibration General Results}

\begin{itemize}
    \item The non realisable case is difficult: showing either negative/positive general results
    \item Further developing the margin loss analysis
\end{itemize}
\subsection{Exploiting the architecture of Neural Networks to get Guarantees}
\begin{itemize}
    \item Exploiting Helmoltz decomposition of flows
    \item Exploiting other flows (Hamiltonian, Momentum, etc.)
\end{itemize}








