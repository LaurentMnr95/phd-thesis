\chapter{Introduction}
\minitoc
\section{Artificial Intelligence foundations}
% The idea of building artificial intelligence dates back to the first automatons that dates back to Ancient Egypt and Greece mythology. At the time, automatons were movable statue as Talos' one in Crete. Artificial Intelligence relies on the idea that human reasoning can be automatized, e.g. building a machine that can replicate human reasoning without external intervention. The study of automatized reasoning has a long stand history, and was the research subject of many mathematicians or philosophers. QUOTE A BIT
% A ENLEVER PE

Artificial Intelligence, as it is understood today, only appears during the 20th century. Its birth is inseparable from the rise of computer science. Computer science finds its foundation in the Church-Turing thesis~\citep{turing1950computing}, which defines the notion of computability, i.e. functions are computable if they can be out as a list of predefined instructions to be followed. Such instructions are called algorithms. Artificial Intelligence research was  ``officially founded'' as a research field in 1956 at the Dartmouth Workshop~\citep{mccarthy2006proposal}, organized by Marvin Minsky, John McCarthy, Claude Shannon and Nathan Rochester. During this conference, the term ``Artificial intelligence'' was proposed and adopted by the community of researchers. Following this conference and thanks to fundings, major advances in the field were made in problem solving and natural language processing for instance before reaching a first ``AI winter'' between 1987 and 1993 due to the lack of fundings and computational power. The second wave of AI occurs with the raise of the first Artificial Neural Networks thanks to the algorithm of backpropagation~\citep{rumelhart1985learning} and the first well performing convolutional networks~\citep{lecun1995convolutional}. 

TALK ABOUT CYBERNETICS
TALK ABOUT THEORETICAL ML/ETC
LACK OF DATA/LACK OF POWER (GPUs)/LACK OF SCALABILITY (SGD)/OPEN SOURCE 
INVERSER LES DEUX PARAGRAPHS

In this manuscript, we will focus on  machine learning~\citep{friedman2001elements}, which is a sub-field of Artificial Intelligence. Machine learning can be defined with the following question: ``How can we build computer systems that automatically improve with experience, and what are the fundamental laws that govern  all learning processes?''. For instance, credit scoring are computed rules that are learnt from previous defaulting consumers. Machine Learning have been widely studied for the $30$ past years using advanced statistical tools and taking profits of more and more powerful computers.  The last $10$ years have seen an exponentially increasing public interest for machine learning systems. Firstly, this incredible growth of interest to AI is also linked to the availability of huge amounts of data at low price, the so-called ``Big Data'' era.  Secondly, the advent of Deep Learning, i.e. Machine Learning using Artificial Neural Networks with a lot of layers, came with the extraordinary success of AlexNet~\citep{krizhevsky2012imagenet} on the ImageNet challenge~\citep{imagenet_cvpr09} in 2012. Since, exceptional progresses were made in generative modeling~\citep{goodfellow2014generative}, natural language processing~\citep{vaswani2017attention}, speech recognition CITE, etc. Machine learning has become the most active research fields in Artificial Intelligence. Therefore the number of industrial application has also exploded. But using such systems is not risk-free and malicious users can take advantages or fool such systems.
\section{Risks with Learning Systems}

Cybersecurity is at the core of computer science. Cryptography has been one of the hottest topics during the last thirty years. Despite their performances, learning systems are subject to many types of vulnerabilities and, by their popularity, are then prone to malicious attacks. Probably, the most known vulnerability that got public attention is privacy. While the amount of available data is exponentially growing, recovering identities by crossing datasets is easier when data are not protected. As it was exhibited in the de-anonymization of the Netflix 1M\$ prize dataset~\citep{narayanan2008robust}, hiding identities in datasets is not sufficient to protect the privacy data. Computer scientists have then worked to propose ways to protect data, creating new research areas as Differential Privacy~\citep{dwork2008differential}. Differential privacy have become a standard by its simplicity and its guarantees in privacy concerns. It grossly consists in adding noise to data to make them unrecoverable. But privacy comes at a price: the quality of data is often lowered while gaining in privacy~\citep{XXX}. CATEGORISER LES ATTAQUES PRIVACY + ATTAQUE MODEL FAILURE
There are many threats to learning systems. One can list (non exhaustively) the following common vulnerabilities:
\begin{itemize}
    \item \textbf{Data poisoning attacks~\citep{kearns1993learning}:} changing some data in the training set so that the model performs very poorly on the hold-out set. 
    \item  \textbf{Model stealing~\citep{tramer2016stealing}:} An attacker aims at stealing the parameters of a given model.
    \item \textbf{Membership inference~\citep{shokri2017membership}:} Inferring whether a data sample was present or not in a training set. This is a attack on the privacy of the model.
    \item \textbf{Evasion attacks~\citep{biggio2013evasion,Szegedy2013IntriguingPO}}: small imperceptible perturbations at inference time.
\end{itemize}
    
\cite{kumar2020adversarial} highlighted that only very few companies are aware of these threats, hence asking potential security issues in a close future. In the manuscript we will focus on adversarial attacks, we introduce this threat more in details in the next paragraph.

\subsection{Adversarial attacks against Machine Learning Systems}

Despite the recent gain of interest in studying adversarial attacks in Machine Learning, the problematic exists for a while and takes its source in SPAM classification where adversaries were spammers whose goal was to evade from the taken decision\footnote{\cite{dalvi2004adversarial} showed that linear classifiers used in spam classification could be fooled by simple ``evasion attacks'' as spammers inserted ``good words'' into their spam emails.}.

With the recent success of Deep Learning algorithms, in particular in computer vision, several authors~\citep{biggio2013evasion,Szegedy2013IntriguingPO} have  highlighted their vulnerability to adversarial attacks. Adversarial attacks in this case are widely understood as ``imperceptible'' perturbations of an image, i.e. slight changes in the pixels, so that this image remains unchanged from human sights. This characteristic might be surprising but is actually a severe curb in applying state-of-the-art deep learning methods in critical systems. There are number of issues that makes difficult building and evaluating robust models for real life applications:
\begin{enumerate}
    \item The notion of imperceptibility is not well understood: numerically measuring human perception is still an open problem. Hence, detecting the change of perception due to adversarial attacks is a ill-posed problem. Most of the  research in the domain have focused on pixel-wise perturbations (e.g. $\ell_p$ norms), while real world threats would be crafted by inserting some misleading objects in the environment (e.g. patches~\citep{brown2017adversarial}, T-shirts~\citep{xu2020adversarial}, textures~\cite{XXX},etc.).
    \item Robustness is often empirically measured: there exist only a few methods with formal guarantees on the robustness and these guarantees are often loose. Robustness is usually measured on a set of possible attacks and not all possibilities of perturbations are spanned by these attacks, leaving rooms for potential blind spots.
    \item There exists a trade-off between robustness and accuracy. Most models that are robust suffer from a performance drop on natural data. For instance, a robustly trained robot will perform much lower on natural tasks than an accurate non-robust robot. That makes robust models unusable in real world applications~\citep{lechner2021adversarial}. 
\end{enumerate}

Consequently there still exists a substantial gap between academic research on adversarial attacks and ``real world'' attacks. This gap is difficult to bridge and it is one of the reasons authorities are reluctant authorizing autonomous systems as driver-free cars~\citep{eykholt2018robust}, automated face recognition as biometry checks~\citep{dong2019efficient}, etc. This possibility of malicious use in learning systems urged European authorities to propose new laws both on data and AI systems in the last decade.
\subsection{Towards a Unified Regulation}




The first and very known regulation is GDPR (General Data Protection Regulation)\footnote{https://eur-lex.europa.eu/eli/reg/2016/679/oj}, adopted in 2016, which defines new rules on the use of data and on privacy. Today, GDPR is widely respected by companies and start-ups today. Indeed, the fine for a company not respecting this law can be up to $4\%$ of the revenues of the company. To comply with this new regulation, companies and government must respect privacy in the implementation of any systems with regards to the data they use.  While the initial aim of such a law was to discourage multinational companies using personal data, these companies has adapted easier than smaller ones thanks to the budget they could allocate for respecting GDPR. To counter this, A second law proposition regarding data sharing from public and private companies and was made by European Commission on The Governance of Data\footnote{https://eur-lex.europa.eu/legal-content/EN/TXT/?uri=CELEX\%3A52020PC0767} in 2020.




More recently, the first regulation text on Artificial Intelligence\footnote{https://eur-lex.europa.eu/legal-content/EN/TXT/?uri=CELEX\%3A52021PC0206} systems was proposed by the European commission in April 2021. This text includes a large section dedicated ``High Risk AI''. High risk AI is referred to any autonomous systems than can endanger human lives.  This text aims at dealing with many threats in Learning Systems. From this text, two direct references are made to adversarial attacks, underlying the need for companies to deal with adversarial attacks. The difficulty is to unify and create precise rules in a domain were results and certificates are mostly empirical. As mentioned earlier, it is known that robust models are often less performing and can make autonomous systems unusable in real world. Thus, this text is a first step towards a unified regulation on autonomous systems but might miss precise requirements for models to be used in production.
\medskip
\begin{tcolorbox}[title=References to adversarial examples in European Commission in law proposal on Artificial Intelligence systems]
\label{ref:adversarial_law}
As part of the introduction: \textit{``Cybersecurity plays a crucial role in ensuring that AI systems are resilient against attempts to alter their use, behaviour, performance or compromise their security properties by malicious third parties exploiting the system’s vulnerabilities. Cyberattacks against AI systems can leverage AI specific assets, such as training data sets (e.g. data poisoning) or trained models (e.g. adversarial attacks), or exploit vulnerabilities in the AI system’s digital assets or the underlying ICT infrastructure. To ensure a level of cybersecurity appropriate to the risks, suitable measures should therefore be taken by the providers of high-risk AI systems, also taking into account as appropriate the underlying ICT infrastructure.''}

\medskip
Title III (High risk AI systems), Chapter II (Requirements for high risk AI system), Article 14.52 (Human oversight): \textit{``High-risk AI systems shall be resilient as regards attempts by unauthorised third parties to alter their use or performance by exploiting the system vulnerabilities.
The technical solutions aimed at ensuring the cybersecurity of high-risk AI systems shall be appropriate to the relevant circumstances and the risks.
The technical solutions to address AI specific vulnerabilities shall include, where appropriate, measures to prevent and control for attacks trying to manipulate the training dataset (‘data poisoning’), inputs designed to cause the model to make a mistake (‘adversarial examples’), or model flaws.''}
\end{tcolorbox}
\medskip


\section{Adversarial Classification in Machine Learning}

In this manuscript, we will focus on the task of classification in Machine Learning. The purpose of this task is to ``learn'' how to classify some input $x$ into some labels. The input can be an image, a text, an audio, etc. For instance, in computer vision, a known dataset is ImageNet where the goal is to learn how to classify high quality images into $1000$ labels~\citep{imagenet_cvpr09}. In natural language processing, the IMDB Movie Review Sentiment Classification dataset~\citep{maas-EtAl:2011:ACL-HLT2011} aims at classifying positive or negative sentiment from movie reviews. To learn a classifier, the task is often supervised~\citep{friedman2001elements}, i.e, we have access to labeled inputs, which constitutes the so-called training set. To assess the quality of the learnt model, we evaluate it on other images that constitutes the test set.

\subsection{A Learning Approach for Classification}
From now, we will assume that the inputs are in some space $\XX$ and the labels form a set $\mathcal{Y}:=\{1,\dots,K\}$. To learn an adequate classification model, we denote $\{(x_1,y_1),\dots,(x_n,y_n)\}$ the $n$ elements of $\XX\times\YY$ forming the training set. We furthermore assume that these inputs are independent and identically distributed (i.i.d.) from some distribution $\PP$ on $\XX\times\YY$. The aim is now to learn a function/hypothesis from these samples $h:\XX\to\YY$ to classify an input $x$ with a label $y$. To assess the quality of a classifier, the metric of interest is often the misclassification rate of the model, or the $0/1$ loss risk, and it is defined as:
\begin{align*}
\risk_{0/1}(h):=\PP(h(x)\neq y) = \EE_{(x,y)\sim\PP}\left[\mathbf{1}_{h(x)\neq y}\right]
\end{align*}
The optimal classifier, minimizing the standard risk is called the Bayes optimal classifier and is defined as $h(x) = \argmaxB_k\PP(y=k\mid x)$.
As the sampling distribution $\PP$ is usually unknown, the optimal Bayes classifier is also unknown. The accuracy is often empirically evaluated on a test set $\{(x'_1,y'_1),\dots,(x'_m,y'_m)\}$ independent from the training set and i.i.d. sampled from $\PP$.  To find this classifier $h$, we learn a function $\mathbf{f}:\XX\to\mathbb{R}^K$ returning scores, or logits, $(f_1(x),\dots,f_K(x))$ corresponding to each label. Then $h$ is set to $h(x)=\argmaxB_k f_k(x)$. The function $\mathbf{f}$ is usually learned by minimizing the empirical risk for a certain convenient loss function $L$ over some class of functions $\mathcal{H}$.
\begin{align*}
\inf_{\mathbf{f}\in\mathcal{H}}\riskemp_{n}(\mathbf{f}):= \frac{1}{n}\sum_{i=1}^nL(\mathbf{f}(x_i),y_i).
\end{align*}

This problem is called Empirical Risk Minimization (ERM). The theory of this problem have been widely studied and is well understood . It is often argued that there is a tradeoff on the ``size'' of $\mathcal{H}$: having a too small $\mathcal{H}$ may lead to underfitting, i.e. not enough parameters to describe the optimal possible function while a too large $\mathcal{H}$ may lead to overfitting, i.e.


added to the ERM objective to prevent from overfitting. This tradeoff was recently questioned by the double descent~\citep{belkin2019reconciling} phenomenon where overparametrized (i.e. number of parameters largely over the number of training samples) regimes lower the risk.
ADD A TRADEOFF GRAPH

The presence of adversaries in classification question the knowledge we have in standard statistical learning. Indeed most standard results do not hold in presence of adversaries, hence, opening a new research area dedicated to studying and understanding the classification problem in presence of adversarial attacks.

\subsection{Classification in Presence of Adversarial Attacks}

Yet a model can be very well performing on natural samples, small perturbations of these natural samples can lead to unexpected and critical behaviours of classification models~\citep{biggio2013evasion,Szegedy2013IntriguingPO}. To formalize that, we will assume the existence of a ``perception'' distance $d:\XX^2\to\mathbb{R}$ such that a perturbation $x'$ of an input $x$ remains imperceptible if $d(x,x')\leq \varepsilon$ for some constant $\varepsilon\geq0$. This ``perception'' distance is difficult to define in practice. For images, the $\lVert\cdot\rVert_\infty$ distance over pixels is often used to describe perception in practice, but is not able to capture all imperceptible perturbations.  This choice is purely arbitrary: for instance, we will highlight in the manuscript that $\lVert\cdot\rVert_2$ perturbations can also be imperceptible while having a large $\lVert\cdot\rVert_\infty$. ADD GEOMETRIC PERTURBATIONS

Therefore, the goal of an attacker is to craft an adversarial input $x'$ from an input $x$ that is imperceptible , i.e. $d(x,x')\leq \varepsilon$ and misclassifies the input, i.e. $h(x')\neq y$. Such a sample $x'$ is called an adversarial attack. The used criterion cannot be the misclassification rate anymore, we need to take into account the possible presence of an adversary that maliciously perturb the input. We then define the robust/adversarial misclassification rate or  misclassification rate or adversarial $0/1$ loss risk: 

\begin{align*}
\risk^\varepsilon_{0/1}(h)&:=\PP_{(x,y)}(\exists x'\in\XX\text{ s.t. } d(x,x')\leq \varepsilon \text{ and } h(x')\neq y)\\
&= \EE_{(x,y)\sim\PP}\left[\sup_{x'\in\XX\text{ s.t. } d(x,x')\leq \varepsilon}\mathbf{1}_{h(x')\neq y}\right]
\end{align*}


Like in standard risk minimization, we aim to learn a function $\mathbf{f}:\mathcal{X}\to\mathbb{R}^K$ such that $h(x)=\argmaxB_k f_k(x)$. Usually in adversarial classification we aim at solving the following optimization problem, that we will call adversarial empirical risk minimization:

\begin{align*}
\inf_{\mathbf{f}\in\mathcal{H}}\riskemp^\varepsilon_{L}(\mathbf{f}):= \frac{1}{n}\sum_{i=1}^n\sup_{x'\in\XX\text{ s.t. } d(x,x')\leq \varepsilon} L(\mathbf{f}(x_i),y_i).
\end{align*}

This problem is a lot more challenging to solve than the standard risk minimization problem since it involves a inner supremum problem that might difficultly tractable~\citep{madry2017towards}. Guarantees in the adversarial setting are therefore difficult to obtain both in terms of optimization and statistical guarantees. The usual technique to solve this problem is called Adversarial Training~\citep{goodfellow2014explaining,madry2017towards} and consists in alternating inner and outer optimization problems. Such a technique improves in practice adversarial robustness but have no theoretical guarantees. So far, most results and advances in understanding on adversarial attacks are empirical~\citep{ilyas2019adversarial,rice2020overfitting}, leaving many theoretical question open.  Moreover, robust models suffers from a performance drop and vulnerablity of models in currently still very high~\ref{table:sota-cifar}, highlighting the problem is far to be solved.

\begin{table}[!ht]
    \centering
    \begin{tabular}{c|c|c|c}
       \textbf{Attacker}  &  \textbf{Paper reference} & \textbf{Standard Acc.} & \textbf{Robust Acc.}  \\ \hline
        None & \citep{ZagoruykoK16} & 94.78\% & 0\%\\
        $\ell_\infty (\varepsilon=8/255)$&  & 89.48\% & 62.76\%\\
    \end{tabular}
    \caption{State of the art accuracieson adversarial tasks on a WideResNet 28x10~\citep{ZagoruykoK16}. Results are reported from~\citep{croce2020robustbench}}
\label{table:sota-cifar}
\end{table}

\section{Outline and Contributions}
We will in a first section introduce some background and essential notions regarding Machine Learning and Adversarial Examples. ADD STH
\subsection{A Game Theoretic Approach to Adversarial Attacks}

In a first part, we will analyse adversarial examples as part of a game theoretic framework. We will define clearly the motivations for both the attacker and the classifier. We will cast it naturally as a zero sum game. When it comes to study a game, a question that naturally arises is the existence of equilibria.
\medskip
\begin{tcolorbox}[title=Question 1]
\textbf{What is the nature of equilibria in the adversarial examples game?}
\end{tcolorbox}
\medskip

In game theory, there are many types of equilibria. In this manuscript, we will focus on Stackelberg and Nash equilibria. We will show the existence of both when both the classifiers and the attacker play randomized strategies. To reach such equilibria, the classifier will be random, and the attacker will move randomly samples at a maximum distance of $\varepsilon$. Then, we will propose two different algorithms to compute the optimal randomized classifier in the case of a finite number of possible classifiers. We will finally propose a heuristic to train a mixture of neural networks and show experimentally improvements with regards to standard methods.



\subsection{Loss Consistency in Classification in Presence of an Adversary}
In standard classification, consistency with regards to $0/1$ loss is a desired property for the loss $L$ used in the learning algorithm for the optimal classifier. In short, a loss $L$ is said to be consistent if for every probability distribution, a sequence of classifier $(f_n)$ that minimizes the risk associated with the loss $L$, it also minimizes the $0/1$ loss risk. Usually, the problem is simplified thanks to the notion of calibration. We will see that the question of consistency is the adversarial problem is much harder.
\medskip
\begin{tcolorbox}[title=Question 2]
\textbf{Which losses are consistent with regards to adversarial classification loss?}
\end{tcolorbox}
\medskip
We tackle this question by showing usual convex losses are not calibrated for the adversarial classification loss. Hence this negative result emphasize the difficulty of understanding the adversarial attack problem. 

\subsection{Building Certifiable Models}

The last problem we tackle in this manuscript is the implementation of robust certifiable models. This problem is challenging and it is often difficult to get non vacuous bounds on adversarial guarantees.
\medskip
\begin{tcolorbox}[title=Question 3]
\textbf{How to efficiently implement certifiable models with non-vacuous guarantees?}
\end{tcolorbox}
\medskip
To this end, we propose two methods that enforces Lipschitzity on the predictions of neural networks:
\begin{enumerate}
    \item The first one consists in noise injection. We show that by adding a noise on an input of a classifier, we are able to get guarantees on the decision up to some level $\varepsilon$.
    \item A second one consists in building contractive blocks in a ResNet architecture. This method takes its inspiration from the continuous flow interpretation of residual networks. More precisely, we show that using a gradient flow of a convex function, our network is $1$-Lipschitz. We then design such function, showing empirically and theoretically the robustness benefits of such an approach.
\end{enumerate}

% \section{Other Works in Appendix}


% In addition to these works, we also add interest in adversarial attacks against online recommendation systems, namely Linear Contextual Bandits. We designed algorithms that provably fool in 

% We also published works in the field of Derivative Free Optimization FOLLOW