\section{A Framework to design Lipschitz Layers}
\label{section:global_framework}


The continuous time interpretation of Definition~\ref{def:flow} allows us to better investigate the robustness properties and assess how a difference of the initial values (the inputs) impacts the inference flow of the model. Let us consider two continuous flows $x_t$ and $z_t$ associated with $F_t$ but differing in their respective initial values $x_0$ and $z_0$. Our goal is to characterize the time evolution of $\lVert x_t-z_t \rVert$ by studying its  time derivative. We recall that  every matrix $M\in\RR^{d\times d}$ can be uniquely decomposed as the sum of a symmetric and skew-symmetric matrix $M = S(M) + A(M)$. By applying this decomposition to the Jacobian matrix $\nabla_x F_t(x)$ of $F_t$, we can show that the time derivative of $\lVert x_t-z_t \rVert$ only involves the symmetric part  $S(\nabla_x F_t(x))$ (see Appendix~\ref{proof:continuous-lip} for details). 

For two symmetric matrices $S_1,S_2\in\RR^{d\times d}$,  we denote $S_1\preceq S_2$ if, for all $x\in\RR^d$, $\langle x,(S_2-S_1)x\rangle\geq 0$. By focusing on the symmetric part of the Jacobian matrix we can show in Appendix~\ref{proof:continuous-lip} the following proposition.
\begin{prop}
\label{prop:continuous-lip}
Let $(F_{t})_{t\in[0,T]}$ be a family of differentiable  functions almost everywhere on $\RR^d$.
Let us assume that there exists two measurable functions $t\mapsto \mu_t$ and  $t\mapsto \lambda_t$ such that
$$\mu_t I\preceq S(\nabla_xF_{t}(x))\preceq \lambda_tI$$
for all $x\in\RR^d$, and $t\in [ 0,T]$. Then the flow associated with $F_t$ satisfies for all initial conditions $x_0$ and $z_0$:
\begin{align*}
  \lVert x_0-z_0 \rVert e^{\int_0^t\mu_s ds}\leq \lVert x_t-z_t \rVert\leq \lVert x_0-z_0 \rVert e^{\int_0^t\lambda_s ds}
\end{align*}
\end{prop}

The symmetric part plays even a more important role since one can show that a function whose Jacobian is always skew-symmetric is actually linear (see Appendix~\ref{sup:skew} for more details). However, constraining $S(\nabla_x F_{t}(x))$ in the general case is technically difficult and a solution resorts to a more intuitive parametrization of  $F_t$ as the sum of two functions $F_{1,t}$ and $F_{2,t}$ whose Jacobian matrix are respectively symmetric  and skew-symmetric.  Thus, such a parametrization enforces $F_{2,t}$  to be linear and skew-symmetric. For the choice of $F_{1,t}$, we propose to rely on potential functions: a function  $F_{1,t}:\RR^d \to \RR^d$ derives from a simpler family of scalar valued function in $\RR^d$, called the \emph{potential}, via the gradient operation. Moreover, since the Hessian of the potential is symmetric, the Jacobian for $F_{1,t}$ is then also symmetric.  If we had the convex property to this potential, its Hessian has positive eigenvalues. Therefore we introduce the following corollary. See proof in Appendix~\ref{proof:conv-skew} 

\begin{corollary} 
\label{cor:conv-skew}Let $(f_{t})_{t\in[0,T]}$ be a family of convex differentiable functions on $\RR^d$ and $(A_t)_{t\in[0,T]}$ a family of skew symmetric matrices. Let us define 
$$F_t(x) = -\nabla_x f_{t}(x)+A_t x,$$ 
then the flow associated with $F_t$ satisfies for all initial conditions $x_0$ and $z_0$:
\begin{align*}
\lVert x_t-z_t \rVert\leq \lVert x_0-z_0 \rVert
\end{align*}
\end{corollary}

This simple property suggests that if we could parameterize $F_t$  with convex potentials, it would be less sensitive to input perturbations and therefore more robust to adversarial examples. We also remark that the skew symmetric part is then norm-preserving.
However, the discretization of such flow is challenging in order to maintain this property of stability. 


\subsection{Discretized Flows}

To study the discretization of  the previous flow, let $t=1,\dots,T$ be the discretized time steps and from now we consider the flow defined by  $F_t(x) = -\nabla f_{t}(x)+A_t x$, with $(f_{t})_{t=1,\dots,T}$  a family of convex differentiable functions on $\RR^d$ and $(A_t)_{t=1,\dots,T}$ a family of skew symmetric matrices. The most basic method the explicit Euler scheme as defined by: 
\begin{align*}
x_{t+1} = x_t+ F_t(x_t)
\end{align*}
However, if $A_t\neq 0$, this discretized system might not satisfy $\lVert x_t-z_t\rVert\leq\lVert x_0-z_0\rVert$. Indeed, consider the simple example where $f_t=0$. We then have:
\begin{align*}
\lVert x_{t+1}-z_{t+1}\rVert - \lVert x_{t}-z_{t}\rVert =\lVert A_t\left(x_{t}-z_{t}\right)\rVert.
\end{align*}
Thus explicit Euler scheme cannot guarantee Lipschitzness when $A_t\neq 0$. To overcome this difficulty, the discretization step can be split in two parts, one for $\nabla_x f_t$ and one for $A_t$: 
\begin{align*}
   \left\{
    \begin{array}{ll}
        x_{t+\frac12} &= \textsc{step1}(x_t, \nabla_x f_t)\\
        x_{t+1}& = \textsc{step2}(x_{t+\frac12}, A_t)
    \end{array} 
    \right.
\end{align*}
This type of discretization scheme  can be found for instance from Proximal Gradient methods where one step is explicit and the other is implicit. Then, we dissociate the Lipschitzness study of both terms of the flow. 

\subsection{Discretization scheme for $\nabla_x f_t$}

To apply the explicit Euler scheme to $\nabla_x f_t$, an  additional smoothness property on the potential functions is required  to generalize the Lipschitzness guarantee to the discretized flows. Recall that a function $f$ is said to be \emph{$L$-smooth} if it is differentiable and if $x\mapsto\nabla_x f(x)$ is $L$-Lipschitz. 
\begin{prop}\label{prop:discrete_convex_potentials}
Let $t\in\{1,\cdots,T\}$ Let us assume that $f_{t}$ is $L_t$-smooth. We  define the following discretized ResNet gradient flow using $h_t$ as a step size:
\begin{align*}
    \begin{array}{ll}
    x_{t+\frac12} &= x_{t}-h_{t}\nabla_xf_{t}(x_{t})\\
  \end{array}
 \end{align*}
Consider now two trajectories $x_t$ and $z_t$ with initial points $x_0=x$ and $z_0=z$ respectively,  if $0\leq h_t\leq \frac{2}{L_t}$,  then 
$$\lVert x_{t+\frac12}-z_{t+\frac12}\rVert_2\leq \lVert x_t-z_t\rVert_2$$
\end{prop}
In Section~\ref{sec:param}, we describe how to parametrize a neural network layer to implement such a discretization step by leveraging the recent work on Input Convex Neural Networks~\cite{amos2017input}. 

\begin{rmq}
Another solution relies on the implicit Euler scheme: $ x_{t+\frac12} = x_{t}-\nabla_xf_{t}(x_{t+\frac12})$. We show in Appendix~\ref{sup:implicit} that this strategy defines a $1$-Lipschitz flow without further assumption on $f_t$ than convexity. We propose an implementation. However preliminary experiments did not show competitive results and the training time is prohibitive. We leave this solution for future work. 

\end{rmq}
\subsection{Discretization scheme for $A_t$}

The second step of discretization involves the term with skew-symmetric matrix $A_t$. As mentioned earlier, the challenge is that the \emph{explicit Euler discretization} is not contractive. More precisely,  the following property 
$$\lVert x_{t+1} - z_{t+1}\rVert\geq \lVert x_{t+\frac12} - z_{t+\frac12}\rVert$$ 
is satisfied with equality only in the special and useless case of $x_{t+\frac12} - z_{t+\frac12} \in \text{ker}(A_t)$. Moreover, the implicit Euler discretization induces an increasing norm and hence does not satisfy the desired property of norm preservation neither. 

\paragraph{Midpoint Euler method.}
We thus propose to use \emph{Midpoint Euler} method, defined as follows:
\begin{align*}
&x_{t+1} = x_{t+\frac12} +A_t \frac{x_{t+1}+x_{t+\frac12}}{2}\\
\iff&x_{t+1} = \left(I-\frac{A_t}{2}\right)^{-1}\left(I+\frac{A_t}{2}\right)x_{t+\frac12}.
\end{align*} 
Since $A_t$ is skew-symmetric, $I-\frac{A_t}{2}$ is invertible. This update corresponds to the Cayley Transform of $\frac{A_t}{2}$ that induces an orthogonal mapping. 
This kind of layers was introduced and extensively studied in~\citet{trockman2021orthogonalizing}.  


\paragraph{Exact Flow.} One can define the simple differential equation corresponding to the flow associated with $A_t$  
\begin{align*}
        \frac{du_t}{ds} = A_t u_s,\quad u_0 = x_{t+\frac12},
\end{align*}
There exists an exact solution exists since $A_t$ is linear. By taking the value at $s=\frac12$, we obtained the following transformation:  
 \begin{align*}
x_{t+1} := u_{\frac12}=e^{\frac{A}{2}} x_{t+\frac12}.
\end{align*}
This step is therefore clearly norm preserving but the matrix exponentiation is challenging and it requires efficient approximations. This trend was recently investigated under the name of Skew Orthogonal Convolution (SOC)~\cite{skew2021sahil}.


\section{Parametrizing Convex Potentials Layers}
\label{sec:param}

As presented in the previous section, parametrizing the skew symmetric updates has been extensively studied by~\citet{trockman2021orthogonalizing,skew2021sahil}. In this paper we focus on  the parametrization of symmetric update with the convex potentials proposed in~\ref{prop:discrete_convex_potentials}. For that purpose, the Input Convex Neural Network (ICNN) \citep{amos2017input} provide a relevant starting point that we will extend. 

\subsection{Gradient of ICNN}
We use $1$-layer ICNN~\citep{amos2017input} to define an efficient computation of Convex Potentials Flows. For any vectors $w_1,\dots w_k\in\mathbb{R}^d$, and bias terms  $b_1,\dots,b_k\in \mathbb{R}$, and for $\phi$ a convex function,  the potential $F$ defined as:
\begin{align*}
    F_{w,b}:x\in\RR^d\mapsto \sum_{i=1}^k\phi( w_i^\top x+b_i)
\end{align*}
defines  a convex function in $x$ as the composition of a linear and a convex function. Its gradient with respect to its input $x$ is then:
\begin{align*}
    x\mapsto \sum_{i=1}^kw_i\phi'(w_i^\top x+b_i) = \mathbf{W}^\top \phi'(\mathbf{W} x+\mathbf{b})
\end{align*}
with $\mathbf{W}\in \mathbb{R}^{k\times d}$ and $\mathbf{b}\in\mathbb{R}^{k}$ are respectively the matrix and vector obtained by the concatenation of, respectively, $w_i^\top$ and $b_i$, and $\phi'$ is applied element-wise.  
Moreover, assuming $\phi'$ is $L$-Lipschitz, we have that $F_{w,b}$ is  $L\lVert\mathbf{W}\rVert_2^2$-smooth. $\lVert\mathbf{W}\rVert_2$ denotes the spectral norm of $\mathbf{W}$.
% , \ie, the greatest singular value of $\mathbf{W}$ defined as:
% \begin{align*}
%    \lVert\mathbf{W}\rVert_2 :=\max_{x\neq 0} \frac{\lVert \mathbf{W}x\rVert_2}{\lVert x\rVert_2}
% \end{align*}
The reciprocal also holds: if $\sigma:\RR\to\RR$ is a non-decreasing $L$-Lipschitz function, $\mathbf{W}\in \RR^{k\times d}$ and $b\in \RR^{k}$, there exists a convex $L\lVert\mathbf{W}\rVert_2^2$-smooth function $F_{w,b}$ such that 
$$\nabla_xF_{w,b}(x) =  \mathbf{W}^\top \sigma(\mathbf{W} x+\mathbf{b}),$$ where $\sigma$ is applied element-wise. The next section shows how this property can be used to implement the building block and training of such layers. 


\subsection{Convex Potential layers}
From the previous section, we derive the following \emph{Convex Potential Layer}: 
\begin{equation*}
\label{equation:stable_block}
  z = x - \frac{2}{\lVert \mathbf{W} \lVert_2^2} \mathbf{W}^\top \sigma(\mathbf{W} x + b)
\end{equation*}
Written in a matrix form, this layer can be implemented with every linear operation $\mathbf{W}$.
In the context of image classification, it is beneficial to use convolutions\footnote{For instance, one can leverage the \texttt{Conv2D} and \texttt{Conv2D\_transpose} functions of the PyTorch framework~\citep{paszke2019pytorch}} instead of generic linear transforms represented by a dense matrix. 

\begin{rmq}
When $\mathbf{W}\in\RR^{1\times d}$, $b =0$ and $\sigma=\textsc{ReLU}$, the \emph{Convex Potential Layer} is equivalent to the HouseHolder activation function introduced in~\citet{singla2021householder}.
\end{rmq}

Residual Networks~\citep{he2016deep} are also composed of other types of layers which increase or decrease the dimensionality of the flow.
Typically, in a classical setting, the number of input channels is gradually increased, while the size of the image is reduced with pooling layers.
In order to build a $1$-Lipschitz Residual Network, all operations need to be properly scale or normalize in order to maintain the Lipschitz constant.

\paragraph{Increasing dimensionsionality.} To increase the number of channels in a convolutional Convex Potential Layer, a zero-padding operation can be easily performed: an input $x$ of size $c\times h \times w$ can be extended to some $x'$ of size  $c'\times h \times w$, where $c'>c$, which equals $x$ on the $c$ first channels and $0$ on the $c'-c$ other channels.
\paragraph{Reducing dimensionsionality.} Dimensionality reduction is another essential operation in neural networks. On one hand, its  goal is to  reduce the number of parameters and thus the amount of computation required to build the network. On the other hand it allows the model to progressively map the input space on the output dimension, which corresponds in many cases to the number of different labels $K$. 
In this context, several operations exist:
pooling layers are used to extract information present in a region of the feature map generated by a convolution layer. One can easily adapt pooling layers (\emph{e.g.} max and average) to make them $1$-Lipschitz~\citep{bartlett2017spectrally}.
Finally, a simple method to reduce the dimension is the product with a non-square matrix. In this paper, we simply implement it as  the truncation of the output. This obviously maintains the Lipschitz constant.


\begin{algorithm}[tb]
\caption{Computation of a Convex Potential Layer}
\label{algorithm:stable_block}
\begin{algorithmic}
  \STATE{Require: \bfseries Input: $x$, vector: $u$, weights: $\mathbf{W}$, $b$}
  \STATE{Ensure: Compute the layer $z$ and return $u$}
  \STATE{$v \gets \mathbf{W} u / \lVert \mathbf{W} u \rVert_2$}
  \STATE{$u \gets \mathbf{W}^\top v / \lVert \mathbf{W}^\top v \rVert_2$
    \rlap{\hspace{0.5cm}\smash{$\left.\begin{array}{@{}c@{}}\\{}\\{}\end{array}\right\}%
      \begin{tabular}{l}1 iter. for training \\100 iter. for inference\end{tabular}$}}}
  \STATE{$h \gets 2 / \left( \sum_i (\mathbf{W} u \cdot v)_i \right)^2$}
  \STATE{\textbf{return} $x - h \left[ \mathbf{W}^\top \sigma( \mathbf{W} x + b) \right], u$}
\end{algorithmic}
\end{algorithm}




\subsection{Computing spectral norms}
Our Convex Potential Layer, described in Equation~\ref{equation:stable_block}, can be adapted to any kind of linear transformations (\emph{i.e.} Dense or Convolutional) but requires the computation of the spectral norm for these transformations.
Given that computation of the spectral norm of a linear operator is known to be NP-hard~\citep{steinberg2005computation}, an efficient approximate method is required during training to keep the complexity tractable. 


Many techniques exist to approximate the spectral norm (or the largest singular value), and most of them exhibit a trade-off between efficiency and accuracy.
Several methods exploit the structure of convolutional layers to build an upper bound on the spectral norm of the linear transform performed by the convolution~\citep{jia2017improving,singla2021fantastic,araujo2021lipschitz}.
While these methods are generally efficient, they can less relevant and adapted to certain settings. For instance in our context, using a loose upper bound of the spectral norm will hinder the expressive power of the layer and make it too contracting.

For these reasons we rely on the Power Iteration Method (PM).  This method converges at a geometric rate towards the largest singular value of a matrix. More precisely the convergence rate for a given matrix $\mathbf{W}$ is $\textstyle O((\frac{\lambda_2}{\lambda_1})^k)$ after $k$ iterations, independently from the choice for the starting vector, where $\lambda_1>\lambda_2$ are the two largest singular values of $\mathbf{W}$. While it can appear to be computationally expensive due to the large number of required iterations for convergence, it is possible to drastically reduce the number of iterations during training. Indeed, as in~\citep{miyato2018spectral}, by considering that the weights' matrices $\mathbf{W}$ change slowly during training, one can perform only one iteration of the PM for each step of the training and let the algorithm slowly converges along with the training process\footnote{Note that a typical training requires approximately 200K steps where 100 steps of PM is usually enough for convergence}.
We describe with more details in Algorithm~\ref{algorithm:stable_block}, the operations performed during a forward pass with a Convex Potential Layer. 

However for evaluation purpose, we need to compute the certified adversarial robustness, and this requires to ensure the convergence of the PM. Therefore, we perform $100$ iterations for each layer\footnote{$100$ iterations of Power Method is sufficient to converge with a geometric rate.} at inference time. Also note that at inference time, the computation of the spectral norm only needs to be performed once for each layer. 


