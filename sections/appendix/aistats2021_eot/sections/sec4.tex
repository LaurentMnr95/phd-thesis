% \section{Primal Formulation of the Problem}
% In fact we manage to obtain the primal formulation of the problem. Let us define $$\Pi_{P,Q}:=\left\{(\gamma_1,\gamma_2)\in\mathcal{M}^+(\mathcal{X}\times\mathcal{Y})^2\text{ s.t. } \Pi_{1\#}\gamma_1+ \Pi_{1\#}\gamma_2=P \text{ and } \Pi_{2\#}\gamma_1+ \Pi_{2\#}\gamma_2=Q \right\}.$$
% \begin{defn}
% Let $P$ and $Q$ two probability measures on respectively $\mathcal{X}$ and $\mathcal{Y}$, 
% then we define 
% \begin{align*}
% \text{POT}_{c_1,c_2}(P,Q):=\inf_{(\gamma_1,\gamma_2)\in \Pi_{P,Q}} \max\left(\int_{\mathcal{X}\times\mathcal{Y}}c_1(x,y) d\gamma_1(x,y) ,\int_{\mathcal{X}\times\mathcal{Y}}c_2(x,y) d\gamma_2(x,y)\right)
% \end{align*}
% \end{defn}

% \begin{thm}
% Let $P$ and $Q$ two probability measures on respectively $\mathcal{X}$ and $\mathcal{Y}$, as soon as  $c_1$, $c_2$ .....  then we have:
% \begin{align}
% \label{thm:duality-GOT}
% \text{POT}_{c_1,c_2}(P,Q) =\text{GOT}_{c_1,c_2}(P,Q)
% \end{align}
% \end{thm}
