\section{Technical lemmas}
\label{sec:teclemmas}
In this section, we prove two technical lemmas on $f$ that will be useful to study the convergence of the algorithm. The first one shows that $f$ can be upper bounded and lower bounded by two spherical functions. 
\begin{lemma}
\label{lem:sandwich}
Under Assumption~\ref{ass:principal}, there exist two real numbers $0<l\leq L$, such that, for all $x\in B(0,r)$:
\begin{align}
\label{eq:lip-cond}
   l\lVert x-x^\star\rVert^2 \leq f(x)\leq L\lVert x-x^\star\rVert^2.
\end{align}
Moreover such $l$ and $L$ must satisfy  $0< l\leq e_1(\mathbf{H})\leq e_d(\mathbf{H})\leq L$.
\end{lemma}
\begin{proof}
As $\mathbf{H}$ is symmetric positive definite, we have the following
classical inequality for the $\mathbf{H}$-norm
\begin{equation}\label{eq:sym-mat-ineq}
e_{1}(\mathbf{H})\lVert x-x^{\star}\rVert^{2}\le\lVert x-x^{\star}\rVert_{\mathbf{H}}^{2}\le e_{d}(\mathbf{H})\lVert x-x^{\star}\rVert^{2}
\end{equation}
Now set for $x\in B(0,r)\setminus\{x^{\star}\}$
\[
\phi(x):=\frac{f(x)-f(x^{\star})}{\lVert x-x^{\star}\rVert^{2}}=\frac{\lVert x-x^{\star}\rVert_{\mathbf{H}}^{2}}{\lVert x-x^{\star}\rVert^{2}}(1+\lVert x-x^{\star}\rVert_{\mathbf{H}}^{\alpha-2}\varepsilon(x-x^{\star})).
\]
By the above inequalities, we have 
\[
e_{1}(\mathbf{H})^{(\alpha-2)/2}\lVert x-x^{\star}\rVert^{\alpha-2}\le\lVert x-x^{\star}\rVert_{\mathbf{H}}^{\alpha-2}\le e_{d}(\mathbf{H})^{(\alpha-2)/2}\lVert x-x^{\star}\rVert^{\alpha-2}.
\]
Thus, as $\alpha>2$, we obtain $\lVert x-x^{\star}\rVert_{\mathbf{H}}^{\alpha-2}\rightarrow_{x\rightarrow x^{\star}}0$.
By assumption, the function $\varepsilon$ is also bounded as $x\rightarrow x^{\star}$.
\\
We thus conclude that there exists $\delta>0$ such that, for all
$x\in\overset{\circ}{B}(x^{\star},\delta)$
\[
\frac{1}{2}e_{1}(\mathbf{H})\le\phi(x)\le2e_{d}(\mathbf{H}).
\]
Now notice that $B(0,r)\setminus\overset{\circ}{B}(x^{\star},\delta)$
is a closed subset of the compact set $B(0,r)$ hence it is also compact.
Moreover, by assumption $f$ is continuous on $B(0,r)$ and $f(x)>0=f(x^{\star})$
for all $x\neq x^{\star}.$ Hence $\phi$ is continuous and positive
on this compact set. Thus it attains its minimum and maximum on this
set and its minimum is positive. In particular, we can write, on this set, for
some $l_{0},L_{0}>0$
\[
l_{0}\le\phi(x)\le L_{0}.
\]
We now set $l=\min\{l_{0},\frac{1}{2}e_{1}(\mathbf{H})\}$. Note that
$l>0$ because $l_{0}>0$ and $e_{1}(\mathbf{H})>0$ (as $\mathbf{H}$
is positive definite). We also set $L=\max\{L_{0},2e_{1}(\mathbf{H})\}$
which is also positive. These are global bounds for $\phi$ which gives the first part of the result.\\
For the second part, let $\mathbf{u}_{1}$ be a normalized eigenvector
respectively associated to $e_{1}(\mathbf{H})$. Then 
\begin{align*}
\frac{f(x^{\star}+\epsilon\mathbf{u}_{1})}{\lVert\epsilon\mathbf{u}_{1}\rVert^{2}}=e_{1}(\mathbf{H})+\epsilon^{\alpha-2}\varepsilon(\epsilon\mathbf{u}_{1})
\end{align*}
Taking the limit as $\epsilon\to0$. we get that, if $l$ satisfies~\eqref{eq:lip-cond},
then $l\leq e_{1}(\mathbf{H})$. Similarly, we can prove that $L\geq e_{d}(\mathbf{H})$.\end{proof}
Secondly, we frame $S_h$ into two ellipsoids as $h\to 0$. This lemma is a consequence of the assumptions we make on $f$.
\begin{lemma}
\label{lemma:sandwich-set}
Under Assumption~\ref{ass:principal}, there exists $h_0\geq 0$ such that for $h\leq h_0$, we have $A_h\subset S_h\subset B_h$ where:
\begin{align*}
A_h:=\{x\mid  \lVert x-x^\star\rVert_{\mathbf{H}}\leq \phi_-(h)\}\\
B_h:=\{x\mid   \lVert x-x^\star\rVert_{\mathbf{H}}\leq \phi_+(h)\}
\end{align*}
with $\phi_-(h)$ and $\phi_+(h)$ two functions satisfying 
\begin{eqnarray*}\phi_-(h)&=&\sqrt{h}-\frac{M}{2}h^{(\alpha-1)/2}+o(h^{(\alpha-1)/2}) \\
\mbox{ and } \phi_+(h)&=&\sqrt{h}+\frac{m}{2}h^{(\alpha-1)/2}+o(h^{(\alpha-1)/2})\end{eqnarray*} when $h\to 0$ for some constants $m>0$ and $M>0$ which are respectively a (specific) lower and upper bound for $\varepsilon$.
\end{lemma}
\begin{proof}
By assumption $\lvert\varepsilon\rvert\leq M$, hence we have: 
\begin{align*}
\{x\in B(0,r) & \mid\lVert x-x^{\star}\rVert_{\mathbf{H}}^{2}+M\lVert x-x^{\star}\rVert_{\mathbf{H}}^{\alpha}\leq h\}\subset S_{h}
\end{align*}
Let $g\colon u\mapsto u^{2}+Mu^{\alpha}$. This is a continuous,
strictly increasing function on $[0,+\infty)$. By a classical consequence
of the intermediate value theorem, this implies that $g$ admits a
continuous, strictly increasing inverse function. Note that $g(0)=0$
hence $g^{-1}(0)=0$. Thus we can write $\{u\geq 0|u^{2}+Mu^{\alpha}\le h\}=[0,g^{-1}(h)]$.
We now denote $g^{-1}$ by $\phi_{-}$. As $\phi_{-}$ is non-decreasing,
we get
\begin{align*}
\{x\in B(0,r) & \mid\lVert x-x^{\star}\rVert_{\mathbf{H}}^{2}+M\lVert x-x^{\star}\rVert_{\mathbf{H}}^{\alpha}\leq h\}=A_{h}\cap B(0,r)
\end{align*}
Now observe that for $h$ sufficiently small
\[
\{x\in B(0,r)\mid\lVert x-x^{\star}\rVert_{\mathbf{H}}^{2}+M\lVert x-x^{\star}\rVert_{\mathbf{H}}^{\alpha}\leq h\}=A_{h}.
\]
Indeed, if $x\in A_{h}$, we have by the triangle inequality and~\eqref{eq:sym-mat-ineq}
\begin{align*}
\lVert x\rVert & \le\lVert x^{\star}\rVert+\lVert x-x^{\star}\rVert\\
 & \le\lVert x^{\star}\rVert+e_{1}(\mathbf{H})^{-1/2}\lVert x-x^{\star}\rVert_{\mathbf{H}}\\
 & \le\lVert x^{\star}\rVert+e_{1}(\mathbf{H})^{-1/2}\phi_{-}(h)
\end{align*}
Recall that by assumption $\lVert x^{\star}\rVert<r$ and let $\delta=r-\lVert x^{\star}\rVert$.
As $\phi_{-}(h)\rightarrow_{h\rightarrow0}0$, for $h$ sufficiently
small, we have $e_{1}(\mathbf{H})^{-1/2}\phi_{-}(h)\le\delta$ hence
$\lVert x\rVert\le r$ for $h$ sufficiently small, which gives the inclusion $A_h \subset S_h$.\\
For the asymptotics of $\phi_{-}$, as we have by definition $\phi_{-}(h)^{2}(1+M\phi_{-}(h)^{\alpha-2})=h$,
and as $\phi_{-}(h)\rightarrow_{h\rightarrow0}0$ we deduce that $\phi_{-}(h)\sim_{0}\sqrt{h}$.
Let us define $u(h)=\phi_{-}(h)-\sqrt{h}$. We have $u(h)\in o(\sqrt{h})$.
We then compute: 
\begin{align*}
(\sqrt{h}+u(h))^{2}+M(\sqrt{h}+u(h))^{\alpha}=h
\end{align*}
This gives
\begin{align*}
u(h)(u(h)+2\sqrt{h}) & =-Mh^{\alpha/2}(1+\frac{u(h)}{\sqrt{h}})^{\alpha}\\
u(h)(\frac{u(h)}{2\sqrt{h}}+1) & =-\frac{M}{2}h^{(\alpha-1)/2}(1+\frac{u(h)}{\sqrt{h}})^{\alpha}
\end{align*}
As $u(h)\in o(\sqrt{h})$ for $h\rightarrow0$, we obtain
\[
u(h)\sim-\frac{M}{2}h^{(\alpha-1)/2}.
\]
which concludes for $\phi_{-}$.

On the other side, we recall that $f(x)>0$ for all $x\neq x^{\star}$
as $x^{\star}$ is the unique minimum of $f$ on $B(0,r)$. Write
\[
0<\lVert x-x^{\star}\rVert_{\mathbf{H}}^{2}(1+\lVert x-x^{\star}\rVert_{\mathbf{H}}^{\alpha-2}\varepsilon(x-x^{\star})).
\]
Now observe that, as $\lVert x^{\star}\rVert<r$, we have for $x\in B(0,r)$,
by the triangle inequality, $\lVert x-x^{\star}\rVert<2r$. Hence,
by the classical inequality for the $\mathbf{H}$-norm~\eqref{eq:sym-mat-ineq}, we get
\begin{align*}
\varepsilon(x-x^{\star}) & >-\frac{1}{\lVert x-x^{\star}\rVert_{\mathbf{H}}^{\alpha-2}}\geq-\left(\sqrt{e_{d}(\mathbf{H})}2r\right)^{-(\alpha-2)}=:-m
\end{align*}
So we have: 
\begin{align*}
S_{h}\subset\{x\in B(0,r) & \mid\lVert x-x^{\star}\rVert_{\mathbf{H}}^{2}-m\lVert x-x^{\star}\rVert_{\mathbf{H}}^{\alpha}\leq h\}
\end{align*}
The function $g\colon u\mapsto u^{2}-mu^{\alpha}$ is differentiable.
A study of the derivative shows that $g$ is continuous, strictly
increasing on $[0,r_{0}]$ and continuous, strictly decreasing on
$[r_{0},+\infty[$ where $r_{0}=(\frac{2}{\alpha m})^{1/(\alpha-2)}$.
Hence $g_{|[0,r_{0}]}$ admits a continuous strictly increasing inverse
$\phi_{+}$ and $g_{|[r_{0},+\infty[}$ a continuous strictly decreasing
inverse $\Tilde{\phi}$. We thus write 
\[
\{u\ge0|u^{2}-mu^{\alpha}\le h\}=[0,\phi_{+}(h)]\cup[\Tilde{\phi}(h),+\infty).
\]
Hence 
\begin{align*}
 \{x\in B(0,r)\mid&\lVert x-x^{\star}\rVert_{\mathbf{H}}^{2}-m\lVert x-x^{\star}\rVert_{\mathbf{H}}^{\alpha}\leq h\}\\
&=\big(B_{h}\cap B(0,r)\big)\cup\big(B(0,r)\cap V_{h}\big)
\end{align*}

with $V_{h}=\{x\in\mathbb{R}^{d}|\ \lVert x-x^{\star}\rVert_{\mathbf{H}}>\tilde{\phi}(h)\}$.
We now show that for $h$ sufficiently small
\[
\{x\in B(0,r)\mid\lVert x-x^{\star}\rVert_{\mathbf{H}}^{2}-m\lVert x-x^{\star}\rVert_{\mathbf{H}}^{\alpha}\leq h\}=B_{h}.
\]
Indeed, note first that if $x\in B(0,r)$, we obtain by~\eqref{eq:sym-mat-ineq}
\[
\lVert x-x^{\star}\rVert_{\mathbf{H}}^{2}\le e_{d}(\mathbf{H})\lVert x-x^{\star}\rVert^{2}<4e_{d}(\mathbf{H})r^{2}.
\]
where we have used that, as $\lVert x\rVert<r$, the triangle inequality
gives $\lVert x-x^{\star}\rVert<2r$. Hence $B(0,r)\subset\{x\in\mathbb{R}^{d}|\ \lVert x-x^{\star}\rVert_{\mathbf{H}}^{2}<4e_{d}(\mathbf{H})r^{2}\}$.
We now show that $B(0,r)\subset\{x\in\mathbb{R}^{d}|\ \lVert x-x^{\star}\rVert_{\mathbf{H}}\le\Tilde{\phi}(h)\}$.
Indeed, at $h=0$, $0=\phi_{+}(0)<\Tilde{\phi}(0)$ are by definition,
the two roots of 
\[
u^{2}-mu^{\alpha}=0.
\]
Hence $\Tilde{\phi}(0)=\sqrt{e_{d}(\mathbf{H})2r}$. By continuity
of $\Tilde{\phi}(h)$ at $h=0$, we obtain that $B(0,r)\subset\{x\in\mathbb{R}^{d}|\ \lVert x-x^{\star}\rVert_{\mathbf{H}}\le\Tilde{\phi}(h)\}$
for $h$ sufficiently small. As $\phi_{+}(h)\le\Tilde{\phi}(h)$,
we thus obtain that, for $h$ sufficiently small, $V_{h}\cap B(0,r)=\emptyset$.
Next, the same line of reasoning as the one for $\phi_{-}$, using
that $\phi_{+}(h)\rightarrow_{h\rightarrow0}0$ and $\lVert x^{\star}\rVert<r$,
shows that $B_{h}\cap B(0,r)=B_{h}$ for $h$ sufficiently small.
\\
Hence, for $h$ small enough we have
\[
\{x\in B(0,r)\mid\lVert x-x^{\star}\rVert_{\mathbf{H}}^{2}-m\lVert x-x^{\star}\rVert_{\mathbf{H}}^{\alpha}\leq h\}=B_{h}.
\]
This gives $S_h \subset B_h$.\\
Finally, similarly to $\phi_{-}$, we can show that $\phi_{+}(h)=\sqrt{h}+\frac{m}{2}h^{(\alpha-1)/2}+o(h^{(\alpha-1)/2})$,
which concludes the proof of this lemma. 
\end{proof}
