\section{Conclusion}
We introduced a new kernel-based statistic for testing CI. We derived its asymptotic null distribution and designed a simple testing procedure that emerges from it. To our knowledge, we are the first article to propose an asymptotic test for CI with a tractable null distribution. Using various synthetic experiments, we demonstrated that our approach is competitive with other SoTA methods both in terms of type-I and type-II errors, even in the high dimensional setting.


% This work offers a new computationally efficient kernel-based test for conditional independence, supported by asymptotic theoretical guarantees. We show that comparing the $L^p$ distance between well chosen mean embeddings at a finite set of locations leads to a simple characterization of the conditional independence relation. Our method is flexible in the sense that the locations and the kernels used to embed the distributions can be chosen in order to maximize the power. 

% A first estimate of the metric can be obtained when one has access to observations from some specific conditional means. However, in practice, such samples are not available; we overcome this by estimating the unknown samples using regularized least-squares approximations of the conditional means. We obtain the asymptotic distribution of the resulting statistic and derive a consistent test from it. Furthermore, we reduce the computational complexity of the proposed method by considering random features expansions of kernels when fitting the regression models. We show that the choice of RLS estimators to estimate samples from the conditional means is valid in order to derive the asymptotic distribution of our statistic. However any generative method which allows to sample from the conditional means can be used to replace our RLS estimators: we leave this as an open question for further work.



% Our results on asymptotic distribution requires additional assumptions compared to the ones in~\cite{zhang2012kernel}. Indeed, we need conditions to ensure the convergence rate of the RLS estimator. However, in~\citep{zhang2012kernel}, the proof of their asymptotic law requires assumptions to ours to hold. In~\citep{strobl1702approximate}, to prove their asymptotic distribution, the authors assume to have access to samples from the conditional mean, which is in practice impossible. {\color{red}[sounds weird, check grammar.]} In our work, by adding Assumptions~\ref{ass:spectrum}-\ref{ass:source}, we show that the asymptotic law still holds with RLS estimate of the conditional expectation. 

% We proposed a new consistent test for conditional independence, for which we proved its asymptotic law. We compared it with several benchmark tests {\color{red}[sounds weird, need to improve the language also we need to see if we are indeed better and faster]}, showing our test is faster and more accurate than other methods. As further work, we plan to deploy our statistic in order to learn invariant representations {\color{red}[reference]}.