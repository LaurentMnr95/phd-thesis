\pdfobjcompresslevel 0
\documentclass{mimosis}

\usepackage[ruled,vlined]{algorithm2e}
\usepackage{algorithmic}
%\usepackage{geometry}
\usepackage{xcolor}
\usepackage{comment}
\usepackage{metalogo}
\usepackage{enumerate}
\usepackage{mathtools,amssymb,amsmath} 
\usepackage{physics}
\usepackage{makecell}
\usepackage{makeidx}
\usepackage{isomath}
\usepackage[framemethod=TikZ]{mdframed}
\usepackage{enumitem}
\usepackage{relsize}
\usepackage{minitoc}
\usepackage{caption}
\usepackage{tcolorbox}
\usepackage{bm}

%\usepackage[pdftex]{graphicx}
\allowdisplaybreaks


\usepackage{amsfonts}
\usepackage{xspace}
\usepackage{array}
\usepackage{multirow}
\usepackage{booktabs}
\usepackage{pgfplots}
\usepackage{pdfpages}
\usepgfplotslibrary{groupplots}
\pgfplotsset{compat=newest}
\usepackage{dsfont}


\newcolumntype{C}[1]{>{\centering\arraybackslash}p{#1}}
\newcolumntype{L}[1]{>{\arraybackslash}p{#1}}

\newcommand{\textred}[1]{\textcolor{red}{#1}}
\newcommand{\rem}[1]{{\color{red}\sout{#1}}}
\newcommand{\add}[1]{{\color{green}#1}}
\newcommand{\replace}[2]{\rem{#1}\add{#2}}
\newcommand{\com}[1]{{\color{gray}\footnotesize ({\it #1})}}
\newcommand{\Omit}[1]{}

\newcommand{\cmark}{\ding{51}}
\newcommand{\xmark}{\ding{55}}



\setkomafont{disposition}{\color{black}\bfseries}
\definecolor{PSLBlue}{RGB}{47,68,134}
\definecolor{grund}{RGB}{238,241,251}          
\definecolor{schrift}{RGB}{0,73,114}
\definecolor{newnavy}{RGB}{0,46,73}
\definecolor{rahmen}{RGB}{0,73,114}
 
\surroundwithmdframed[
   topline=false,
   rightline=false,
   bottomline=false,
   leftline=false,
   skipabove=\medskipamount,
   skipbelow=\medskipamount,
   backgroundcolor=grund,
]{proof}


\def\table{\def\figurename{Table}\figure}
\let\endtable\endfigure

%%%%%%%%%%%%%%%%%%%%%%%%
% Table setup
%%%%%%%%%%%%%%%%
\renewcommand{\arraystretch}{1.2}



%%%%%%%%%%%%%%%%%%%%%%%%%%%%%%%%%%%%%%%%%%%%%%%%%%%%%%%%%%%%%%%%%%%%%%%%
% Some of my favourite personal adjustments
%%%%%%%%%%%%%%%%%%%%%%%%%%%%%%%%%%%%%%%%%%%%%%%%%%%%%%%%%%%%%%%%%%%%%%%%
%
% These are the adjustments that I consider necessary for typesetting
% a nice thesis. However, they are *not* included in the template, as
% I do not want to force you to use them.

% This ensures that I am able to typeset bold font in table while still aligning the numbers
% correctly.
\usepackage{etoolbox}
\usepackage[binary-units=true]{siunitx}
\DeclareSIUnit\px{px}

\sisetup{%
  detect-all           = true,
  detect-family        = true,
  detect-mode          = true,
  detect-shape         = true,
  detect-weight        = true,
  detect-inline-weight = math,
}

%%%%%%%%%%%%%%%%%%%%%%%%%%%%%%%%%%%%%%%%%%%%%%%%%%%%%%%%%%%%%%%%%%%%%%%%
% Hyperlinks & bookmarks
%%%%%%%%%%%%%%%%%%%%%%%%%%%%%%%%%%%%%%%%%%%%%%%%%%%%%%%%%%%%%%%%%%%%%%%%


\usepackage[%
  colorlinks = true,
  citecolor  = PSLBlue,
  linkcolor  = PSLBlue,
  urlcolor   = blue,
  unicode,
  ]{hyperref}



% \renewcommand*{\backref}[1]{}
% \renewcommand*{\backrefalt}[4]{%
%     \ifcase #1 (Not cited.)%
%     \or        (Cited on page~#2.)%
%     \else      (Cited on pages~#2.)%
%     \fi}


\usepackage{bookmark}

%%%%%%%%%%%%%%%%%%%%%%%%%%%%%%%%%%%%%%%%%%%%%%%%%%%%%%%%%%%%%%%%%%%%%%%%
% Bibliography
%%%%%%%%%%%%%%%%%%%%%%%%%%%%%%%%%%%%%%%%%%%%%%%%%%%%%%%%%%%%%%%%%%%%%%%%
%
% I like the bibliography to be extremely plain, showing only a numeric
% identifier and citing everything in simple brackets. The first names,
% if present, will be initialized. DOIs and URLs will be preserved.

% \usepackage[%
%   autocite     = plain,
%   backend      = biber,
%   doi          = false,
%   url          = true,
%   giveninits   = true,
%   hyperref     = true,
%   maxbibnames  = 200,
%   maxcitenames = 200,
%   sortcites    = true,
%   style        = numeric,
%   backref,
%   backrefstyle = none,
%   hyperref,
%   ]{biblatex}
\usepackage{natbib}
 
% \input{bibliography-mimosis}
% \addbibresource{biblio.bib}


\pdfoutput=1


% \newtheorem{coro}{Corollary}[section]
% \newtheorem{lemme}{Lemma}[section]
% \newtheorem{defn}{Definition}[section]
% \newtheorem{prop}{Proposition}[section]
% \newtheorem{thm}{Theorem}[section]
% \newtheorem{property}{Property}[section]
% \newtheorem{rmq}{Remark}
% \newtheorem{corol}{Corollary}
% \newtheorem{proposition}{Proposition}[section]
% \newtheorem{prv}{Proof}[section]
\newtheorem{coro}{Corollary}
\newtheorem{prop}{Proposition}
\newtheorem{thm}{Theorem}
\newtheorem{property}{Property}
\newtheorem{corol}{Corollary}
\newtheorem{assump}{Assumption}
\newtheorem{corollary}{Corollary}
\newtheorem{lemma}{Lemma}
\newtheorem{definition}{Definition}
\newtheorem{example}{Example}
\newtheorem{rmq}{Remark}
\newtheorem*{counterexample*}{Counterexample}
\newtheorem*{consequence*}{Consequence}

\newtheorem*{assump*}{Assumption}

\newtheorem*{example*}{Example}
\newtheorem*{prop*}{Proposition}
\newtheorem*{thm*}{Theorem}
\newtheorem*{prv*}{Proof}

% Notations

%\newcommand{\GOT}{\textsc{MS}\xspace}
\newcommand{\MOT}{\textsc{EOT}\xspace}
\newcommand{\MOTe}{\textsc{EOT}^{\bm{\varepsilon}}\xspace}
% \newcommand{\DOT}{\textsc{MOT}\xspace}
\newcommand{\KL}{\textsc{KL}\xspace}
\newcommand{\TV}{\textsc{TV}\xspace}
\newcommand{\ent}{\textsc{H}\xspace}
\newcommand{\wass}{\textsc{W}\xspace}
\newcommand{\riskemp}{\widehat{\mathcal{R}}}

\newcommand{\riskadv}{\mathcal{R}_{\epsilon}}
\newcommand{\valuerand}{\mathcal{V}^{rand}}
\newcommand{\valuedet}{\mathcal{V}^{det}}
\newcommand{\dualvalue}{\mathcal{D}}

\DeclareMathOperator*{\supp}{supp}   
\DeclareMathOperator*{\sign}{sign}   
\DeclareMathOperator*{\essinf}{essinf}   

\newcommand{\QQ}{\mathbb{Q}}
\newcommand{\PP}{\mathbb{P}}
\newcommand{\EE}{\mathbb{E}}
\newcommand{\RR}{\mathbb{R}}

\newcommand{\XX}{\mathcal{X}}
\newcommand{\YY}{\mathcal{Y}}
\newcommand{\ZZ}{\mathcal{Z}}


\newcommand{\risk}{\mathcal{R}}
\newcommand{\loss}{L}

\newcommand{\probmap}{h}




\newcommand{\bookboxx}[1]{\small
\par\medskip\noindent
\framebox[0.99\textwidth]{
\begin{minipage}{0.97\dimexpr\textwidth-\parindent\relax} {#1} \end{minipage} } \par\medskip }
\newcommand{\forceindent}{\leavevmode{\parindent=1em\indent}}

\DeclareMathOperator*{\argminB}{argmin}   
\DeclareMathOperator*{\esssup}{ess\text{  }sup}  
\DeclareMathOperator*{\argmaxB}{argmax} 
\DeclareMathOperator*{\Vol}{Vol}


\newcommand{\lints}{\textsc{LinTS}\xspace}
\newcommand{\expfour}{\textsc{Exp}$4$\xspace}
\newcommand{\expthree}{\textsc{Exp}$3$\xspace}
\newcommand{\oful}{\textsc{OFUL}\xspace}
\newcommand{\linucb}{\textsc{LinUCB}\xspace}
\newcommand{\ucb}{\textsc{UCB}\xspace}
\newcommand{\epsgreedy}{\textsc{$\varepsilon$-greedy}\xspace}


\DeclareRobustCommand{\eg}{e.g.,\@\xspace}
\DeclareRobustCommand{\ie}{i.e.,\@\xspace}
\DeclareRobustCommand{\aka}{a.k.a.\@\xspace}
\DeclareRobustCommand{\wrt}{w.r.t.\@\xspace}
\DeclareRobustCommand{\wp}{w.p.\@\xspace}
\DeclareRobustCommand{\st}{s.t.\@\xspace}
\newcommand{\wt}[1]{\widetilde{#1}}
\newcommand{\wh}[1]{\widehat{#1}}
\newcommand{\wb}[1]{\overline{#1}}


\newcommand{\PCadvRisk}{\text{PC-Risk}_{\alpha}}
\newcommand{\EoT}{\text{EoT}}


\newcommand{\numsamples}{N}

% TODO: numlabels, inputspace, labelspace, 
%%%%%%%%%%%%%%%%%%%%%%%%%%%%%%%%%%%%%%%%%%%%%%%%%%%%%%%%%%%%%%%%%%%%%%%%
% Fonts
%%%%%%%%%%%%%%%%%%%%%%%%%%%%%%%%%%%%%%%%%%%%%%%%%%%%%%%%%%%%%%%%%%%%%%%%

\ifxetexorluatex
  \setmainfont{Minion Pro}
\else
  \usepackage[lf]{ebgaramond}
  \usepackage[oldstyle,scale=0.7]{sourcecodepro}
  \singlespacing
\fi


%%%%%%%%%%%%%%%%%%%%%%%%%%%%%%%%%%%%%%%%%%%%%%%%%%%%%%
%Acronyms
%%%%%%%%%%%%%%%%%%%%%%%%%%%%%%%%%%%%%%%%%%%%%%%%%%%%%%


\newacronym[description={Empirical Risk Minimization}]{ERM}{ERM}{Empirical Risk Minimization}
\newacronym[description={Projected Gradient Decent}]{PGD}{PGD}{Projected Gradient Descent}


%%%%%%%%%%%%%%%%%%%%%%%%%%%%%%%%%%%%%%%%%%%%%%%%%%%%%%
%Glossary
%%%%%%%%%%%%%%%%%%%%%%%%%%%%%%%%%%%%%%%%%%%%%%%%%%%%%%

\newglossaryentry{Integers}{%
  name        = {$\mathbb{N}$},
  description = {The set of natural integers},
  sort        = {Natural Integers},
}
\newglossaryentry{Real numbers}{%
  name        = {$\mathbb{R}$},
  description = {The set of real numbers},
  sort        = {Real numbers},
}


\makeindex
\makeglossaries

%%%%%%%%%%%%%%%%%%%%%%%%%%%%%%%%%%%%%%%%%%%%%%%%%%%%%%
%Operators and commands
%%%%%%%%%%%%%%%%%%%%%%%%%%%%%%%%%%%%%%%%%%%%%%%%%%%%%%

%%%%%%%%%%%%%%%%%%%%%%%%%%%%%%%%%%%%%%%%%%%%%%%%%%%%%%%%%%%%%%%%%%%%%%%%%




%%%%%%%%%%%%%%%%%%%%%%%%%%%%%%%%%%%%%%%%%%%%%%%%%%%%%%%%%%%%%%%%%%%%%%%%
% Incipit
%%%%%%%%%%%%%%%%%%%%%%%%%%%%%%%%%%%%%%%%%%%%%%%%%%%%%%%%%%%%%%%%%%%%%%%%

\title{xxx}
\subtitle{xxx}
\author{Laurent Meunier}

\begin{document}
 

\frontmatter
\chapter*{Remerciements}


% Pour commencer, je souhaiterais remercier Julien Mairal et Panayotis Mertikopoulos d’avoir accepté d’être rapporteur de cette thèse ainsi qu xxx  de s’être intéressés à mes travaux de recherche
% et d’avoir accepté d’être exam du jury. Nos échanges pendant la relecture ainsi
% que la soutenance ont été très enrichissants.


% Je remercie chaleureusement et amicalement mes maîtres de thèse Jamal Atif et Olivier Teytaud pour les trois ans effectués de travail ensemble et leur précieux encadrement. C'est grâce à eux que j'ai pu m'épanouir d'un point de vue personnel et professionel au cours de cette thèse. Jamal 

% Cette thèse n’aurait pas été possible sans les financements de Meta. Ainsi, je
% souhaite remercier Facebook AI Research pour m’avoir donné cette opportunité. 
\chapter*{Abstract}
This thesis investigates the problem of classification in presence of advesarial attacks. An adversarial attack is a small and humanly imperceptible perturbation of input designed to fool start-of-the-art machine learning classifiers. In particular, deep learning systems, used in safety critical AI systems as self-driving cars are at stake with the eventuality of such attacks. What is even more striking is the ease to create such adversarial examples and the difficulty to defend against them while keeping a high level of accuracy. Robustness to adversarial perturbations is a still misunderstood field in academics. In this thesis, we aim at understanding better the nature of the adversarial attacks problem from a theoretical perspective.


\begin{tcolorbox}[colback=grund,colframe=rahmen]
\begin{center}
    Can we find a principled way to defend against adversarial examples?
\end{center}
\end{tcolorbox}


In a first part, we tackle the problem of adversarial examples from a game theoretic point of view. We study the open question of the existence of mixed Nash equilibria in the zero-sum game formed by the attacker and the classifier. To that extent, we consider a randomized classifier and we introduce a more general attacker that can move each point randomly in the vinicity of original points. While previous game theoretic approaches usually allow only one player to use randomized strategies, we show the necessity of considering randomization for both the classifier and the attacker. We demonstrate that this game has no duality gap, meaning that it always admits approximate Nash equilibria. We also provide the first optimization algorithms to learn a mixture of a finite number of classifiers that approximately realizes the value of this game, i.e. procedures to build an optimally robust randomized classifier.



In a second part, we study the problem of surrogate losses in the adversarial examples case. In classification, the goal is to maximize the accuracy, but in practice, the accuracy is not efficiently optimizable. Instead, it is usual to minimize a convex and continuous loss that satisfy what is called the \emph{consistency property}. In the adversarial case, we tackle this problem and show that a wide range of usually consistent losses cannot be consistent. In particular, convex losses are not good  surrogate losses for the adversarial attack problem.  Finally, we pave a way towards designing a class of consistent losses, but this question is partially treated and left as further work.

In a final section, we study the robustness of neural networks from a dynamical system perspective. Residual Networks can indeed be interepreted as a discretization of a first order parametric differential equation. By studying this system, we provide a generic method to build 1-Lipschitz Neural Networks and show that some previous approaches are special cases of this framework. We extend this reasoning and show that ResNet flows derived from convex potentials define 1-Lipschitz transformations, that lead us to define the Convex Potential Layer (CPL). 

%  Besides security issues, this shows how
% little we know about the worst-case behaviors of models the industry uses daily. Accordingly, it
% became increasingly important for the machine learning community to understand the nature
% of this failure mode to mitigate the attacks. One can always build trivial classiers that will not
% change decision under adversarial manipulation – e.g. constant classers– but this comes at odds
% with standard accuracy of the model. This raises several questions. Among them, we tackle the
\input{sections/abstract_fr}




\dominitoc
\tableofcontents

 \renewcommand\listfigurename{List of Figures and Tables}
 \listoffigures
 
%----------------------------------------------------------------------------------------
%	SYMBOLS
%----------------------------------------------------------------------------------------
%\begin{comment}
 
% \chapter*{Notations and Symbols}
% \markboth{Notations and Symbols}{Notations and Symbols}

% We use bold lower-case to denote vectors and functions with multidimensional outputs and standard lower-case to denote scalars and real-value functions. Depending on the context, we either use calligraphic font or upper-case to denote ensembles -- most of the times calligraphic, sometimes upper-case to denote sub-sets or elements of a set of sets. 

% \section*{Algebra}
% \begin{tabular}{lll} % Include a list of Symbols (a three column table)
% $\mathbb{R}$  & Set of real numbers & \\
% $\mathbb{N}$ & Set of natural integers & \\
% $\mathbb{R}^d$ & Set of $d$-dimensional real-valued vectors & \\
% $\mathcal{M}_{d \times d'} (\R)$ & Set of $d \times d'$ real-valued matrices \\
% $I_d$ & $d \times d$ identity matrix \\
% $[a]$ & Set of integers between $1$ and $a$ & $[a] \equaldef \{1, \ldots, a \}$ \\
% $\Simplex(K)$ & $K$ dimensional simplex & $\Simplex(K) \equaldef \{ \vectorsym{z} \in \R^K ~\st~ \norm{\vectorsym{z}}_1 =1 \}$ \\
% $\| \vectorsym{v} \|_{p}$ & $\ell_p$-norm of $\vectorsym{v} \in \mathbb{R}^d$ for $p \in [1,+\infty)$ & $ \norm{\vectorsym{v}}_p = \left(\sum_{i = 1}^d \abs{\vectorsym{v}_i}^p \right)^{1/p}$ \\
% $\| \vectorsym{v} \|_{\infty}$ & Infinite norm of $\vectorsym{v} \in \mathbb{R}^d$ & $\| \vectorsym{v} \|_{\infty} = \max_{i \in [d]} ( |\vectorsym{v}_i|) $ \\ 
% $\norm{\vectorsym{v}}_{M}$ & Mahalanobis norm of $\vectorsym{v} \in \mathbb{R}^d$ with $M \in \mathcal{M}_{d \times d}(\mathbb{R})$  & $\norm{\vectorsym{v}}_{M} = \sqrt{ \vectorsym{v}^\intercal M \vectorsym{v}}$\\
% $B_p(\vectorsym{v} ,\alpha)$  & $\ell_p$ ball with center $\vectorsym{v} \in \R^d$ and radius $\alpha \geq 0$ & $\{\vectorsym{u} ~\st~ \norm{\vectorsym{u} -\vectorsym{v}}_p \leq \alpha \}$ \\
% $B_p(\alpha)$  & $\ell_p$ ball with center $0$ and radius $\alpha \geq 0$ & $\{\vectorsym{u} ~\st~ \norm{\vectorsym{u}}_p \leq \alpha \}$ \\
% $\Vol(B)$ & Volume of the sub-space $B \subset \R^d$
% \end{tabular}
% \section*{Probability}
% \begin{tabular}{lll} 
% $\mathcal{A}\left(\mathcal{Z}\right)$ & $\sigma$-algebra of an arbitrary space $\mathcal{Z}$ \\
% $\mathcal{P}\left(\mathcal{Z} \right)$ & Set of probability distribution over $(\mathcal{A}\left(\mathcal{Z}\right),\mathcal{Z})$ \\
% $\mathcal{F}_{\mathcal{Z} \times \mathcal{Z}' }$ & Set of measurable functions from $\mathcal{Z}$ to $\mathcal{Z}'$ \\
% $\vectorsym{\psi} \# \rho$ & Push-forward of $\rho \in \mathcal{P}\left(\mathcal{Z} \right)$ by $\vectorsym{\psi} \in \mathcal{F}_{\mathcal{Z} \times \mathcal{Z}' }$ \\
% $\expect[.]$ & Expectation of a random event \\
% $\proba[.]$ & Probability of a random event \\
% $\mathcal{N}(. , .)$ & Gaussian distribution \\
% $\text{Lap}(.,.)$ & Laplace distribution \\
% $\Phi$ & cdf of the standard Gaussian distribution $\mathcal{N}(0,1)$\\ 
% \end{tabular}

% \section*{Classification and Learning theory}
% \begin{tabular}{lll} 
% $\inputspace$ & Input space \\
% $d$ & Dimension of the input space \\
% $\outputspace$ & Output space \\
% $K$ & Number of classes \\
% $\groundDistrib$ & Ground-truth distribution \\
% $\fullSample$ & Training sample \\
% $\Hypothesisspace$ & Hypothesis space \\ 
% $\loss$ & Loss function \\
% \end{tabular}

% \section*{Functions}
% \begin{tabular}{lll}
%      $\1\{.\}$ & Indicator function of an event & $\1\{A\} = 1$ if $A$ is true, $0$ otherwise  \\
%      $\sign(x)$ & Sign function applied on $x$ & $\sign(x)= 1$ if $x>0$, $-1$ if $x<0$ and $0$ if $x=0$ \\
% \end{tabular}

 
% %----------------------------------------------------------------------------------------
% %	ABBREVIATIONS
% %----------------------------------------------------------------------------------------
% %\begin{comment}
% \chapter*{Abbreviations}
% \markboth{Abbreviations}{Abbreviations}

% \begin{tabular}{ll} 
% \emph{\textbf{a}.\textbf{k}.\textbf{a}.} & \textbf{a}lso \textbf{k}nown \textbf{a}s\\
% \textbf{cdf} & \textbf{c}umulative \textbf{d}ensity \textbf{f}unction \\
% \textbf{C \& W} & \textbf{C}arlini and \textbf{W}agner (attack)\\
% \emph{\textbf{e}.\textbf{g}.} & \emph{\textbf{e}xempli \textbf{g}ratia} \\
% \textbf{Eq.} & \textbf{Eq}uation \\
% \textbf{ERM} & \textbf{E}mpirical \textbf{R}isk \textbf{M}inimization\\
% \textbf{FGM} & \textbf{F}ast \textbf{G}radient \textbf{M}ethod (attack)\\
% \emph{\textbf{i}.\textbf{e}.} & \emph{\textbf{i}d \textbf{e}st} \\
% \emph{\textbf{i}.\textbf{i}.\textbf{d}.} & \textbf{i}dentically and \textbf{i}ndependently \textbf{d}istributed \\
% \textbf{PGD} & \textbf{P}rojected \textbf{G}radient \textbf{D}escent (attack)\\
% \textbf{resp.} & \textbf{resp}ectively \\
% \emph{\textbf{s}.\textbf{t}.} & \textbf{s}uch \textbf{t}hat \\
% \textbf{SRM} & \textbf{S}tructural \textbf{R}isk \textbf{M}inimization \\  
% \textbf{std} & \textbf{st}andard \textbf{d}eviation \\
% \textbf{w}.\textbf{r}.\textbf{t}. & \textbf{w}ith \textbf{r}espect \textbf{t}o 
% \end{tabular}



\dominitoc
\tableofcontents
\mainmatter
\chapter{Introduction}
\minitoc
\section{Artificial Intelligence foundations}

Machine Learning, the computer science subdomain dedicated to building and studying computer systems that automatically improve with experience, is at the very core of the recent advances in Artificial Intelligence. Finding its roots in statistical analysis, it has been widely studied over the past thirty years from algorithmic and mathematical perspectives, giving rise to a new discipline, computational learning theory. With the availability of massive amounts of data and computing power at low price, the last two decades have witnessed a growing interest in  real-world applications of the domain. This interest is even stronger since 2012, with the remarkable success of of AlexNet~\citep{krizhevsky2012imagenet} on the ImageNet challenge~\citep{imagenet_cvpr09}, using neural networks with several layers. The era of Deep Learning started then, with  unexpected achievements in several domains: generative modeling~\citep{goodfellow2014generative}, natural language processing~\citep{vaswani2017attention}, etc. The success of Deep Learning (artificial neural networks with a large number of layers) can be explained by the conjunction of the following factors: 

%Machine learning can be defined with the following question: ``How can we build computer systems that automatically improve with experience, and what are the fundamental laws that govern  all learning processes?''. For instance, credit scoring are computed rules that are learnt from previous defaulting consumers. Machine Learning have been widely studied for the $30$ past years using advanced statistical tools and taking profits of more and more powerful computers.  The last $10$ years have seen an exponentially increasing public interest for machine learning systems. Firstly, this incredible growth of interest to AI is also linked to the availability of huge amounts of data at low price, the so-called ``Big Data'' era.  Secondly, the advent of Deep Learning, i.e. Machine Learning using Artificial Neural Networks with a lot of layers, came with the extraordinary success of AlexNet~\citep{krizhevsky2012imagenet} on the ImageNet challenge~\citep{imagenet_cvpr09} in 2012. Since, exceptional progresses were made in generative modeling~\citep{goodfellow2014generative}, natural language processing~\citep{vaswani2017attention}, etc. Machine learning has become the most active research fields in Artificial Intelligence. Therefore the number of industrial application has also exploded. The recent renewal in Machine Learning is due to the conjunction of many factors:
\begin{itemize}
    \item \textbf{Availability of data:} the amount and the cost of data have largely decreased since the emergence of web platforms, and tools for large-scale data management.
    \item \textbf{Computational power:} new specialised hardware architectures such as GPUs and TPUs allow faster and larger training algorithms.
    \item \textbf{Algorithmic scalability:} algorithms are scalable to large models (Distributed Computing, etc.) and large number of data (Stochastic Gradient Descent~\citep{bottou2010large}, etc.)
    \item \textbf{Open Source projects:} Large projects in Machine Learning are nowadays open-sourced (TensorFlow~\citep{abadi2016deep}, PyTorch~\citep{paszke2017automatic}, Scikit Learn~\citep{pedregosa2011scikit}, etc.) stimulating the emergence of large communities.
\end{itemize}

It is worth noting here that Artificial Intelligence, as a scientific domain, exists since early 20th century. Protean in nature, it encompasses several notions and fields, beyond Machine Learning, and Deep Learning.  Its birth is inseparable from the development of computer science. The first efficient computer was built by Charles Babbage and ran Ada Lovelace's algorithm.  Computer Science was formalized and theoretized in the Church-Turing thesis~\citep{turing1950computing}, which defines the notion of computability, i.e. functions are computable if they can be out as a list of predefined instructions to be followed. Such instructions are called algorithms. Artificial Intelligence, or at the least the term,  was  ``officially founded'' as a research field in 1956 at the Dartmouth Workshop~\citep{mccarthy2006proposal}, organized by Marvin Minsky, John McCarthy, Claude Shannon and Nathan Rochester. During this conference, the term ``Artificial intelligence'' was proposed and adopted by the community of researchers. Since then, the field has oscillated between hype and disappointment, with no less than two major period of disinterest  as the AI winters. This thesis is clearly developed during the third hype's period, but we keep in mind the very enlightening  history of the discipline.  

%Following this conference and thanks to (military) fundings, substantial advances in the field were made in problem solving and natural language processing. However the results, being far from meeting the funders' expectations, the domain   entered in a first ``AI winter'' during late sixties and the seventies. The second wave of AI occured thanks to the fundamental limits of symbolic AI and expert systems that allow focus on new designs of Artificial Neural Networks thanks to the algorithm of backpropagation~\citep{rumelhart1985learning} and the first well performing convolutional networks~\citep{lecun1995convolutional}. In the 1980s, theoretical analysis of machine learning also appeared with the theory of learning~\citep{valiant1984theory,vapnik1998} with the first generalization bounds on learning algorithms. Until the success of Deep Learning, Support Vector Machines~\citep{vapnik1998} and kernel methods~\citep{vert2004primer} were the most popular methods in Learning problems.


% TALK ABOUT CYBERNETICS

\section{Risks with Learning Systems}
% The idea of building artificial intelligence dates back to the first automatons that dates back to Ancient Egypt and Greece mythology. At the time, automatons were movable statue as Talos' one in Crete. Artificial Intelligence relies on the idea that human reasoning can be automatized, e.g. building a machine that can replicate human reasoning without external intervention. The study of automatized reasoning has a long stand history, and was the research subject of many mathematicians or philosophers. QUOTE A BIT
% A ENLEVER PE





\subsection{Common Threats}
Cybersecurity is at the core of computer science. Cryptography has been one of the hottest topics during the last thirty years. Despite their performances, learning systems are subject to many types of vulnerabilities and, by their popularity, are then prone to malicious attacks. Probably, the most known vulnerability that got public attention is privacy. While the amount of available data is exponentially growing, recovering identities by crossing datasets is easier when data are not protected. As it was exhibited in the de-anonymization of the Netflix 1M\$ prize dataset~\citep{narayanan2008robust}, hiding identities in datasets is not sufficient to protect the privacy data. Computer scientists have then intensified their effort so as to propose ways to protect data, leading to the emergence to what is considered as a gold standard for data protection: Differential Privacy~\citep{dwork2008differential}. It barely consists in adding noise to data to make them unrecoverable without too much deteriorating the their utility. It is appealing because it comes with strong theoretical guarantees, while being simple to manipulate,  allowing to tradeoff between the degree of privacy through noise injection and the quality of the information one can infer from the data.  Common privacy attacks are:
\begin{itemize}

    \item  \textbf{Model stealing~\citep{tramer2016stealing}:} An attacker aims at stealing the parameters of a given model.
    \item \textbf{Membership inference~\citep{shokri2017membership}:} Inferring whether a data sample was present or not in a training set. 

\end{itemize}
    

Consequently to privacy threats, European authorities conceived the GDPR (General Data Protection Regulation)\footnote{\url{https://eur-lex.europa.eu/eli/reg/2016/679/oj}}, adopted in 2016, which defines new rules on the use of data and on privacy. Today, GDPR is part of any data management plan of private companies. 
%Indeed, the fine for a company not respecting this law can be up to $4\%$ of the revenues of the company. To comply with this new regulation, companies and government must respect privacy in the implementation of any system with regards to the data they use.  While the initial aim of such a law was to discourage multinational companies using personal data, these companies has adapted easier than smaller ones thanks to the budget they could allocate for respecting GDPR. To counter this,
As an update of the GDPR, a second law proposition regarding data sharing from public and private companies has been introduced by the European Commission on The Governance of Data\footnote{\url{https://eur-lex.europa.eu/legal-content/EN/TXT/?uri=CELEX\%3A52020PC0767}} in 2020.

    
    
Another type of vulnerability in Machine Learning is model failure. A malicious user, by modifying either the model or the data, can make it performs very poorly. The most known attacks aiming at model failures are:
\begin{itemize}
    \item \textbf{Data poisoning attacks~\citep{kearns1993learning}:} changing some data in the training set so that the model performs very poorly on the hold-out set. 
    \item \textbf{Evasion attacks~\citep{biggio2013evasion,Szegedy2013IntriguingPO}}: small imperceptible perturbations at inference time. We will refer them to \emph{``adversarial attacks''}.
\end{itemize}
    
Known and gaining interest in academia, these threats are not very known by most of the companies~\cite{kumar2020adversarial}. More importantly, such vulnerabilities  hinder the use of state of the art models in critical systems (autonomous vehicles, healthcare, etc.). In the manuscript we will focus on adversarial attacks.  We introduce this threat more in details in the next paragraph.

\begin{tcolorbox}[title=References to adversarial examples in European Commission in law proposal on Artificial Intelligence systems]
\label{ref:adversarial_law}
As part of the introduction: \textit{``Cybersecurity plays a crucial role in ensuring that AI systems are resilient against attempts to alter their use, behaviour, performance or compromise their security properties by malicious third parties exploiting the system’s vulnerabilities. Cyberattacks against AI systems can leverage AI specific assets, such as training data sets (e.g. data poisoning) or trained models (e.g. adversarial attacks), or exploit vulnerabilities in the AI system’s digital assets or the underlying ICT infrastructure. To ensure a level of cybersecurity appropriate to the risks, suitable measures should therefore be taken by the providers of high-risk AI systems, also taking into account as appropriate the underlying ICT infrastructure.''}

\medskip
Title III (High risk AI systems), Chapter II (Requirements for high risk AI system), Article 14.52 (Human oversight): \textit{``High-risk AI systems shall be resilient as regards attempts by unauthorised third parties to alter their use or performance by exploiting the system vulnerabilities.
The technical solutions aimed at ensuring the cybersecurity of high-risk AI systems shall be appropriate to the relevant circumstances and the risks.
The technical solutions to address AI specific vulnerabilities shall include, where appropriate, measures to prevent and control for attacks trying to manipulate the training dataset (‘data poisoning’), inputs designed to cause the model to make a mistake (‘adversarial examples’), or model flaws.''}
\end{tcolorbox}
\medskip
A first regulation text on Artificial Intelligence\footnote{\url{https://eur-lex.europa.eu/legal-content/EN/TXT/?uri=CELEX\%3A52021PC0206}} systems was proposed by the European commission in April 2021. This text includes a large section dedicated to ``High Risk AI''. High risk AI is referred to any autonomous system than can endanger human lives.  This text aims at dealing with many threats in Learning Systems. Two direct references are made to adversarial attacks, underlying the need for companies to deal with them. The difficulty is to unify and create precise rules in a domain where results and certificates are mostly empirical. As mentioned earlier, it is known that robust models are often less performing and can make autonomous systems unusable in real world scenarii. Thus, this text is a first step towards a unified regulation on autonomous systems but might miss precise requirements for models to be used in production.


\subsection{Adversarial attacks against Machine Learning Systems}

Despite the recent gain of interest in studying adversarial attacks in Machine Learning, the problematic exists however for a while and takes its source in SPAM classification where adversaries were spammers whose goal was to evade from the taken decision\footnote{\cite{dalvi2004adversarial} showed that linear classifiers used in spam classification could be fooled by simple ``evasion attacks'' as spammers inserted ``good words'' into their spam emails.}.

With the recent success of Deep Learning algorithms, in particular in computer vision, several authors~\citep{biggio2013evasion,Szegedy2013IntriguingPO} have  highlighted their vulnerability to adversarial attacks. Adversarial attacks in this case are widely understood as ``imperceptible'' perturbations of an image, i.e. slight changes in the pixels, so that this image remains unchanged from human sights. This characteristic might be surprising but is actually a severe curb in applying state-of-the-art deep learning methods in critical systems. There are number of issues that makes difficult building and evaluating robust models for real life applications:
\begin{enumerate}
    \item The notion of imperceptibility is not well understood: numerically measuring human perception is still an open problem. Hence, detecting the change of perception due to adversarial attacks is an ill-posed problem. Most of the  research in the domain focused on pixel-wise perturbations (e.g. $\ell_p$ norms), while real world threats would be crafted by inserting some misleading objects in the environment (e.g. patches~\citep{brown2017adversarial}, T-shirts~\citep{xu2020adversarial}, textures~\citep{wiyatno2019physical},etc.).
    \item Robustness is often empirically measured: there exist only a few methods with formal guarantees on the robustness and these guarantees are often loose. Robustness is usually measured on a set of possible attacks and not all possible perturbations are spanned by these attacks, leaving rooms for potential blind spots.
    \item There exists a trade-off between robustness and accuracy. Most models that are robust suffer from a performance drop on natural data. For instance, a robustly trained robot will perform much lower on natural tasks than an accurate non-robust robot. That makes robust models unusable in real world applications~\citep{lechner2021adversarial}. 
\end{enumerate}

%ENLEVER CE QU Il Y A EN DESSOUS ?

%The substantial gap between academic research on adversarial attacks and ``real world'' attacks is difficult to bridge and it is one of the reasons authorities are reluctant authorizing autonomous systems as driver-free cars~\citep{eykholt2018robust}, automated face recognition as biometry checks~\citep{dong2019efficient}, etc.  The proposal from European Commission on AI highlights the risk of Adversarial Examples on algorithms  in production. 



\section{Adversarial Classification in Machine Learning}

In this manuscript, we will focus on the task of classification in Machine Learning. The purpose of this task is to ``learn'' how to classify some input $x$ into some label(s). The input can be an image, a text, an audio, etc. For instance, in computer vision, a known dataset is ImageNet where the goal is to learn how to classify high quality images into $1000$ labels~\citep{imagenet_cvpr09}. In natural language processing, the IMDB Movie Review Sentiment Classification dataset~\citep{maas-EtAl:2011:ACL-HLT2011} aims at classifying positive or negative sentiments from movie reviews. To learn a classifier, the task is often supervised, i.e, we have access to labeled inputs, which constitutes the so-called training set. To assess the quality of the learnt model, we evaluate it on other images that constitute the test set.

\subsection{A Learning Approach for Classification}
From now, we will assume that the inputs are in some space $\XX$ and the labels form a set $\mathcal{Y}:=\{1,\dots,K\}$. To learn an adequate classification model, we denote $\{(x_1,y_1),\dots,(x_\numsamples,y_\numsamples)\}$ the $n$ elements of $\XX\times\YY$ forming the training set. We furthermore assume that these inputs are independent and identically distributed (i.i.d.) from some distribution $\PP$ on $\XX\times\YY$. The aim is now to learn a function/hypothesis from these samples $h:\XX\to\YY$ to classify an input $x$ with a label $y$. To assess the quality of a classifier, the metric of interest is often the misclassification rate of the model, or the $0/1$ loss risk, and it is defined as:
\begin{align*}
\risk_{0/1}(h):=\PP(h(x)\neq y) = \EE_{(x,y)\sim\PP}\left[\mathbf{1}_{h(x)\neq y}\right]
\end{align*}
The optimal classifier, minimizing the standard risk is called the Bayes optimal classifier and is defined as $h(x) = \argmaxB_k\PP(y=k\mid x)$.
As the sampling distribution $\PP$ is usually unknown, the optimal Bayes classifier is also unknown. The accuracy is often empirically evaluated on a test set $\{(x'_1,y'_1),\dots,(x'_m,y'_m)\}$ independent from the training set and i.i.d. sampled from $\PP$.  To find this classifier $h$, we learn a function $\mathbf{f}:\XX\to\mathbb{R}^K$ returning scores, or logits, $(f_1(x),\dots,f_K(x))$ corresponding to each label. Then $h$ is set to $h(x)=\argmaxB_k f_k(x)$. The function $\mathbf{f}$ is usually learned by minimizing the empirical risk for a certain convenient loss function $\loss$ over some class of functions $\mathcal{H}$.
\begin{align*}
\inf_{\mathbf{f}\in\mathcal{H}}\riskemp_{\numsamples}(\mathbf{f}):= \frac{1}{\numsamples}\sum_{i=1}^\numsamples \loss(\mathbf{f}(x_i),y_i).
\end{align*}

This problem is called Empirical Risk Minimization (ERM). The theory of this problem has been widely studied and is well understood. It is often argued that there is a tradeoff on the ``size'' of $\mathcal{H}$: having a too small $\mathcal{H}$ may lead to underfitting, i.e. not enough parameters to describe the optimal possible function while a too large $\mathcal{H}$ may lead to overfitting, i.e. fitting too much training data. We often talk about bias-variance tradeoff (see Figure~\ref{fig:tradeoff_bias_accuracy}. A penalty term $\Omega_{\mathcal{H}}(f)$ can also be added to the ERM objective to prevent from overfitting. This tradeoff was recently questioned by the double descent~\citep{belkin2019reconciling} phenomenon where overparametrized (i.e. number of parameters largely over the number of training samples) regimes lower the risk.
\begin{figure}
    \centering
    \includegraphics[width=0.5\textwidth]{Images/tradeoff_bias_variance.png}
    \caption{Bias-Variance tradeoff. A model with low complexity will have a low variance but an high bias. A model with high complexity will have a low bias but an high variance.}
    \label{fig:tradeoff_bias_accuracy}
\end{figure}
\begin{tcolorbox}
The presence of adversaries in classification questions the knowledge we have in standard statistical learning. Indeed most standard results do not hold in presence of adversaries, hence, opening a new research area dedicated to studying and understanding the classification problem in presence of adversarial attacks, and more importantly, deepen our understanding of  machine learning/deep learning in high dimensional regimes.
\end{tcolorbox}


\subsection{Classification in Presence of Adversarial Attacks}

Yet a model can be very well performing on natural samples, small perturbations of these natural samples can lead to unexpected and critical behaviours of classification models~\citep{biggio2013evasion,Szegedy2013IntriguingPO}. To formalize that, we will assume the existence of a ``perception'' distance $d:\XX^2\to\mathbb{R}$ such that a perturbation $x'$ of an input $x$ remains imperceptible if $d(x,x')\leq \varepsilon$ for some constant $\varepsilon\geq0$. This ``perception'' distance is difficult to define in practice. For images, the $\lVert\cdot\rVert_\infty$ distance over pixels is often used, but is not able to capture all imperceptible perturbations.  This choice is purely arbitrary: for instance, we will highlight in the manuscript that $\lVert\cdot\rVert_2$ perturbations can also be imperceptible while having a large $\lVert\cdot\rVert_\infty$. Image classification algorithms are also vulnerable to geometric perturbations, i.e. rotations and translations~\citep{xxx}.

Therefore, the goal of an attacker is to craft an adversarial input $x'$ from an input $x$ that is imperceptible , i.e. $d(x,x')\leq \varepsilon$ and misclassifies the input, i.e. $h(x')\neq y$. Such a sample $x'$ is called an adversarial attack. The used criterion cannot be the misclassification rate anymore, we need to take into account the possible presence of an adversary that maliciously perturbs the input. We then define the robust/adversarial misclassification rate or robust/adversarial $0/1$ loss risk: 

\begin{align*}
\risk_{0/1}^{\varepsilon}(h)&:=\PP_{(x,y)}(\exists x'\in\XX\text{ s.t. } d(x,x')\leq \varepsilon \text{ and } h(x')\neq y)\\
&= \EE_{(x,y)\sim\PP}\left[\sup_{x'\in\XX\text{ s.t. } d(x,x')\leq \varepsilon}\mathbf{1}_{h(x')\neq y}\right]
\end{align*}


Akin standard risk minimization, we aim to learn a function $\mathbf{f}:\mathcal{X}\to\mathbb{R}^K$ such that $h(x)=\argmaxB_k f_k(x)$. Usually in adversarial classification we aim at solving the following optimization problem, that we will call adversarial empirical risk minimization:

\begin{align*}
\inf_{\mathbf{f}\in\mathcal{H}}\riskemp^\varepsilon_{\numsamples}(\mathbf{f}):= \frac{1}{\numsamples}\sum_{i=1}^\numsamples\sup_{x'\in\XX\text{ s.t. } d(x,x')\leq \varepsilon} \loss(\mathbf{f}(x_i),y_i).
\end{align*}

This problem is a  more challenging to tackle than the standard risk minimization  since it involves a hard inner supremum problem~\citep{madry2017towards}. Guarantees in the adversarial setting are therefore difficult to obtain both in terms of convergence and statistical guarantees. The usual technique to solve this problem is called Adversarial Training~\citep{goodfellow2014explaining,madry2017towards}. It consists in alternating inner and outer optimization problems. Such a technique improves in practice adversarial robustness but lack theoretical guarantees. So far, most results and advances in understanding and harnessing adversarial attacks are empirical~\citep{ilyas2019adversarial,rice2020overfitting}, leaving many theoretical and practical questions open.  Moreover, robust models suffer from a performance drop and vulnerablity of models is currently still very high (see Table~\ref{table:sota-cifar}), which leaves room for substantial improvements.

\begin{table}[ht]
    \centering
    \begin{tabular}{c|c|c|c}
       \textbf{Attacker}  &  \textbf{Paper reference} & \textbf{Standard Acc.} & \textbf{Robust Acc.}  \\ \hline
        None & \citep{ZagoruykoK16} & 94.78\% & 0\%\\
        $\ell_\infty (\varepsilon=8/255)$&  \citep{rebuffi2021fixing}& 89.48\% & 62.76\%\\
        $\ell_2 (\varepsilon=0.5)$&  \citep{rebuffi2021fixing}& 91.79\% & 78.80\%\\
    \end{tabular}
    \caption{State of the art accuracies on adversarial tasks on a WideResNet 28x10~\citep{ZagoruykoK16}. Results are reported from~\citep{croce2020robustbench}}
\label{table:sota-cifar}
\end{table}

\section{Outline and Contributions}
We will first introduce in Chapter~\ref{chap:background} the necessary  background regarding Machine Learning and Adversarial Examples. We will then analyze  adversarial attacks from three complementary point of views outlined as follows.
\subsection{A Game Theoretic Approach to Adversarial Attacks}

A line of research, following~\cite{pinot2020randomization}, to understand adversarial classification is to rely on game theory. In Chapter~\ref{chap:game},  we will build on this approach and define precisely the motivations for both the attacker and the classifier. We will cast it naturally as a zero sum game. We will in particular, study the problem  of the existence of equilibria. More precisely, we will answer the following open question.
\medskip
\begin{tcolorbox}[title=Question 1]
\textbf{What is the nature of equilibria in the adversarial examples game?}
\end{tcolorbox}
\medskip

In game theory, there are many types of equilibria. In this manuscript, we will focus on Stackelberg and Nash equilibria. We will show the existence of both when both the classifier and the attacker play randomized strategies. To reach such equilibria, the classifier will be random, and the attacker will move randomly the samples at a maximum distance of $\varepsilon$. Then, we will propose two different algorithms to compute the optimal randomized classifier in the case of a finite number of possible classifiers. We will finally propose a heuristic algorithm to train a mixture of neural networks and show experimentally the improvements we achieve over standard methods.

This work \textbf{Mixed Nash Equilibria in the Adversarial Examples Game} was published at ICML2021.



\subsection{Loss Consistency in Classification in Presence of an Adversary}
In standard classification, consistency with regards to $0/1$ loss is a desired property for the surrogate loss $\loss$ used to train the model. In short, a loss $\loss$ is said to be consistent if for every probability distribution, a sequence of classifiers $(f_n)$ that minimizes the risk associated with the loss $\loss$, it also minimizes the $0/1$ loss risk. Usually, in standard classification, the problem is simplified thanks to the notion of calibration. We will see that the question of consistency in the adversarial problem is much harder. 
\medskip
\begin{tcolorbox}[title=Question 2]
\textbf{Which losses are consistent with regards to the $0/1$ loss in the adversarial classification setting?}
\end{tcolorbox}
\medskip
We tackle this question by showing that usual convex losses are not calibrated for the adversarial classification loss. Hence this negative result emphasizes the difficulty of understanding the adversarial attack problem, and building provable defense mechanisms. 

This work is not submitted yet. We plan to submit it in the coming months
\subsection{Building Certifiable Models}

The last problem we deal with in this manuscript is the implementation of robust certifiable models. This problem is challenging since it is far from trivial to come up with non vacuous bounds that are exploitable in practice.
\medskip
\begin{tcolorbox}[title=Question 3]
\textbf{How to efficiently implement certifiable models with non-vacuous guarantees?}
\end{tcolorbox}
\medskip
To this end, we propose two methods that enforce Lipschitzness on the predictions of neural networks:
\begin{enumerate}
    \item The first one consists in noise injection. We show that by adding a noise on an input of a classifier, we are able to get guarantees on the decision up to some level $\varepsilon$. This work \textbf{Theoretical evidence for adversarial robustness through randomization} was published at NeurIPS2019.
    \item A second one consists in building contractive blocks in a ResNet architecture. This method draws its inspiration from the continuous flow interpretation of residual networks. More precisely, we show that using a gradient flow of a convex function, our network is $1$-Lipschitz. We then design such a function, showing empirically and theoretically the robustness benefits of this approach. This work is not submitted yet. We plan to submit it in the coming months

\end{enumerate}

% \section{Other Works in Appendix}
\subsection{Additional Works}
Additionally to the works we present in the main document, we also present some other contributions we made during the thesis. These are deferred to the appendices. 

Regarding adversarial examples, we will present:
\begin{itemize}
    \item \textbf{Adversarial Attacks on Linear Contextual Bandits (see Appendix~\ref{paper:banditsattacks}):} we build provable attacks against online recommendation systems, namely Linear Contextual Bandits. This work was published at NeurIPS2020.
    \item \textbf{ROPUST: Improving Robustness through Fine-tuning with Photonic Processors and Synthetic Gradients (see Appendix~\ref{paper:ropust}):} we use an Optical Processor Unit over existing defenses to improve adversarial robustness. This work was published at a workshop on Adversarial Attacks at ICML2021.
\end{itemize}
We  published a paper in optimal transport named \textbf{Equitable and Optimal Transport with Multiple Agents (see Appendix~\ref{paper:eot})} where we introduce a way to deal with multiple costs in optimal transport by equitably partitioning transport among costs.
We also published many works in the field of evolutionary algorithms:
\begin{itemize}

    \item \textbf{Variance Reduction for Better Sampling in Continuous Domains (see Appendix~\ref{paper:rescaling})}: we show that, in one shot optimization, the optimal search distribution, used for the sampling, might be more peaked around the center of the distribution than the prior distribution modelling our uncertainty about the location of the optimum. This work was published at PPSN2020.
    \item \textbf{On averaging the best samples in evolutionary computation (see Appendix~\ref{paper:kbest}):}  we prove mathematically that a single parent leads to a sub-optimal simple regret in the case of the sphere function. We provide a theoretically-based selection rate that leads to better progress rates. This work was published at PPSN2020.
    \item \textbf{Asymptotic convergence rates for averaging strategies (see Appendix~\ref{paper:kbestgen}):} we extend the results from the previous papers to a wide class of functions including $C^3$ functions with unique optima. This work was published at FOGA2021.
    \item  \textbf{Black-Box Optimization Revisited: Improving Algorithm Selection Wizards through Massive Benchmarking (see Appendix~\ref{paper:benchmark}):} We propose a wide range of benchmarks integrated in Nevergrad~\citep{nevergrad} platform. This work was published in TEVC.
    
\end{itemize}
% In addition to these works, we also add interest in adversarial attacks against online recommendation systems, namely Linear Contextual Bandits. We designed algorithms that provably fool in 

% We also published works in the field of Derivative Free Optimization FOLLOW
\chapter{Background}
\minitoc
\section{Classification}
First, we formalize the classification task:
\begin{itemize}
    \item Consider an input space $\mathcal{X}$, typically images. We assume this space is endowed an arbitrary metric $d$ possibly the perception distance or any $\ell_p$ norm. In the remaining of the manuscript, unless it is specified, $(\mathcal{X},d)$ will be a \textit{proper} (i.e. closed balls are compact) \textit{Polish} (i.e. completely separable) metric space. Note that for any norm $\lVert\cdot\rVert$,  $(\mathbb{R}^d,\lVert\cdot\rVert)$ is a proper Polish metric space.
    \item Each image $x\in \mathcal{X}$ has to be classified to a label $y$. We describe the set of labels $\mathcal{Y}:=\{1,\dots,K\}$ as descriptors of an input. For instance the label of an image will be the description of it. $\mathcal{Y}$ will be endowed with the trivial metric  $d'(y,y') = \mathbf{1}_{y\neq y'}$. Note that $(\mathcal{X}\times\mathcal{Y},d\oplus d')$ is a proper Polish space.
\end{itemize}
With such spaces, the space $(\mathcal{X}\times\mathcal{Y},d\oplus d')$ is also a proper Polish space. As in every classification problem, the data is sampled from probability distribution $\PP$. We will assume from now that the distribution we consider are Borel. For any Polish Space $\mathcal{Z}$, we will denot $\mathcal{B}(\mathcal{Z})$ the Borel $\sigma$-algebra and the set of Borel distributions $\mathcal{Z}$ over will be denoted $\mathcal{M}_+^1(\mathcal{Z})$. We also recall the notion of \textit{universal measurability}: a set $A\subset \mathcal{Z}$ is said to be universally measurable if it measurable for every \textit{complete} Borel probability measures.

In standard classification, we aim at learning a (universally or Borel) measurable function $h:\mathcal{X}\to\mathcal{Y}$ minimizing the $0/1$ loss risk:
\begin{align}
   \risk_{0/1}(h):=\PP(h(x)\neq y) = \EE_{(x,y)\sim\PP}\left[\mathbf{1}_{h(x)\neq y}\right]
\end{align}
Note that this quantity is well defined when $h$ is measurable. The optimal classifier is called the Bayes Optimal classifier and is defined as $h(x) = \argmaxB_k\PP(y=k\mid x)$. One can note, using disintegration theorem that $h$ is indeed Borel measurable.

However in practice, the access to the Bayes Optimal classifier is impossible because it requires full knowledge on the probability distribution $\PP$ which is not the case in general. Let assume having access to a training set of $n$ data points $\{(x_1,y_1),\dots,(x_n,y_n)\}$. The knowledge of the Bayes classifier on training points would not be sufficient to have generalization properties for the classifier on out-of-sample data points because such functions would overfit the training set. Hence one need to reduce the search space of measurable functions to a much smaller one, we will denote $\mathcal{H}$. More precisely, for binary classification (i.e $\mathcal{Y}:=\{-1,+1\})$, we aim at learning a function $\mathbf{f}:\XX\to\mathbb{R}$ such that $h(x)=\sign(f(x))$ (with a convention on $sign(0)$). In multilabel classification (i.e $|\mathcal{Y}|\geq2$), we learn a function $\mathbf{f}:\XX\to\mathbb{R}^K$ and $h$ is set to $h(x)=\argmaxB_k f_k(x)$. Minimizing directly the $0/1$ loss risk is a NP-hard problem in general CITE. Then one needs to minimize a well-chosen loss function $L$. A \textit{loss function} $L:\mathbb{R}^K\times\mathcal{Y}\to\mathbb{R}$ will be without loss of generality a non negative Borel measurable function. An example of such a loss is the cross entropy loss defined as:
\begin{align*}
    L(\mathbf{f}(x),y)=\sum_{i=1}^n
\end{align*}

The study of which loss is suited for classification has been a widely studied topic. Hence the learning objective is then defined as:

\begin{align*}
\inf_{\mathbf{f}\in\mathcal{H}}\riskemp_{n}(\mathbf{f}):= \frac{1}{n}\sum_{i=1}^nL(\mathbf{f}(x_i),y_i).
\end{align*}

define 


% Instead, we aim at learning either a classifier $h\in\mathcal{H}$ such that:



\section{Background in adversarial classification}
Adversarial classification, we aim at learning a a (universally or Borel) measurable function $h:\mathcal{X}\to\mathcal{Y}$ minimizing the $0/1$ loss risk: 
\begin{align*}
\risk^\varepsilon_{0/1}(h)&:=\PP_{(x,y)}(\exists x'\in\XX\text{ s.t. } d(x,x')\leq \varepsilon \text{ and } h(x')\neq y)\\
&= \EE_{(x,y)\sim\PP}\left[\sup_{x'\in\XX\text{ s.t. } d(x,x')\leq \varepsilon}\mathbf{1}_{h(x')\neq y}\right]
\end{align*}
The definition of this quantity is not immediate and requires a proposition.

\begin{prop}
For any $h$ Borel measurable, the adversarial risk is well defined $\risk^\varepsilon_{0/1}(h)$.
\end{prop}

The existence of a minimizer for adversarial risk is a difficult question, that was partially answered in CITE, which states, under some mild conditions, that the minimum is attained over the set of universally measurable functions.


 define general adversarial loss.



The question of the loss function 


\subsection{Standard datasets in Classification}
Images are embedding in pixels laying in $[0,255]$ and then normalized to $[0,1]$. These images can be black and white, hence encoded on only one channel, or colorful and then encoded on three channels, often, Red, Green, Blue (RGB). The images are of diverse qualities, the number of pixels quantifies this quality.  In image classification evaluation, three datasets are mainly used:
\begin{itemize}
    \item \textbf{MNIST~\citep{lecun1998mnist}:} A dataset of black and white low-quality images representing the $10$ digits. The training set contains $50000$ images and test set $10000$ images. These images are of dimension $28\times28\times 1$ ($784$ in total). This dataset is known to be easy ($>99\%$ can be obtained using simple classifiers). In adversarial classification, the problem is also easy to be solved. Evaluation MNIST is not sufficient to assess the performance of a classifier or even a defense against adversarial examples.
    \item \textbf{CIFAR10 and CIFAR100~\citep{krizhevsky2009learning}:} Datasets of colored low-quality images representing the $10$ labels and $100$ labels for respectively CIFAR10 and CIFAR100. Each training set contains $50000$ images and test set $10000$ images. These images are of dimension $32\times32\times 1$ ($3072$ in total). The current state-of-the-art on CIFAR10 in standard classification is $>99\%$ of accuracy, but asks advanced methods to reach such a score. On CIFAR100, the current state-of-the-art is around $94\%$. In adversarial classification both datasets are challenging and difficult. The evolution of state-of-the-art in adversarial classification is available in RobustBench\footnote{\url{https://robustbench.github.io/}}. Benchmark in adversarial classification are often made on these datasets.
    \item \textbf{ImageNet~\citep{imagenet_cvpr09}:} ImageNet refers to a dataset containing $1.2$ million of images labeled into $1000$ classes. Images are of diverse qualities, but often $224\times224\times 3$ (dimension $150528$ in total). The current state-of-the-art on ImageNet is about $87\%$. There is no need to say that adversarial classification on ImageNet is still a very-challenging task. Further than the standard dataset, ImageNet project is still in development: the project gathers $14197122$ images and $21841$ labels on August 31th, 2021.   


\end{itemize}
\section{Background on adversarial examples}
\subsection{Crafting adversarial examples}

What is the more striking about adversarial examples is the facility to craft them. Let consider an attacker that aim  finding an adversarial perturbation $x'$ of an input $x$ for a given classifier $\mathbf{f}$.  Given a differentiable loss $L$, typically the cross-entropy, the attacker usually maximize the following objective:

\begin{align}
    \max_{x'\in\XX\text{ s.t. } d(x,x')\leq \varepsilon}L\left(\mathbf{f}(x'), y)\right).
\end{align}
To do so, many attacks were proposed that we will categorize in two parts: 
\begin{itemize}
    \item \textbf{White box attacks:} the attacker have the full knowledge of the function $\mathbf{f}$ and its parameters. Hence this attacks are often based on the gradient of the former objective. The most popular white box attacks are CITE
    \item \textbf{Black box attacks:} the attacker have no knowledge on the classifier parameters. The attacker have limited access to the classifier, e.g. he can only access logits or predicted class for instance.
    
\end{itemize}

\subsection{Which Perception Distance to Use?}

The choice of the 
% \section{Optimal Transportation}

% As seen Optimal Transportation seems to play a central role in understanding adversarial attacks, 
% For any Polish space $\mathcal{Z}$, we denote $\mathcal{M}_+^1(\mathcal{Z})$ the Polish space of Borel probability measures on $\mathcal{Z}$. Let us assume the data is drawn from $\PP\in\mathcal{M}_+^1(\mathcal{X}\times\mathcal{Y})$. Let $(\Theta,d_\Theta)$ be a Polish space (not necessarily proper) representing the set of classifier parameters (for instance neural networks). We also define a loss function: $l:\Theta\times (\mathcal{X}\times\mathcal{Y})\to [0,\infty)$ satisfying the following set of assumptions.
\chapter{Related Work}
\label{chap:rw}
\minitoc
\section{Q1}

\subsection{Optimal Transport}
The optimal 

\begin{definition}[Couplings between distributions] 
    Let $\ZZ$ be a Polish space. Let $\PP$ and $\QQ$ be two Borel probability distributions over $\ZZ$. The set of coupling distributions between $\PP$ and $\QQ$ is defined as:
    \begin{align*}
        \Gamma_{\PP,\QQ}:=\left\{\gamma\in\mathcal{M}^1_+(\ZZ^2)\mid~ \Pi_{1,\sharp}\gamma = \PP,~\Pi_{2,\sharp}\gamma = \QQ\right\}
    \end{align*}
where $\Pi_{i,\sharp}$ represents the push-forward of the projection on the $i$-th component.
\end{definition}
\begin{definition}[Optimal Transport]
Let $\ZZ$ be a Polish space. Let $c:\ZZ\to\bar{\RR}_+$ be a lower semi-continuous non-negative function. Let $\PP$ and $\QQ$ be two  Borel probability distributions over $\ZZ$. The Optimal Transport problem or Wasserstein problem between $\PP$ and $\QQ$ associated with cost function $c$ is defined as:
\begin{align*}
    W_c(\PP,\QQ):=\inf_{\gamma\in\Gamma_{\PP,\QQ}}\int c(x,y) d\gamma(x,y) = \inf_{\gamma\in\Gamma_{\PP,\QQ}}\mathbb{E}_{(x,y)\sim\gamma}\left[c(x,y) \right]
\end{align*} 
\end{definition}
The infimum is attained.

\begin{thm}[Kantorovich duality]
    Let $\ZZ$ be a Polish space. Let $c:\ZZ\to\bar{\RR}_+$ be a lower semi-continuous non-negative function. Let $\PP$ and $\QQ$ be two Borel probability distributions over $\ZZ$. Then   the following strong duality theorem holds:
\begin{align*}
    W_c(\PP,\QQ)=\sup_{f,g\in C(\ZZ),~f\oplus g\leq c}   \int fd\PP+\int fd\QQ
\end{align*}
where for all $x,y\in\ZZ$, $f\oplus g(x,y):=f(x)+g(y)$.
\end{thm}


\subsection{Distributionally Robust Optimization}


Let us consider the following problem. Let $\ZZ$ and $\Theta$ be Polish spaces. Let $\PP$ be a Borel probability distribution over $\ZZ$. Let $f:\Theta\times\ZZ\to\RR$ be an upper semi continuous function in its second variable. Let us consider the following problem:
\begin{align}
    \label{eq:min-objective}
    \min_{\theta\in\Theta} \mathbb{E}_{z\sim\PP}\left[f(z)\right] = \min_{\theta\in\Theta} \int f(\theta,z)d\PP(z)
\end{align}



This problem can typically cast a risk minimization problem in Machine Learning when $\PP$ is a distribution over input-label pairs. A distributionally robust optimization (DRO) problem is a problem similar to Equation~\eqref{eq:min-objective}, but the learner aims at being robust to a change in the distribution $\PP$. Typically if $D$ is an uncertainty metric for distribubtions. Formally, the DRO problem is casted as follows:
\begin{align*}
    \min_{\theta\in\Theta}\sup_{\QQ\in\mathcal{M}^1_+(\ZZ)\mid~D(\PP,\QQ)\leq \varepsilon}\mathbb{E}_{z\sim\QQ}\left[f(z)\right]
\end{align*}
For instance, $D$ be a Kullback-Leibler etc CITE

In the case of Wasserstein uncertainty sets, let $c:\ZZ\to\bar{\RR}_+$ be a lower semi-continuous non-negative function. Then a  Wasserstein distributionally robust optimization (DRO) problem is defined as follows:
\begin{align*}
    \min_{\theta\in\Theta}\sup_{\QQ\in\mathcal{M}^1_+(\ZZ)\mid~W_c(\PP,\QQ)\leq \varepsilon}\mathbb{E}_{z\sim\QQ}\left[f(z)\right]
\end{align*}

Then we can define the Wasserstein balls as 
\begin{align*}
    \mathcal{B}_{c}(\PP,\eta) := \left\{\QQ\in \mathcal{M}^+_1(\mathcal{Z})\mid W_c(\PP,\QQ)\leq \eta\right\}
\end{align*}
\paragraph{Properties of Wasserstein balls.} The Wasserstein balls inherits from nice properties. Since $\QQ\mapsto  W_c(\PP,\QQ)$ is convex, they are convex sets. Moreover the function $\QQ\mapsto  W_c(\PP,\QQ)$ is lower semi-continuous for the narrow topology of measures, then the set $\mathcal{B}_{c}(\PP,\eta) $ is closed for the narrow topology too. Concerning the compactness of this set, if $\ZZ$ is compact then the set $\mathcal{B}_{c}(\PP,\eta) $ is also compact as a closed subset of the compact set $\mathcal{M}^+_1(\mathcal{Z})$.~\cite{yue2020linear} proved the compactness for $l^p$ distances. In general, compactness is a case by case question. 


\paragraph{Duality results} The problem of computing DRO solutions is difficult become it concerns optimization over distribution. A strong duality leading to a relaxation of the problem was proved by~\cite{blanchet2019quantifying}. We state this theorem as follows.


\begin{thm}[Duality for Wasserstein DRO]
    Let $\PP$ be a Borel probability distribution over $\ZZ$. Let $f:\ZZ\to\RR$ be an upper semi continuous function. Let $c:\ZZ\to\RR_+$ be a lower semi-continuous non-negative function. 
    \begin{align*}
        \sup_{\QQ\in\mathcal{M}^1_+(\ZZ)\mid~W_c(\PP,\QQ)\leq \varepsilon}\mathbb{E}_{z\sim\QQ}\left[f(z)\right] = \inf_{\lambda\geq 0}\mathbb{E}_{z\sim\PP}\left[\sup_{z'\in\ZZ}f(z')-\lambda c(z,z')\right] +\lambda\varepsilon
    \end{align*}
  
\end{thm}


\subsection{Optimal Transport and Adversarial Learning}
\subsection{Game Theory and Adversarial Learning}
\section{Q2}
\newpage
\section{Q3}

We recall a classifier $h$ is \emph{certifiably robust at level $\varepsilon$} at input $x$ with label $y$ if there exist a property depending on $h$, $x$, $y$ and $\varepsilon$ that implies that for all $x'$ such that $d(x,x')\leq\varepsilon$, $h(x') = y$.




\subsection{Lipschitz Property of Neural Networks}


The Lipschitz constant has seen a growing interest in the last few years in the field of deep learning~\citep{scaman2018lipschitz,fazlyab2019efficient,combettes2020lipschitz,bethune2021many}.
Indeed, numerous results have shown that neural networks with a small Lipschitz constant exhibit better generalization~\citep{bartlett2017spectrally}, higher robustness to adversarial attacks~\citep{szegedy2014intriguing,farnia2018generalizable,tsuzuku2018lipschitz}, better training stability~\citep{xiao2018dynamical,trockman2021orthogonalizing}, improved Generative Adversarial Networks~\citep{arjovsky2017wasserstein}, etc.
Formally, we define the Lipschitz constant with respect to the $\ell_2$ norm of a Lipschitz continuous function $f$ as follows:
\begin{equation*}
  Lip_{2}{(f)} = \sup_{\substack{x, x' \in \XX \\ x \neq x'}} \frac{\lVert f(x) - f(x') \rVert_2}{\lVert x - x' \rVert_2} \enspace.
\end{equation*}

Intuitively, if a classifier is Lipschitz, one can bound the impact of a given input variation on the output, hence obtaining guarantees on the adversarial robustness.
We can formally characterize the robustness of a neural network with respect to its Lipschitz constant with the following proposition:
\begin{prop}[\citet{tsuzuku2018lipschitz}] \label{proposition:tsuzuku}
Let $f:\XX\to\RR^K$ be an $L$-Lipschitz continuous classifier for the $\ell_2$ norm.
Let $\varepsilon > 0$, $x \in \XX$ and $y \in \YY$ the label of $x$.
If at point $x$, the margin $\mathcal{M}_{\mathbf{f}}(x)$ satisfies:
\begin{equation*}
  \mathcal{M}_{\mathbf{f}}(x):=\max(0,f_y(x)-\max_{y'\neq y}f_{y'}(x)) > \sqrt{2} L \varepsilon
\end{equation*}
then we have for every $\tau$ such that $\lVert \tau \rVert_2 \leq \varepsilon$:
\begin{equation*}
  \argmaxB_{k}f_k(x + \tau) = y
\end{equation*}
\end{prop}
From Proposition~\ref{proposition:tsuzuku}, it is straightforward to compute a robustness certificate for a given point.
Consequently, in order to build robust neural networks the margin needs to be large and the Lipschitz constant small to get optimal guarantees on the robustness for neural networks.

\paragraph{Lipschitz Constant of Neural Networks.}  A neural network is a function $f$ defined succession of linear and non-linear activation functions $\sigma$:
\begin{align*}
  f(x) = \left(A_L\sigma\left(A_{L-1}\dots \sigma\left(A_1x+b_1\right)\dots\right)+b_L\right)
\end{align*}
Assuming that $\sigma$ is $1$-Lipschitz, we have:
\begin{align*}
  \lVert f(x)-f(y)\rVert_2\leq \lVert A_1\rVert_2\dots \lVert A_L\rVert_2\lVert x-y\rVert_2
\end{align*}
with $\lVert A\rVert_2$ is the spectral norm of $A$ defined as
\begin{align*}
  \lVert A\rVert_2 = \max_{x\neq 0} \frac{\lVert Ax\rVert_2}{\lVert x\rVert_2} = \lambda_{max}(A^TA)\quad.
\end{align*}
where $\lambda_{max}(A^TA)$ denotes the greatest eigen value of $A^TA$. Note that $\lVert A\rVert_2$ is also the greatest singular value of $A$.  
Then the Lipschitz constant of $f$ is upperbounded by $\lVert A_1\rVert_2\dots \lVert A_L\rVert_2$. Hence to control the Lipschitz constant of a neural network, it is usual to control the spectral norm of each layer. It could be done either in penalizing this upperbound or imposing a spectral norm equals smaller than $1$ for each layer. 

\subsection{Learning Lipschitz layers}

Many research proposed methods to build 1-Lipschitz layers in order to boost adversarial robustness. These approaches provide deterministic guarantees for adversarial robustness. One can either normalize the weight matrices by their largest singular values making the layer $1$-Lipschitz, \emph{e.g.}~\citep{yoshida2017spectral,miyato2018spectral,farnia2018generalizable,anil2019sorting} or project the weight matrices on the Stiefel manifold \citep{li2019preventing,trockman2021orthogonalizing,skew2021sahil}.

The first natural idea to learn $1$-Lipschitz layers is to normalize the matrices in the forward pass of a Neural Networks : $A_i\leftarrow \frac{A_i}{\lVert A_i\rVert_2}$. This natural idea was exploited by~\citet{miyato2018spectral}. A key difficulty is the computation of the spectral norm $\lVert A_i\rVert_2$. The authors proposed to use the power iteration method to compute the spectral norm (see Algorithm~\ref{xxx}). The number of iterations might be prohibitive, hence the authors proposed to use only one step in the training phase to make it faster. This method effectively approximated well the sprectral norm of the last layer. However, this method present some disadvantages. The spectral normalization has for effect crushing all smaller singular values. A consequence is the gradient vanishing that is very present in this structure. 

Also, several works~\cite{anil2019sorting,singla2021householder,huang2021local} proposed methods leveraging the properties of activation functions to constraints the Lipschitz of Neural Networks. These works are usually useful to help  improving the performance of linear orthogonal layers.



\paragraph{Learning Orthogonal layers} A workaround for the limitations of previously presented methods is to build norm preserving linear layers, i.e. orthogonal layers. We recall a matrix $\Omega\in\RR^{d\times d}$ is said to be orthogonal if for every $x\in\RR^d$, $\lVert\Omega x\rVert_2 = \lVert x\rVert_2$. Indeed such layers exactly preserve the norm, hence avoid crushing all singular values and gradient vanishing issues. Recently, there have been a trend in aiming at learning Orthogonal Layers in neural networks.  The following approaches consist of projecting the weights matrices onto an orthogonal space in order to preserve gradient norms and enhance adversarial robustness by guaranteeing low Lipschitz constants. While both works have similar objectives, their execution is different .It is a difficult question to conciliate the convolution structure with orthogonality of linear layers. The presented works of \citet{li2019preventing}, \citet{trockman2021orthogonalizing} and \citet{skew2021sahil} (denoted BCOP, Cayley and SOC respectively) present the advantage of being ``compatible''  with convolutional structure in layers. 

The BCOP layer (Block Convolution Orthogonal Parameterization) uses an iterative algorithm proposed by \citet{bjorck1971iterative} to orthogonalize a linear transformation. The BCOP layer relies on the following algorithm to orthonormalize a linear operator $M$:
\begin{align*}
    xxxx
\end{align*}
To build a ``convolutional layer'' from the BCOP the matrix $M$ can be structured as a convolutional operator is a standard Deep Learning framework as Tensorflow~\citep{abadi2016deep} or PyTorch~\citep{paszke2019pytorch}. TODO DESCRIBE

Two other alternatives, the SOC layer (Skew Orthogonal Convolution) and the Cayley layer, used two different parametrization of the Special Orthogonal Group $SO_n(\RR)$ using skew-symmetric matrices. Indeed, in Riemmanian geometry, the space skew-symmetric matrices is isomorphic to the tangent space of $SO_n(\RR)$ at any point. 

SOC layers uses the expontial mapping of a skew symmetric matrix defined using the following Taylor expansion:
\begin{align*}
  \exp{A}:=\sum_{k=0}^{\infty}\frac{A^k}{k!}
\end{align*}
which defines an orthogonal matrix, indeed $(\exp{A})^T\exp{A} = \exp(A^T)\exp(A) = \exp(-A)\exp(A) = \exp(A-A)=I$ . More precisely, the application $A\mapsto\exp{A}$ defines a surjective mapping of $SO_n(\RR)$ from the space of skew-symmetric matrices. To approximate the exponential of a matrix, the authors proposed to use a finite number of terms in its Taylor series expansion.  To be adapted to convolutions, a skew-symmetric linear transformation $A = M-M^T$ can be  computed in a Deep Learning Framework using the  convolution and convolution-transpose operators.



The Cayley method proposed by~\citet{trockman2021orthogonalizing} use the Cayley transform to orthogonalize the weights matrices. Given a skew symmetric matrix $A$, the Cayley transform consists in computing the orthogonal matrix:
\begin{align*}
   \text{Cayley}(A)= (I - A)^{-1} (I + A) \quad.
\end{align*}
Like exponential mapping, the Cayley Tranform defines  a surjective mapping of $SO_n(\RR)$ from the space of skew-symmetric matrices. TODO: adaptabiltilty to skew





\subsection{Residual Networks}

During the training phase in neural networks, it may occur some issues as gradient vanishing or gradient expolosion~\citep{hochreiter2001gradient}. These issues limited the emergence of scalable and very deep neural networks until~\cite{he2016deep} proposed the Residual Network (ResNet) architecture defined as follows.
\begin{align*}
  \left\{
    \begin{array}{ll}
    x_0 &= x\in\XX\\
    x_{t+1} &= x_t+F_{{t}}(x_{t}) \  \text{for } \ t\in\{0, \dots,T\}
  \end{array}
  \right.
\end{align*}
where $F_{{t}}(x_{t})$ is typically a two layer neural networks. The ResNet uses residual connection that have the effect of limiting gradient vanishing issues. Combined with batch normalization, the issue of gradient explosion can also be mitigated, hence opening the possibility to very deep and stable architecture. 

To theoretically analyse the ResNet architecture, several works~\citep{haber2017stable,e17Proposal,lu18beyond,chen2018neural} proposed a ``continuous time'' interpretation inspired by dynamical systems that can be defined as follows.

\begin{definition}\label{def:flow}
Let $(F_{t})_{t\in[0,T]}$ be a family of functions on $\RR^d$, we define the continuous time Residual Networks flow associated with $F_t$ as:
\begin{align*}
  \left\{
    \begin{array}{ll}
    x_0 &= x\in\mathcal{X}\\
    \frac{dx_{t}}{dt} &= F_{{t}}(x_{t}) \  \text{for } \ t\in[0, T]
  \end{array}
  \right.
\end{align*}
\end{definition}

This continuous time interpretation helps as it allows us to consider the stability of the forward propagation through the stability of the associated dynamical system.
A dynamical system is said to be \emph{stable} if two trajectories starting from an input and another one remain sufficiently close to each other all along the propagation. This stability property takes all its sense in the context of adversarial classification.

It was argued by~\citet{haber2017stable} that when $F_{t}$ does not depend on $t$ or vary slowly with time\footnote{This blurry definition of "vary slowly" makes the property difficult to apply.}, the stability can be characterized by the eigenvalues of the Jacobian matrix $\nabla_x F_{t}(x_t)$: 
the dynamical system is stable if the real part of the eigenvalues of the Jacobian stay negative throughout the propagation.
This property however only relies on intuition and this condition might be difficult to  verify in practice.
In the following, in order to derive stability properties, we study gradient flows and convex potentials, which are sub-classes of Residual networks.

Other works~\citep{huang2020adversarial,li2020implicit} also proposed to enhance adversarial robustness using dynamical systems interpretations of Residual Networks. Both works argues that using particular discretization scheme would make gradient attacks more difficult to compute due to numerical stability. These works did not provide any provable guarantees for such approaches.





% \section{Q2}
\newpage
% \section{Q3}

We recall a classifier $h$ is \emph{certifiably robust at level $\varepsilon$} at input $x$ with label $y$ if there exist a property depending on $h$, $x$, $y$ and $\varepsilon$ that implies that for all $x'$ such that $d(x,x')\leq\varepsilon$, $h(x') = y$.




\subsection{Lipschitz Property of Neural Networks}


The Lipschitz constant has seen a growing interest in the last few years in the field of deep learning~\citep{scaman2018lipschitz,fazlyab2019efficient,combettes2020lipschitz,bethune2021many}.
Indeed, numerous results have shown that neural networks with a small Lipschitz constant exhibit better generalization~\citep{bartlett2017spectrally}, higher robustness to adversarial attacks~\citep{szegedy2014intriguing,farnia2018generalizable,tsuzuku2018lipschitz}, better training stability~\citep{xiao2018dynamical,trockman2021orthogonalizing}, improved Generative Adversarial Networks~\citep{arjovsky2017wasserstein}, etc.
Formally, we define the Lipschitz constant with respect to the $\ell_2$ norm of a Lipschitz continuous function $f$ as follows:
\begin{equation*}
  Lip_{2}{(f)} = \sup_{\substack{x, x' \in \XX \\ x \neq x'}} \frac{\lVert f(x) - f(x') \rVert_2}{\lVert x - x' \rVert_2} \enspace.
\end{equation*}

Intuitively, if a classifier is Lipschitz, one can bound the impact of a given input variation on the output, hence obtaining guarantees on the adversarial robustness.
We can formally characterize the robustness of a neural network with respect to its Lipschitz constant with the following proposition:
\begin{prop}[\citet{tsuzuku2018lipschitz}] \label{proposition:tsuzuku}
Let $f:\XX\to\RR^K$ be an $L$-Lipschitz continuous classifier for the $\ell_2$ norm.
Let $\varepsilon > 0$, $x \in \XX$ and $y \in \YY$ the label of $x$.
If at point $x$, the margin $\mathcal{M}_{\mathbf{f}}(x)$ satisfies:
\begin{equation*}
  \mathcal{M}_{\mathbf{f}}(x):=\max(0,f_y(x)-\max_{y'\neq y}f_{y'}(x)) > \sqrt{2} L \varepsilon
\end{equation*}
then we have for every $\tau$ such that $\lVert \tau \rVert_2 \leq \varepsilon$:
\begin{equation*}
  \argmaxB_{k}f_k(x + \tau) = y
\end{equation*}
\end{prop}
From Proposition~\ref{proposition:tsuzuku}, it is straightforward to compute a robustness certificate for a given point.
Consequently, in order to build robust neural networks the margin needs to be large and the Lipschitz constant small to get optimal guarantees on the robustness for neural networks.

\paragraph{Lipschitz Constant of Neural Networks.}  A neural network is a function $f$ defined succession of linear and non-linear activation functions $\sigma$:
\begin{align*}
  f(x) = \left(A_L\sigma\left(A_{L-1}\dots \sigma\left(A_1x+b_1\right)\dots\right)+b_L\right)
\end{align*}
Assuming that $\sigma$ is $1$-Lipschitz, we have:
\begin{align*}
  \lVert f(x)-f(y)\rVert_2\leq \lVert A_1\rVert_2\dots \lVert A_L\rVert_2\lVert x-y\rVert_2
\end{align*}
with $\lVert A\rVert_2$ is the spectral norm of $A$ defined as
\begin{align*}
  \lVert A\rVert_2 = \max_{x\neq 0} \frac{\lVert Ax\rVert_2}{\lVert x\rVert_2} = \lambda_{max}(A^TA)\quad.
\end{align*}
where $\lambda_{max}(A^TA)$ denotes the greatest eigen value of $A^TA$. Note that $\lVert A\rVert_2$ is also the greatest singular value of $A$.  
Then the Lipschitz constant of $f$ is upperbounded by $\lVert A_1\rVert_2\dots \lVert A_L\rVert_2$. Hence to control the Lipschitz constant of a neural network, it is usual to control the spectral norm of each layer. It could be done either in penalizing this upperbound or imposing a spectral norm equals smaller than $1$ for each layer. 

\subsection{Learning Lipschitz layers}

Many research proposed methods to build 1-Lipschitz layers in order to boost adversarial robustness. These approaches provide deterministic guarantees for adversarial robustness. One can either normalize the weight matrices by their largest singular values making the layer $1$-Lipschitz, \emph{e.g.}~\citep{yoshida2017spectral,miyato2018spectral,farnia2018generalizable,anil2019sorting} or project the weight matrices on the Stiefel manifold \citep{li2019preventing,trockman2021orthogonalizing,skew2021sahil}.

The first natural idea to learn $1$-Lipschitz layers is to normalize the matrices in the forward pass of a Neural Networks : $A_i\leftarrow \frac{A_i}{\lVert A_i\rVert_2}$. This natural idea was exploited by~\citet{miyato2018spectral}. A key difficulty is the computation of the spectral norm $\lVert A_i\rVert_2$. The authors proposed to use the power iteration method to compute the spectral norm (see Algorithm~\ref{xxx}). The number of iterations might be prohibitive, hence the authors proposed to use only one step in the training phase to make it faster. This method effectively approximated well the sprectral norm of the last layer. However, this method present some disadvantages. The spectral normalization has for effect crushing all smaller singular values. A consequence is the gradient vanishing that is very present in this structure. 

Also, several works~\cite{anil2019sorting,singla2021householder,huang2021local} proposed methods leveraging the properties of activation functions to constraints the Lipschitz of Neural Networks. These works are usually useful to help  improving the performance of linear orthogonal layers.



\paragraph{Learning Orthogonal layers} A workaround for the limitations of previously presented methods is to build norm preserving linear layers, i.e. orthogonal layers. We recall a matrix $\Omega\in\RR^{d\times d}$ is said to be orthogonal if for every $x\in\RR^d$, $\lVert\Omega x\rVert_2 = \lVert x\rVert_2$. Indeed such layers exactly preserve the norm, hence avoid crushing all singular values and gradient vanishing issues. Recently, there have been a trend in aiming at learning Orthogonal Layers in neural networks.  The following approaches consist of projecting the weights matrices onto an orthogonal space in order to preserve gradient norms and enhance adversarial robustness by guaranteeing low Lipschitz constants. While both works have similar objectives, their execution is different .It is a difficult question to conciliate the convolution structure with orthogonality of linear layers. The presented works of \citet{li2019preventing}, \citet{trockman2021orthogonalizing} and \citet{skew2021sahil} (denoted BCOP, Cayley and SOC respectively) present the advantage of being ``compatible''  with convolutional structure in layers. 

The BCOP layer (Block Convolution Orthogonal Parameterization) uses an iterative algorithm proposed by \citet{bjorck1971iterative} to orthogonalize a linear transformation. The BCOP layer relies on the following algorithm to orthonormalize a linear operator $M$:
\begin{align*}
    xxxx
\end{align*}
To build a ``convolutional layer'' from the BCOP the matrix $M$ can be structured as a convolutional operator is a standard Deep Learning framework as Tensorflow~\citep{abadi2016deep} or PyTorch~\citep{paszke2019pytorch}. TODO DESCRIBE

Two other alternatives, the SOC layer (Skew Orthogonal Convolution) and the Cayley layer, used two different parametrization of the Special Orthogonal Group $SO_n(\RR)$ using skew-symmetric matrices. Indeed, in Riemmanian geometry, the space skew-symmetric matrices is isomorphic to the tangent space of $SO_n(\RR)$ at any point. 

SOC layers uses the expontial mapping of a skew symmetric matrix defined using the following Taylor expansion:
\begin{align*}
  \exp{A}:=\sum_{k=0}^{\infty}\frac{A^k}{k!}
\end{align*}
which defines an orthogonal matrix, indeed $(\exp{A})^T\exp{A} = \exp(A^T)\exp(A) = \exp(-A)\exp(A) = \exp(A-A)=I$ . More precisely, the application $A\mapsto\exp{A}$ defines a surjective mapping of $SO_n(\RR)$ from the space of skew-symmetric matrices. To approximate the exponential of a matrix, the authors proposed to use a finite number of terms in its Taylor series expansion.  To be adapted to convolutions, a skew-symmetric linear transformation $A = M-M^T$ can be  computed in a Deep Learning Framework using the  convolution and convolution-transpose operators.



The Cayley method proposed by~\citet{trockman2021orthogonalizing} use the Cayley transform to orthogonalize the weights matrices. Given a skew symmetric matrix $A$, the Cayley transform consists in computing the orthogonal matrix:
\begin{align*}
   \text{Cayley}(A)= (I - A)^{-1} (I + A) \quad.
\end{align*}
Like exponential mapping, the Cayley Tranform defines  a surjective mapping of $SO_n(\RR)$ from the space of skew-symmetric matrices. TODO: adaptabiltilty to skew





\subsection{Residual Networks}

During the training phase in neural networks, it may occur some issues as gradient vanishing or gradient expolosion~\citep{hochreiter2001gradient}. These issues limited the emergence of scalable and very deep neural networks until~\cite{he2016deep} proposed the Residual Network (ResNet) architecture defined as follows.
\begin{align*}
  \left\{
    \begin{array}{ll}
    x_0 &= x\in\XX\\
    x_{t+1} &= x_t+F_{{t}}(x_{t}) \  \text{for } \ t\in\{0, \dots,T\}
  \end{array}
  \right.
\end{align*}
where $F_{{t}}(x_{t})$ is typically a two layer neural networks. The ResNet uses residual connection that have the effect of limiting gradient vanishing issues. Combined with batch normalization, the issue of gradient explosion can also be mitigated, hence opening the possibility to very deep and stable architecture. 

To theoretically analyse the ResNet architecture, several works~\citep{haber2017stable,e17Proposal,lu18beyond,chen2018neural} proposed a ``continuous time'' interpretation inspired by dynamical systems that can be defined as follows.

\begin{definition}\label{def:flow}
Let $(F_{t})_{t\in[0,T]}$ be a family of functions on $\RR^d$, we define the continuous time Residual Networks flow associated with $F_t$ as:
\begin{align*}
  \left\{
    \begin{array}{ll}
    x_0 &= x\in\mathcal{X}\\
    \frac{dx_{t}}{dt} &= F_{{t}}(x_{t}) \  \text{for } \ t\in[0, T]
  \end{array}
  \right.
\end{align*}
\end{definition}

This continuous time interpretation helps as it allows us to consider the stability of the forward propagation through the stability of the associated dynamical system.
A dynamical system is said to be \emph{stable} if two trajectories starting from an input and another one remain sufficiently close to each other all along the propagation. This stability property takes all its sense in the context of adversarial classification.

It was argued by~\citet{haber2017stable} that when $F_{t}$ does not depend on $t$ or vary slowly with time\footnote{This blurry definition of "vary slowly" makes the property difficult to apply.}, the stability can be characterized by the eigenvalues of the Jacobian matrix $\nabla_x F_{t}(x_t)$: 
the dynamical system is stable if the real part of the eigenvalues of the Jacobian stay negative throughout the propagation.
This property however only relies on intuition and this condition might be difficult to  verify in practice.
In the following, in order to derive stability properties, we study gradient flows and convex potentials, which are sub-classes of Residual networks.

Other works~\citep{huang2020adversarial,li2020implicit} also proposed to enhance adversarial robustness using dynamical systems interpretations of Residual Networks. Both works argues that using particular discretization scheme would make gradient attacks more difficult to compute due to numerical stability. These works did not provide any provable guarantees for such approaches.





\chapter{Game Theory of Adversarial Examples}
\label{chap:game}
\minitoc

In this chapter, we answer \textbf{Question 1: ``What is the nature of equilibria in the adversarial examples game?''} by proving the existence of Mixed Nash equilibria in the adversarial example game when both the adversary and the classifier can use randomized strategies. First, we motivate in Section~\ref{sec:adv-problem} the necessity for using randomized strategies both with the attacker and the classifier. Then, we extend the work of~\cite{pydi2019adversarial}, by rigorously reformulating the adversarial risk as a linear optimization problem over distributions. In fact, we cast the adversarial risk minimization problem as a Distributionally Robust Optimization (DRO)~\citep{blanchet2019quantifying} problem for a well suited cost function. This formulation naturally leads us, in Section~\ref{sec:nash-eq}, to analyze adversarial risk minimization as a zero-sum game. We demonstrate that, in this game, the duality gap always equals $0$, meaning that it always admits approximate mixed Nash equilibria.  

Afterwards, we aim at designing an efficient algorithm to learn an optimally robust randomized classifier.
We focus on learning a finite mixture of classifiers. Drawing inspiration from robust optimization~\cite{sinha2017certifying} and subgradient methods~\cite{boyd2003subgradient}, we derive in Section~\ref{sec:algo} a first oracle algorithm to optimize a finite mixture. Then, following the line of work of~\citep{cuturi2013sinkhorn}, we introduce an entropic regularization to effectively compute an approximation of the optimal mixture. We validate our findings with experiments on simulated and real  datasets, namely CIFAR-10 an CIFAR-100~\cite{krizhevsky2009learning}.






\section{The Adversarial Attack Problem}
\label{sec:adv-problem}
\subsection{A Motivating Example}
\label{sec:motiv-ex}

\begin{figure*}[!ht]
    \centering
\includegraphics[width=\textwidth]{Images/Drawing-Intro-Mixte-Nash-on-a-line.pdf}   \caption{Motivating example: blue distribution represents label $-1$ and the red one, label $+1$.  The height of columns represents their mass. The red and blue arrows represent the attack on the given classifier. On left: deterministic classifiers ($f_1$ on the left, $f_2$ in the middle) for whose, the blue point can always be attacked. On right: a randomized classifier, where the attacker has a probability $1/2$ of failing, regardless of the attack it selects. }
    \label{fig:motivating_ex}
\end{figure*}
Consider the binary classification task illustrated in Figure~\ref{fig:motivating_ex}. We assume that all input-output pairs $(X,Y)$ are sampled from a distribution $\PP$ defined as follows
$$ 
\PP\left(Y =\pm 1\right)=1/2 \ \mbox{ and }\left\{
    \begin{array}{ll}
        \PP\left(X=0 \mid Y=-1\right) = 1 \\
        \PP\left(X= \pm 1 \mid Y=1\right) = 1/2 
    \end{array}
\right.
$$ 
Given access to $\PP$, the adversary aims to maximize the expected risk, but can only move each point by at most $1$ on the real line. In this context, we study two classifiers: $f_1(x) = -x -1/2$ and $f_2(x)=x-1/2$\footnote{$(X,Y) \sim \PP$ is misclassified by $f_i$ if and only if $f_i(X)Y \leq 0$}. Both $f_1$ and $f_2$ have a standard risk of $1/4$. In the presence of an adversary, the risk (\emph{a.k.a.} the adversarial risk) increases to $1$. Here, using a randomized classifier can make the system more robust. Consider $f$ where $f=f_1$ w.p. $1/2$ and $f_2$ otherwise. The standard risk of $f$ remains $1/4$ but its adversarial risk is $3/4<1$. Indeed, when attacking $f$, any adversary will have to choose between moving points from $0$ to $1$ or to $-1$. Either way, the attack only works half of the time; hence an overall adversarial risk of $3/4$. Furthermore, if $f$ knows the strategy the adversary uses, it can always update the probability it gives to $f_1$ and $f_2$ to get a better (possibly deterministic) defense. For example, if the adversary chooses to always move $0$ to $1$, the classifier can set $f=f_1$ w.p. $1$ to retrieve an adversarial risk of $1/2$ instead of $3/4$. %In other words, if the adversary can only use a deterministic attack, then there exists no equilibrium in the game.

Now, what happens if the adversary can use randomized strategies, meaning that for each point it can flip a coin before deciding where to move? In this case, the adversary could decide to move points from $0$ to $1$ w.p. $1/2$ and to $-1$ otherwise. This strategy is still optimal with an adversarial risk of $3/4$ but now the classifier cannot use its knowledge of the adversary's strategy to lower the risk. We are in a state where neither the adversary nor the classifier can benefit from unilaterally changing its strategy. In the game theory terminology, this state is called a Mixed Nash equilibrium. %Previous works studying adversarial examples from the scope of game theory investigated the existence of Mixed Nash equilibria in restricted settings where the adversary can by reduced to a set of parameters~\citep{7533509,DBLP:journals/corr/abs-1906-02816,bose2021adversarial}. But, as pointed out in \citep{pinot2020randomization} and \citep{pydi2019adversarial}, studying the existence of a Mixed Nash equilibrium in a more general framework is a challenging open problem. In the present paper we tackle the following question.

\subsection{General setting}
Let us consider a loss function: $\loss:\Theta\times (\XX\times\YY)\to [0,\infty)$ satisfying the following set of assumptions.
\begin{assump}[Loss function]
\label{ass:loss}
1) The loss function $\loss$ is a non negative Borel measurable function. 2) For all $\theta\in\Theta$, $\loss(\theta,\cdot)$ is upper-semi continuous. 3) There exists $M>0$ such that for all $\theta\in\Theta$, $(x,y)\in\XX\times\YY$, $0\leq \loss(\theta,(x,y))\leq M$.
\end{assump}
It is usual to assume upper-semi continuity when studying optimization over distributions~\citep{villani2003topics,blanchet2019quantifying}. Furthermore, considering bounded (and positive) loss functions is also very common in learning theory~\citep{bartlett2002rademacher} and is not restrictive. 

In the adversarial examples framework, the loss of interest is the $0/1$ loss, for whose surrogates are misunderstood  and is the object of Chapter~\ref{chap:calibration}; hence it is essential that a $0/1$ loss satisfies Assumption~\ref{ass:loss}. In the binary classification setting (\emph{i.e.} $\YY=\{-1,+1\}$) a possible $0/1$ loss writes $\loss_{0/1}(\theta,(x,y)) = \mathbf{1}_{yf_\theta(x)\leq 0}$. Then, assuming that for all $\theta$, $f_\theta(\cdot)$ is continuous and for all $x$, $f_\cdot(x)$ is continuous, the $0/1$ loss satisfies Assumption~\ref{ass:loss}. In particular, it is the case for neural networks with continuous activation functions.


\subsection{Measure Theoretic Lemmas}

We first recall and prove some important lemmas about theoretic measure.
\begin{lemma}[Fubini's theorem]
\label{lem:fubini}
Let $l:\Theta\times(\XX\times\YY)\rightarrow [0,\infty)$ satisfying Assumption~\ref{ass:loss}. Then for all $\mu\in\mathcal{M}^1_+(\Theta)$, $\int \loss(\theta,\cdot)d\mu(\theta)$ is Borel measurable; for  $\QQ\in\mathcal{M}^1_+(\XX\times\YY)$, $\int \loss(\cdot,(x,y))d\QQ(x,y)$ is Borel measurable. Moreover: $\int \loss(\theta,(x,y))d\mu(\theta)d\QQ(x,y)=\int \loss(\theta,(x,y))d\QQ(x,y)d\mu(\theta)$
\end{lemma}

\begin{lemma}
\label{lem:usc1}
Let $\loss:\Theta\times(\XX\times\YY)\rightarrow [0,\infty)$ satisfying Assumption~\ref{ass:loss}.
Then for all $\mu\in\mathcal{M}^1_+(\Theta)$, $(x,y)\mapsto\int \loss(\theta,(x,y))d\mu(\theta)$ is upper semi-continuous and hence Borel measurable.  
\end{lemma}
\begin{proof}
Let $(x_n,y_n)_n$ be a sequence of $\XX\times\YY$ converging to $(x,y)\in\XX\times\YY$.  For all $\theta\in\Theta$, $M-\loss(\theta,\cdot)$ is non negative and lower semi-continuous. Then by Fatou's Lemma applied:
\begin{align*}
   \int M-\loss(\theta,(x,y))d\mu(\theta)&\leq\int \liminf_{n\to\infty}  M-\loss(\theta,(x_n,y_n))d\mu(\theta)\\
   &\leq  \liminf_{n\to\infty}  \int M-\loss(\theta,(x_n,y_n))d\mu(\theta) 
\end{align*}

Then we deduce that: $\int M- \loss(\theta,\cdot)d\mu(\theta)$ is lower semi-continuous and then $\int \loss(\theta,\cdot)d\mu(\theta)$ is upper-semi continuous.
\end{proof}


\begin{lemma}
\label{lem:usc2}

Let $\loss:\Theta\times(\XX\times\YY)\rightarrow [0,\infty)$ satisfying Assumption~\ref{ass:loss}
Then for all $\mu\in\mathcal{M}^1_+(\Theta)$, $\QQ\mapsto\int \loss(\theta,(x,y))d\mu(\theta)d\QQ(x,y)$ is upper semi-continuous for weak topology of measures. 
\end{lemma}
\begin{proof}
 $-\int \loss(\theta,\cdot)d\mu(\theta) $ is lower semi-continuous from Lemma~\ref{lem:usc1}. Then $M-\int \loss(\theta,\cdot)d\mu(\theta) $ is lower semi-continuous and non negative. Let denote $v$ this function. Let $(v_n)_n$ be a non-decreasing sequence of continuous bounded functions such that $v_n\to v$. Let $(\QQ_k)_k$ converging weakly towards $\QQ$. Then by monotone convergence:
 
 \begin{align*}
     \int vd\QQ = \lim_n \int v_nd\QQ =\lim_n \lim_k\int v_nd\QQ_k\leq \liminf_k \int vd\QQ_k
 \end{align*}
 Then $\QQ\mapsto\int vd\QQ$ is lower semi-continuous and then $\QQ\mapsto\int \loss(\theta,(x,y))d\mu(\theta)d\QQ(x,y)$ is upper semi-continuous for weak topology of measures. 
 \end{proof}



\begin{lemma}
\label{lem:measure-sup}
Let $\loss:\Theta\times(\XX\times\YY)\rightarrow [0,\infty)$ satisfying Assumption~\ref{ass:loss}.
Then for all $\mu\in\mathcal{M}^1_+(\Theta)$, $(x,y)\mapsto \sup_{(x',y'),d(x,x')\leq\varepsilon,y=y'} \int \loss(\theta,(x',y'))d\mu(\theta)$ is universally measurable (i.e. measurable for all Borel probability measures). And hence the adversarial risk is well defined. 
\end{lemma}
\begin{proof}
Let $\phi :(x,y)\mapsto \sup_{(x',y'),d(x,x')\leq\varepsilon,y=y'} \int \loss(\theta,(x',y'))d\mu(\theta)$. Then for $u\in\bar{\mathbb{R}}$:
\begin{align*}
\left\{\phi(x,y)>u\right\}=\text{Proj}_1\left\{((x,y),(x',y'))\mid\int \loss(\theta,(x',y'))d\mu(\theta)-c_\varepsilon((x,y),(x',y'))>u\right\}
\end{align*}
By Lemma~\ref{lem:usc2}: $((x,y),(x',y'))\mapsto \int \loss(\theta,(x',y'))d\mu(\theta)-c_\varepsilon((x,y),(x',y'))$ is upper-semicontinuous hence Borel measurable. So its level sets are Borel sets, and by~\citep[Proposition 7.39]{bertsekas2004stochastic}, the projection of a Borel set is analytic. And then $\left\{\phi(x,y)>u\right\}$ universally measurable thanks to~\citep[Corollary 7.42.1]{bertsekas2004stochastic}. We deduce that $\phi$ is universally measurable.
\end{proof}



\subsection{Adversarial Risk Minimization}
The standard risk for a single classifier $\theta$ associated with the loss $\loss$ satisfying Assumption~\ref{ass:loss} writes: $\risk(\theta):=\mathbb{E}_{(x,y)\sim \PP}\left[\loss(\theta,(x,y))\right]$. Similarly, the adversarial risk of $\theta$ at level $\varepsilon$ associated with the loss $\loss$ is defined as\footnote{For the well-posedness, see Lemma~\ref{lem:measure-sup}.}
\begin{align*}
    \risk_\varepsilon(\theta):=\mathbb{E}_{(x,y)\sim \PP}\left[\sup_{x'\in\XX,~d(x,x')\leq\varepsilon}\loss(\theta,(x',y))\right].
\end{align*}
 It is clear that $\risk_0(\theta) =\risk(\theta)$ for all $\theta$. We can generalize these notions with distributions of classifiers. In other terms the classifier is then randomized according to some distribution $\mu\in\mathcal{M}^1_+(\Theta)$. A classifier is randomized if for a given input, the output of the classifier is a probability distribution.
 The standard risk of a randomized classifier $\mu$ writes $\risk(\mu) = \mathbb{E}_{\theta\sim\mu}\left[\risk (\theta)\right]$. Similarly, the adversarial risk of the randomized classifier $\mu$ at level $\varepsilon$ is\footnote{This risk is also well posed (see Lemma~\ref{lem:measure-sup}).}
\begin{align*}
    \risk_\varepsilon(\mu):=\mathbb{E}_{(x,y)\sim \PP}\left[\sup_{x'\in\XX,~d(x,x')\leq\varepsilon}\mathbb{E}_{\theta\sim\mu}\left[\loss(\theta,(x',y))\right]\right].
\end{align*}
For instance, for the $0/1$ loss, the inner maximization problem, consists in maximizing the probability of misclassification for a given couple $(x,y)$. Note that $\risk(\delta_\theta)=\risk(\theta)$ and $\risk_\varepsilon(\delta_\theta)=\risk_\varepsilon(\theta)$. In the remainder of this section, we study the adversarial risk minimization problems with randomized and deterministic classifiers and denote
\begin{align}
\label{eq:advriskmin}
    \valuerand_\varepsilon:=\inf_{\mu\in\mathcal{M}^1_+(\Theta)} \risk_\varepsilon(\mu),~\valuedet_\varepsilon:=\inf_{\theta\in\Theta} \risk_\varepsilon(\theta)
\end{align}
Note that we can show that the standard risk infima are equal :  $\valuerand_0=\valuedet_0$. 
\begin{prop}
\label{prop:eqstandardrisk}
Let $\PP$ be a Borel probability distribution on $\XX\times\YY$, and $l$ a loss satisfying Assumption~\ref{ass:loss}, then:
\begin{align*}
        \inf_{\mu\in\mathcal{M}^1_+(\Theta)} \risk(\mu) =\inf_{\theta\in\Theta} \risk(\theta)
\end{align*}
\end{prop}
\begin{proof}
It is clear that:         $\inf_{\mu\in\mathcal{M}^1_+(\Theta)} \risk(\mu) \leq \inf_{\theta\in\Theta} \risk(\theta)$. Now, let $\mu\in\mathcal{M}^1_+(\Theta)$, then:
\begin{align*}
    \risk(\mu)= \mathbb{E}_{\theta\sim\mu}(\risk(\theta))&\geq \essinf_\mu \mathbb{E}_{\theta\sim\mu} \left(\risk(\theta)\right)\\
    &\geq\inf_{\theta\in\Theta} \risk(\theta).
\end{align*}
where $\essinf$ denotes the essential infimum.
\end{proof}
\begin{rmq}
No randomization is needed for minimizing the standard risk. Denoting $\mathcal{V}$ this common value, we also have the following inequalities for any $\varepsilon>0$, $\mathcal{V}\leq \valuerand_\varepsilon\leq \valuedet_\varepsilon$.
\end{rmq}



\subsection{Distributional Formulation of the Adversarial Risk} 

To account for the possible randomness of the adversary, we rewrite the adversarial attack problem as a convex optimization problem over distributions. Let us first introduce the set of adversarial distributions.
\begin{definition}[Set of adversarial distributions]
Let $\PP$ be a Borel probability distribution on $\XX\times\YY$ and $\varepsilon>0$. We define the set of adversarial distributions as
\begin{align*}
\mathcal{A}_{\varepsilon}&(\PP) := \left\{\QQ\in\mathcal{M}^+_1(\XX\times\YY)\mid\exists \gamma\in\mathcal{M}^+_1\left((\XX\times\YY)^2\right),\right.\\
&\left.d(x,x')\leq\varepsilon,~y=y'~~ \gamma\text{-a.s.},~\Pi_{1\sharp}\gamma=\PP,~\Pi_{2\sharp}\gamma=\QQ\right\} 
\end{align*}
where $\Pi_i$ denotes the projection on the $i$-th component, and $g_\sharp$ the push-forward measure by a measurable function $g$.
\end{definition}
An attacker that can move the initial distribution $\PP$ anywhere in $\mathcal{A}_\varepsilon(\PP)$ is not applying a point-wise deterministic perturbation as considered in the standard adversarial risk. %For every point $(x,y)$ in the support of $\PP$, the attacker is allowed to move $(x,y)$ according to a ``random mapping'' in the ball of radius $\varepsilon$, and not to a single other point $(x',y)$ like the usual attacker in adversarial attacks. 
In other words, for a point $(x,y)\sim\PP$, the attacker could choose a distribution $q(\cdot\mid(x,y))$ whose support is included in $\{(x',y')\mid d(x,x')\leq \epsilon,~y=y'\}$ from which he will sample the adversarial attack. In this sense, we say the attacker is allowed to be randomized.
% This set allows to move every single point in the support of $\PP$ and moving its mass everywhere as far as the perturbation is smaller than $\varepsilon$ and does not change the `true' label. In this problem, the attacker can then play randomized strategies to move the samples inside the balls of radius $\varepsilon$. \textcolor{red}{Pas hyper limpide je trouve}

\textbf{Link with DRO.} We immediately remark that $\mathcal{A}_\varepsilon(\PP)$ correspond in the Wasserstein-$\infty$ set associated with
the cost
\begin{align*}
    d'((x,y),(x',y'))\mapsto \left\{
        \begin{array}{ll}
            d(x,x') & \mbox{if } y = y'\\
            +\infty & \mbox{otherwise.}
        \end{array}
    \right.
\end{align*}
We also remark, such a set can be defined from usual (not $\infty$) Wasserstein uncertainty sets:  for an arbitrary $\varepsilon>0$, we define the cost $c_\varepsilon$ as follows
\begin{align*}
c_\varepsilon((x,y),(x',y')) := \left\{
    \begin{array}{ll}
        0 & \mbox{if } d(x,x')\leq\varepsilon\mbox{ and }y = y'\\
        +\infty & \mbox{otherwise.}
    \end{array}
\right.
\end{align*}
This cost is lower semi-continuous and penalizes to infinity perturbations that change the label or move the input by a distance greater than $\varepsilon$. As Proposition~\ref{prop:wass_ball} shows, the Wasserstein ball associated with $c_\varepsilon$ is equal to $\mathcal{A}_{\varepsilon}(\PP)$.
\begin{prop}
\label{prop:wass_ball}
Let $\PP$ be a Borel probability distribution on $\XX\times\YY$ and $\varepsilon>0$ and $\eta\geq 0$, then
    $\mathcal{B}_{c_\varepsilon}(\PP,\eta) =\mathcal{A}_{\varepsilon}(\PP)$.
Moreover, $\mathcal{A}_{\varepsilon}(\PP)$ is convex and compact for the weak topology of $\mathcal{M}^+_1(\XX\times\YY)$.
\end{prop}
\begin{proof}
Let $\eta>0$. Let $\QQ\in\mathcal{A}_\varepsilon(\PP)$. There exists $\gamma\in
\mathcal{M}^+_1\left((\XX\times\YY)^2\right)$ such that, $d(x,x')\leq\varepsilon$, $y=y'$ $\gamma$-almost surely, and $\Pi_{1\sharp}\gamma=\PP$, and $\Pi_{2\sharp}\gamma=\QQ$. Then $\int c_\varepsilon d \gamma = 0\leq \eta$. Then, we deduce that $W_{c_\varepsilon}(\PP,\QQ)\leq \eta$, and $\QQ\in\mathcal{B}_{c_\varepsilon}(\PP,\eta)$. Reciprocally, let $\QQ\in\mathcal{B}_{c_\varepsilon}(\PP,\eta)$. Then, since the infimum is attained in the Wasserstein definition, there exists $\gamma\in
\mathcal{M}^+_1\left((\XX\times\YY)^2\right)$ such that $\int c_\varepsilon d \gamma \leq \eta$. Since $c_\varepsilon((x,x'),(y,y'))=+\infty$ when $d(x,x')>\varepsilon$ and $y\neq y'$, we deduce that, $d(x,x')\leq\varepsilon$ and $y=y'$, $\gamma$-almost surely. Then $\QQ\in\mathcal{A}_\varepsilon(\PP)$. We have then shown that: $\mathcal{A}_\varepsilon(\PP)=\mathcal{B}_{c_\varepsilon}(\PP,\eta)$.

The convexity of $\mathcal{A}_\varepsilon(\PP)$ is then immediate from the relation with the Wasserstein uncertainty set.

Let us show first that $\mathcal{A}_\varepsilon(\PP)$ is relatively compact for weak topology. To do so we will show that $\mathcal{A}_\varepsilon(\PP)$ is tight and apply Prokhorov's theorem. Let $\delta>0$, $(\XX\times \YY,d\oplus d')$ being a Polish space, $\{\PP\}$ is tight then there exists $K_\delta$ compact such that $\PP(K_\delta)\geq1-\delta$.
Let $\Tilde{K}_\delta:=\left\{(x',y')\mid \exists (x,y)\in K_\delta,~ d(x',x)\leq\varepsilon,~y=y'\right\}$.  Recalling that $(\XX,d)$ is proper (i.e. the closed balls are compact), so $\Tilde{K}_\delta$ is compact. Moreover for $\QQ\in\mathcal{A}_\varepsilon(\PP)$, $\QQ(\Tilde{K}_\delta)\geq \PP(K_\delta)\geq 1-\delta$. And then, Prokhorov's theorem holds, and $\mathcal{A}_\varepsilon(\PP)$ is relatively compact for weak topology.

Let us now prove that $\mathcal{A}_\varepsilon(\PP)$ is closed to conclude.  Let $(\QQ_n)_n$ be a sequence of $\mathcal{A}_\varepsilon(\PP)$ converging towards some $\QQ$ for weak topology. For each $n$, there exists $\gamma_n\in \mathcal{M}^1_+(\XX\times\YY)$ such that $d(x,x')\leq\varepsilon$ and $y=y'$ $\gamma_n$-almost surely and $\Pi_{1\sharp}\gamma_n=\PP$, $\Pi_{2\sharp}\gamma_n=\QQ_n$. $\{\QQ_n,n\geq0\}$ is relatively compact, then tight, then $\bigcup_n \Gamma_{\PP,\QQ_n}$ is tight, then relatively compact by Prokhorov's theorem. $(\gamma_n)_n\in\bigcup_n \Gamma_{\PP,\QQ_n}$, then up to an extraction,  $\gamma_n\to\gamma$. Then $d(x,x')\leq\varepsilon$ and $y=y'$ $\gamma$-almost surely, and by continuity, $\Pi_{1\sharp}\gamma=\PP$ and by continuity, $\Pi_{2\sharp}\gamma=\QQ$. And hence $\mathcal{A}_\varepsilon(\PP)$ is closed.

Finally $\mathcal{A}_\varepsilon(\PP)$ is a convex compact set for the weak topology. 
\end{proof}




% Next proposition shows that the set $\mathcal{A}_{\varepsilon}(\PP)$ satisfies interesting topological properties. See proof in Appendix~\ref{prv:a_eps}.
% \begin{prop}
% \label{prop:a_eps}
% Let $\PP$ be a Borel probability distribution on $\XX\times\YY$ and $\varepsilon>0$. $\mathcal{A}_{\varepsilon}(\PP)$ is convex and compact for the weak topology of $\mathcal{M}^+_1(\XX\times\YY)$.
% \end{prop}
Thanks to this result, we can reformulate the adversarial risk as the value of a convex problem over $\mathcal{A}_\varepsilon(\PP)$. %See proof in Appendix~\ref{prv:duality-rand}.
\begin{prop} 
\label{prop:dro_adv}
Let $\PP$ be a Borel probability distribution on $\XX\times\YY$ and $\mu$ a Borel probability distribution on $\Theta$. Let $\loss:\Theta\times(\XX\times\YY)\to [0,\infty)$ satisfying Assumption~\ref{ass:loss}. Let $\varepsilon>0$. Then:
\begin{align}
\label{eq:dro-adv}
\risk_\varepsilon(\mu)= \sup_{\QQ\in \mathcal{A}_{\varepsilon}(\PP)}\mathbb{E}_{(x',y')\sim\QQ,\theta\sim\mu}\left[\loss(\theta,(x',y'))\right].
\end{align}
The supremum is attained. Moreover $\QQ^*\in \mathcal{A}_{\varepsilon}(\PP)$ is an optimum of Problem~\eqref{eq:dro-adv} if and only if there exists $\gamma^*\in\mathcal{M}^+_1\left((\XX\times\YY)^2\right)$ such that: $\Pi_{1\sharp}\gamma^*=\PP$, $\Pi_{2\sharp}\gamma^*=\QQ^*$, $d(x,x')\leq\varepsilon$, $y=y'$ and  $\loss(x',y')=\sup_{\substack{u\in\XX,d(x,u)\leq\varepsilon}}\loss(u,y)$ $\gamma^*$-almost surely.
\end{prop}

\begin{proof}
Let $\mu\in\mathcal{M}^1_+(\Theta)$. Let $\Tilde{f}:((x,y),(x',y'))\mapsto \mathbb{E}_{\theta\sim\mu}\left[\loss(\theta,(x,y))\right]-c_\varepsilon((x,y),(x',y'))$. $\Tilde{f}$ is upper-semi continuous, hence upper semi-analytic. Then, by upper semi continuity of $\mathbb{E}_{\theta\sim\mu}\left[\loss(\theta,\cdot)\right]$ on the compact $\{(x',y')\mid~d(x,x')\leq\varepsilon,y=y'\}$ and~\citep[Proposition 7.50]{bertsekas2004stochastic}, there exists a universally measurable mapping $T$ such that $\mathbb{E}_{\theta\sim\mu}\left[\loss(\theta,T(x,y))\right]=\sup_{(x',y'),~d(x,x')\leq\varepsilon,y=y'}\mathbb{E}_{\theta\sim\mu}\left[\loss(\theta,(x,y))\right]$.  Let $\QQ = T_{\sharp}\PP$, then $\QQ\in\mathcal{A}_\varepsilon(\PP)$. And then $$\mathbb{E}_{(x,y)\sim\PP}\left[\sup_{(x',y'),~d(x,x')\leq\varepsilon,y=y'}\mathbb{E}_{\theta\sim\mu}\left[\loss(\theta,(x',y'))\right]\right]\leq \sup_{\QQ\in\mathcal{A}_\varepsilon(\PP)}\mathbb{E}_{(x,y)\sim\QQ}\left[\mathbb{E}_{\theta\sim\mu}\left[\loss(\theta,(x,y))\right]\right]$$.

Reciprocally, let $\QQ\in\mathcal{A}_\varepsilon(\PP)$. There exists $\gamma\in\mathcal{M}^1_+((\XX\times\YY)^2)$, such that $d(x,x')\leq\varepsilon$ and $y=y'$ $\gamma$-almost surely, and, $\Pi_{1\sharp}\gamma=\PP$ and  $\Pi_{2\sharp}\gamma=\QQ$. Then:
$\mathbb{E}_{\theta\sim\mu}\left[\loss(\theta,(x',y'))\right]\leq\sup_{(u,v),~d(x,u)\leq\varepsilon,y=v}\mathbb{E}_{\theta\sim\mu}\left[\loss(\theta,(u,v))\right]$ $\gamma$-almost surely. Then, we deduce that:
\begin{align*}
    \mathbb{E}_{(x',y')\sim\QQ}\left[\mathbb{E}_{\theta\sim\mu}\left[\loss(\theta,(x',y'))\right]\right]& =     \mathbb{E}_{(x,y,x',y')\sim\gamma}\left[\mathbb{E}_{\theta\sim\mu}\left[\loss(\theta,(x',y'))\right]\right] \\
    &\leq\mathbb{E}_{(x,y,x',y')\sim\gamma}\left[\sup_{(u,v),~d(x,u)\leq\varepsilon,y=v}\mathbb{E}_{\theta\sim\mu}\left[\loss(\theta,(u,v))\right]\right]\\
    &\leq\mathbb{E}_{(x,y)\sim\PP}\left[\sup_{(u,v),~d(x,u)\leq\varepsilon,y=v}\mathbb{E}_{\theta\sim\mu}\left[\loss(\theta,(u,v))\right]\right]
\end{align*}

Then we deduce the expected result:
\begin{align*}
\risk_\varepsilon(\mu)= \sup_{\QQ\in\mathcal{A}_\varepsilon(\PP)}\mathbb{E}_{(x,y)\sim\QQ}\left[\mathbb{E}_{\theta\sim\mu}\left[\loss(\theta,(x,y))\right]\right]
\end{align*}
Let us show that the optimum is attained. $\QQ\mapsto\mathbb{E}_{(x,y)\sim\QQ}\left[\mathbb{E}_{\theta\sim\mu}\left[\loss(\theta,(x,y))\right]\right]$ is upper semi continuous by Lemma~\ref{lem:usc2} for the weak topology of measures, and $\mathcal{A}_\varepsilon(\PP)$ is compact by Proposition~\ref{prop:wass_ball}, then by~\citep[Proposition 7.32]{bertsekas2004stochastic}, the supremum is attained for a certain $\QQ^*\in\mathcal{A}_\varepsilon(\PP)$. 

\end{proof}

The adversarial attack problem is a DRO problem for the cost $c_\varepsilon$.
Proposition~\ref{prop:dro_adv} means that, against a fixed classifier $\mu$, the randomized attacker that can move the distribution in $\mathcal{A}_\varepsilon(\PP)$ has exactly the same power as an attacker that moves every single point $x$ in the ball of radius $\varepsilon$.  By Proposition~\ref{prop:dro_adv}, we also  deduce that the adversarial risk can be casted as a linear optimization problem over distributions.

\begin{rmq}
  In a recent work,~\citep{pydi2019adversarial} proposed a similar adversary using Markov kernels but left as an open question the link with the classical adversarial risk, due to measurability issues. Proposition~\ref{prop:dro_adv} solves these issues. The result is similar to~\citep{blanchet2019quantifying}. Although we believe its proof might be extended for infinite valued costs,~\citep{blanchet2019quantifying} did not treat that case. We provide an alternative proof in this special case. 
\end{rmq}


 %In the next section we show how Proposition~\ref{prop:dro_adv} helps casting the adversarial risk minimization problem as a game between the classifier and the attacker for which we study the Nash equilibria.

% \paragraph{Link with distributionally robust optimization (DRO). } The optimization problem over $\mathcal{A}_\varepsilon(\PP)$ is very close to a DRO problem ~\citep{blanchet2019quantifying}. 
% When $(\mathcal{Z},d)$ a Polish space and $c:\mathcal{Z}^2\rightarrow\mathbb{R}^+\cup\{+\infty\}$ be a lower semi-continuous function, for $\PP,\QQ\in\mathcal{M}^+_1(\mathcal{Z})$ , the primal Optimal Transport problem is defined as:
% \begin{align*}
%   W_c(\PP,\QQ):=\inf_{\gamma\in\Gamma_{\PP,\QQ}}  \int_{\mathcal{Z}^2} c(z,z')d\gamma(z,z')
% \end{align*}
% with $\Gamma_{\PP,\QQ}:=\left\{\gamma\in\mathcal{M}^+_1(\mathcal{Z}^2)\mid~\Pi_{1\sharp}\gamma = \PP,~\Pi_{2\sharp}\gamma = \QQ \right\}$. When $\delta>0$ and for $\PP\in\mathcal{M}^+_1(\mathcal{Z})$, the associated Wasserstein uncertainty set is defined as: 
% \begin{align*}
%     \mathcal{B}_{c}(\PP,\delta) := \left\{\QQ\in \mathcal{M}^+_1(\mathcal{Z})\mid W_c(\PP,\QQ)\leq \delta\right\}
% \end{align*}
% A DRO problem is a linear optimization problem over Wasserstein uncertainty sets. We can show that $\mathcal{A}_{\varepsilon}(\PP)$ is actually a Wasserstein uncertainty set for a well-suited cost on $(\XX\times\YY)^2$. For $\varepsilon$, we define $c_\varepsilon$ the cost defined as $c_\varepsilon((x,y),(x',y')) = d(x,x')$ if $d(x,x')\leq\varepsilon$ and $y = y'$, and $+\infty$ otherwise. The cost for perturbations that changing the label and the input by a distance greater than $\varepsilon$ would be infinite. Next proposition link the Wasserstein ball associated with $c_\varepsilon$ and $\mathcal{A}_{\varepsilon}(\PP)$.
% \begin{prop}
% Let $\PP$ be a Borel probability distribution on $\XX\times\YY$ and $\varepsilon>0$ and $\delta>\varepsilon$, then:
% \begin{align*}
%     \mathcal{B}_{c_\varepsilon}(\PP,\delta) =\mathcal{A}_{\varepsilon}(\PP)
% \end{align*}
% \end{prop}
% One may have thought to apply directy Theorem 1 from~\citep{blanchet2019quantifying} to prove Proposition~\ref{prop:dro_adv}; however, the result holds for real-valued costs (i.e. $<\infty$). We believe it can be extended to infinite valued costs, but it is out of the scope of this paper. 
% \begin{prop} 
% \label{prop:dro_adv}
% Let $\PP$ be a Borel probability distribution on $\XX\times\YY$ and $\mu$ a Borel probability distribution on $\Theta$. Let $l:\Theta\times(\XX\times\YY)\to [0,+\infty]$ satisfying Assumption~\ref{ass:loss}. Let $\varepsilon>0$. Then:
% \begin{align*}
% \risk^\epsilon(\mu)= \sup_{\QQ\in \mathcal{A}_{\varepsilon}(\PP)}\mathbb{E}_{(x',y')\sim\QQ,\theta\sim\mu}\left[\loss(\theta,(x',y'))\right]
% \end{align*}
% \begin{align*}
% \mathbb{E}_{(x,y)\sim \PP}\left[\sup_{\substack{u\in\XX,\\d(x,u)\leq\varepsilon}}l(u,y)\right] = \sup_{\QQ\in \mathcal{A}_{\varepsilon}(\PP)}\mathbb{E}_{(x',y')\sim\QQ}\left[\loss(x',y')\right]
% \end{align*}
% The supremum is attained. Moreover $\QQ^*\in \mathcal{A}_{\varepsilon}(\PP)$ is an optimum if and only if there exists $\gamma^*\in\mathcal{M}^+_1\left((\XX\times\YY)^2\right)$ such that: $\Pi_{1\sharp}\gamma^*=\PP$, $\Pi_{2\sharp}\gamma^*=\QQ^*$, $d(x,x')\leq\varepsilon$, $y=y'$ and  $l(x',y')=\sup_{\substack{u\in\XX,d(x,u)\leq\varepsilon}}l(u,y)$ $\gamma^*$-almost surely.
% \end{prop}


% say that the program is linear blabla + link with latest pydi
% One can also recast the adversarial problem as an infinite linear problem:
% \begin{align*}
%   \risk^\epsilon(\mu) = \sup_{\gamma\in \Tilde{\mathcal{A}}_{\varepsilon}(\PP)}\mathbb{E}_{(x,y,x',y')\sim\gamma}\left[\loss(x',y')\right]
% \end{align*}

% where 
% \begin{align*}
% \Tilde{\mathcal{A}}_{\varepsilon}&(\PP) := \left\{\gamma\in\mathcal{M}^+_1\left((\XX\times\YY)^2\right)\mid~\Pi_{1\sharp}\gamma=\PP\right.\\
% &\left.\int \delta\{d(x,x')\leq\varepsilon,~y=y'\}d\gamma(x,y,x'y')=0\right\}
% \end{align*}
% with $\delta$ is the indicator function. Remark that $\mathcal{A}_{\varepsilon}(\PP)=\Pi_{2\sharp}\Tilde{\mathcal{A}}_{\varepsilon}(\PP)$

% \paragraph{A primal-dual formulation of risk minimization.}Let $\mu\in\mathcal{M}^1_+(\Theta)$ a possibly randomized classifier. Thanks to Proposition~\ref{prop:dro_adv}, we have that for $\varepsilon>0$:
% \begin{align*}
%     \risk^\varepsilon(\mu)=\sup_{\QQ\in\mathcal{A}_{\varepsilon}(\PP)}\mathbb{E}_{(x,y)\sim\QQ,\theta\sim\mu}\left[\loss(\theta,(x,y))\right]
% \end{align*}
% Following the definition of the risk minimization objective,  one can define the dual of this problem for randomized classifiers:
% \begin{align}
%     \sup_{\QQ\in\mathcal{A}_{\varepsilon}(\PP)}\inf_{\mu\in\mathcal{M}^1_+(\Theta)}\mathbb{E}_{(x,y)\sim\QQ,\theta\sim\mu}\left[\loss(\theta,(x,y))\right]
% \label{eq:dual_pb}
% \end{align}
% and its analog for deterministic classifiers:
% \begin{align*}
%     \sup_{\QQ\in\mathcal{A}_{\varepsilon}(\PP)}\inf_{\theta\in\Theta}\mathbb{E}_{(x,y)\sim\QQ}\left[\loss(\theta,(x,y))\right]
% \end{align*}
% Then, one can notice that the dual problems for deterministic and randomized classifiers are equivalent\footnote{See Appendix XXX for more details}. We denote $\dualvalue^\varepsilon$ the value of the dual problem.
% Weak duality is always satisfied:
% \begin{align}
% \label{eq:weak_duality}
% \dualvalue^\varepsilon\leq \valuerand^\varepsilon\leq \valuedet^\varepsilon
% \end{align}
% We will show in the next section, that, under Assumption~\ref{ass:loss}, strong duality always holds for this min-max problem.


% \begin{definition}[Optimal transport problem] Let $(\mathcal{Z},d)$ be a Polish space. Let $c:\mathcal{Z}^2\rightarrow\mathbb{R}^+\cup\{+\infty\}$ be a lower semi-continuous function. For $\PP,\QQ\in\mathcal{M}^+_1(\mathcal{Z})$, the primal optimal transport problem is defined as:
% \begin{align*}
%   W_c(\PP,\QQ):=\inf_{\gamma\in\Gamma_{\PP,\QQ}}  \int_{\mathcal{Z}^2} c(z,z')d\gamma(z,z')
% \end{align*}
% with $\Gamma_{\PP,\QQ}:=\left\{\gamma\in\mathcal{M}^+_1(\mathcal{Z}^2)\mid~\Pi_{1\sharp}\gamma = \PP,~\Pi_{2\sharp}\gamma = \QQ \right\}$ 
% \end{definition}

% We now define the Wasserstein uncertainty set.
% \begin{definition}[Wasserstein uncertainty set]Let $(\mathcal{Z},d)$ be a Polish space. Let $c:\mathcal{Z}^2\rightarrow\mathbb{R}^+\cup\{+\infty\}$ be a lower semi-continuous function. For $\PP\in\mathcal{M}^+_1(\mathcal{Z})$ and $\delta>0$, we define the Wasserstein uncertainty set:
% \begin{align*}
%     \mathcal{B}_{c}(\PP,\delta) := \left\{\QQ\in \mathcal{M}^+_1(\mathcal{Z})\mid W_c(\PP,\QQ)\leq \delta\right\}
% \end{align*}

% \end{definition}
% We recall that since for all $\PP$, $\QQ\mapsto W_c(\PP,\QQ)$ is convex, the Wasserstein uncertainty sets are convex. Moreover, $\QQ\mapsto W_c(\PP,\QQ)$ is a lower semi-continuous function for the weak topology then $\mathcal{B}_{c}(\PP,\delta)$ is a closed set. Under some mild assumptions the Wasserstein uncertainty balls are compact for weak topology. See details in App. 

% We now connect this problem to that of designing adversarial attacks for a given classification task. Let $(\XX,d_{\XX})$ be a proper Polish metric space representing the input space. Let $\YY=\{1,\dots,K\}$ be the labels set, endowed with trivial metric  $d_{\YY}(y,y') = \mathrm{1}_{y\neq y'}$. Then the space  $(\XX\times\YY,d_\XX\oplus d_\YY)$ is a proper Polish space. We denote $c_\varepsilon$ the cost defined as $c_\varepsilon((x,y),(x',y')) = d(x,x')$ if $d(x,x')\leq\varepsilon$ and $y = y'$, and $+\infty$. The cost for perturbations that changing the label and the input by a distance greater than $\varepsilon$ would be infinite. For $\delta>\epsilon$, next proposition shows that $\mathcal{B}_{c_{\varepsilon}}(\PP,\delta)$ allows all perturbations smaller than $\epsilon$.

% \begin{prop}
% Let $\PP$ be a distribution on $\XX\times\YY$ and $\varepsilon>0$ and $\delta>\varepsilon$, then:
% \begin{align*}
% \mathcal{B}_{c_{\varepsilon}}&(\PP,\delta) = \left\{\QQ\in\mathcal{M}^+_1(\XX\times\YY)\mid\exists \gamma\in\mathcal{M}^+_1\left((\XX\times\YY)^2\right),\right.\\
% &\left.d(x,x')\leq\varepsilon,~y=y'~~ \gamma\text{-a.s.},~\Pi_{1\sharp}\gamma=\PP,~\Pi_{2\sharp}\gamma=\QQ\right\} 
% \end{align*}
% This set will be denoted $\mathcal{A}_\varepsilon(\PP)$ this set. The set $\mathcal{A}_\varepsilon(\PP)$ is a convex and compact for weak topology.
% \end{prop}
 
% \begin{prop} 
% \label{prop:dro_adv}
% Let $\PP$ be a distribution on $\XX\times\YY$. Let $l:\XX\times\YY\mapsto\mathbb{R}$ be an upper semi continuous measurable function. We suppose that $l\in L^1(\PP)$. Let $\varepsilon>0$. Then:
% \begin{align*}
% \mathbb{E}_{(x,y)\sim \PP}\left[\sup_{\substack{u\in\XX,\\d(x,u)\leq\varepsilon}}l(u,y)\right] = \sup_{\QQ\in \mathcal{A}_{\varepsilon}(\PP)}\mathbb{E}_{(x',y')\sim\QQ}\left[\loss(x',y')\right]
% \end{align*}
% The supremum is attained. Moreover $\QQ^*\in \mathcal{A}_{\varepsilon}(\PP)$ is an optimum if and only if there exists $\gamma^*\in\mathcal{M}^+_1\left((\XX\times\YY)^2\right)$ such that: $\Pi_{1\sharp}\gamma^*=\PP$, $\Pi_{2\sharp}\gamma^*=\QQ^*$, $d(x,x')\leq\varepsilon$, $y=y'$ and  $l(x',y')=\sup_{\substack{u\in\XX,d(x,u)\leq\varepsilon}}l(u,y)$ $\gamma^*$-almost surely.
% \end{prop}

% \laurent{can we prove this result without using blanchet, i think so no? It is the result of blanchet}

\section{Nash Equilibria in the Adversarial Game}
\label{sec:nash-eq}

\subsection{Adversarial Attacks as a Zero-Sum Game}

Thanks to Proposition~\ref{sec:adv-problem}, the adversarial risk minimization problem can  be seen as a two-player zero-sum game that writes as follows,
\begin{align}
    \inf_{\mu\in\mathcal{M}^1_+(\Theta)} \sup_{\QQ\in\mathcal{A}_{\varepsilon}(\PP)}\mathbb{E}_{(x,y)\sim\QQ,\theta\sim\mu}\left[\loss(\theta,(x,y))\right].
\label{eq:primal_pb}
\end{align}
In this game the classifier objective is to find the best distribution $\mu \in \mathcal{M}_1^+(\Theta)$ while the adversary is manipulating the data distribution. For the classifier, solving the infimum problem in Equation~\eqref{eq:primal_pb} simply amounts to solving the adversarial risk minimization problem -- Problem~\eqref{eq:advriskmin}, whether the classifier is randomized or not. Then, given a randomized classifier $\mu \in \mathcal{M}_1^+(\Theta)$, the goal of the attacker is to find a new data-set distribution $\mathbb{Q}$ in the set of adversarial distributions $\mathcal{A}_{\varepsilon}(\PP)$ that maximizes the risk of $\mu$. More formally, the adversary looks for $$\mathbb{Q} \in \argmaxB_{\QQ\in\mathcal{A}_{\varepsilon}(\PP)} \mathbb{E}_{(x,y)\sim\QQ,\theta\sim\mu}\left[\loss(\theta,(x,y))\right]. $$
In the game theoretic terminology, $\mathbb{Q}$ is also called the best response of the attacker to the classifier $\mu$.

\begin{rmq}
Note that for a given classifier $\mu$ there always exists a ``deterministic'' best response, i.e. every single point $(x,y)$ is mapped to another single point $T(x,y)$. Let $T:\XX\times\YY\to\XX\times\YY$ be defined such that for all $(x,y)\in\XX\times\YY$, $\EE_{\theta\sim\mu}\left[\loss(T(x,y))\right] = \sup_{x',~d(x,x')\leq\varepsilon}\EE_{\theta\sim\mu}\left[ L(x',y)\right]$. Thanks to~\citep[Proposition 7.50]{bertsekas2004stochastic}, $T$ is $\PP$-measurable. Moreover, we get that $\QQ = (T,id)_\sharp \PP$ belongs to the best response to $\mu$. Therefore, $T$ is the optimal ``deterministic'' attack against the classifier $\mu$.
\end{rmq}


%The classifier cannot play a best response against the attacker since the attacker is usually unknown. Thanks to the inequality~\eqref{eq:weak_duality}, the lowest risk, the classifier can hope is $\mathcal{D}^\varepsilon$. But as seen in the motivating example, it might not been attained by a deterministic classifier. We show in this section, that randomized classifiers can approach $\mathcal{D}^\varepsilon$ arbitrary closely.

%\begin{itemize}
%    \item 3.1 0 sum game , mininmizer = jeu
%    \item in this game: classifier oblivious to the attacker -> standard setting, attacker is whitebox
%    \item 3.2 dual formulation: sup inf, attackant adaptif, and Black box attacks
%    \item 3.3 evaluating duality gap
%\end{itemize}

\subsection{Dual Formulation of the Game}

Every zero sum game has a dual formulation that allows a deeper understanding of the framework. Here, from Proposition~\ref{prop:dro_adv}, we can define the dual problem of adversarial risk minimization for randomized classifiers. This dual problem also characterizes a two-player zero-sum game that writes as follows,
\begin{align}
    \sup_{\QQ\in\mathcal{A}_{\varepsilon}(\PP)}\inf_{\mu\in\mathcal{M}^1_+(\Theta)}\mathbb{E}_{(x,y)\sim\QQ,\theta\sim\mu}\left[\loss(\theta,(x,y))\right].
\label{eq:dual_pb}
\end{align}
In this dual game problem, the adversary plays first and seeks an adversarial distribution that has the highest possible risk when faced with an arbitrary classifier. This means that it has to select an adversarial perturbation for every input $x$, without seeing the classifier first. In this case, as pointed out by the motivating example in Section~\ref{sec:motiv-ex}, the attack can (and should) be randomized to ensure maximal harm against several classifiers. Then, given an adversarial distribution, the classifier objective is to find the best possible classifier on this distribution. Let us denote $\dualvalue^\varepsilon$ the value of the dual problem. Since the weak duality is always satisfied, we get
\begin{align}
\label{eq:weak_duality}
\dualvalue_\varepsilon\leq \valuerand_\varepsilon\leq \valuedet_\varepsilon.
\end{align}
Inequalities in Equation~\eqref{eq:weak_duality} mean that the lowest risk the classifier can get (regardless of the game we look at) is $\mathcal{D}^\varepsilon$. In particular, this means that the primal version of the game, \emph{i.e.} the adversarial risk minimization problem, will always have a value greater or equal to $\mathcal{D}^\varepsilon$. As we discussed in Section~\ref{sec:motiv-ex}, this lower bound may not be attained by a deterministic classifier. As we will demonstrate in the next section, optimizing over randomized classifiers allows to approach $\mathcal{D}^\varepsilon$ arbitrary closely.

Note that, we can always define the dual problem when the classifier is deterministic, \begin{align*}
    \sup_{\QQ\in\mathcal{A}_{\varepsilon}(\PP)}\inf_{\theta\in\Theta}\mathbb{E}_{(x,y)\sim\QQ}\left[\loss(\theta,(x,y))\right].
\end{align*}


We can deduce an immediate corollary from Proposition~\ref{prop:eqstandardrisk}
that the dual problems for deterministic and randomized classifiers have the same value.
\begin{corollary}
Under Assumption~\ref{ass:loss}, the dual for randomized and deterministic classifiers are equal.
\end{corollary}


%This primal-dual formulation highlights the profound link between  game theory and adversarial examples. Indeed, the adversarial risk minimization can be seen as a zero-sum game. In this section we study the Nash equilibrium of this game in the randomized setting. \textcolor{red}{pas tres clair, en quoi c'est la formulation dual qui fait apparaitre le jeu?} %We show in the next section, that, under Assumption~\ref{ass:loss}, strong duality always holds for this min-max problem, and hence study the Nash equilibrium of the related game.
%We now define the adversarial examples game, as a zero-sum game, where one player is the attacker and the other is the classifier. We study the Nash equilibria of this game in both randomized and deterministic cases.




%\textbf{Attacker Objective.}
%Given a possibly  randomized classifier $\mu\in \mathcal{M}^1_+(\Theta)$, the goal of the attacker is to find a possibly random perturbation for each single pair $(x,y)\in \supp(\PP)$, such that this perturbation maximizes $\mathbb{E}_{\theta\sim\mu}[\loss(\theta,(x',y))]$ over $x'$ under the constraint that $d(x,x')\leq\varepsilon$. In other words, thanks to Proposition $1$, the set of best responses to $\mu$ writes
%\begin{align*}
%    \text{BR}(\mu) := \argmaxB_{\QQ\in\mathcal{A}_{\varepsilon}(\PP)} \mathbb{E}_{(x,y)\sim\QQ,\theta\sim\mu}\left[ l(\theta,(x,y))\right] 
%\end{align*}




%In a white-box setting, the adversary knows the classifier $\mu$ before attacking then he only needs to perform a best response. In the black-box setting, it is different: since the adversary does not know the classifier $\mu$, he needs to compute a solution to the sup-inf problem to get its attack the most effective. Note that in this case, the attack can be possibly random: a single point $x$ can be mapped to a distribution of attacks.



%\textbf{Classifier Objective.} In the standard adversarial examples setting, since the attacker plays after the classifier, the goal of the classifier is to find the best classifier against every possible attacks at level $\varepsilon$, i.e. solving the adversarial risk minimization problem~\eqref{eq:advriskmin}, whether the classifier is randomized or not. The classifier cannot play a best response against the attacker since the attacker is usually unknown. Thanks to the inequality~\eqref{eq:weak_duality}, the lowest risk, the classifier can hope is $\mathcal{D}^\varepsilon$. But as seen in the motivating example, it might not been attained by a deterministic classifier. We show in this section, that randomized classifiers can approach $\mathcal{D}^\varepsilon$ arbitrary closely.

\subsection{Nash Equilibria for Randomized Strategies}

In the adversarial examples game, a Nash equilibrium is a couple $(\mu^*,\QQ^*)\in\mathcal{M}^1_+(\Theta)\times\mathcal{A}_\varepsilon(\PP)$ where both the classifier and the attacker have no incentive to deviate unilaterally from their strategies $\mu^*$ and $\QQ^*$. More formally, $(\mu^*,\QQ^*)$ is a Nash equilibrium of the adversarial examples game if $(\mu^*,\QQ^*)$ is a saddle point of the objective function $$(\mu,\QQ)\mapsto \mathbb{E}_{(x,y)\sim\QQ,\theta\sim\mu}\left[\loss(\theta,(x,y))\right].$$ Alternatively, we can say that $(\mu^*,\QQ^*)$ is a Nash equilibrium if and only if $\mu^*$ solves the adversarial risk minimization problem -- Problem~\eqref{eq:advriskmin}, $\QQ^*$ the dual problem -- Problem~\eqref{eq:duality}, and $\mathcal{D}^\varepsilon=\mathcal{V}_{rand}^\varepsilon$. In our problem, $\QQ^*$ always exists but it might not be the case for $\mu^*$. Then for any $\delta>0$, we say that  $(\mu_\delta,\QQ^*)$ is a $\delta$-approximate Nash equilibrium if $\QQ^*$ solves the dual problem and $\mu_\delta$ satisfies $\mathcal{D}^\varepsilon\geq\risk_\varepsilon(\mu_\delta)-\delta$. 
% \begin{align*}
%     \text{BR}(\QQ) := \argmaxB_{\mu\in\mathcal{A}_{\varepsilon}(\PP)} \int l(\theta,(x,y))d\QQ(x,y)d\mu(\theta)
% \end{align*}

% \begin{align*}
%     \inf_{\theta\in\Theta} \sup_{\QQ\in \mathcal{A}_{\varepsilon}(\PP))} \mathbb{E}_{z\sim\QQ }\left[\loss(\theta,z)\right]:=\int l(\theta,z)d\QQ(z)
% \end{align*}
% where: 
% \begin{align*}
%     \mathcal{A}_{\varepsilon}(\PP)) := \left\{\QQ\in \mathcal{M}^+_1(\mathcal{Z})\text{ s.t. } D(\PP,\QQ)\leq \varepsilon\right\}
% \end{align*}

% We will study:

% \begin{align*}
%     \inf_{\mu\in \mathcal{M}^+_1(\Theta)} \sup_{\QQ\in \mathcal{A}_{\varepsilon}(\PP))} \mathbb{E}_{\theta \sim \mu, z\sim\QQ }\left[\loss(\theta,z)\right] \\
%     =\int l(\theta,z)d\mu(\theta)d\QQ(z):=L(\mu,\QQ)
% \end{align*}

%Meyer{generalise to any sub convex set of the set of distributions of $\Theta$}

We now state our main result: the existence of approximate Nash equilibria in the adversarial examples game when both the classifier and the adversary can use randomized strategies. More precisely, we demonstrate that the duality gap between the adversary and the classifier problems is zero, which gives as a corollary the existence of Nash equilibria. 

\begin{thm}
\label{thm:duality-rand}
Let $\PP\in\mathcal{M}^1_+(\XX\times\YY)$. Let $\varepsilon>0$. Let $\loss:\Theta\times(\XX\times\YY)\to [0,\infty)$ satisfying Assumption~\ref{ass:loss}. %Let assume that for all $\mu\in\mathcal{M}^1_+(\Theta)$ and $(x,y)\in\XX\times\YY$, $l(\cdot,(x,y))$ is $\mu$-measurable.
Then strong duality always holds in the randomized  setting:
\begin{align}
 \label{eq:duality}\inf_{\mu\in \mathcal{M}^+_1(\Theta)} \max_{\QQ\in \mathcal{A}_{\varepsilon}(\PP)} \mathbb{E}_{\theta \sim \mu, (x,y)\sim\QQ }\left[\loss(\theta,(x,y))\right]\\
=
\nonumber\max_{\QQ\in \mathcal{A}_{\varepsilon}(\PP)}\inf_{\mu\in \mathcal{M}^+_1(\Theta)}  \mathbb{E}_{\theta \sim \mu, (x,y)\sim\QQ }\left[\loss(\theta,(x,y))\right]
\end{align}
The supremum is always attained. If $\Theta$ is a compact set, and for all $(x,y)\in\XX\times\YY$, $\loss(\cdot,(x,y))$ is lower semi-continuous, the infimum is also attained.
\end{thm}


\begin{proof}
$\mathcal{A}_\varepsilon(\PP)$, endowed with the weak topology of measures, is a Hausdorff compact convex space, thanks to Proposition~\ref{prop:wass_ball}. Moreover, $\mathcal{M}^1_+(\Theta)$ is clearly convex and $(\QQ,\mu)\mapsto \int ld\mu d\QQ$ is bilinear, hence concave-convex. Moreover thanks to Lemma~\ref{lem:usc2}, for all $\mu$, $\QQ\mapsto \int ld\mu d\QQ$ is upper semi-continuous. Then Fan's theorem applies and strong duality holds.
\end{proof}
 \begin{corollary}
\label{cor:nash-eq}
Under Assumption~\ref{ass:loss}, for any $\delta>0$, there exists a $\delta$-approximate Nash-Equibilrium $(\mu_\delta,\QQ^*)$. Moreover, if the infimum is attained, there exists a Nash equilibrium $(\mu^*,\QQ^*)$ to the adversarial examples game.
\end{corollary}



\cite{bose2021adversarial} mentioned a particular form of Theorem~\ref{thm:duality-rand} for convex cases.  It is still a direct corollary of Fan's theorem. This theorem can be stated as follows: 
\begin{thm}Let $\PP\in\mathcal{M}^1_+(\XX\times\YY)$, $\varepsilon>0$ and $\Theta$ a convex set. Let $\loss$ be a loss satisfying Assumption~\ref{ass:loss}, and also, $(x,y)\in\XX\times\YY$, $\loss(\cdot,(x,y))$ is a convex function, then we have the following:
\begin{align*}
\inf_{\theta\in\Theta} \sup_{\QQ\in \mathcal{A}_{\varepsilon}(\PP)} \mathbb{E}_{ \QQ }\left[\loss(\theta,(x,y))\right]
=
\sup_{\QQ\in \mathcal{A}_{\varepsilon}(\PP)}\inf_{\theta\in \Theta}  \mathbb{E}_{\QQ }\left[\loss(\theta,(x,y))\right]
\end{align*}
The supremum is always attained. If $\Theta$ is a compact set then, the infimum is also attained.
\end{thm}


Theorem~\ref{thm:duality-rand} shows that $\mathcal{D}^\varepsilon=\mathcal{V}_{rand}^\varepsilon$. From a game theoretic perspective, this means that the minimal adversarial risk for a randomized classifier against any attack (primal problem) is the same as the maximal risk an adversary can get by using an attack strategy that is oblivious to the classifier it faces (dual problem). This suggests that playing randomized strategies for the classifier could substantially improve robustness to adversarial examples.
%Second, Corollary~\ref{cor:nash-eq} says that there always exists a randomized classifier that gets arbitrarily close this minimal adversarial risk. Hence, if we can design an algorithm that efficiently learn this classifier, we will get improve adversarial robustness over classical deterministic defenses. 
In the next section, we will design an algorithm that efficiently learn a randomized classifier and show improved adversarial robustness over classical deterministic defenses.


\begin{rmq}
Theorem~\ref{thm:duality-rand} remains true if one replaces $\mathcal{A}_\varepsilon(\PP)$ with any other Wasserstein compact uncertainty sets (see~\citep{yue2020linear} for conditions of compactness). 
\end{rmq}

%Meyer{Add discrete case also}
% \paragraph{Case of $0/1$ loss.} The $0/1$ loss is particularly interesting in adversarial attacks since in this case the goal of the attacker would be to maximize the probability of misclassification. We can prove that our duality result also holds for the $0/1$ loss. 

% For $\theta\in\Theta$, we define $f_\theta:x\in\XX\mapsto (f_\theta(x)_1,\dots,f_\theta(x)_K)\in\mathbb{R}^K$ such that the predicted class is $y\in\YY$ if and only if: $f_\theta(x)_y>\max_{i\neq y} f_\theta(x)_i$. The corresponding $0/1$ loss is then: 
% \begin{align*}
% l(\theta,(x,y))=\mathbf{1}_{f_\theta(x)_y\leq\max_{i\neq y} f_\theta(x)_i}
% \end{align*}

% Assuming that for all $\theta$, $f_\theta(\cdot)$ is continuous, and that for all $\mu\in\mathcal{M}^1_+(\Theta)$ and $(x,y)\in\XX\times\YY$, $l(\cdot,(x,y))$ is $\mu$-measurable (for instance, for all $x\in\XX$, $f_{\cdot}(x)$ is continuous),  then the duality result~\eqref{eq:duality} holds for the $0/1$ loss. However the existence of a minimizing distribution $\mu^*\in\mathcal{M}^1_+(\Theta)$ is not guaranteed. See Appendix XXX for more details.
% \begin{prop}Let $\lambda>0$. For all $\theta\in \Theta$, $f_\theta:\XX\mapsto \mathbb{R}^k$. be classifier 
% \begin{align*}
% \sigma_\lambda(\theta,(x,y)) =\left(1+\exp\left\{-\lambda\left(f_\theta^y(x)-\max_{i\neq y}f_\theta^i(x)\right)\right\}\right)^{-1}
% \end{align*}
% \end{prop}






% \paragraph{Example where randomization is needed}Let now exhibit a simple example where randomization is needed. We assume that $\XX = \mathbb{R}$ and $\YY=\{-1,1\}$. Let assume the following distribution $\PP$: $Y = 1$ with probability $1/2$ and $-1$ with probability $1/2$. Conditionally to $Y=1$, $X$ always equals $0$ and conditionally to $Y=-1$, $X=1$ with probability $1/2$ and $-1$ otherwise.

% We assume that $d_\XX(x,x') =\lvert x-x'\rvert$. We now fix $\varepsilon=1$. Let consider the $0/1$ loss: $l(f,(x,y)) =\mathbf{1}_{f(x)\neq y}$.

% In this example, every Bayes optimal classifier have a zero natural risk, but the its adversarial risk is actually equal to $1$. Let now reduce the hypothesis class to two classifiers: $f_1(x) = \sign\left\{x\leq 1/2\right\}$ and $f_2(x) = \sign\left\{x\geq -1/2\right\}$. Then for $i=1,2$, $\risk(f_i)=1/4$ and $\risk(f_i)=3/4$. Let now define the randomized classifier $f=f_1$ with probability $1/2$ and $f_2$ with probability $1/2$. In this case we have $\risk(f)=1/4$ and $\risk(f)=1/2$.


% \section{todo}
% \begin{itemize}

%     \item unicity of Nash equilbrium?  no a priori
%     \item statistical study of the nash equilibrium
%     \item big part: algo...
% \end{itemize}



\section{Finding the Optimal Classifiers}
\label{sec:algo}

\subsection{An Entropic Regularization}
\label{sec:entropic-reg}


% From now on, we focus on finite class of classifiers. Let $\Theta = \{\theta_1,\dots,\theta_K\}$, we aim to learn the optimal mixture of classifiers in this case. The adversarial risk  is defined as:
% \begin{align*}
%     \risk(\lambda)=\mathbb{E}_{(x,y)\sim \PP}\left[\sup_{x'\in\mathcal{X},~d(x,x')\leq\varepsilon}\sum_{k=1}^K\lambda_kl(\theta_k,(x',y))\right]
% \end{align*}
% for $\lambda\in\Delta_K: = \{\lambda\in\mathbb{R}_+^K~\mathrm{s.t.}~\sum_{i=1}^K\lambda_i=1\}$, the probability simplex of $\mathbb{R}^K$. One can notice $\risk(\cdot)$ is a continuous convex function, hence $\min_{\lambda\in\Delta_K}\risk(\lambda)$ is attained for a certain $\lambda^\star$. Then, thanks to Corollary XXX, there always exists a Nash equilbrium to the advarsarial game when $\Theta$ is finite. In this section, we present two algorithms to learn the find the optimum of the adversarial risk minimization problem.
% \subsection{Algorithm}
% The first algorithm we present is inspired from~\citep{sinha2017certifying} and the convergence of projected sub-gradient methods CITE. The computation of the inner supremum problem is usually NP-hard~\footnote{See App XXX for details.}, but one may assume the existence of an approximate oracle to this supremum. The algorithm is presented in Algorithm~\ref{algo:duchi}. We get the following guarantees on this algorithm. See proof in Appendix XXX.
% \begin{prop}
% Let $\PP\in\mathcal{M}^1_+(\mathcal{X}\times\mathcal{Y})$. Let $l:\Theta\times(\mathcal{X}\times\mathcal{Y})\to [0,\infty)$ satisfying Assumption~\ref{ass:loss}. Then, Algorithm~\ref{algo:duchi} satisfies:  
% \begin{align*}
%     \min_{t\in[T]}\risk(\lambda_t)-\risk(\lambda^\star)\leq2\delta+ 2\sqrt{\frac{M}{T}}
%     % \delta\sqrt{K}+ \frac{M\sum_{t=1}^T \eta_t^2+\rVert\lambda_0-\lambda^\star\lVert^2}{\sum_{t=1}^T\eta_t}
% \end{align*}
% % In particular for $\eta_t =\frac{\eta}{t^{1/2}}$, we get: $\min_{t\in[T]}\risk(\lambda_t)-\risk(\lambda^\star)\in O\left(\delta+\frac{\log T}{\sqrt{T}}\right)$
% \end{prop}
% \begin{algorithm}[H]
% \SetAlgoLined
%  $\lambda_0 = \frac{\mathbf{1}_K}{K}; T;~\eta = \sqrt{\frac{4}{MT}}$\\
%  \For{$t=1,\dots,T$}{

%   $\Tilde{\QQ}$ s.t. $\exists\QQ^\star\in\mathcal{A}_\varepsilon(\PP)$,  $\mathbb{E}_{\QQ^\star,\lambda}(l(\theta_k,(x,y)))= \max_{\QQ\in \mathcal{A}_\varepsilon(\PP)}\mathbb{E}_{\QQ,\lambda}(l(\theta_k,(x,y)))$ and for all $k\in[K]$, $\lvert\mathbb{E}_{\Tilde{\QQ}}(l(\theta_k,(x,y)))-\mathbb{E}_{\QQ^\star}(l(\theta_k,(x,y))) \rvert\leq\delta$\\
  
%   $g_t=\left(\mathbb{E}_{\Tilde{\QQ}}(l(\theta_1,(x,y)),\dots,\mathbb{E}_{\Tilde{\QQ}}(l(\theta_K,(x,y))\right)^T$\\
%   $\lambda_t = \Pi_{\Delta_K}\left(\lambda_{t-1}-\eta g_t\right)$
%   }
%  \caption{Mixture optimization by PGD}
 
%  \label{algo:duchi}

% \end{algorithm}
% \subsection{Entropic regularization}
Let $\{(x_i,y_i)\}_{i=1}^\numsamples$ samples independently drawn from $\PP$ and denote  $\widehat{\mathbb
{P}}:=\frac{1}{\numsamples}\sum_{i=1}^\numsamples \delta_{(x_i,y_i)}$ the associated empirical distribution. %In fact for such distribution we have a simple characterization of the set  of adversarial distributions which is:
% \begin{align*}
%     \mathcal{A}_{\varepsilon}&(\widehat{\mathbb
% {P}})=\Big\{\QQ~|\exists \QQ_1,\dots,\QQ_N\in\mathcal{M}_{+}(\mathcal{X}\times\mathcal{Y}),\QQ=\sum_{i=1}^N \QQ_i,\\
% & \int_{\mathcal{X}\times\mathcal{Y}}d\QQ_i=\frac{1}{N},~\int c_{\varepsilon}((x_i,y_i),\cdot)d\QQ_i=0\Big\}.
% \end{align*}
% where 
% \begin{align*}
% \delta_{B((x_i,y_i),\varepsilon)}(x,y) = \left\{
%     \begin{array}{ll}
%         0 & \mbox{if } d(x_i,x)\leq\varepsilon\mbox{ and }y_i = y\\
%         +\infty & \mbox{otherwise.}
% %     \end{array}
% % \right.
% % \end{align*}
% Therefore by denoting
% \begin{align*}
%     \Gamma_{\varepsilon}&(\widehat{\mathbb
%  {P}})=\Big\{(\QQ_1,\dots,\QQ_N)\mid\lvert \QQ_i\rvert =\frac{1}{N},~\int c_{\varepsilon}((x_i,y_i),\cdot)d\QQ_i=0\Big\}
% \end{align*}
% the problem of interest can be written as
% \begin{align*}
% \inf_{\mu\in \mathcal{M}^+_1(\Theta)}\sup_{(Q_1,\dots,Q_N)\in\Gamma_{\varepsilon}(\widehat{\mathbb
% {P}})}\sum_{i=1}^N\mathbb{E}_{(x,y)\sim Q_i,\theta \sim \mu}\left[\loss(\theta,(x,y))\right]
% \end{align*}
One can show the adversarial empirical risk minimization can be cast as:
\begin{align*}
\widehat{\mathcal{R}}_{\varepsilon}^{\star}:=\inf_{\mu\in \mathcal{M}^+_1(\Theta)}\sum_{i=1}^\numsamples\sup_{\QQ_i\in\Gamma_{i,\varepsilon}}\mathbb{E}_{(x,y)\sim \QQ_i,\theta \sim \mu}\left[\loss(\theta,(x,y))\right]
\end{align*}
where $\Gamma_{i,\varepsilon}$ is defined as : 
\begin{align*}
    \Gamma_{i,\varepsilon}:=\Big\{\QQ_i\mid~\int d\QQ_i=\frac{1}{\numsamples},~\int c_{\varepsilon}((x_i,y_i),\cdot) d\QQ_i=0\Big\}.
\end{align*}
\begin{prop}
    Let $\hat{\PP}:=\frac1\numsamples\sum_{i=1}^\numsamples \delta_{(x_i,y_i)}$. Let $l$ be a loss satisfying Assumption~\ref{ass:loss}. Then we have:
    \begin{align*}
    \frac{1}{N}\sum_{i=1}^N\sup_{x,~d(x,x_i)\leq\varepsilon}\mathbb{E}_{\theta \sim \mu}\left[\loss(\theta,(x,y))\right]=\sum_{i=1}^N\sup_{\QQ_i\in\Gamma_{i,\varepsilon}}\mathbb{E}_{(x,y)\sim \QQ_i,\theta \sim \mu}\left[\loss(\theta,(x,y))\right]
    \end{align*}
    where $\Gamma_{i,\varepsilon}$ is defined as : 
    \begin{align*}
        \Gamma_{i,\varepsilon}:=\Big\{\QQ_i\mid~\int d\QQ_i=\frac{1}{N},~\int c_{\varepsilon}((x_i,y_i),\cdot) d\QQ_i=0\Big\}.
    \end{align*}\end{prop}
    
    \begin{proof}
    This proposition is a direct application of Proposition~\ref{prop:dro_adv} for Dirac distributions $\delta_{(x_i,y_i)}$.
    \end{proof}
    
In the following, we regularize the above objective by adding an entropic term to each inner supremum problem. Let $\bm{\alpha}:=(\alpha_i)_{i=1}^\numsamples\in\mathbb{R}_+^\numsamples$ such that for all $i\in\{1,\dots,\numsamples\}$, and let us consider the following optimization problem:
\begin{equation*}
\begin{aligned}
\label{eq-legendre-KL}
\widehat{\mathcal{R}}_{\varepsilon,\bm{\alpha}}^{\star}:=\inf_{\mu\in \mathcal{M}^+_1(\Theta)}\sum_{i=1}^\numsamples&\sup_{\QQ_i\in\Gamma_{i,\varepsilon}}\mathbb{E}_{ \QQ_i, \mu}\left[\loss(\theta,(x,y))\right]\\
&-\alpha_i\text{KL}\left(\QQ_i\Big|\Big|\frac{1}{\numsamples}\mathbb{U}_{(x_i,y_i)}\right)
\end{aligned}
\end{equation*}
where $\mathbb{U}_{(x,y)}$ is an arbitrary distribution of support equal to:
\begin{align*}
    S_{(x,y)}^{(\varepsilon)}:=\Big\{(x',y')\mid c_{\varepsilon}((x,y),(x',y'))=0\Big\},
\end{align*}
and for all $\QQ,\mathbb{U}\in\mathcal{M}_{+}(\mathcal{X}\times\mathcal{Y})$,
\begin{align*}
\text{KL}(\QQ||\mathbb{U}):=  \left\{
    \begin{array}{lll}
        \int\log(\frac{d\QQ}{d\mathbb{U}})d\QQ+|\mathbb{U}| - |\QQ| &  \mbox{if } \QQ\ll \mathbb{U}\\
        +\infty & \mbox{otherwise.}
    \end{array}
\right.
\end{align*}
Note that when $\bm{\alpha}=0$, we recover the problem of interest $\widehat{\mathcal{R}}_{\varepsilon}^{\star}=\widehat{\mathcal{R}}_{\varepsilon,\bm{0}}^{\star}$. Moreover, we show the regularized supremum tends to the standard supremum when $\bm{\alpha}\to 0$.
\begin{prop}
\label{prop:limit-eps}
For $\mu\in\mathcal{M}_{1}^{+}(\Theta)$, one has
\begin{align*}
    &\lim_{\alpha_i\rightarrow 0}\sup_{\QQ_i\in\Gamma_{i,\varepsilon}}\mathbb{E}_{\QQ_i,\mu}\left[\loss(\theta,(x,y))\right]-\alpha_i\text{KL}\left(\QQ\Big|\Big|\frac{1}{\numsamples}\mathbb{U}_{(x_i,y_i)}\right)\\
    &=\sup_{\QQ_i\in\Gamma_{i,\varepsilon}}\mathbb{E}_{(x,y)\sim \QQ_i,\theta \sim \mu}\left[\loss(\theta,(x,y))\right].
\end{align*}
\end{prop}
\begin{proof}
Let us first show that for $\alpha\geq 0$, $\sup_{\QQ_i\in\Gamma_{i,\varepsilon}}\mathbb{E}_{\QQ_i,\mu}\left[\loss(\theta,(x,y))\right]-\alpha\text{KL}\left(\QQ_i\Big|\Big|\frac{1}{\numsamples}\mathbb{U}_{(x_i,y_i)}\right)$ admits a solution. Let $\alpha\geq 0$,
$(\QQ_{\alpha,i}^{n})_{n\geq 0}$ a sequence such that
\begin{align*}
  \mathbb{E}_{\QQ_{\alpha,i}^{n},\mu}\left[\loss(\theta,(x,y))\right]&-\alpha\text{KL}\left(\QQ_{\alpha,i}^{n}\Big|\Big|\frac{1}{\numsamples}\mathbb{U}_{(x_i,y_i)}\right)\\
  &\xrightarrow[n]{} \sup_{\QQ_i\in\Gamma_{i,\varepsilon}}\mathbb{E}_{\QQ_i,\mu}\left[\loss(\theta,(x,y))\right]-\alpha\text{KL}\left(\QQ_i\Big|\Big|\frac{1}{\numsamples}\mathbb{U}_{(x_i,y_i)}\right).
\end{align*}
As $\Gamma_{i,\varepsilon}$ is tight ($(\mathcal{X},d)$ is a proper metric space therefore all the closed ball are compact) and by Prokhorov's theorem, we can extract a subsequence which converges  toward  $\QQ^{\star}_{\alpha,i}$. Moreover, $\loss$ is upper semi-continuous (u.s.c), thus $\QQ\rightarrow \mathbb{E}_{\QQ,\mu}\left[\loss(\theta,(x,y))\right]$ is also u.s.c.\footnote{Indeed, by considering a decreasing sequence of continuous and bounded functions which converge towards $\mathbb{E}_{\mu}\left[\loss(\theta,(x,y))\right]$ and by definition of the weak convergence the result follows.} Moreover, 
$\QQ\rightarrow - \alpha \text{KL}\left(\QQ\Big|\Big|\frac{1}{\numsamples}\mathbb{U}_{(x_i,y_i)}\right)$ is also u.s.c. \footnote{For $\alpha=0$ the result is clear, and if $\alpha>0$, note that $\text{KL}\left(\cdot\Big|\Big|\frac{1}{\numsamples}\mathbb{U}_{(x_i,y_i)}\right)$ is lower semi-continuous}, therefore, by considering the limit superior as $n$ goes to infinity we obtain that
\begin{align*}
    &\limsup_{n\to+\infty}\mathbb{E}_{\QQ_{\alpha,i}^{n},\mu}\left[\loss(\theta,(x,y))\right]-\alpha\text{KL}\left(\QQ_{\alpha,i}^{n}\Big|\Big|\frac{1}{\numsamples}\mathbb{U}_{(x_i,y_i)}\right)\\
    &=\sup_{\QQ_i\in\Gamma_{i,\varepsilon}}\mathbb{E}_{\QQ_i,\mu}\left[\loss(\theta,(x,y))\right]-\alpha\text{KL}\left(\QQ_i\Big|\Big|\frac{1}{\numsamples}\mathbb{U}_{(x_i,y_i)}\right)\\
    &\leq \mathbb{E}_{\QQ_{\alpha,i}^{\star},\mu}\left[\loss(\theta,(x,y))\right]-\alpha\text{KL}\left(\QQ_{\alpha,i}^{\star}\Big|\Big|\frac{1}{\numsamples}\mathbb{U}_{(x_i,y_i)}\right)
\end{align*}
from which we deduce that $\QQ_{\alpha,i}^{\star}$ is optimal.

Let us now show the result. We consider a positive sequence of $(\alpha_i^{(\ell)})_{\ell\geq0}$ such that $\alpha_i^{(\ell)}\to 0$.
Let us denote $\QQ^{\star}_{\alpha_i^{(\ell)},i}$ and $\QQ^{\star}_i$ the solutions of  respectively:
$$\max_{\QQ_i\in\Gamma_{i,\varepsilon}}\mathbb{E}_{\QQ_i,\mu}\left[\loss(\theta,(x,y))\right]-\alpha_i^{(\ell)}\text{KL}\left(\QQ_i\Big|\Big|\frac{1}{\numsamples}\mathbb{U}_{(x_i,y_i)}\right)$$
and 
$$\max_{\QQ_i\in\Gamma_{i,\varepsilon}}\mathbb{E}_{\QQ_i,\mu}\left[\loss(\theta,(x,y))\right]\quad.$$  Since $\Gamma_{i,\varepsilon}$ is tight, $(\QQ^{\star}_{\alpha_i^{(\ell)},i})_{\ell\geq 0}$ is also tight and we can extract by Prokhorov's theorem a subsequence which converges towards $\QQ^{\star}$. Moreover we have
\begin{align*}
 \mathbb{E}_{\QQ^{\star}_i,\mu}\left[\loss(\theta,(x,y))\right] &-\alpha_i^{(\ell)}\text{KL}\left(\QQ^{\star}_i\Big|\Big|\frac{1}{\numsamples}\mathbb{U}_{(x_i,y_i)}\right)\\
 &\leq \mathbb{E}_{\QQ^{\star}_{\alpha_i^{(\ell)},i},\mu}\left[\loss(\theta,(x,y))\right] -\alpha_i^{(\ell)}\text{KL}\left(\QQ^{\star}_{\alpha_i^{(\ell)},i}\Big|\Big|\frac{1}{\numsamples}\mathbb{U}_{(x_i,y_i)}\right)
\end{align*}
from which follows that
\begin{align*}
0\leq \mathbb{E}_{\QQ^{\star}_i,\mu}\left[\loss(\theta,(x,y))\right] &-  \mathbb{E}_{\QQ^{\star}_{\alpha_i^{(\ell)},i},\mu}\left[\loss(\theta,(x,y))\right]\\
&\leq \alpha_i^{(\ell)}\left(\text{KL}\left(\QQ^{\star}_i\Big|\Big|\frac{1}{\numsamples}\mathbb{U}_{(x_i,y_i)}\right)- \text{KL}\left(\QQ^{\star}_{\alpha_i^{(\ell)},i}\Big|\Big|\frac{1}{\numsamples}\mathbb{U}_{(x_i,y_i)}\right)\right)
\end{align*}
Then by considering the limit superior we obtain that
\begin{align*}
    \limsup_{\ell\to+\infty}\mathbb{E}_{\QQ^{\star}_{\alpha_i^{(\ell)},i},\mu}\left[\loss(\theta,(x,y))\right] = \mathbb{E}_{\QQ^{\star}_i,\mu}\left[\loss(\theta,(x,y))\right]
\end{align*}
from which follows that 
\begin{align*}
 \mathbb{E}_{\QQ^{\star}_i,\mu}\left[\loss(\theta,(x,y))\right]\leq \mathbb{E}_{\QQ^{\star},\mu}\left[\loss(\theta,(x,y))\right]
\end{align*}
and by optimality of $\QQ^{\star}_i$ we obtain the desired result. 
\end{proof}



By adding an entropic term to the objective, we obtain an explicit formulation of the supremum involved in the sum: as soon as $\bm{\alpha}>0$ (which means that each $\alpha_i>0$), each sub-problem becomes just the Fenchel-Legendre transform of $\text{KL}(\cdot|\mathbb{U}_{(x_i,y_i)}/\numsamples)$ which has the following closed form:
\begin{align*}
 &\sup_{\QQ_i\in\Gamma_{i,\varepsilon}}\mathbb{E}_{\QQ_i, \mu}\left[\loss(\theta,(x,y))\right]-\alpha_i\text{KL}\left(\QQ_i||\frac{1}{\numsamples}\mathbb{U}_{(x_i,y_i)}\right)\\
 &=\frac{\alpha_i}{\numsamples}\log\left( \int_{\mathcal{X}\times\mathcal{Y}}\exp\left(\frac{\mathbb{E}_{\theta \sim \mu}\left[\loss(\theta,(x,y))\right]}{\alpha_i}\right)d\mathbb{U}_{(x_i,y_i)}\right).
\end{align*}
Finally, we end up with the following problem: 
\begin{align*}
  \inf_{\mu\in \mathcal{M}^+_1(\Theta)}\sum_{i=1}^\numsamples  \frac{\alpha_i}{\numsamples}\log\left( \int\exp\frac{\mathbb{E}_{ \mu}\left[\loss(\theta,(x,y))\right]}{\alpha_i}d\mathbb{U}_{(x_i,y_i)}\right).
\end{align*}
In order to solve the above problem, one needs to compute the integral involved in the objective. To do so, we estimate it by randomly sampling $m_i\geq 1$ samples $(u_1^{(i)},\dots,u_{m_i}^{(i)})\in(\mathcal{X}\times\mathcal{Y})^{m_i}$ from $\mathbb{U}_{(x_i,y_i)}$ for all $i\in\{1,\dots,\numsamples\}$ which leads to the following optimization problem
\begin{align}
\label{eq-obj-sample}
  \inf_{\mu\in \mathcal{M}^+_1(\Theta)}\sum_{i=1}^\numsamples  \frac{\alpha_i}{\numsamples}\log\left( \frac{1}{m_i}\sum_{j=1}^{m_i}\exp\frac{\mathbb{E}_{ \mu}\left[\loss(\theta,u_j^{(i)})\right]}{\alpha_i}\right)
\end{align}
denoted $\widehat{\mathcal{R}}_{\varepsilon,\bm{\alpha}}^{\bm{m}}$ where $\bm{m}:=(m_i)_{i=1}^\numsamples$ in the following. Now we aim at controlling the error made with our approximations. We decompose the error into two terms
\begin{align*}
  |\widehat{\mathcal{R}}_{\varepsilon,\bm{\alpha}}^{\bm{m}} - \widehat{\mathcal{R}}_{\varepsilon}^{\star}|
   \leq |\widehat{\mathcal{R}}_{\varepsilon,\bm{\alpha}}^{\star} - \widehat{\mathcal{R}}_{\varepsilon,\bm{\alpha}}^{\bm{m}}| +|\widehat{\mathcal{R}}_{\varepsilon,\bm{\alpha}}^{\star} - \widehat{\mathcal{R}}_{\varepsilon}^{\star}|
\end{align*}
where the first one corresponds to the statistical error made by our estimation of the integral, and the second to the approximation error made by the entropic regularization of the objective. First, we show a control of the statistical error using Rademacher complexities~\citep{bartlett2002rademacher}. %See proof in Appendix~\ref{prv:control-error-stat}.
\begin{prop}
\label{prop:control-error-stat}
Let $m\geq 1$ and $\alpha>0$ and denote $\bm{\alpha}:=(\alpha,\dots,\alpha)\in\mathbb{R}^\numsamples$ and $\bm{m}:=(m,\dots,m)\in\mathbb{R}^\numsamples$.  \textcolor{black}{Then by denoting $\tilde{M}=\max(M,\alpha)$} with $M$ as in Assumption~\ref{ass:loss}, we have with a probability of at least $1-\delta$
\begin{align*}
|\widehat{\mathcal{R}}_{\varepsilon,\bm{\alpha}}^{\star} - \widehat{\mathcal{R}}_{\varepsilon,\bm{\alpha}}^{\bm{m}}|\leq& \frac{2e^{M/\alpha}}{\numsamples}\sum_{i=1}^\numsamples C_i + \color{black}{6\tilde{M}}\color{black}e^{M/\alpha}\sqrt{\frac{\log(\frac4\delta)}{2m\numsamples}}
\end{align*}
where $C_i:=\frac{1}{m}\mathbb{E}_{\bm{\sigma}}\left[\sup_{\theta\in\Theta}\sum_{j=1}^m \sigma_j \loss(\theta,u_j^{(i)})\right]$ and $\bm{\sigma}:=(\sigma_1,\dots,\sigma_m)$ with $\sigma_i$ i.i.d. sampled as $\mathbb{P}[\sigma_i=\pm1]=1/2$.
\end{prop}

\begin{proof}
Let us denote for all $\mu\in\mathcal{M}_1^{+}(\Theta)$,
\begin{align*}
  \widehat{\mathcal{R}}^{\textbf{m}}_{\varepsilon,\bm{\alpha}}(\mu):=  \sum_{i=1}^\numsamples  \frac{\alpha_i}{\numsamples}\log\left( \frac{1}{m}\sum_{j=1}^{m}\exp\frac{\mathbb{E}_{ \mu}\left[\loss(\theta,u_j^{(i)})\right]}{\alpha_i}\right).
\end{align*}
Let us also consider $(\mu^{(\textbf{m})}_n)_{n\geq 0}$ and $(\mu_n)_{n\geq 0}$ two sequences such that
\begin{align*}
 \widehat{\mathcal{R}}^{\textbf{m}}_{\varepsilon,\bm{\alpha}}(\mu^{(\textbf{m})}_n) \xrightarrow[n \to +\infty]{}\widehat{\mathcal{R}}^{\textbf{m}}_{\varepsilon,\bm{\alpha}},~\quad
\widehat{\mathcal{R}}_{\varepsilon,\bm{\alpha}}(\mu_n)\xrightarrow[n \to +\infty]{}\widehat{\mathcal{R}}^{\star}_{\varepsilon,\bm{\alpha}}.
\end{align*}
Since $\widehat{\mathcal{R}}^{\textbf{m}}_{\varepsilon,\bm{\alpha}} \leq \widehat{\mathcal{R}}^{\textbf{m}}_{\varepsilon,\bm{\alpha}}(\mu_n)$, we remark that
\begin{align*}
\widehat{\mathcal{R}}^{\textbf{m}}_{\varepsilon,\bm{\alpha}}- \widehat{\mathcal{R}}^{\star}_{\varepsilon,\bm{\alpha}}& =\widehat{\mathcal{R}}^{\textbf{m}}_{\varepsilon,\bm{\alpha}} - \widehat{\mathcal{R}}^{\textbf{m}}_{\varepsilon,\bm{\alpha}}(\mu_n)\\
& + \widehat{\mathcal{R}}^{\textbf{m}}_{\varepsilon,\bm{\alpha}}(\mu_n) - \widehat{\mathcal{R}}_{\varepsilon,\bm{\alpha}}(\mu_n)\\
&+ \widehat{\mathcal{R}}_{\varepsilon,\bm{\alpha}}(\mu_n)-
\widehat{\mathcal{R}}^{\star}_{\varepsilon,\bm{\alpha}} \\
&\leq \sup_{\mu\in \mathcal{M}^+_1(\Theta)}\Big|\widehat{\mathcal{R}}^{\textbf{m}}_{\varepsilon,\bm{\alpha}}(\mu) - \widehat{\mathcal{R}}_{\varepsilon,\bm{\alpha}}(\mu) \Big|\\
& + \widehat{\mathcal{R}}_{\varepsilon,\bm{\alpha}}(\mu_n)-
\widehat{\mathcal{R}}^{\star}_{\varepsilon,\bm{\alpha}},
\end{align*}
and by considering the limit, we obtain that
\begin{align*}
  \widehat{\mathcal{R}}^{\textbf{m}}_{\varepsilon,\bm{\alpha}}- \widehat{\mathcal{R}}^{\star}_{\varepsilon,\bm{\alpha}}&\leq  \sup_{\mu\in \mathcal{M}^+_1(\Theta)}\Big|\widehat{\mathcal{R}}^{\textbf{m}}_{\varepsilon,\bm{\alpha}}(\mu) - \widehat{\mathcal{R}}_{\varepsilon,\bm{\alpha}}(\mu) \Big| 
\end{align*}
Similarly we have that
\begin{align*}
\widehat{\mathcal{R}}^{\star}_{\varepsilon,\bm{\alpha}} - \widehat{\mathcal{R}}^{\textbf{m}}_{\varepsilon,\bm{\alpha}}&\leq \widehat{\mathcal{R}}^{\star}_{\varepsilon,\bm{\alpha}} -
\widehat{\mathcal{R}}_{\varepsilon,\bm{\alpha}}(\mu_n^{(\bm{m})})\\
&+\widehat{\mathcal{R}}_{\varepsilon,\bm{\alpha}}(\mu_n^{(\bm{m})}) - \widehat{\mathcal{R}}^{\textbf{m}}_{\varepsilon,\bm{\alpha}}(\mu_n^{(\bm{m})}) \\
&+ \widehat{\mathcal{R}}^{\textbf{m}}_{\varepsilon,\bm{\alpha}}(\mu_n^{(\bm{m})}) - \widehat{\mathcal{R}}^{\textbf{m}}_{\varepsilon,\bm{\alpha}}
\end{align*}
from which follows that 
\begin{align*}
\widehat{\mathcal{R}}^{\star}_{\varepsilon,\bm{\alpha}} - \widehat{\mathcal{R}}^{\textbf{m}}_{\varepsilon,\bm{\alpha}}&\leq  \sup_{\mu\in \mathcal{M}^+_1(\Theta)}\Big|\widehat{\mathcal{R}}^{\textbf{m}}_{\varepsilon,\bm{\alpha}}(\mu) - \widehat{\mathcal{R}}_{\varepsilon,\bm{\alpha}}(\mu) \Big| 
\end{align*}
Therefore we obtain that 
\begin{align*}
\Big| \widehat{\mathcal{R}}^{\star}_{\varepsilon,\bm{\alpha}} - \widehat{\mathcal{R}}^{\textbf{m}}_{\varepsilon,\bm{\alpha}}\Big |\leq 
\sum_{i=1}^\numsamples\frac{\alpha}{\numsamples}  &\sup_{\mu\in \mathcal{M}^+_1(\Theta)}\Big|\log\left(\frac{1}{m_i}\sum_{j=1}^{m_i}\exp\left(\frac{\mathbb{E}_{\theta \sim \mu}\left[\loss(\theta,u_j^{(i)}))\right]}{\alpha}\right)\right)\\
    &- \log\left(\int_{\mathcal{X}\times\mathcal{Y}}\exp\left(\frac{\mathbb{E}_{\theta \sim \mu}\left[\loss(\theta,(x,y))\right]}{\alpha}\right) d\mathbb{U}_{(x_i,y_i)} \right)\Big|.
\end{align*}
Observe that $\loss$ is non negative, therefore because the $\log$ function is 1-Lipschitz on $[1,+\infty)$, we obtain that 
\begin{align*}
\Big| \widehat{\mathcal{R}}^{\star}_{\varepsilon,\bm{\alpha}} - \widehat{\mathcal{R}}^{\textbf{m}}_{\varepsilon,\bm{\alpha}}\Big |
\leq 
\sum_{i=1}^\numsamples\frac{\alpha}{\numsamples}  &\sup_{\mu\in \mathcal{M}^+_1(\Theta)}\Big |\frac{1}{m}\sum_{j=1}^{m}\exp\left(\frac{\mathbb{E}_{\theta \sim \mu}\left[\loss(\theta,u_j^{(i)}))\right]}{\alpha}\right)\\
    &- \int_{\mathcal{X}\times\mathcal{Y}}\exp\left(\frac{\mathbb{E}_{\theta \sim \mu}\left[\loss(\theta,(x,y))\right]}{\alpha}\right) d\mathbb{U}_{(x_i,y_i)} \Big|.
\end{align*}
Let us now denote for all $i=1,\dots,\numsamples$,
\begin{align*}
    \widehat{C}_i(\mu,\bm{u}^{(i)})&:=\frac1m \sum_{j=1}^{m}\exp\left(\frac{\mathbb{E}_{\theta \sim \mu}\left[\loss(\theta,u_j^{(i)}))\right]}{\alpha}\right)\\
    C_i(\mu)&:= \int_{\mathcal{X}\times\mathcal{Y}}\exp\left(\frac{\mathbb{E}_{\theta \sim \mu}\left[\loss(\theta,(x,y))\right]}{\alpha}\right) d\mathbb{U}_{(x_i,y_i)}.
\end{align*}
and let us define 
\begin{align*}
    f(\bm{u}^{(1)},\dots,\bm{u}^{(\numsamples)}):=\sum_{i=1}^\numsamples\frac{\alpha}{\numsamples}\sup_{\mu\in \mathcal{M}^+_1(\Theta)}\Big |\widehat{C}_i(\mu,\bm{u}^{(i)}) -C_i(\mu)\Big |
\end{align*}
where $\bm{u}^{(i)}:=(u_1^{(i)},\dots,u_1^{(m)})$. By denoting $z^{(i)}=(u_1^{(i)},\dots,u_{k-1}^{(i)},z,u_{k+1}^{(i)},\dots,u_m^{(i)})$, we have that
\begin{align*}
  |f(\bm{u}^{(1)},\dots,\bm{u}^{(\numsamples)})& - f(\bm{u}^{(1)},\dots,\bm{u}^{(i-1)},\bm{z}^{(i)},\bm{u}^{(i+1)},\dots,\bm{u}^{(\numsamples)})|\\
  &\leq \frac{\alpha}{\numsamples}\Big | \sup_{\mu\in \mathcal{M}^+_1(\Theta)}\Big |\widehat{C}_i(\mu,\bm{u}^{(i)}) -C_i(\mu)\Big |\\
 & - \sup_{\mu\in \mathcal{M}^+_1(\Theta)}\Big |\widehat{C}_i(\mu,\bm{z}^{(i)}) -C_i(\mu)\Big | \Big |\\
 &\leq\sup_{\mu\in \mathcal{M}^+_1(\Theta)} \lvert \widehat{C}_i(\mu,\bm{u}^{(i)}) -\widehat{C}_i(\mu,\bm{z}^{(i)}) \rvert\\
  &= \frac{\alpha}{\numsamples}\Big |\frac{1}{m}\left[\exp\left(\frac{\mathbb{E}_{\theta \sim \mu}\left[\loss(\theta,u_k^{(i)}))\right]}{\alpha}\right) - \exp\left(\frac{\mathbb{E}_{\theta \sim \mu}\left[\loss(\theta,z^{(i)}))\right]}{\alpha}\right) \right]\Big| \\
  &\leq \frac{2\alpha\exp(M/\alpha)}{\numsamples m}
\end{align*}
where the last inequality comes from the fact that the loss  $\loss$ is upper bounded by $M$. Then by appling the McDiarmid Inequality, we obtain that with a probability of at least $1-\delta$,
\begin{align*}
 \Big| \widehat{\mathcal{R}}^{\star}_{\varepsilon,\bm{\alpha}} - \widehat{\mathcal{R}}^{\textbf{m}}_{\varepsilon,\bm{\alpha}}\Big |\leq\mathbb{E}(f(\bm{u}^{(1)},\dots,\bm{u}^{(\numsamples)}))+\frac{2\alpha\exp(M/\alpha)}{\sqrt{m\numsamples}}\sqrt{\frac{\log(2/\delta)}{2}}.
\end{align*}
We have that 
\begin{align*}
    \mathbb{E}(f(\bm{u}^{(1)},\dots,\bm{u}^{(\numsamples)}))=\frac{\alpha}{n} \sum_{i=1}^\numsamples \mathbb{E}\left(\sup_{\mu\in \mathcal{M}^+_1(\Theta)}\Big |\widehat{C}_i(\mu,\bm{u}^{(i)}) -C_i(\mu)\Big |\right) \quad.   
\end{align*}
From the properties of Rademacher complexity (see Section~\ref{sec:erm-results}), we have for every $i$ :
\begin{align*}
    \mathbb{E}\left(\sup_{\mu\in \mathcal{M}^+_1(\Theta)}\Big |\widehat{C}_i(\mu) -C_i(\mu)\Big |\right)\leq 2 \mathbb{E}(\text{Rad}(\mathcal{T}\circ \bm{u}^{(i)}))   
\end{align*}

where we recall for any class of functions $\mathcal{H}$ defined on $\mathcal{Z}$  and point $\bm{z}:(z_1,\dots,z_q)\in\mathcal{Z}^q$
\begin{align*}
    &\mathcal{H}\circ \bm{z}:=\Big\{(f(z_1),\dots,f(z_q)),f\in\mathcal{F}\Big\} \quad,\\
    &\quad \text{Rad}(\mathcal{T}\circ \bm{z}):=\frac{1}{q}\mathbb{E}_{\bm{\sigma}\sim\{\pm 1\}}\left[\sup_{f\in\mathcal{H}}\sum_{i=1}^q\sigma_if(z_i)\right],\\
    &\mathcal{T}:=\Big\{u\rightarrow\exp\left(\frac{\mathbb{E}_{\theta \sim \mu}\left[\loss(\theta,u))\right]}{\alpha}\right),~\mu\in\mathcal{M}_{1}^{+}(\Theta) \Big\}.
      \end{align*}

Moreover, as mentioned in Section~\ref{sec:erm-results}, $x\mapsto\exp(x/\alpha)$ is $\frac{\exp(M/\alpha)}{\alpha}$-Lipschitz on $(-\infty,M]$, we have 
\begin{align*}
   \text{Rad}(\mathcal{T}\circ \mathbf{u^{(i)}})\leq \frac{\exp(M/\alpha)}{\alpha} \text{Rad}(\mathcal{H}\circ \mathbf{u^{(i)}}) 
\end{align*}
where 
\begin{align*}
    \mathcal{H}:=\Big\{u\rightarrow \mathbb{E}_{\theta \sim \mu}\left[\loss(\theta,u))\right],~\mu\in\mathcal{M}_{1}^{+}(\Theta) \Big\}.
\end{align*}
Let us now define
\begin{align*}
    g(\bm{u}^{(1)},\dots,\bm{u}^{(\numsamples)}):=\sum_{j=1}^\numsamples\frac{2\exp(M/\alpha)}{\numsamples}\text{Rad}(\mathcal{H}\circ \mathbf{u^{(j)}}).
\end{align*}
We observe that 
\begin{align*}
|g(\bm{u}^{(1)},\dots,\bm{u}^{(\numsamples)}) &- g(\bm{u}^{(1)},\dots,\bm{u}^{(i-1)},\bm{z}^{(i)},\bm{u}^{(i+1)},\dots,\bm{u}^{(\numsamples)})|\\
&\leq \frac{2\exp(M/\alpha)}{\numsamples}|\text{Rad}(\mathcal{H}\circ \mathbf{u^{(i)}}) - \text{Rad}(\mathcal{H}\circ \mathbf{z^{(i)}})|\\
&\leq \frac{2\exp(M/\alpha)}{\numsamples}\frac{M}{m}.
\end{align*}
By Applying the McDiarmid’s Inequality, we have that with a probability of at least $1-\delta$
\begin{align*}
\mathbb{E}(g(\bm{u}^{(1)},\dots,\bm{u}^{(\numsamples)}))\leq g(\bm{u}^{(1)},\dots,\bm{u}^{(\numsamples)}) +\frac{2\exp(M/\alpha)M}{\sqrt{m\numsamples}}\sqrt{\frac{\log(2/\delta)}{2}}.
\end{align*}
Remarks also that 
\begin{align*}
    \text{Rad}(\mathcal{H}\circ \mathbf{u^{(i)}})&=\frac{1}{m}\mathbb{E}_{\bm{\sigma}\sim\{\pm 1\}}\left[\sup_{\mu\in\mathcal{M}_1^{+}(\Theta)}\sum_{j=1}^m\sigma_i\mathbb{E}_{\mu}(\loss(\theta,u^{(i)}_j))\right]\\
    &=\frac{1}{m}\mathbb{E}_{\bm{\sigma}\sim\{\pm 1\}}\left[\sup_{\theta\in\Theta}\sum_{j=1}^m\sigma_i \loss(\theta,u^{(i)}_j)\right]
\end{align*}
Finally, applying a union bound leads to the desired result.

\end{proof}




We deduce from the above Proposition that in the particular case where $\Theta$ is finite such that $|\Theta|= l$, with probability of at least $1-\delta$
\begin{align*}
   |\widehat{\mathcal{R}}_{\varepsilon,\bm{\alpha}}^{\star} - \widehat{\mathcal{R}}_{\varepsilon,\bm{\alpha}}^{\bm{m}}| \in \mathcal{O}\left(Me^{M/\alpha}\sqrt{\frac{\log(l)}{m}} \right).
\end{align*}
This case is of particular interest when one wants to learn the optimal mixture of some given classifiers in order to minimize the adversarial risk. In the following proposition, we control the approximation error made by adding an entropic term to the objective. %See proof in Appendix~\ref{prv:control-error-approx}.
\begin{prop}
\label{prop:control-error-approx}
Denote for $\beta>0$, $(x,y)\in\mathcal{X}\times\mathcal{Y}$ and $\mu\in\mathcal{M}_{1}^{+}(\Theta)$,
    $$A_{\beta,\mu}^{\left(  x,y\right)}:=\{u|\sup_{v\in  S_{(x,y)}^{(\varepsilon)}}\mathbb{E}_{\mu}[\loss(\theta,v)]\leq \mathbb{E}_{\mu}[\loss(\theta,u)]+\beta\}$$ 
    where 
    \begin{align*}
        S_{(x,y)}^{(\varepsilon)}:=\Big\{(x',y')\mid c_{\varepsilon}((x,y),(x',y'))=0\Big\},
    \end{align*}
    If there exists $C_{\beta}$ such that for all $(x,y)\in\mathcal{X}\times\mathcal{Y}$ and $\mu\in\mathcal{M}_{1}^{+}(\Theta)$, $\mathbb{U}_{(x,y)}\left(A_{\beta,\mu}^{\left(  x,y\right)}\right)\geq C_\beta$ then we have
\begin{align*}
   |\widehat{\mathcal{R}}_{\varepsilon,\bm{\alpha}}^{\star} - \widehat{\mathcal{R}}_{\varepsilon}^{\star}|\leq 2\alpha |\log(C_\beta)| + \beta.
\end{align*}
\end{prop}




The assumption made in the above Proposition states that for any given random classifier $\mu$, and any given point $(x,y)$, the set of $\beta$-optimal attacks at this point has at least a certain amount of mass depending on the $\beta$ chosen. This assumption is always true when $\beta$ is sufficiently large. However, in order to obtain a tight control of the error, a trade-off exists between $\beta$ and the smallest amount of mass $C_{\beta}$ of $\beta$-optimal attacks.

\begin{proof}
Following the same steps as for the proof of Proposition~\ref{prop:control-error-stat}, let $(\mu_n^{\varepsilon})_{n\geq 0}$ and $(\mu_n)_{n\geq 0}$ be two sequences such that
\begin{align*}
    \widehat{\mathcal{R}}_{\varepsilon,\bm{\alpha}}^{\varepsilon}(\mu_n^{\varepsilon})\xrightarrow[n \to +\infty]{}\widehat{\mathcal{R}}_{\varepsilon,\bm{\alpha}}^{\star},~\quad \widehat{\mathcal{R}}_{\varepsilon}^{\varepsilon}(\mu_n)\xrightarrow[n \to +\infty]{}\widehat{\mathcal{R}}_{\varepsilon}^{\star}.
\end{align*}
Remarks that 
\begin{align*}
  \widehat{\mathcal{R}}_{\varepsilon,\bm{\alpha}}^{\star} - \widehat{\mathcal{R}}_{\varepsilon}^{\star}&\leq \widehat{\mathcal{R}}_{\varepsilon,\bm{\alpha}}^{\star} - \widehat{\mathcal{R}}_{\varepsilon,\bm{\alpha}}^{\varepsilon}(\mu_n)\\
  & + \widehat{\mathcal{R}}_{\varepsilon,\bm{\alpha}}^{\varepsilon}(\mu_n) -   \widehat{\mathcal{R}}_{\varepsilon}^{\varepsilon}(\mu_n)\\
  &+ \widehat{\mathcal{R}}_{\varepsilon}^{\varepsilon}(\mu_n)-\widehat{\mathcal{R}}_{\varepsilon}^{\star}\\
  &\leq \sup_{\mu\in\mathcal{M}_1^{+}(\Theta)}\Big|\widehat{\mathcal{R}}_{\varepsilon,\bm{\alpha}}^{\varepsilon}(\mu) -   \widehat{\mathcal{R}}_{\varepsilon}^{\varepsilon}(\mu)  \Big|\\
  & + \widehat{\mathcal{R}}_{\varepsilon}^{\varepsilon}(\mu_n)-\widehat{\mathcal{R}}_{\varepsilon}^{\star}
\end{align*}
Then by considering the limit we obtain that 
\begin{align*}
    \widehat{\mathcal{R}}_{\varepsilon,\bm{\alpha}}^{\star} - \widehat{\mathcal{R}}_{\varepsilon}^{\star}&\leq \sup_{\mu\in\mathcal{M}_1^{+}(\Theta)}\Big|\widehat{\mathcal{R}}_{\varepsilon,\bm{\alpha}}^{\varepsilon}(\mu) -   \widehat{\mathcal{R}}_{\varepsilon}^{\varepsilon}(\mu)  \Big|.
\end{align*}
Similarly, we obtain that 
\begin{align*}
     \widehat{\mathcal{R}}_{\varepsilon}^{\star}-\widehat{\mathcal{R}}_{\varepsilon,\bm{\alpha}}^{\star}&\leq \sup_{\mu\in\mathcal{M}_1^{+}(\Theta)}\Big|\widehat{\mathcal{R}}_{\varepsilon,\bm{\alpha}}^{\varepsilon}(\mu) -   \widehat{\mathcal{R}}_{\varepsilon}^{\varepsilon}(\mu)  \Big|,
\end{align*}
from which follows that
\begin{align*}
 \Big| \widehat{\mathcal{R}}_{\varepsilon,\bm{\alpha}}^{\star} - \widehat{\mathcal{R}}_{\varepsilon}^{\star}\Big|\leq \frac{1}{\numsamples}\sum_{i=1}^\numsamples&\sup_{\mu\in\mathcal{M}_1^{+}(\Theta)}\Big|\alpha\log\left(\int_{\mathcal{X}\times\mathcal{Y}}\exp\left(\frac{\mathbb{E}_{\mu}[\loss(\theta,(x,y))]}{\alpha} \right) d\mathbb{U}_{(x_i,y_i)}\right)\\
 &-\sup_{u\in S^{\varepsilon}_{(x_i,y_i)}}\mathbb{E}_{\mu}[\loss(\theta,u)] \Big|.
\end{align*}
Let $\mu\in\mathcal{M}_1^{+}(\Theta)$ and $i\in\{1,\dots,\numsamples\}$, then we have
\begin{align*}
 \Big|\alpha&\log\left(\int_{\mathcal{X}\times\mathcal{Y}}\exp\left(\frac{\mathbb{E}_{\mu}[\loss(\theta,(x,y))]}{\alpha} \right) d\mathbb{U}_{(x_i,y_i)}\right)-\sup_{u\in S^{\varepsilon}_{(x_i,y_i)}}\mathbb{E}_{\mu}[\loss(\theta,u)] \Big|\\
 &=\Big|\alpha\log\left(\int_{\mathcal{X}\times\mathcal{Y}}\exp\left(\frac{\mathbb{E}_{\mu}[\loss(\theta,(x,y))]-\sup_{u\in S^{\varepsilon}_{(x_i,y_i)}}\mathbb{E}_{\mu}[\loss(\theta,u)]}{\alpha} \right) d\mathbb{U}_{(x_i,y_i)}\right) \Big|  \\
 &=\alpha  \Big| \log\left(\int_{A_{\beta,\mu}^{(x_i,y_i)}}\exp\left(\frac{\mathbb{E}_{\mu}[\loss(\theta,(x,y))]-\sup_{u\in S^{\varepsilon}_{(x_i,y_i)}}\mathbb{E}_{\mu}[\loss(\theta,u)]}{\alpha} \right) d\mathbb{U}_{(x_i,y_i)}\right. \\
 &+ \left.\int_{(A_{\beta,\mu}^{(x_i,y_i)})^{c}}\exp\left(\frac{\mathbb{E}_{\mu}[\loss(\theta,(x,y))]-\sup_{u\in S^{\varepsilon}_{(x_i,y_i)}}\mathbb{E}_{\mu}[\loss(\theta,u)]}{\alpha} \right) d\mathbb{U}_{(x_i,y_i)}\right)  \Big|\\
 &\leq \alpha \Big | \log\left(\exp(-\frac\beta\alpha)\mathbb{U}_{(x_i,y_i)}\left(A_{\beta,\mu}^{(x_i,y_i)}\right) \right)\Big | \\
 &+ \alpha  \Big|\log\left(1+\right.\\
 &\left.\frac{\exp(\beta/\alpha)}{\mathbb{U}_{(x_i,y_i)}\left(A_{\beta,\mu}^{(x_i,y_i)}\right)}\int_{(A_{\beta,\mu}^{(x_i,y_i)})^{c}}\exp\left(\frac{\mathbb{E}_{\mu}[\loss(\theta,(x,y))]-\sup_{u\in S^{\varepsilon}_{(x_i,y_i)}}\mathbb{E}_{\mu}[\loss(\theta,u)]}{\alpha} \right) d\mathbb{U}_{(x_i,y_i)}\right)  \Big|\\
 &\leq \alpha\log(1/C_\beta)+\beta +\frac{\alpha}{C_\beta}\\
 &\leq 2\alpha\log(1/C_\beta)+\beta
\end{align*}
Note that $(A_{\beta,\mu}^{(x_i,y_i)})^{c}$ denotes the complementary set of $A_{\beta,\mu}^{(x_i,y_i)}$.
\end{proof}



Now that we have shown that solving~\eqref{eq-obj-sample} allows to obtain an approximation of the true solution $\widehat{\mathcal{R}}_{\varepsilon}^{\star}$, we next aim at deriving an algorithm to compute it. 

\subsection{Proposed Algorithms}
\label{sec:proposed-algo}
From now on, we focus on finite class of classifiers. Let $\Theta = \{\theta_1,\dots,\theta_l\}$, we aim to learn the optimal mixture of classifiers in this case. The adversarial  empirical risk  is therefore defined as:
\begin{align*}
    \widehat{\mathcal{R}}_{\varepsilon}(\bm{\lambda})= \sum_{i=1}^\numsamples\sup_{\QQ_i\in\Gamma_{i,\varepsilon}}\mathbb{E}_{(x,y)\sim \QQ_i}\left[\sum_{k=1}^l \lambda_k \loss(\theta_k,(x,y))\right]
\end{align*}
for $\bm{\lambda}\in\Delta_l: = \{\bm{\lambda}\in\mathbb{R}_+^l~\mathrm{s.t.}~\sum_{i=1}^l\lambda_i=1\}$, the probability simplex of $\mathbb{R}^l$. One can notice that $ \widehat{\mathcal{R}}_{\varepsilon}(\cdot)$ is a continuous convex function, hence $\min_{\bm{\lambda}\in\Delta_l}\risk(\bm{\lambda})$ is attained for a certain $\bm{\lambda}^\star$. Then there exists a non-approximate Nash equilibrium $(\bm{\lambda}^\star,\QQ^\star)$ in the adversarial game when $\Theta$ is finite. Here, we present two algorithms to learn the optimal mixture of the adversarial risk minimization problem.


\begin{algorithm}[ht]
\SetAlgoLined
 $\bm{\lambda}_0 = \frac{\mathbf{1}_l}{l}; T;~\textcolor{black}{\eta=\frac{2}{M\sqrt{lT}}}$\\
 \For{$t=1,\dots,T$}{

  $\Tilde{\QQ}$ s.t. $\exists\QQ^\star\in\mathcal{A}_\varepsilon(\PP)$ best response to \textcolor{black}{$\bm{\lambda}_{t-1}$} and for all $k\in[l]$, $\lvert\mathbb{E}_{\Tilde{\QQ}}(\loss(\theta_k,(x,y)))-\mathbb{E}_{\QQ^\star}(\loss(\theta_k,(x,y))) \rvert\leq\delta$\\
  $\bm{g}_t=\left(\mathbb{E}_{\Tilde{\QQ}}(\loss(\theta_1,(x,y)),\dots,\mathbb{E}_{\Tilde{\QQ}}(\loss(\theta_l,(x,y))\right)^T$\\
  $\bm{\lambda}_t = \Pi_{\Delta_l}\left(\bm{\lambda}_{t-1}-\eta \bm{g}_t\right)$
  }
 \caption{Oracle-based Algorithm}
 \label{algo:duchi}
\end{algorithm}
% To validate 
\begin{figure*}[!ht]
    \centering
    \includegraphics[width=0.32\textwidth]{Images/illustration.pdf}  \includegraphics[width=0.32\textwidth]{Images/convergence_toy.pdf}     \includegraphics[width=0.32\textwidth]{Images/risk_toy.pdf}
    \caption{On left, $40$ data samples with their set of possible attacks represented in shadow and the optimal randomized classifier, with a color gradient representing the probability of the classifier. \textcolor{black}{In the middle}, convergence of the oracle ($\alpha=0$) and regularized algorithm for different values of regularization parameters. On right, in-sample and out-sample risk for randomized and deterministic minimum risk in function of the perturbation size $\varepsilon$. In the latter case, the randomized classifier is optimized with oracle Algorithm~\ref{algo:duchi}.}
    \label{fig:toy_example}
\end{figure*}


\textbf{An Entropic Relaxation.} Using the results from Section~\ref{sec:entropic-reg}, adding an entropic term to the objective allows to have a simple reformulation of the problem, as follows:
\begin{align*}
  \inf_{\bm{\lambda}\in \Delta_l}\sum_{i=1}^\numsamples  \frac{\alpha}{\numsamples}\log\left( \frac{1}{m_i}\sum_{j=1}^{m_i}\exp\left(\frac{\sum_{k=1}^l \lambda_k\loss(\theta_k,u_j^{(i)})}{\alpha}\right)\right)
\end{align*}
Note that in $\bm{\lambda}$, the objective is convex and smooth. One can  apply the accelerated PGD~\citep{beck2009fast,tseng2008accelerated} which enjoys an optimal convergence rate for first order methods of $\mathcal{O}(T^{-2})$ for $T$ iterations.

\textbf{A First Oracle Algorithm.} Besides entropic regularization, we present an oracle-based algorithm inspired from~\citep{sinha2017certifying} and the convergence of projected subgradient methods~\citep{boyd2003subgradient}. The computation of the inner supremum problem is usually NP-hard. Let us justify it on a mixture of linear classifiers in binary classification: $f_{\theta_k,b_k}(x) = \langle \theta_k,x\rangle+b_k$ for $k\in \llbracket l\rrbracket$ and $\bm{\lambda}=\mathbf{1}_l/l$. Let us consider the $\ell_2$ norm and $x=0$ and $y=1$. Then the problem of attacking $x$ is the following:
\begin{align*}
    \sup_{\tau,~\lVert \tau\rVert\leq\varepsilon} \frac{1}{l}\sum_{k=1}^l\mathbf{1}_{\langle \theta_k,x+\tau\rangle+b_k\leq0}
\end{align*}
This problem is equivalent to a linear binary classification problem on $\tau$, which is known to be NP-hard. Assuming the existence of a $\delta$-approximate oracle to this supremum, the algorithm is presented in Algorithm~\ref{algo:duchi}. We get the following guarantee for this algorithm. %See proof in Appendix~\ref{prv:algo-oracle}.
\begin{prop}
\label{prop:algo-oracle}
% Let $\PP\in\mathcal{M}^1_+(\mathcal{X}\times\mathcal{Y})$.
Let $\Theta = (\theta_1,\dots,\theta_l)$, $\loss:\Theta\times(\mathcal{X}\times\mathcal{Y})\to [0,\infty)$ be a loss satisfying Assumption~\ref{ass:loss}, $M$ bed defined as in Assumption~\ref{ass:loss} and $T\geq 1$. Then, Algorithm~\ref{algo:duchi} satisfies:  
\begin{align*}
    \min_{t\in\llbracket T-1\rrbracket} \widehat{\mathcal{R}}_{\varepsilon}(\bm{\lambda}_t)-\widehat{\mathcal{R}}_{\varepsilon}^{\star}\leq2\delta+\textcolor{black}{ \frac{2M\sqrt{l}}{\sqrt{T}}}
    % \delta\sqrt{K}+ \frac{M\sum_{t=1}^T \eta_t^2+\rVert\lambda_0-\lambda^\star\lVert^2}{\sum_{t=1}^T\eta_t}
\end{align*}
% In particular for $\eta_t =\frac{\eta}{t^{1/2}}$, we get: $\min_{t\in[T]}\risk(\lambda_t)-\risk(\lambda^\star)\in O\left(\delta+\frac{\log T}{\sqrt{T}}\right)$
\end{prop}

\begin{proof}
Thanks to Danskin theorem, if $\QQ^\star$ is a best response to $\bm{\lambda}$, then $$\bm{g}^\star:=\left(\mathbb{E}_{\QQ^\star}\left[\loss(\theta_1,(x,y))\right],\dots,\mathbb{E}_{\QQ^\star}\left[\loss(\theta_l,(x,y))\right]\right)^T$$ is a subgradient of $\bm{\lambda}\to \risk(\bm{\lambda})$. In particular for every $\bm{\lambda}^\star$ optimal classifier:
$$\langle\bm{g}_t,\bm{\lambda}^\star- \bm{\lambda}_{t-1}\rangle\leq \risk_\varepsilon(\bm{\lambda}^\star)-\risk_\varepsilon(\bm{\lambda}_{t-1})\quad.$$
Moreover, we also have
\begin{align*}
    \lvert\langle\bm{g}^\star_t-\bm{g}_t, \bm{\lambda}_{t-1}-\bm{\lambda}^\star\rangle\vert&\leq \lVert \bm{g}^\star_t-\bm{g}_t\rVert_\infty \lVert \bm{\lambda}_{t-1}-\bm{\lambda}^\star\rVert_1\\
    &\leq \delta\left(\lVert{\lambda}_{t-1}\rVert_1+\lVert{\lambda}^\star\rVert_1\right)\\
    &\leq 2\delta\quad.
\end{align*}
We also have that $\lVert \bm{g}_t\rVert_2\leq \sqrt{l}\delta$.
Let $\eta\geq 0$ be the learning rate. Then we have for all $t\geq 1$:
\begin{align*}
\lVert \bm{\lambda}_t-\bm{\lambda}^\star\rVert^2&\leq \lVert \bm{\lambda}_{t-1}-\eta \bm{g}_t-\bm{\lambda}^\star\rVert^2\\
&=\lVert \bm{\lambda}_{t-1}-\bm{\lambda}^\star\rVert^2-2\eta \langle\bm{g}_t, \bm{\lambda}_{t-1}-\bm{\lambda}^\star\rangle+ \eta^2\lVert \bm{g}_t\rVert^2_2\\
&\leq \lVert \bm{\lambda}_{t-1}-\bm{\lambda}^\star\rVert^2-2\eta \langle\bm{g}^\star_t, \bm{\lambda}_{t-1}-\bm{\lambda}^\star\rangle\\
&+2\eta\langle\bm{g}^\star_t-\bm{g}_t, \bm{\lambda}_{t-1}-\bm{\lambda}^\star\rangle+\eta^2 M^2 l\\
&\leq \lVert \bm{\lambda}_{t-1}-\bm{\lambda}^\star\rVert^2-2\eta\left(\risk_\varepsilon(\bm{\lambda}_{t-1})-\risk_\varepsilon(\bm{\lambda}^\star)\right) +4\eta\delta+\eta^2  M^2 l%\lVert \bm{\lambda}_{t-1}-\bm{\lambda}^\star\rVert_1
\end{align*}
We then deduce by summing:
\begin{align*}
   2\eta \sum_{t=0}^{T-1} \risk_\varepsilon(\bm{\lambda}_t)-\risk_\varepsilon(\bm{\lambda}^\star) \leq 4\delta\eta T +\lVert \bm{\lambda}_{0}-\bm{\lambda}^\star\rVert^2+\eta^2 M^2 lT
\end{align*}
Then we have:
\begin{align*}
    \min_{t\in\llbracket T-1\rrbracket }\risk_\varepsilon(\bm{\lambda}_t)-\risk_\varepsilon(\bm{\lambda}^\star)\leq 2\delta+\frac{4}{\eta T}+M^2l\eta
\end{align*}
The left-hand term is minimal for $\eta=\frac{2}{M\sqrt{lT}}$, and for this value:
\begin{align*}
    \min_{t\in\llbracket T-1\rrbracket}\risk_\varepsilon(\bm{\lambda}_t)-\risk_\varepsilon(\bm{\lambda}^\star)\leq 2\delta+\frac{2M\sqrt{l}}{\sqrt{T}}
\end{align*}
\end{proof}. 

The main drawback of the above algorithm is that one needs to have access to an oracle to guarantee the convergence of the proposed algorithm. The entropic regularized algorithm is made  to find an approximate the solution and do not require access to an oracle.

 %In practice, we can change the distribution of sampling, to be more likely to find adversaries. 
% \begin{rmq}
% In general, one can use the exact same tools to obtain a proxy of the general DRO problem. Indeed thanks to~\citep{blanchet2019quantifying}, the dual can be approximated by adding an entropic term to the objective which leads to. We end-up with a minimization problem of a convex objective over the set of distribution. 
% \end{rmq}

\subsection{A General Heuristic Algorithm}

So far, our algorithms are not easily practicable in the case of deep learning. Adversarial examples are known to be easily transferrable from one model to another~\citep{tramer2017space,papernot2016transferability}. So we aim at learning diverse models. To this end, and support our theoretical claims, we propose an heuristic algorithm (see Algorithm~\ref{algo:heuristic}) to train a robust mixture of $l$ classifiers.   We alternatively train these classifiers with adversarial examples against the current mixture and update the probabilities of the mixture according to the algorithms we proposed in Section~\ref{sec:proposed-algo}. 


\begin{algorithm}[h!]
\SetAlgoLined
$l$: number of models, $T$: number of iterations,\\
$T_\theta$: number of updates for the models $\bm{\theta}$,\\
$T_\lambda$: number of updates for the mixture $\bm{\lambda}$,\\ $\bm{\lambda}_0=(\lambda_0^1,\dots\lambda_0^l),~\bm{\theta}_0=(\theta_0^1,\dots\theta_0^l)$\\
 \For{$t=1,\dots,T$}{
 Let $B_t$ be a batch of data.\\
\eIf{$t \mod (T_\theta l+1)\neq 0$}{
$k$ sampled uniformly in $\{1,\dots,l\}$\\
$\Tilde{B}_t\leftarrow$ Attack of images in $B_t$ for the  model $(\bm{\lambda}_t,\bm{\theta}_t)$\\
$\theta^t_k\leftarrow$ Update $\theta^{t-1}_k$ with $\Tilde{B}_t$ for fixed $\bm{\lambda}_t$ with a SGD step}{
$\bm{\lambda}_t\leftarrow$Update $\bm{\lambda}_{t-1}$ on $B_t$ for fixed $\bm{\theta}_t$
with oracle-based or regularized algorithm with $T_\lambda$ iterations.
}
  }
 \caption{Adversarial Training for Mixtures}
 
 \label{algo:heuristic}
\end{algorithm}


\section{Experiments}

\subsection{Synthetic Dataset}


To illustrate our theoretical findings, we start by testing our learning algorithm on the following synthetic two-dimensional problem. Let us consider the distribution $\PP$ defined as  $\PP\left(Y =\pm 1\right)=1/2$, $\PP\left(X\mid Y=-1\right) = \mathcal{N}(0,I_2)$ and $\PP\left(X \mid Y=1\right) = \frac12\left[\mathcal{N}((-3,0),I_2)+\mathcal{N}((3,0),I_2) \right]$.
% \begin{align*}
%   \left\{
%     \begin{array}{ll}
%   \\
%         \PP\left(X\mid Y=-1\right) = \mathcal{N}(0,I_2) \\
%         \PP\left(X \mid Y=1\right) = \frac12\left[\mathcal{N}((-3,0),I_2)+\mathcal{N}((3,0),I_2) \right].
%     \end{array}
% \right. 
% \end{align*}
We sample $1000$ training points from this distribution and randomly generate $10$ linear classifiers that achieves a standard training risk lower than $0.4$. To simulate an adversary with budget $\varepsilon$ in $\ell_2$ norm, we proceed as follows. For every sample $(x,y)\sim \PP$ we generate $1000$ points uniformly at random in the ball of radius $\varepsilon$ and select the one maximizing the risk for the $0/1$ loss. Figure~\ref{fig:toy_example} (left) illustrates the type of mixture we get after convergence of our algorithms. Note that in this toy problem, we are likely to find the optimal adversary with this sampling strategy if we sample enough attack points. 
% \begin{figure}[ht]
%     \centering
%     \includegraphics[width=0.46\textwidth]{Images/convergence_toy.pdf}
%     \caption{Caption}
%     \label{fig:toy_example_cvgence}
% \end{figure}
\begin{figure*}[!ht]
\begin{center}

\textbf{Adversarial Training, CIFAR-10 dataset results}
 \begin{small}
\begin{tabular}{c|c|ccc} 
\textbf{ Models} & \textbf{Acc. }&\textbf{$\textrm{APGD}_\textrm{CE}$}& \textbf{$\textrm{APGD}_\textrm{DLR}$} & \textbf{Rob. Acc.} \\ \hline
 1 & $81.9\%$ &	$47.6\%$ & $47.7\%$ & $45.6\%$ \\ 
 2 & $81.9\%$ & $49.0\%$ & ${49.6\%}$ & ${47.0\%}$\\ 
  3 & ${81.7\%}$& ${49.0\%}$ & $49.3\%$ & ${46.9\%}$\\
    4 & $\bm{82.6\%}$& $\bm{49.7\%}$ & $\bm{49.8}\%$ & $\bm{47.2\%}$\\

\end{tabular}
\end{small}\\
\includegraphics[width=0.35\textwidth]{Images/robust_acc_finalrun_ResNet18_1024_200_0.001.pdf}\includegraphics[width=0.35\textwidth]{Images/standard_acc_finalrun_ResNet18_1024_200_0.001.pdf} 
  

\textbf{TRADES, CIFAR-10 dataset results}

 \begin{small}
\begin{tabular}{c|c|ccc} 
\textbf{ Models} & \textbf{Acc. }&\textbf{$\textrm{APGD}_\textrm{CE}$}& \textbf{$\textrm{APGD}_\textrm{DLR}$} & \textbf{Rob. Acc.} \\ \hline
 1 &  $79.6\%$ &$50.9\%$& $48.9\%$ &$48.3\%$ \\ 
 2 & $80.3\%$& $52.3\%$ &$51.2\%$ &$50.2\%$\\ 
  3 & $80.7\%$& $52.8\%$ &$51.7\%$ &$50.7\%$\\
    4 & \bm{$80.9\%$} & \bm{$53.0\%$}& \bm{$51.8\%$}& \bm{$50.8\%$}\\

\end{tabular}
\end{small}

\includegraphics[width=0.35\textwidth]{Images/robust_acc_CIFAR10_final_cam_ready_bisss_ResNet18_1024_200_0.001.pdf}\includegraphics[width=0.35\textwidth]{Images/standard_acc_CIFAR10_final_cam_ready_bisss_ResNet18_1024_200_0.001.pdf} 
  

\textbf{Adversarial Training, CIFAR-100 dataset results}
 \begin{small}
\begin{tabular}{c|c|ccc} 
\textbf{ Models} & \textbf{Acc. }&\textbf{$\textrm{APGD}_\textrm{CE}$}& \textbf{$\textrm{APGD}_\textrm{DLR}$} & \textbf{Rob. Acc.} \\ \hline
 1 & $55.2\%$& $24.1\%$& $23.8\%$ & $22.5\%$\\ 
 2 & $55.2\%$ & $25.3\%$ &$26.1\%$ &$23.6\%$\\ 
  3 & \bm{$55.4\%$} & $25.7\%$ &$26.8\%$ &$24.2\%$\\
    4 & $55.3\%$ & \bm{$26.0\%$} & \bm{$27.5\%$}& \bm{$24.5\%$}\\

\end{tabular}
\end{small}
\includegraphics[width=0.35\textwidth]{Images/robust_acc_CIFAR100_finalrun_ResNet18_1024_200_0.001.pdf}\includegraphics[width=0.35\textwidth]{Images/standard_acc_CIFAR100_finalrun_ResNet18_1024_200_0.001.pdf} 


 \caption{Upper plots: Adversarial Training, CIFAR-10 dataset results. Middle plots:  TRADES, CIFAR-10 dataset results. Bottom plots: CIFAR-100 dataset results. {On left}: Comparison of our algorithm with a standard adversarial training (one model). We reported the results for the model with the best robust accuracy obtained over two independent runs because adversarial training might be unstable. Standard and Robust accuracy (respectively in the middle and on right) on CIFAR-10 test images in function of the number of epochs per classifier with $1$ to $3$ ResNet18 models. The performed attack is PGD with $20$ iterations and $\varepsilon=8/255$.}
\label{fig:results_cifar}

\end{center}
\end{figure*}

To evaluate the convergence of our algorithms, we compute the adversarial risk of our mixture for each iteration of both the oracle and regularized algorithms. Figure~\ref{fig:toy_example} illustrates the convergence of the algorithms w.r.t the regularization parameter. We observe that the risk for both algorithms converge. Moreover, they converge towards the oracle minimizer when the regularization parameter $\alpha$ goes to $0$.

Finally, to demonstrate the improvement randomized techniques offer against deterministic defenses, we plot in Figure~\ref{fig:toy_example} (right) the minimum adversarial risk for both randomized and deterministic classifiers w.r.t. $\varepsilon$. The adversarial risk is strictly better for randomized classifier whenever the adversarial budget $\varepsilon$ is bigger than $2$. This illustration validates our analysis of Theorem~\ref{thm:duality-rand}, and motivates a in depth study of a more challenging framework, namely image classification with neural networks.

% \begin{figure}[ht]
%     \centering
%     \includegraphics[width=0.46\textwidth]{Images/risk_toy.pdf}
%     \caption{Caption}
%     \label{fig:toy_example_risk}
% \end{figure}

\subsection{CIFAR Datasets}

% \begin{figure*}[!ht]
% \begin{center}

% \vskip 0.15in
%  \begin{minipage}[ht!]{0.39\textwidth}
%  \begin{scriptsize}
% \begin{tabular}{c|c|ccc} 
% \textbf{ Models} & \textbf{Acc. }&\textbf{$\textrm{APGD}_\textrm{CE}$}& \textbf{$\textrm{APGD}_\textrm{DLR}$} & \textbf{Rob. Acc.} \\ \hline
%  1 & $81.9\%$ &	$47.6\%$ & $47.7\%$ & $45.6\%$ \\ 
%  2 & $81.9\%$ & $49.0\%$ & ${49.6\%}$ & ${47.0\%}$\\ 
%   3 & ${81.7\%}$& ${49.0\%}$ & $49.3\%$ & ${46.9\%}$\\
%     4 & $\bm{82.6\%}$& $\bm{49.7\%}$ & $\bm{49.8}\%$ & $\bm{47.2\%}$\\

% \end{tabular}
% \end{scriptsize}
%   \end{minipage}\begin{minipage}[!ht]{0.61\textwidth}
% \includegraphics[width=0.49\textwidth]{Images/robust_acc_finalrun_ResNet18_1024_200_0.001.pdf}\includegraphics[width=0.49\textwidth]{Images/standard_acc_finalrun_ResNet18_1024_200_0.001.pdf} 
%   \end{minipage}
  
% Adversarial Training, CIFAR-10 dataset results

% \vskip 0.15in
%  \begin{minipage}[ht!]{0.39\textwidth}
%  \begin{scriptsize}
% \begin{tabular}{c|c|ccc} 
% \textbf{ Models} & \textbf{Acc. }&\textbf{$\textrm{APGD}_\textrm{CE}$}& \textbf{$\textrm{APGD}_\textrm{DLR}$} & \textbf{Rob. Acc.} \\ \hline
%  1 &  $79.6\%$ &$50.9\%$& $48.9\%$ &$48.3\%$ \\ 
%  2 & $80.3\%$& $52.3\%$ &$51.2\%$ &$50.2\%$\\ 
%   3 & $80.7\%$& $52.8\%$ &$51.7\%$ &$50.7\%$\\
%     4 & \bm{$80.9\%$} & \bm{$53.0\%$}& \bm{$51.8\%$}& \bm{$50.8\%$}\\

% \end{tabular}
% \end{scriptsize}
%   \end{minipage}\begin{minipage}[!ht]{0.61\textwidth}
% \includegraphics[width=0.49\textwidth]{Images/robust_acc_CIFAR10_final_cam_ready_bisss_ResNet18_1024_200_0.001.pdf}\includegraphics[width=0.49\textwidth]{Images/standard_acc_CIFAR10_final_cam_ready_bisss_ResNet18_1024_200_0.001.pdf} 
%   \end{minipage}
  
% TRADES, CIFAR-10 dataset results
% \vskip 0.15in
%  \begin{minipage}[ht!]{0.39\textwidth}
%  \begin{scriptsize}
% \begin{tabular}{c|c|ccc} 
% \textbf{ Models} & \textbf{Acc. }&\textbf{$\textrm{APGD}_\textrm{CE}$}& \textbf{$\textrm{APGD}_\textrm{DLR}$} & \textbf{Rob. Acc.} \\ \hline
%  1 & $55.2\%$& $24.1\%$& $23.8\%$ & $22.5\%$\\ 
%  2 & $55.2\%$ & $25.3\%$ &$26.1\%$ &$23.6\%$\\ 
%   3 & \bm{$55.4\%$} & $25.7\%$ &$26.8\%$ &$24.2\%$\\
%     4 & $55.3\%$ & \bm{$26.0\%$} & \bm{$27.5\%$}& \bm{$24.5\%$}\\

% \end{tabular}
% \end{scriptsize}
%   \end{minipage}\begin{minipage}[!ht]{0.61\textwidth}
% \includegraphics[width=0.49\textwidth]{Images/robust_acc_CIFAR100_finalrun_ResNet18_1024_200_0.001.pdf}\includegraphics[width=0.49\textwidth]{Images/standard_acc_CIFAR100_finalrun_ResNet18_1024_200_0.001.pdf} 
%   \end{minipage}
%   Adversarial Training, CIFAR-100 dataset results

%  \caption{Upper plots: Adversarial Training, CIFAR-10 dataset results. Middle plots:  TRADES, CIFAR-10 dataset results. Bottom plots: CIFAR-100 dataset results. {On left}: Comparison of our algorithm with a standard adversarial training (one model). We reported the results for the model with the best robust accuracy obtained over two independent runs because adversarial training might be unstable. Standard and Robust accuracy (respectively in the middle and on right) on CIFAR-10 test images in function of the number of epochs per classifier with $1$ to $3$ ResNet18 models. The performed attack is PGD with $20$ iterations and $\varepsilon=8/255$.}
% \label{fig:results_cifar}

% \end{center}
% \end{figure*}

% Adversarial examples are known to be easily transferrable from one model to another~\cite{tramer2017space}. To counter this and support our theoretical claims, we propose an heuristic algorithm (see Algorithm~\ref{algo:heuristic}) to train a robust mixture of $L$ classifiers. We alternatively train these classifiers with adversarial examples against the current mixture and update the probabilities of the mixture according to the algorithms we proposed in Section~\ref{sec:proposed-algo}. More details on the heuristic algorithm are available in Appendix~\ref{sec:additional-xp}. 
% \begin{algorithm}[h!]
% \small
% \SetAlgoLined
% $L$: number of models, $T$: number of iterations,\\
% $T_\theta$: number of updates for the models $\bm{\theta}$,\\
% $T_\lambda$: number of updates for the mixture $\bm{\lambda}$,\\ $\bm{\lambda}_0=(\lambda_0^1,\dots\lambda_0^L),~\bm{\theta}_0=(\theta_0^1,\dots\theta_0^L)$\\
%  \For{$t=1,\dots,T$}{
%  Let $B_t$ be a batch of data.\\
% \eIf{$t \mod (T_\theta L+1)\neq 0$}{
% $k$ sampled uniformly in $\{1,\dots,L\}$\\
% $\Tilde{B}_t\leftarrow$ Attack of images in $B_t$ for the  model $(\bm{\lambda}_t,\bm{\theta}_t)$\\
% $\theta^t_k\leftarrow$ Update $\theta^{t-1}_k$ with $\Tilde{B}_t$ for fixed $\bm{\lambda}_t$ with a SGD step}{
% $\bm{\lambda}_t\leftarrow$Update $\bm{\lambda}_{t-1}$ on $B_t$ for fixed $\bm{\theta}_t$
% with oracle or regularized algorithm with $T_\lambda$ iterations.
% }
%   }
%  \caption{Adversarial Training for Mixtures}
 
%  \label{algo:heuristic}
% \end{algorithm}

\paragraph{Experimental Setup.} We now implement our heuristic algorithm (Alg.~\ref{algo:heuristic}) on CIFAR-10 and CIFAR-100 datasets for both Adversarial Traning~\citep{madry2018towards} and TRADES~\citep{zhang2019theoretically} loss. To evaluate the performance of Algorithm~\ref{algo:heuristic}, we trained from $1$ to $4$ ResNet18~\citep{he2016deep} models on $200$ epochs per model\footnote{$L\times200$ epochs in total, where $L$ is the number of models.}. We study the robustness with regards to $\ell_\infty$ norm and fixed adversarial budget $\varepsilon=8/255$. The attack we used in the inner maximization of the training is an adapted (adaptative) version of PGD for mixtures of classifiers with $10$ steps. Note that for one single model, Algorithm~\ref{algo:heuristic} exactly corresponds to adversarial training~\citep{madry2018towards} or TRADES. For each of our setups, we made two independent runs and select the  best one. The training time of our algorithm is around four times longer than a standard Adversarial Training (with PGD 10 iter.) with two models, eight times with three models and twelve times with four models. We trained our models with a batch of size  $1024$ on $8$ Nvidia V100 GPUs. 

\paragraph{Optimizer.} For each of our models, The optimizer we used in all our implementations is SGD with learning rate set to $0.4$ at epoch $0$ and is divided by $10$ at half training then by $10$ at the three quarters of training. The momentum is set to $0.9$ and the weight decay to $5\times10^{-4}$. The batch size is set to $1024$. 
\paragraph{Adaptation of Attacks.} Since our classifier is randomized, we need to adapt the attack accordingly. To do so we used the expected loss:
\begin{align*}
\Tilde{\loss}\left((\bm{\lambda},\bm{\theta}),(x,y)\right) = \sum_{k=1}^L \lambda_k \loss(\theta_k,(x,y))
\end{align*}
to compute the gradient in the attacks, regardless the loss (DLR or cross-entropy). For the inner maximization at training time, we used a PGD attack on the cross-entropy loss with $\varepsilon=0.03$. For the final evaluation, we used the untargeted $DLR$ attack with default parameters.
\paragraph{Regularization in Practice.} The entropic regularization in higher dimensional setting need to be adapted to be more likely to find adversaries. To do so, we computed PGD attacks with only $3$ iterations with $5$ different restarts instead of sampling uniformly $5$ points  in the $\ell_\infty$-ball. In our experiments in the main paper, we use a regularization parameter $\alpha=0.001$. The learning rate for the minimization on $\bm{\lambda}$ is always fixed to $0.001$. 
\paragraph{Alternate Minimization Parameters.} Algorithm~\ref{algo:heuristic} implies an alternate minimization algorithm. We set the number of updates of $\bm{\theta}$ to $T_\theta = 50$ and, the update of $\bm{\lambda}$ to $T_\lambda = 25$. 

\subsection{Effect of the Regularization}
In this subsection, we experimentally investigate the effect of the regularization. In Figure~\ref{fig:xp-regularization}, we notice, that the regularization has the effect of stabilizing, reducing the variance and improving the level of the robust accuracy for adversarial training for mixtures (Algorithm~\ref{algo:heuristic}). The standard accuracy curves are very similar in both cases.



\paragraph{Evaluation Protocol.} At each epoch, we evaluate the current mixture on test data against PGD attack  with $20$ iterations. To select our model and avoid overfitting~\citep{rice2020overfitting}, we kept the most robust against this PGD attack.
To make a final evaluation of our mixture of models, we used an adapted version of $\textrm{AutoPGD}$ untargeted attacks~\citep{croce2020reliable} for randomized classifiers with both Cross-Entropy (CE) and Difference of Logits Ratio (DLR) loss. For both attacks, we made $100$ iterations and $5$ restarts.

\paragraph{Results.} The results are presented in Figure~\ref{fig:results_cifar}. We remark our algorithm outperforms a standard adversarial training in all the cases by more $1\%$ on CIFAR-10 and CIFAR-100, without additional loss of standard accuracy as it is attested by the left figures. On TRADES, the gain is even more important by more than $2\%$ in robust accuracy. Moreover, it seems our algorithm, by adding more and more models, reduces the overfitting of adversarial training. It also appears that robustness increases as the number of models increases. So far, experiments are computationally very costful and it is difficult to raise precise conclusions. Further, hyperparameter tuning ~\citep{gowal2020uncovering} such as architecture, unlabeled data~\citep{carmon2019unlabeled} or activation function may still increase the results.





% \begin{figure*}[!ht]
% \begin{center}


% \label{table:results}
% \vskip 0.15in
%  \begin{minipage}[!ht]{0.39\textwidth}
%  \begin{scriptsize}
% \begin{tabular}{c|c|ccc} 
% \textbf{ Models} & \textbf{Acc. }&\textbf{$\textrm{APGD}_\textrm{CE}$}& \textbf{$\textrm{APGD}_\textrm{DLR}$} & \textbf{Rob. Acc.} \\ \hline
%  1 & $81.9\%$ &	$47.6\%$ & $47.7\%$ & $45.6\%$ \\ 
%  2 & $81.9\%$ & $49.0\%$ & ${49.6\%}$ & ${47.0\%}$\\ 
%   3 & ${81.7\%}$& ${49.0\%}$ & $49.3\%$ & ${46.9\%}$\\
%     4 & $\bm{82.6\%}$& $\bm{49.7\%}$ & $\bm{49.8}\%$ & $\bm{47.2\%}$\\

% \end{tabular}
% \end{scriptsize}
%   \end{minipage}\begin{minipage}[!ht]{0.61\textwidth}
% \includegraphics[width=0.49\textwidth]{Images/robust_acc_finalrun_ResNet18_1024_200_0.001.pdf}\includegraphics[width=0.49\textwidth]{Images/standard_acc_finalrun_ResNet18_1024_200_0.001.pdf} 
%   \end{minipage}
  
% \caption{On left: Comparison of our algorithm with a standard adversarial training (one model). We reported the results for the model with the best robust accuracy obtained over two independent runs because adversarial training might be unstable. Standard and Robust accuracy ( respectively in the center and on left) on CIFAR-10 test images in function of the number of epochs per classifier with $1$ to $3$ ResNet18 models. The performed attack is PGD with $20$ iterations and $\varepsilon=8/255$.}
% \label{fig:results_cifar}
% \end{center}
% \end{figure*}











% \begin{figure}[!ht]
% \includegraphics[width=0.46\textwidth]{Images/standard_acc_finalrun_ResNet18_1024_200_0.001.pdf}    \caption{Standard accuracy on CIFAR-10 test images in function of the number of epochs per classifier with $1$ to $3$ ResNet18 models.}
%     \label{fig:plot_standard_acc}
% \end{figure}

% \begin{figure}[!ht]
% \includegraphics[width=0.46\textwidth]{Images/robust_acc_finalrun_ResNet18_1024_200_0.001.pdf}    \caption{Robust accuracy on CIFAR-10 test images in function of the number of epochs per classifier with $1$ to $3$ ResNet18 models. The attack performed is PGD with $20$ iterations and $\varepsilon=8/255$.}
%     \label{fig:plot_robust_acc}
% \end{figure}



% mettre ca au propre mais les resultats sont la!!

% setup: ResNet18 ou WRN28x10

% Details in supmat: loss +lr + format of training etc

% In supmat: 
% \begin{enumerate}
%     \item details HP (attack + training)
%     \item details runtime
%     \item additional results (with other attacks maybe + other archis...)
    
    
% \end{enumerate}
% Epochs 100


% \subsection{Useful Lemmas}
% \begin{lemma}[Fubini's theorem]
% \label{lem:fubini}
% Let $l:\Theta\times(\mathcal{X}\times\mathcal{Y})\rightarrow [0,\infty)$ satisfying Assumption~\ref{ass:loss}. Then for all $\mu\in\mathcal{M}^1_+(\Theta)$, $\int l(\theta,\cdot)d\mu(\theta)$ is Borel measurable; for  $\QQ\in\mathcal{M}^1_+(\mathcal{X}\times\mathcal{Y})$, $\int l(\cdot,(x,y))d\QQ(x,y)$ is Borel measurable. Moreover: $\int l(\theta,(x,y))d\mu(\theta)d\QQ(x,y)=\int l(\theta,(x,y))d\QQ(x,y)d\mu(\theta)$
% \end{lemma}

% \begin{lemma}
% \label{lem:usc1}
% Let $l:\Theta\times(\mathcal{X}\times\mathcal{Y})\rightarrow [0,\infty)$ satisfying Assumption~\ref{ass:loss}.
% Then for all $\mu\in\mathcal{M}^1_+(\Theta)$, $(x,y)\mapsto\int l(\theta,(x,y))d\mu(\theta)$ is upper semi-continuous and hence Borel measurable.  
% \end{lemma}
% \begin{proof}
% Let $(x_n,y_n)_n$ be a sequence of $\mathcal{X}\times\mathcal{Y}$ converging to $(x,y)\in\mathcal{X}\times\mathcal{Y}$.  For all $\theta\in\Theta$, $M-l(\theta,\cdot)$ is non negative and lower semi-continuous. Then by Fatou's Lemma applied:
% \begin{align*}
%   \int M-l(\theta,(x,y))d\mu(\theta)&\leq\int \liminf_{n\to\infty}  M-l(\theta,(x_n,y_n))d\mu(\theta)\\
%   &\leq  \liminf_{n\to\infty}  \int M-l(\theta,(x_n,y_n))d\mu(\theta) 
% \end{align*}

% Then we deduce that: $\int M- l(\theta,\cdot)d\mu(\theta)$ is lower semi-continuous and then $\int l(\theta,\cdot)d\mu(\theta)$ is upper-semi continuous.
% \end{proof}


% \begin{lemma}
% \label{lem:usc2}

% Let $l:\Theta\times(\mathcal{X}\times\mathcal{Y})\rightarrow [0,\infty)$ satisfying Assumption~\ref{ass:loss}
% Then for all $\mu\in\mathcal{M}^1_+(\Theta)$, $\QQ\mapsto\int l(\theta,(x,y))d\mu(\theta)d\QQ(x,y)$ is upper semi-continuous for weak topology of measures. 
% \end{lemma}
% \begin{proof}
%  $-\int l(\theta,\cdot)d\mu(\theta) $ is lower semi-continuous from Lemma~\ref{lem:usc1}. Then $M-\int l(\theta,\cdot)d\mu(\theta) $ is lower semi-continuous and non negative. Let denote $v$ this function. Let $(v_n)_n$ be a non-decreasing sequence of continuous bounded functions such that $v_n\to v$. Let $(\QQ_k)_k$ converging weakly towards $\QQ$. Then by monotone convergence:
 
%  \begin{align*}
%      \int vd\QQ = \lim_n \int v_nd\QQ =\lim_n \lim_k\int v_nd\QQ_k\leq \liminf_k \int vd\QQ_k
%  \end{align*}
%  Then $\QQ\mapsto\int vd\QQ$ is lower semi-continuous and then $\QQ\mapsto\int l(\theta,(x,y))d\mu(\theta)d\QQ(x,y)$ is upper semi-continuous for weak topology of measures. 
%  \end{proof}



% \begin{lemma}
% \label{lem:measure-sup}
% Let $l:\Theta\times(\mathcal{X}\times\mathcal{Y})\rightarrow [0,\infty)$ satisfying Assumption~\ref{ass:loss}.
% Then for all $\mu\in\mathcal{M}^1_+(\Theta)$, $(x,y)\mapsto \sup_{(x',y'),d(x,x')\leq\varepsilon,y=y'} \int l(\theta,(x',y'))d\mu(\theta)$ is universally measurable (i.e. measurable for all Borel probability measures). And hence the adversarial risk is well defined. 
% \end{lemma}
% \begin{proof}
% Let $\phi :(x,y)\mapsto \sup_{(x',y'),d(x,x')\leq\varepsilon,y=y'} \int l(\theta,(x',y'))d\mu(\theta)$. Then for $u\in\bar{\mathbb{R}}$:
% \begin{align*}
% \left\{\phi(x,y)>u\right\}=\text{Proj}_1\left\{((x,y),(x',y'))\mid\int l(\theta,(x',y'))d\mu(\theta)-c_\varepsilon((x,y),(x',y'))>u\right\}
% \end{align*}
% By Lemma~\ref{lem:usc2}: $((x,y),(x',y'))\mapsto \int l(\theta,(x',y'))d\mu(\theta)-c_\varepsilon((x,y),(x',y'))$ is upper-semicontinuous hence Borel measurable. So its level sets are Borel sets, and by~\citep[Proposition 7.39]{bertsekas2004stochastic}, the projection of a Borel set is analytic. And then $\left\{\phi(x,y)>u\right\}$ universally measurable thanks to~\citep[Corollary 7.42.1]{bertsekas2004stochastic}. We deduce that $\phi$ is universally measurable.
% \end{proof}



% \paragraph{Optimizer.} For each of our models, The optimizer we used in all our implementations is SGD with learning rate set to $0.4$ at epoch $0$ and is divided by $10$ at half training then by $10$ at the three quarters of training. The momentum is set to $0.9$ and the weight decay to $5\times10^{-4}$. The batch size is set to $1024$. 
% \paragraph{Adaptation of Attacks.} Since our classifier is randomized, we need to adapt the attack accordingly. To do so we used the expected loss:
% \begin{align*}
% \Tilde{l}\left((\bm{\lambda},\bm{\theta}),(x,y)\right) = \sum_{k=1}^L \lambda_k l(\theta_k,(x,y))
% \end{align*}
% to compute the gradient in the attacks, regardless the loss (DLR or cross-entropy). For the inner maximization at training time, we used a PGD attack on the cross-entropy loss with $\varepsilon=0.03$. For the final evaluation, we used the untargeted $DLR$ attack with default parameters.
% \paragraph{Regularization in Practice.} The entropic regularization in higher dimensional setting need to be adapted to be more likely to find adversaries. To do so, we computed PGD attacks with only $3$ iterations with $5$ different restarts instead of sampling uniformly $5$ points  in the $\ell_\infty$-ball. In our experiments in the main paper, we use a regularization parameter $\alpha=0.001$. The learning rate for the minimization on $\bm{\lambda}$ is always fixed to $0.001$. 
% \paragraph{Alternate Minimization Parameters.} Algorithm~\ref{algo:heuristic} implies an alternate minimization algorithm. We set the number of updates of $\bm{\theta}$ to $T_\theta = 50$ and, the update of $\bm{\lambda}$ to $T_\lambda = 25$. 

% \subsection{Effect of the Regularization}
% In this subsection, we experimentally investigate the effect of the regularization. In Figure~\ref{fig:xp-regularization}, we notice, that the regularization has the effect of stabilizing, reducing the variance and improving the level of the robust accuracy for adversarial training for mixtures (Algorithm~\ref{algo:heuristic}). The standard accuracy curves are very similar in both cases.


\begin{figure}[ht]
    \centering
    \includegraphics[width=0.40\textwidth]{Images/standard_acc_finalrun_ResNet18_1024_200_-1_bis.pdf}    \includegraphics[width=0.40\textwidth]{Images/standard_acc_finalrun_ResNet18_1024_200_0.001_bis.pdf}\\

    \includegraphics[width=0.40\textwidth]{Images/robust_acc_finalrun_ResNet18_1024_200_-1_bis.pdf}    \includegraphics[width=0.40\textwidth]{Images/robust_acc_finalrun_ResNet18_1024_200_0.001_bis.pdf}
    \caption{On top: Standard accuracies over epochs with respectively no regularization and regularization set to $\alpha=0.001$. On bottom: Robust accuracies for the same parameters against PGD attack with $20$ iterations and $\varepsilon=0.03$.}
    \label{fig:xp-regularization}
\end{figure}

\subsection{Additional Experiments on WideResNet28x10}

We now evaluate our algorithm on WideResNet28x10~\cite{ZagoruykoK16} architecture. Due to computation costs, we limit ourselves to $1$ and $2$ models, with regularization parameter set to $0.001$ as in the paper experiments section. Results are reported in Figure~\ref{fig:xp-wideresnet}. We remark this architecture can lead to more robust models, corroborating the results from~\cite{gowal2020uncovering}.
\begin{figure*}[!ht]
\begin{center}

\begin{small}
\begin{tabular}{c|c|ccc} 
\textbf{ Models} & \textbf{Acc. }&\textbf{$\textrm{APGD}_\textrm{CE}$}& \textbf{$\textrm{APGD}_\textrm{DLR}$} & \textbf{Rob. Acc.} \\ \hline
 1 & $85.2\%$ &	$49.9\%$ & $50.2\%$ & $48.5\%$ \\ 
 2 & $\bm{86.0\%}$ & $\bm{51.5\%}$ & $\bm{52.1\%}$ & $\bm{49.6\%}$\\ 

\end{tabular}
\end{small}

\includegraphics[width=0.35\textwidth]{Images/robust_acc_finalrun_WideResNet28x10_1024_200_0.001.pdf}\includegraphics[width=0.35\textwidth]{Images/standard_acc_finalrun_WideResNet28x10_1024_200_0.001.pdf} 
  
\caption{Comparison of our algorithm with a standard adversarial training (one model) on WideResNet28x10. We reported the results for the model with the best robust accuracy obtained over two independent runs because adversarial training might be unstable. Standard and Robust accuracy (respectively in the middle and on right) on CIFAR-10 test images in function of the number of epochs per classifier with $1$ and $2$ WideResNet28x10 models. The performed attack is PGD with $20$ iterations and $\varepsilon=8/255$.}
\label{fig:xp-wideresnet}
\end{center}
\end{figure*}


\subsection{Overfitting in Adversarial Robustness}
We further investigate the overfitting of our heuristic algorithm. We plotted in Figure~\ref{fig:overfitting} the robust accuracy on ResNet18 with $1$ to $5$ models. The most robust mixture of $5$ models against PGD with $20$ iterations arrives at epoch $198$, \emph{i.e.} at the end of the training, contrary to $1$ to $4$ models, where the most robust mixture occurs around epoch $101$. However, the accuracy against AGPD with 100 iterations in lower than the one at epoch $101$ with global robust accuracy of $47.6\%$ at epoch $101$ and $45.3\%$ at epoch 198. This strange phenomenon would suggest that the more powerful the attacks are, the more the models are subject to overfitting. We leave this question to further works.


\begin{figure*}[!ht]
\begin{center}
\includegraphics[width=0.49\textwidth]{Images/5_standard_acc_finalrun_ResNet18_1024_200_0.001.pdf}\includegraphics[width=0.49\textwidth]{Images/5_robust_acc_finalrun_ResNet18_1024_200_0.001.pdf} 

\caption{Standard and Robust accuracy (respectively on  left and on right) on CIFAR-10 test images in function of the number of epochs per classifier with $1$ to $5$ ResNet18 models. The performed attack is PGD with $20$ iterations and $\varepsilon=8/255$. The best mixture for $5$ models occurs at the end of training (epoch $198$).}
\label{fig:overfitting}
\end{center}
\end{figure*}
% \subsection{Additional Results}
% \label{sec:complements}
% \subsection{Equality of Standard Randomized and Deterministic Minimal Risks}
% \begin{prop}
% Let $\PP$ be a Borel probability distribution on $\mathcal{X}\times\mathcal{Y}$, and $l$ a loss satisfying Assumption~\ref{ass:loss}, then:
% \begin{align*}
%         \inf_{\mu\in\mathcal{M}^1_+(\Theta)} \risk(\mu) =\inf_{\theta\in\Theta} \risk(\theta)
% \end{align*}
% \end{prop}
% \begin{proof}
% It is clear that:         $\inf_{\mu\in\mathcal{M}^1_+(\Theta)} \risk(\mu) \leq \inf_{\theta\in\Theta} \risk(\theta)$. Now, let $\mu\in\mathcal{M}^1_+(\Theta)$, then:
% \begin{align*}
%     \risk(\mu)= \mathbb{E}_{\theta\sim\mu}(\risk(\theta))&\geq \essinf_\mu \mathbb{E}_{\theta\sim\mu} \left(\risk(\theta)\right)\\
%     &\geq\inf_{\theta\in\Theta} \risk(\theta).
% \end{align*}
% where $\essinf$ denotes the essential infimum.
% \end{proof}
% We can deduce an immediate corollary. 
% \begin{corollary}
% Under Assumption~\ref{ass:loss}, the dual for randomized and deterministic classifiers are equal.
% \end{corollary}

% \subsection{Decomposition of the Empirical Risk for Entropic Regularization}

% \begin{prop}
% Let $\hat{\PP}:=\frac1N\sum_{i=1}^N \delta_{(x_i,y_i)}$. Let $l$ be a loss satisfying Assumption~\ref{ass:loss}. Then we have:
% \begin{align*}
% \frac{1}{N}\sum_{i=1}^N\sup_{x,~d(x,x_i)\leq\varepsilon}\mathbb{E}_{\theta \sim \mu}\left[l(\theta,(x,y))\right]=\sum_{i=1}^N\sup_{\QQ_i\in\Gamma_{i,\varepsilon}}\mathbb{E}_{(x,y)\sim \QQ_i,\theta \sim \mu}\left[l(\theta,(x,y))\right]
% \end{align*}
% where $\Gamma_{i,\varepsilon}$ is defined as : 
% \begin{align*}
%     \Gamma_{i,\varepsilon}:=\Big\{\QQ_i\mid~\int d\QQ_i=\frac{1}{N},~\int c_{\varepsilon}((x_i,y_i),\cdot) d\QQ_i=0\Big\}.
% \end{align*}\end{prop}

% \begin{proof}
% This proposition is a direct application of Proposition~\ref{prop:dro_adv} for diracs $\delta_{(x_i,y_i)}$.
% \end{proof}

% % \subsection{On the NP-Hardness of Attacking a Mixture of Classifiers}
% % In general, the problem of finding a best response to a mixture of classifiers is in general NP-hard. Let us justify it on a mixture of linear classifiers in binary classification: $f_{\theta_k}(x) = \langle \theta,x\rangle$ for $k\in [L]$ and $\bm{\lambda}=\mathbf{1}_L/L$. Let us consider the $\ell_2$ norm and $x=0$ and $y=1$. Then the problem of attacking $x$ is the following:
% % \begin{align*}
% %     \sup_{\tau,~\lVert \tau\rVert\leq\varepsilon} \frac{1}{L}\sum_{k=1}^L\mathbf{1}_{\langle \theta_k,x+\tau\rangle\leq0}
% % \end{align*}
% % This problem is equivalent to a linear binary classification problem on $\tau$, which is known to be NP-hard.




% \subsection{Case of Separated Conditional Distribtions}
% \begin{prop} Let $\mathcal{Y} = \{-1,+1\}$. Let $\PP\in\mathcal{M}^1_+(\mathcal{X}\times\mathcal{Y})$. Let $\varepsilon>0$. For $i\in\mathcal{Y}$, let us denote $\PP_i$ the distribution of $\PP$ conditionally to $y=i$. Let us assume that  $d_\mathcal{X}(\supp(\PP_{1+1}),\supp(\PP_{-1}))>2\varepsilon$. Let us consider the nearest neighbor deterministic classifier : $f(x) =  d(x,\supp(\PP_{+1}))-d(x,\supp(\PP_{-1}))$ and the $0/1$ loss $l(f,(x,y))=\mathbf{1}_{yf(x)\leq 0}$. Then $f$ satisfies both optimal standard and adversarial risks: $\risk(f)=0$ and $\riskadv^\varepsilon(f)=0$.
% \end{prop}

% \begin{proof} Let 
% Let denote $p_i =\PP(y=i)$. Then we have
% \begin{align*}
%  \riskadv^\varepsilon(f)= p_{+1}\mathbb{E}_{\PP_{+1}}\left[\sup_{x',~d(x,x')\leq \varepsilon}\mathbf{1}_{ f(x')\leq 0}\right]+p_{-1}\mathbb{E}_{\PP_{-1}}\left[\sup_{x',~d(x,x')\leq \varepsilon}\mathbf{1}_{ f(x')\geq 0}\right]
% \end{align*}
% For $x\in \supp(\PP_{+1})$, we have, for all $x'$ such that $d(x,x')\neq 0$, $f(x')>0$, then: $\mathbb{E}_{\PP_{+1}}\left[\sup_{x',~d(x,x')\leq \varepsilon}\mathbf{1}_{ f(x')\leq 0}\right]=0$. Similarly, we have $\mathbb{E}_{\PP_{-1}}\left[\sup_{x',~d(x,x')\leq \varepsilon}\mathbf{1}_{ f(x')\geq 0}\right]=0$. We then deduce the result.
% \end{proof}

\chapter{Consistency Study of Adversarial loss}
\label{chap:calibration}

\minitoc
\textbf{Disclaimer: This section is still an unachieved work}

In this section we assume a binary classification task, i.e. an output space $\mathcal{Y}=\{-1,+1\}$. W extend the notion of loss functions to general cost functions.  A cost function is a function $\loss:\mathcal{X}\times\mathcal{Y}\times \mathcal{F}(\mathcal{X})\to \mathbb{R}$ such that $\loss(\cdot,\cdot,f)$ is universally measurable for all $f\in\mathcal{F}(\mathcal{X})$. We will denote $\mathcal{R}_{\loss}$ the risk associated with a loss $\loss$. 


\section{Loss Consistency in Classification}

\subsection{Consistency in Standard Classification}
In a standard classification setting,  given $(x,y)$, we recall  the classification loss is defined as:
\begin{align*}
    \loss_{0/1}(x,y,f) = \mathbf{1}_{y\sign(f(x))\leq 0}
\end{align*}
In the literature the notion of consistency with regards to the $0/1$ loss is defined as follows.
\begin{definition}[Classification Consistency]
A cost function $\loss$ is said to be consistent with respect to $0/1$ loss for a a probability distribution $\PP\in\mathcal{M}_1^+(\mathcal{X}\times\mathcal{Y})$ if and only if for all sequences $(f_n)_n $ of measurable functions:
\begin{align}
    \mathcal{R}_{\loss}(f_n)\to \mathcal{R}_{\loss}^\star\implies\mathcal{R}_{0/1}(f_n)\to \mathcal{R}_{0/1}^\star
\end{align}
\end{definition}


\paragraph{Standard results in classification.} Previous literature have focused, in standard classification, on margin losses, and it is defined as follows.

\begin{definition}[Margin loss] We say a loss $\loss$ is a margin loss if there exists a measurable function $\phi:\mathbb{R}\to\mathbb{R}_+$, such that for all $x\in\mathcal{X}$, $y\in\mathcal{Y}$, $f:\mathcal{X}\to\mathbb{R}$ a measurable function,
$$\loss(x,y,f)=\phi(yf(x))$$
\end{definition}
Without loss of generality, we will denote $\mathcal{R}_\phi$ the risk associated with a margin loss. Standard classification consistency highly relies on the fact that given a loss, the minimization can be made pointwisely, independently from the considered distribution $\PP$. The notion of consistency exactly matches with the notion of calibration, defined as follows:
\begin{definition}[Classification Calibration] A margin loss $\phi$ is said to be classification-calibrated if for all $\eta\in[0,1]$, $\eta\neq \frac12$:
\begin{align*}
    H(\eta)>H^-(\eta)
\end{align*}
where $H(\eta):=\inf_{\alpha\in\RR}C_\eta(\alpha)$,  $H^-(\eta) = \inf_{\alpha,~\alpha(2\eta-1)\leq0}C_\eta(\alpha)$ and $C_\eta(\alpha):=\eta\phi(\alpha)+(1-\eta)\phi(-\alpha)$.
\end{definition}

\cite{bartlett2006convexity} and \cite{steinwart2007compare} show that consistency and calibration are equivalent notions in standard classification.

\begin{thm} In standard calibration, a margin loss $\phi$ is consistent with regards to classification loss if and only if $\phi$ is classification calibrated. 
\end{thm}

In particular, one can state the following results on convex margin losses.
\begin{thm} Let $\phi$ be a convex margin differentiable in $0$. $\phi$ is consistent with regards to  classification loss if and only if $\phi'(0)<0$.
\end{thm}
\paragraph{What is the good 0/1 loss?}
In this paragraph, we precise why this $0/1$ loss is used in consistency/calibration study. Let $f:\mathcal{X}\to\mathbb{R}$ be a measurable function. In the literature,  we generally find three ways for defining the $0/1$ loss:
\begin{itemize}
    \item $\loss_{\leq}(x,y,f)=\mathbf{1}_{yf(x)\leq 0}$, the loss that penalizes indecision.
    \item $\loss_{<}(x,y,f)=\mathbf{1}_{yf(x)< 0}$, the loss that encourages indecision.
    \item $\loss_{0/1}(x,y,f)=\mathbf{1}_{y\text{sign}(f(x))\leq 0}$, with a sign convention ($sign(0) = 1$ for instance).
\end{itemize}

The original results from~\citep{bartlett2006convexity,steinwart2007compare} tackles the problem of the consistency with respect to the latter one. We can even remark that usual consistency results stated in the two previous papers are false for the two first losses, as shown in the following simple counterexample.

\begin{counterexample*}
Let $\PP$ defined as follows: $\PP(Y=-1)=\PP(Y=1)=\frac12$, and $\PP(X=0\mid Y)=1$. Let $\phi:\mathbb{R}\to\mathbb{R}$ be a margin  loss. The $\phi$-risk minimization problem writes $\inf_{\alpha} \frac{1}{2} (\phi(\alpha)+\phi(-\alpha))$. For a convex margin loss $\phi$ the optimum is attained in $\alpha=0$. 
\begin{enumerate}
    \item $f_n:x\mapsto 0$ is a minimizing sequence for the $\phi$-risk. However $R_{\loss_{\leq}}(f_n)=1$ for all $n$ and $R_{\loss_{\leq}}^*=\frac{1}{2}$. Then we deduce that no convex margin based loss can be calibrated for $\loss_{\leq}$.
    \item $f_n:x\mapsto \frac{1}{n}$ is a minimizing sequence for the $\phi$-risk. However $R_{\loss_{<}}(f_n)=1/2$ for all $n$ and $R_{\loss_{<}}^*=0$. Then we deduce that no convex margin based loss can be calibrated for $\loss_{<}$.
\end{enumerate}

\end{counterexample*}



\subsection{Consistency in adversarial setting}


Consistency in standard classification is studied with respect to the cost $\loss_{0/1}(x,y,f)=\mathbf{1}_{y\text{sign}(f(x))\leq 0}$. In the adversarial setting, the natural loss to consider then is  $$\loss_{0/1,\varepsilon}(x,y,f)=\sup_{x'\in B_\varepsilon(x)}\mathbf{1}_{y\text{sign}(f(x'))\leq 0}$$ 
with a sign convention (e.g. $sign(0)=1$). Otherwise, when $\varepsilon=0$, the consistency results would be wrong as stated in the previous counterexample. Previous works~\citep{pmlr-v125-bao20a,awasthi2021calibration,awasthi2021finer} focused on the loss $\sup_{x'\in B_\varepsilon(x)}\mathbf{1}_{yf(x')\leq 0}$ that consequently lead to misleading results because not compatible with standard classification when $\varepsilon=0$. We define the notion of consistency with regards to adversarial $0/1$ loss as follows.
\begin{definition}[Adversarial Consistency]
A cost function $\loss$ is said to be consistent with respect to $0/1$ loss for a distribution  $\PP\in\mathcal{M}_1^+(\mathcal{X}\times\mathcal{Y})$ if and only if for all sequences $(f_n)_n $ of measurable functions:
\begin{align}
    \mathcal{R}_{\loss}(f_n)\to \mathcal{R}_{\loss}^\star\implies\mathcal{R}_{0/1}^\varepsilon(f_n)\to \mathcal{R}_{0/1}^{\varepsilon,\star}
\end{align}
\end{definition}

\paragraph{Calibration in adversarial classification} One need to note that the notion of calibration in the adversarial setting is misleading.  

TODO.



Hence, one need to focus specifically on the notion of consistency in the adversarial setting. In the next section, we prove some results on adversarial consistency.

\section{Adversarial Consistency Results}

\subsection{Convex Losses are not Consistent}
Following the work on standard classification, natural candidate losses for adversarial consistency would be suprema of standardly calibrared margin losses:

$$\loss(x,y,f)=\sup_{x', d(x,x')\leq\varepsilon}\phi(yf(x))$$
including a wide range of  convex functions $\phi$. Next theorem shows the counter-intuitive result that no calibrated convex loss can define a consistent loss for adversarial classification
\begin{prop}
Let $\phi$ be a convex classification-calibrated margin loss (i.e. $\phi'(0) <0$). In $\RR$, for any $\varepsilon > 0$, there exists a distribution $\PP$ such that $x\to\sup_{x', d(x,x')\leq\varepsilon}\phi(yf(x))$ is not consistent for $\PP$ with regards to $\loss_{0/1,\varepsilon}$.
\end{prop}

\begin{proof}

Let $\epsilon$ > 0. We will construct a distribution $\mathcal{D}$ and a sequence of classifier $h_n$ so that the $\phi$-risk converges toward the optimal, while the $0-1$ loss remains constant at a non-optimal value.

\paragraph{}Let a such that $\epsilon < a < 2 \epsilon$. We define D by :
\[
\left\{ \begin{array}{ll}
\mathbb{P}(Y=1, X=0) = \frac{1}{2} & \\
\mathbb{P}(Y=-1, X=-a) = \frac{1}{4} & \\
\mathbb{P}(Y=-1, X=a) = \frac{1}{4} & \\
\mathbb{P}(X, Y) = 0 & \mbox{otherwise.} \\
\end{array} \right.
\]

Since $\phi$ is classification-calibrated, it is either non-increasing, or it has a minimum, attained on a point $u>0$. Let us consider both cases :

\paragraph{First case : $\phi$ non-increasing}

We define $Z_1 = [-\epsilon, -a + \epsilon]$ and $Z_2 = [a-\epsilon, \epsilon]$, the zones that can be attacked by two points at once.
Let us then define : 
\[
\forall x \in \mathbb{R}, h_n(x) = \left\{ \begin{array}{ll}
\frac{1}{n} & \mbox{if } x \in Z_1 \\
-1/n & \mbox{if } x \in Z_2 \\
1 & \mbox{if }x \in [-a+\epsilon, a-\epsilon] \\
-1 & \mbox{otherwise} \\
\end{array} \right.
\]

\begin{figure}
    \centering
    \includegraphics[scale=0.6]{sections/3_calibration/images/example.png}
    \caption{The distribution we use as a counter-example. For the $0-1$ loss, there are two optimal classifiers, putting either both $Z_1$ and $Z_2$ to $1$ (and saving point A), or both to $-1$ (saving points B and C). With convex surrogate losses however, the optimal is to bring values in both $Z_1$ and $Z_2$ to zero, which can be done while maintaining opposite signs on $Z_1$ and $Z_2$, and so ensuring the loss of both A and one of B and C for the $0-1$ loss. }
    \label{fig:my_label}
\end{figure}


Since the central point and the point $x=-a$ are attackable, but not the point $x=a$, we have $\mathbb{E}_{(x,y)\sim \mathcal{D}}\left[ \sup\limits_{z \in \mathcal{B}(x,\epsilon)} l_{0/1}(y,h_n(z))\right] = \frac{3}{4}$, which is worse than the constant classifier h=1, which gives a score of $\frac{1}{2}$. So $h_n$ is not a minimizing sequence for the $0-1$ loss.

Let us now show that $h_n$ is a minimizing sequence for the loss $\phi$ under attack, which will be proof of the non-consistency.

\begin{align*}
    \mathbb{E}_{(x,y)\sim \mathcal{D}}\left[ \sup\limits_{z \in \mathcal{B}(x,\epsilon)} \phi(y*h_n(z))\right] &= \frac{1}{2} \phi(\frac{-1}{n}) +\frac{1}{4}\phi(\frac{1}{n}) + \frac{1}{4}\phi(\frac{1}{n})\\
    &=\frac{1}{2}\phi(\frac{-1}{n}) + \frac{1}{2}\phi(\frac{1}{n}) \\
    &\underset{n\to +\infty}{\longrightarrow}\phi(0)
\end{align*}

We now need to show that $\phi(0)$ is a lower bound of the optimal adversarial risk for the loss $\phi$.
Let h be any  classifier. We define :
\begin{align*}
b &= \inf\limits_{z \in \left[ -a + \epsilon, a- \epsilon \right]} h(z) \\
c &= \sup\limits_{z \in \left[ -a - \epsilon, -\epsilon \right]} h(z) \\
d &= \sup\limits_{z \in \left[\epsilon, a + \epsilon \right]} h(z) \\
m_i &= \inf\limits_{z \in \mathcal{Z}_i} h(z) \mbox{  for  }i \in \left\{ 1,2\right\} \\
M_i &= \sup\limits_{z \in \mathcal{Z}_i} h(z) \mbox{  for  }i \in \left\{ 1,2\right\} \\
m &= \min(m_1,m_2)
\end{align*}

We then have :

\begin{align*}
    \mathbb{E}_{(x,y)\sim \mathcal{D}}\left[ \sup\limits_{z \in \mathcal{B}(x,\epsilon)} \phi(y*h_n(z))\right] &= \frac{1}{2} \sup\limits_{ z \in \left[ -\epsilon, \epsilon \right]} \phi(h(z))
            + \frac{1}{4}\sup\limits_{z \in \left[ -a-\epsilon, -a+\epsilon \right]} \phi(-h(z)) 
            + \frac{1}{4}\sup\limits_{z \in \left[ a-\epsilon, a+\epsilon \right]} \phi(-h(z)) \\
    &= \frac{1}{2} \max \left[ \sup\limits_{ z \in \left[ -a+\epsilon, a-\epsilon \right]} \phi(h(z)), \sup\limits_{ z \in \mathcal{Z}_1} \phi(h(z)), \sup\limits_{ z \in \mathcal{Z}_2} \phi(h(z)) \right] \\
    &+ \frac{1}{4} \max \left[ \sup\limits_{ z \in \left[ -a-\epsilon, -\epsilon \right]} \phi(-h(z)), \sup\limits_{ z \in \mathcal{Z}_1} \phi(-h(z)) \right] \\
    &+ \frac{1}{4} \max \left[ \sup\limits_{ z \in \left[ \epsilon, a+\epsilon \right]} \phi(-h(z)), \sup\limits_{ z \in \mathcal{Z}_2} \phi(-h(z)) \right] \\
    &= \frac{1}{2} \max \left[ \phi(b), \phi(m_1), \phi(m_2) \right]
    + \frac{1}{4} \max \left[ \phi(-M_1), \phi(-c) \right] \\
    &+ \frac{1}{4} \max \left[ \phi(-M_2), \phi(-c) \right] \\
    &\geq \frac{1}{2} \max \left[\phi(m_1), \phi(m_2) \right] + \frac{1}{4}\phi(-M_1) + \frac{1}{4}\phi(-M_2) \\
    &\geq \frac{1}{2} \phi(m) + \frac{1}{4}\phi(-m) + \frac{1}{4}\phi(-m) \\
    &\geq \phi(0)
\end{align*}

Hence the result.
\end{proof}


\subsection{Realisable case}
In this realisable case, i.e. $\mathcal{R}^{\varepsilon}_{0/1} = 0$, results about standard classification consistency still hold.
\begin{prop}
Let $\PP$ be a Borel probability distribution over $\mathcal{X}\times\mathcal{Y}$ such that $\mathcal{R}^{\varepsilon}_{0/1} = 0$. Then, let $\phi$ be a standard classification calibrated margin-based loss. Then $\Tilde{\phi}:(x,y,f)\mapsto\sup_{x'\in B_\varepsilon(x)}\phi(yf(x'))$ is adversarially consistent at level $\varepsilon$.
\end{prop}

\begin{proof}
Coming soon
\end{proof}
\chapter{A Dynamical System Perspective for Lipschitz Neural Networks}
\minitoc
%

\section{Introduction}

Modern neural networks have been known to be sensible against small, imperceptible and adversarially-chosen perturbations of their inputs~\citep{biggio2013evasion,szegedy2014intriguing}.
This vulnerability  has become a major issue as more and more neural networks have been deployed into production applications.
Over the past decade, the research progress plays out like a cat-and-mouse game between the development of more and more powerful attacks~\citep{goodfellow2014explaining,kurakin2016adversarial,carlini2017adversarial,croce2020reliable} and the design of empirical defense mechanisms~\citep{madry2017towards,moosavi2019robustness,cohen2019certified}.
Finishing the game calls for certified adversarial robustness~\citep{raghunathan2018certified,wong2018scaling}.
While recent work devised defenses with theoretical guarantees against adversarial perturbations, they share the same limitation, \ie, the tradeoffs between expressivity and robustness, and between scalability and accuracy.

A natural approach to provide robustness guarantees on a classifier is to enforce Lipschitzness properties. 
To achieve such properties, researchers mainly focused on two different kinds of approaches.
The first one is based on randomization~\citep{lecuyer2018certified,cohen2019certified,pinot2019theoretical} and consists in convolving the input with with a predefined probability distribution.
While this approach offers some level of scalability (\ie, currently the only certified defense on the ImageNet dataset), it suffers from significant impossibility results~\cite{yang2020randomized}.
A second approach consists in building $1$-Lipschitz layers using specific linear transform~\citep{cisse2017parseval,li2019preventing,anil2019sorting,trockman2021orthogonalizing,skew2021sahil,li2019preventing,singla2021householder}.
Knowing the Lipschitz constant of the network, it is then possible to compute a certification radius around any points. 

A large line of work explored the interpretation of residual neural networks \cite{he2016deep} as a parameter estimation problem of nonlinear dynamical systems~\citep{haber2017stable,e17Proposal,lu18beyond}.
Reconsidering the ResNet architecture as an Euler discretization of a continuous dynamical system yields to the trend around Neural Ordinary Differential Equation~\citep{chen2018neural}.
For instance, in the seminal work of~\citet{haber2017stable}, the continuous formulation offers more flexibility to investigate the stability of neural networks during inference, knowing that the discretization will be then implemented by the architecture design.
The notion of stability, in our context, quantifies how a small perturbation on the initial value impacts the trajectories of the dynamical system. 

From this continuous and dynamical interpretation, we  analyze the Lipschitzness property of Neural Networks. We then study the discretization schemes that can preserve the Lipschitzness properties. With this point of view, we can readily recover several previous methods that build 1-Lipschitz neural networks~\citep{trockman2021orthogonalizing,skew2021sahil}.
Therefore, the dynamical system perspective offers a general and flexible framework to build Lipschitz Neural Networks facilitating the discovery of new approaches.
In this vein, we introduce convex potentials in the design of the Residual Network flow and show that this choice of parametrization yields to by-design $1$-Lipschitz neural networks.
At the very core of our approach lies a new $1$-Lipschitz non-linear operator that we call {\em Convex Potential Layer} which allows us to adapt convex potential flows to the discretized case. 
These blocks enjoy the desirable property of stabilizing the training of the neural network by controlling the gradient norm, hence overcoming the exploding gradient issue.
We experimentally demonstrate our approach by training large-scale neural networks on several datasets, reaching state-of-the art results in terms of under-attack and certified accuracy.


%
\section{Background and Related Work}
\label{section:background_rw}

In this paper, we aim at devising {\em certified} defense mechanisms against adversarial attacks, in the following, we formally define an adversarial attacks and a robustness certificate.
We consider a classification task from an input space $\mathcal{X}\subset\RR^d$ to a label space $\mathcal{Y}:=\{1,\dots,K\}$.
To this end, we aim at learning a classifier function $\mathbf{f}:=(f_1,\dots,f_K):\mathcal{X}\to \RR^K$ such that the predicted label for an input $x$ is $\argmaxB_k f_k(x)$.
For a given couple input-label $(x,y)$, we say that $x$ is correctly classified if $\argmaxB_k f_k(x)=y$.


\begin{definition}[{\bf Adversarial Attacks}]
Let $x \in \mathcal{X}$, $y \in \mathcal{Y}$ the label of $x$ and let $\mathbf{f}$ be a classifier.
An adversarial attack at level $\varepsilon$ is a perturbation $\tau$ \st $\lVert\tau\rVert\leq\varepsilon$ such that:
\begin{equation*}
  \argmaxB_k f_k(x+\tau) \neq y
\end{equation*}
\end{definition}
Let us now define the notion of  robust certification. For $x \in \mathcal{X}$, $y \in \mathcal{Y}$ the label of $x$ and let $\mathbf{f}$ be a classifier, a classifier $\mathbf{f}$ is said to be \emph{certifiably robust at radius $\varepsilon\geq 0$} at point $x$ if for all $\tau$ such that ${\lVert\tau\rVert \leq \varepsilon}$ :
\begin{equation*}
  \argmaxB_k f_k(x+\tau) = y
\end{equation*}
The task of robust certification is then to find methods that ensure the previous property. A key quantity in this case is the Lipschitz constant of the classifier.



\subsection{Lipschitz property of Neural Networks}

The Lipschitz constant has seen a growing interest in the last few years in the field of deep learning~\citep{scaman2018lipschitz,fazlyab2019efficient,combettes2020lipschitz,bethune2021many}.
Indeed, numerous results have shown that neural networks with a small Lipschitz constant exhibit better generalization~\citep{bartlett2017spectrally}, higher robustness to adversarial attacks~\citep{szegedy2014intriguing,farnia2018generalizable,tsuzuku2018lipschitz}, better training stability~\citep{xiao2018dynamical,trockman2021orthogonalizing}, improved Generative Adversarial Networks~\citep{arjovsky2017wasserstein}, etc.
Formally, we define the Lipschitz constant with respect to the $\ell_2$ norm of a Lipschitz continuous function $f$ as follows:
\begin{equation*}
  Lip_{2}{(f)} = \sup_{\substack{x, x' \in \mathcal{X} \\ x \neq x'}} \frac{\lVert f(x) - f(x') \rVert_2}{\lVert x - x' \rVert_2} \enspace.
\end{equation*}

Intuitively, if a classifier is Lipschitz, one can bound the impact of a given input variation on the output, hence obtaining guarantees on the adversarial robustness.
We can formally characterize the robustness of a neural network with respect to its Lipschitz constant with the following proposition:
\begin{prop}[\citet{tsuzuku2018lipschitz}] \label{proposition:tsuzuku}
Let $\mathbf{f}$ be an $L$-Lipschitz continuous classifier for the $\ell_2$ norm.
Let $\varepsilon > 0$, $x \in \mathcal{X}$ and $y \in \mathcal{Y}$ the label of $x$.
If at point $x$, the margin $\mathcal{M}_{\mathbf{f}}(x)$ satisfies:
\begin{equation*}
  \mathcal{M}_{\mathbf{f}}(x):=\max(0,f_y(x)-\max_{y'\neq y}f_{y'}(x)) > \sqrt{2} L \varepsilon
\end{equation*}
then we have for every $\tau$ such that $\lVert \tau \rVert_2 \leq \varepsilon$:
\begin{equation*}
  \argmaxB_{k}f_k(x + \tau) = y
\end{equation*}
\end{prop}
From Proposition~\ref{proposition:tsuzuku}, it is straightforward to compute a robustness certificate for a given point.
Consequently, in order to build robust neural networks the margin needs to be large and the Lipschitz constant small to get optimal guarantees on the robustness for neural networks.



\subsection{Certified Adversarial Robustness}

Mainly two kinds of methods have been developed to come up with certified adversarial robustness.
The first category relies on randomization and consists of convolving the input with a predefined probability distribution during both training and inference phases.
Several works that rely on the method have proposed empirical~\cite{cao2017mitigating,liu2018towards,pinot2019theoretical,pinot2020randomization} and certified defenses~\cite{lecuyer2018certified,li2019certified,cohen2019certified,salman2019provably,yang2020randomized}. These methods are model-agnostic, in the sense they do not depend on the architecture of the classifier, and provide ``high probability'' certificates.
However, this approach suffers from significant impossibility results: the maximum radius that can be certified for a given smoothing distribution vanishes as the dimension increases~\cite{yang2020randomized}.
Furthermore, in order to get non-vacuous provable guarantees, such approaches often require to query the network hundreds of times to infer the label of a single image.
This computational cost naturally limits the use of these methods in practice.

The second approach directly exploits the Lipschitzness property with the design of built-in $1$-Lipschitz layers. Contrarily to previous methods,  these approaches provide deterministic guarantees.
Following this line, one can either normalize the weight matrices by their largest singular values making the layer $1$-Lipschitz, \emph{e.g.}~\citep{yoshida2017spectral,miyato2018spectral,farnia2018generalizable,anil2019sorting} or project the weight matrices on the Stiefel manifold \citep{li2019preventing,trockman2021orthogonalizing,skew2021sahil}.
The work of \citet{li2019preventing}, \citet{trockman2021orthogonalizing} and \citet{skew2021sahil} (denoted BCOP, Cayley and SOC respectively) are considered the most relevant approach to our work.
Indeed, their approaches consist of projecting the weights matrices onto an orthogonal space in order to preserve gradient norms and enhance adversarial robustness by guaranteeing low Lipschitz constants. 
While both works have similar objectives, their execution is different.
The BCOP layer (Block Convolution Orthogonal Parameterization) uses an iterative algorithm proposed by \citet{bjorck1971iterative} to orthogonalize the linear transform performed by a convolution.
The SOC layer (Skew Orthogonal Convolutions) uses the property that if $A$ is a skew symmetric matrix then $Q=\exp{A}$ is an orthogonal matrix. To approximate the exponential, the authors proposed to use a finite number of terms in its Taylor series expansion.
Finally, the method proposed by~\citet{trockman2021orthogonalizing} use the Cayley transform to orthogonalize the weights matrices.
Given a skew symmetric matrix $A$, the Cayley transform consists in computing the orthogonal matrix $Q = (I - A)^{-1} (I + A)$. Both methods are well adapted to convolutional layers and are able to reach high accuracy levels on CIFAR datasets. Also, several works~\cite{anil2019sorting,singla2021householder,huang2021local} proposed methods leveraging the properties of activation functions to constraints the Lipschitz of Neural Networks. These works are usually useful to help  improving the performance of linear orthogonal layers.


\subsection{Residual Networks}

To prevent from gradient vanishing issues in neural networks during the training phase~\citep{hochreiter2001gradient},~\cite{he2016deep} proposed the Residual Network (ResNet) architecture.
Based on this architecture, several works~\citep{haber2017stable,e17Proposal,lu18beyond,chen2018neural} proposed a ``continuous time'' interpretation inspired by dynamical systems that can be defined as follows.

\begin{definition}\label{def:flow}
Let $(F_{t})_{t\in[0,T]}$ be a family of functions on $\RR^d$, we define the continuous time Residual Networks flow associated with $F_t$ as:
\begin{align*}\label{eq:resnet_c0}
  \left\{
    \begin{array}{ll}
    x_0 &= x\in\mathcal{X}\\
    \frac{dx_{t}}{dt} &= F_{{t}}(x_{t}) \  \text{for } \ t\in[0, T]
  \end{array}
  \right.
\end{align*}
\end{definition}

This continuous time interpretation helps as it allows us to consider the stability of the forward propagation through the stability of the associated dynamical system.
A dynamical system is said to be \emph{stable} if two trajectories starting from an input and another one remain sufficiently close to each other all along the propagation.
This stability property takes all its sense in the context of adversarial classification.

It was argued by~\citet{haber2017stable} that when $F_{t}$ does not depend on $t$ or vary slowly with time\footnote{This blurry definition of "vary slowly" makes the property difficult to apply.}, the stability can be characterized by the eigenvalues of the Jacobian matrix $\nabla_x F_{t}(x_t)$: 
the dynamical system is stable if the real part of the eigenvalues of the Jacobian stay negative throughout the propagation.
This property however only relies on intuition and this condition might be difficult to  verify in practice.
In the following, in order to derive stability properties, we study gradient flows and convex potentials, which are sub-classes of Residual networks.

Other works~\citep{huang2020adversarial,li2020implicit} also proposed to enhance adversarial robustness using dynamical systems interpretations of Residual Networks. Both works argues that using particular discretization scheme would make gradient attacks more difficult to compute due to numerical stability. These works did not provide any provable guarantees for such approaches.



\section{A Framework to design Lipschitz Layers}
\label{section:global_framework}


The continuous time interpretation of Definition~\ref{def:flow} allows us to better investigate the robustness properties and assess how a difference of the initial values (the inputs) impacts the inference flow of the model. Let us consider two continuous flows $x_t$ and $z_t$ associated with $F_t$ but differing in their respective initial values $x_0$ and $z_0$. Our goal is to characterize the time evolution of $\lVert x_t-z_t \rVert$ by studying its  time derivative. We recall that  every matrix $M\in\RR^{d\times d}$ can be uniquely decomposed as the sum of a symmetric and skew-symmetric matrix $M = S(M) + A(M)$. By applying this decomposition to the Jacobian matrix $\nabla_x F_t(x)$ of $F_t$, we can show that the time derivative of $\lVert x_t-z_t \rVert$ only involves the symmetric part  $S(\nabla_x F_t(x))$ (see Appendix~\ref{proof:continuous-lip} for details). 

For two symmetric matrices $S_1,S_2\in\RR^{d\times d}$,  we denote $S_1\preceq S_2$ if, for all $x\in\RR^d$, $\langle x,(S_2-S_1)x\rangle\geq 0$. By focusing on the symmetric part of the Jacobian matrix we can show in Appendix~\ref{proof:continuous-lip} the following proposition.
\begin{prop}
\label{prop:continuous-lip}
Let $(F_{t})_{t\in[0,T]}$ be a family of differentiable  functions almost everywhere on $\RR^d$.
Let us assume that there exists two measurable functions $t\mapsto \mu_t$ and  $t\mapsto \lambda_t$ such that
$$\mu_t I\preceq S(\nabla_xF_{t}(x))\preceq \lambda_tI$$
for all $x\in\RR^d$, and $t\in [ 0,T]$. Then the flow associated with $F_t$ satisfies for all initial conditions $x_0$ and $z_0$:
\begin{align*}
  \lVert x_0-z_0 \rVert e^{\int_0^t\mu_s ds}\leq \lVert x_t-z_t \rVert\leq \lVert x_0-z_0 \rVert e^{\int_0^t\lambda_s ds}
\end{align*}
\end{prop}

The symmetric part plays even a more important role since one can show that a function whose Jacobian is always skew-symmetric is actually linear (see Appendix~\ref{sup:skew} for more details). However, constraining $S(\nabla_x F_{t}(x))$ in the general case is technically difficult and a solution resorts to a more intuitive parametrization of  $F_t$ as the sum of two functions $F_{1,t}$ and $F_{2,t}$ whose Jacobian matrix are respectively symmetric  and skew-symmetric.  Thus, such a parametrization enforces $F_{2,t}$  to be linear and skew-symmetric. For the choice of $F_{1,t}$, we propose to rely on potential functions: a function  $F_{1,t}:\RR^d \to \RR^d$ derives from a simpler family of scalar valued function in $\RR^d$, called the \emph{potential}, via the gradient operation. Moreover, since the Hessian of the potential is symmetric, the Jacobian for $F_{1,t}$ is then also symmetric.  If we had the convex property to this potential, its Hessian has positive eigenvalues. Therefore we introduce the following corollary. See proof in Appendix~\ref{proof:conv-skew} 

\begin{corollary} 
\label{cor:conv-skew}Let $(f_{t})_{t\in[0,T]}$ be a family of convex differentiable functions on $\RR^d$ and $(A_t)_{t\in[0,T]}$ a family of skew symmetric matrices. Let us define 
$$F_t(x) = -\nabla_x f_{t}(x)+A_t x,$$ 
then the flow associated with $F_t$ satisfies for all initial conditions $x_0$ and $z_0$:
\begin{align*}
\lVert x_t-z_t \rVert\leq \lVert x_0-z_0 \rVert
\end{align*}
\end{corollary}

This simple property suggests that if we could parameterize $F_t$  with convex potentials, it would be less sensitive to input perturbations and therefore more robust to adversarial examples. We also remark that the skew symmetric part is then norm-preserving.
However, the discretization of such flow is challenging in order to maintain this property of stability. 


\subsection{Discretized Flows}

To study the discretization of  the previous flow, let $t=1,\dots,T$ be the discretized time steps and from now we consider the flow defined by  $F_t(x) = -\nabla f_{t}(x)+A_t x$, with $(f_{t})_{t=1,\dots,T}$  a family of convex differentiable functions on $\RR^d$ and $(A_t)_{t=1,\dots,T}$ a family of skew symmetric matrices. The most basic method the explicit Euler scheme as defined by: 
\begin{align*}
x_{t+1} = x_t+ F_t(x_t)
\end{align*}
However, if $A_t\neq 0$, this discretized system might not satisfy $\lVert x_t-z_t\rVert\leq\lVert x_0-z_0\rVert$. Indeed, consider the simple example where $f_t=0$. We then have:
\begin{align*}
\lVert x_{t+1}-z_{t+1}\rVert - \lVert x_{t}-z_{t}\rVert =\lVert A_t\left(x_{t}-z_{t}\right)\rVert.
\end{align*}
Thus explicit Euler scheme cannot guarantee Lipschitzness when $A_t\neq 0$. To overcome this difficulty, the discretization step can be split in two parts, one for $\nabla_x f_t$ and one for $A_t$: 
\begin{align*}
   \left\{
    \begin{array}{ll}
        x_{t+\frac12} &= \textsc{step1}(x_t, \nabla_x f_t)\\
        x_{t+1}& = \textsc{step2}(x_{t+\frac12}, A_t)
    \end{array} 
    \right.
\end{align*}
This type of discretization scheme  can be found for instance from Proximal Gradient methods where one step is explicit and the other is implicit. Then, we dissociate the Lipschitzness study of both terms of the flow. 

\subsection{Discretization scheme for $\nabla_x f_t$}

To apply the explicit Euler scheme to $\nabla_x f_t$, an  additional smoothness property on the potential functions is required  to generalize the Lipschitzness guarantee to the discretized flows. Recall that a function $f$ is said to be \emph{$L$-smooth} if it is differentiable and if $x\mapsto\nabla_x f(x)$ is $L$-Lipschitz. 
\begin{prop}\label{prop:discrete_convex_potentials}
Let $t\in\{1,\cdots,T\}$ Let us assume that $f_{t}$ is $L_t$-smooth. We  define the following discretized ResNet gradient flow using $h_t$ as a step size:
\begin{align*}
    \begin{array}{ll}
    x_{t+\frac12} &= x_{t}-h_{t}\nabla_xf_{t}(x_{t})\\
  \end{array}
 \end{align*}
Consider now two trajectories $x_t$ and $z_t$ with initial points $x_0=x$ and $z_0=z$ respectively,  if $0\leq h_t\leq \frac{2}{L_t}$,  then 
$$\lVert x_{t+\frac12}-z_{t+\frac12}\rVert_2\leq \lVert x_t-z_t\rVert_2$$
\end{prop}
In Section~\ref{sec:param}, we describe how to parametrize a neural network layer to implement such a discretization step by leveraging the recent work on Input Convex Neural Networks~\cite{amos2017input}. 

\begin{rmq}
Another solution relies on the implicit Euler scheme: $ x_{t+\frac12} = x_{t}-\nabla_xf_{t}(x_{t+\frac12})$. We show in Appendix~\ref{sup:implicit} that this strategy defines a $1$-Lipschitz flow without further assumption on $f_t$ than convexity. We propose an implementation. However preliminary experiments did not show competitive results and the training time is prohibitive. We leave this solution for future work. 

\end{rmq}
\subsection{Discretization scheme for $A_t$}

The second step of discretization involves the term with skew-symmetric matrix $A_t$. As mentioned earlier, the challenge is that the \emph{explicit Euler discretization} is not contractive. More precisely,  the following property 
$$\lVert x_{t+1} - z_{t+1}\rVert\geq \lVert x_{t+\frac12} - z_{t+\frac12}\rVert$$ 
is satisfied with equality only in the special and useless case of $x_{t+\frac12} - z_{t+\frac12} \in \text{ker}(A_t)$. Moreover, the implicit Euler discretization induces an increasing norm and hence does not satisfy the desired property of norm preservation neither. 

\paragraph{Midpoint Euler method.}
We thus propose to use \emph{Midpoint Euler} method, defined as follows:
\begin{align*}
&x_{t+1} = x_{t+\frac12} +A_t \frac{x_{t+1}+x_{t+\frac12}}{2}\\
\iff&x_{t+1} = \left(I-\frac{A_t}{2}\right)^{-1}\left(I+\frac{A_t}{2}\right)x_{t+\frac12}.
\end{align*} 
Since $A_t$ is skew-symmetric, $I-\frac{A_t}{2}$ is invertible. This update corresponds to the Cayley Transform of $\frac{A_t}{2}$ that induces an orthogonal mapping. 
This kind of layers was introduced and extensively studied in~\citet{trockman2021orthogonalizing}.  


\paragraph{Exact Flow.} One can define the simple differential equation corresponding to the flow associated with $A_t$  
\begin{align*}
        \frac{du_t}{ds} = A_t u_s,\quad u_0 = x_{t+\frac12},
\end{align*}
There exists an exact solution exists since $A_t$ is linear. By taking the value at $s=\frac12$, we obtained the following transformation:  
 \begin{align*}
x_{t+1} := u_{\frac12}=e^{\frac{A}{2}} x_{t+\frac12}.
\end{align*}
This step is therefore clearly norm preserving but the matrix exponentiation is challenging and it requires efficient approximations. This trend was recently investigated under the name of Skew Orthogonal Convolution (SOC)~\cite{skew2021sahil}.


\section{Parametrizing Convex Potentials Layers}
\label{sec:param}

As presented in the previous section, parametrizing the skew symmetric updates has been extensively studied by~\citet{trockman2021orthogonalizing,skew2021sahil}. In this paper we focus on  the parametrization of symmetric update with the convex potentials proposed in~\ref{prop:discrete_convex_potentials}. For that purpose, the Input Convex Neural Network (ICNN) \citep{amos2017input} provide a relevant starting point that we will extend. 

\subsection{Gradient of ICNN}
We use $1$-layer ICNN~\citep{amos2017input} to define an efficient computation of Convex Potentials Flows. For any vectors $w_1,\dots w_k\in\mathbb{R}^d$, and bias terms  $b_1,\dots,b_k\in \mathbb{R}$, and for $\phi$ a convex function,  the potential $F$ defined as:
\begin{align*}
    F_{w,b}:x\in\RR^d\mapsto \sum_{i=1}^k\phi( w_i^\top x+b_i)
\end{align*}
defines  a convex function in $x$ as the composition of a linear and a convex function. Its gradient with respect to its input $x$ is then:
\begin{align*}
    x\mapsto \sum_{i=1}^kw_i\phi'(w_i^\top x+b_i) = \mathbf{W}^\top \phi'(\mathbf{W} x+\mathbf{b})
\end{align*}
with $\mathbf{W}\in \mathbb{R}^{k\times d}$ and $\mathbf{b}\in\mathbb{R}^{k}$ are respectively the matrix and vector obtained by the concatenation of, respectively, $w_i^\top$ and $b_i$, and $\phi'$ is applied element-wise.  
Moreover, assuming $\phi'$ is $L$-Lipschitz, we have that $F_{w,b}$ is  $L\lVert\mathbf{W}\rVert_2^2$-smooth. $\lVert\mathbf{W}\rVert_2$ denotes the spectral norm of $\mathbf{W}$, \ie, the greatest singular value of $\mathbf{W}$ defined as:
\begin{align*}
   \lVert\mathbf{W}\rVert_2 :=\max_{x\neq 0} \frac{\lVert \mathbf{W}x\rVert_2}{\lVert x\rVert_2}
\end{align*}
The reciprocal also holds: if $\sigma:\RR\to\RR$ is a non-decreasing $L$-Lipschitz function, $\mathbf{W}\in \RR^{k\times d}$ and $b\in \RR^{k}$, there exists a convex $L\lVert\mathbf{W}\rVert_2^2$-smooth function $F_{w,b}$ such that 
$$\nabla_xF_{w,b}(x) =  \mathbf{W}^\top \sigma(\mathbf{W} x+\mathbf{b}),$$ where $\sigma$ is applied element-wise. The next section shows how this property can be used to implement the building block and training of such layers. 


\subsection{Convex Potential layers}
From the previous section, we derive the following \emph{Convex Potential Layer}: 
\begin{equation*}
\label{equation:stable_block}
  z = x - \frac{2}{\lVert \mathbf{W} \lVert_2^2} \mathbf{W}^\top \sigma(\mathbf{W} x + b)
\end{equation*}
Written in a matrix form, this layer can be implemented with every linear operation $\mathbf{W}$.
In the context of image classification, it is beneficial to use convolutions\footnote{For instance, one can leverage the \texttt{Conv2D} and \texttt{Conv2D\_transpose} functions of the PyTorch framework~\citep{paszke2019pytorch}} instead of generic linear transforms represented by a dense matrix. 

\begin{rmq}
When $\mathbf{W}\in\RR^{1\times d}$, $b =0$ and $\sigma=\textsc{ReLU}$, the \emph{Convex Potential Layer} is equivalent to the HouseHolder activation function introduced in~\citet{singla2021householder}.
\end{rmq}

Residual Networks~\citep{he2016deep} are also composed of other types of layers which increase or decrease the dimensionality of the flow.
Typically, in a classical setting, the number of input channels is gradually increased, while the size of the image is reduced with pooling layers.
In order to build a $1$-Lipschitz Residual Network, all operations need to be properly scale or normalize in order to maintain the Lipschitz constant.

\paragraph{Increasing dimensionsionality.} To increase the number of channels in a convolutional Convex Potential Layer, a zero-padding operation can be easily performed: an input $x$ of size $c\times h \times w$ can be extended to some $x'$ of size  $c'\times h \times w$, where $c'>c$, which equals $x$ on the $c$ first channels and $0$ on the $c'-c$ other channels.
\paragraph{Reducing dimensionsionality.} Dimensionality reduction is another essential operation in neural networks. On one hand, its  goal is to  reduce the number of parameters and thus the amount of computation required to build the network. On the other hand it allows the model to progressively map the input space on the output dimension, which corresponds in many cases to the number of different labels $K$. 
In this context, several operations exist:
pooling layers are used to extract information present in a region of the feature map generated by a convolution layer. One can easily adapt pooling layers (\emph{e.g.} max and average) to make them $1$-Lipschitz~\citep{bartlett2017spectrally}.
Finally, a simple method to reduce the dimension is the product with a non-square matrix. In this paper, we simply implement it as  the truncation of the output. This obviously maintains the Lipschitz constant.


\begin{algorithm}[tb]
\caption{Computation of a Convex Potential Layer}
\label{algorithm:stable_block}
\begin{algorithmic}
  \STATE{Require: \bfseries Input: $x$, vector: $u$, weights: $\mathbf{W}$, $b$}
  \STATE{Ensure: Compute the layer $z$ and return $u$}
  \STATE{$v \gets \mathbf{W} u / \lVert \mathbf{W} u \rVert_2$}
  \STATE{$u \gets \mathbf{W}^\top v / \lVert \mathbf{W}^\top v \rVert_2$
    \rlap{\hspace{0.5cm}\smash{$\left.\begin{array}{@{}c@{}}\\{}\\{}\end{array}\right\}%
      \begin{tabular}{l}1 iter. for training \\100 iter. for inference\end{tabular}$}}}
  \STATE{$h \gets 2 / \left( \sum_i (\mathbf{W} u \cdot v)_i \right)^2$}
  \STATE{\textbf{return} $x - h \left[ \mathbf{W}^\top \sigma( \mathbf{W} x + b) \right], u$}
\end{algorithmic}
\end{algorithm}




\subsection{Computing spectral norms}
Our Convex Potential Layer, described in Equation~\ref{equation:stable_block}, can be adapted to any kind of linear transformations (\emph{i.e.} Dense or Convolutional) but requires the computation of the spectral norm for these transformations.
Given that computation of the spectral norm of a linear operator is known to be NP-hard~\citep{steinberg2005computation}, an efficient approximate method is required during training to keep the complexity tractable. 


Many techniques exist to approximate the spectral norm (or the largest singular value), and most of them exhibit a trade-off between efficiency and accuracy.
Several methods exploit the structure of convolutional layers to build an upper bound on the spectral norm of the linear transform performed by the convolution~\citep{jia2017improving,singla2021fantastic,araujo2021lipschitz}.
While these methods are generally efficient, they can less relevant and adapted to certain settings. For instance in our context, using a loose upper bound of the spectral norm will hinder the expressive power of the layer and make it too contracting.

For these reasons we rely on the Power Iteration Method (PM).  This method converges at a geometric rate towards the largest singular value of a matrix. More precisely the convergence rate for a given matrix $\mathbf{W}$ is $\textstyle O((\frac{\lambda_2}{\lambda_1})^k)$ after $k$ iterations, independently from the choice for the starting vector, where $\lambda_1>\lambda_2$ are the two largest singular values of $\mathbf{W}$. While it can appear to be computationally expensive due to the large number of required iterations for convergence, it is possible to drastically reduce the number of iterations during training. Indeed, as in~\citep{miyato2018spectral}, by considering that the weights' matrices $\mathbf{W}$ change slowly during training, one can perform only one iteration of the PM for each step of the training and let the algorithm slowly converges along with the training process\footnote{Note that a typical training requires approximately 200K steps where 100 steps of PM is usually enough for convergence}.
We describe with more details in Algorithm~\ref{algorithm:stable_block}, the operations performed during a forward pass with a Convex Potential Layer. 

However for evaluation purpose, we need to compute the certified adversarial robustness, and this requires to ensure the convergence of the PM. Therefore, we perform $100$ iterations for each layer\footnote{$100$ iterations of Power Method is sufficient to converge with a geometric rate.} at inference time. Also note that at inference time, the computation of the spectral norm only needs to be performed once for each layer. 



\section{Experiments}
\label{section:experiments}
\begin{table}
\begin{center}
    \begin{tabular}{cccccccc}
    \toprule
    \textbf{\#} & \textbf{S} &  & \textbf{M} & & \textbf{L} & & \textbf{XL} \\
    \midrule
    \textbf{Conv. Layers}      & 20 & & 30 & & 90 & & 120 \\
    \textbf{Channels}  &45 & & 60 & & 60 & & 70 \\ 
    \textbf{Lin. Layers}        &7 & & 10 & & 15 & & 15 \\
    \textbf{Lin. Features} & 2048 & & 2048 & & 4096 & & 4096 \\
    \bottomrule
    \end{tabular}%

\end{center}
\caption{\label{table:model-desc}
Architectures description for our Convex Potential Layers (CPL) neural networks with different capacities. We vary the number of Convolutional Convex Potential Layers, the number of Linear Convex Potential Layers, the number of channels in the convolutional layers and the width of fully
connected layers. In the paper, they will be reported respectively as CPL-S, CPL-M, CPL-L and CPL-XL.}
\end{table}

To evaluate our new $1$-Lipschitz Convex Potential Layers, we carry out an extensive set of experiments. In this section, we first describe  the details of our experimental setup. We then recall  the concurrent approaches that build $1$-Lipschitz Neural Networks and stress their limitations. Our experimental results are finally summarized in ection~\ref{sec:setting-xp}. By computing the certified and empirical adversarial  accuracy of our networks on CIFAR10 and CIFAR100 classification tasks~\citep{krizhevsky2009learning}, we show that our architecture is competitive with state-of-the-art methods (Sections~\ref{sec:results}). In Appendix~\ref{app:xp-supp}, we also study the influence of some hyperparameters and demonstrate the stability and the scalability of our approach by training very deep neural networks up to 1000 layers without normalization tricks or gradient clipping.










\subsection{Training and Architectural Details}
\label{sec:setting-xp}

We demonstrate the effectiveness of our approach on a classification task with CIFAR10 and CIFAR100 datasets~\citep{krizhevsky2009learning}. We use a similar training configuration to the one proposed in~\citep{trockman2021orthogonalizing}
We trained our networks with a batch size of $256$ over $200$ epochs.
We use standard data augmentation (\ie, random cropping and flipping), a learning rate of $0.001$ with Adam optimizer \citep{diederik2014adam} without weight decay and a piecewise triangular learning rate scheduler. We used a margin parameter in the loss set to $0.7$.

As other usual convolutional neural networks, we first stack few Convolutional CPLs and then stack some Linear CPLs for classification tasks. To validate the performance  and the scalability of our layers,  we will evaluate four different variations of different hyperparameters as described in Table~\ref{table:model-desc}, respectively named CPL-S, CPL-M, CPL-L and CPL-XL, ranked according to the number of parameters they have. In all our experiments, we made $3$ independent trainings to evaluate accurately the models. All reported results are the average of these $3$ runs.

\subsection{Concurrent Approaches} We compare our networks with SOC~\citep{skew2021sahil} and Cayley~\cite{trockman2021orthogonalizing} networks which are to our knowledge the best performing approaches for deterministic $1$-Lipschitz Neural Networks. Since our layers are fundamentally different from these ones, we cannot compare with the same architectures. We reproduced SOC results for with $10$ and $20$ layers, that we call respectively SOC-$10$ and SOC-$20$ in the same training setting, \emph{i.e.} normalized inputs, cross entropy loss, SGD optimizer with learning rate $0.1$ and multi-step learning rate scheduler. For Cayley layers networks, we reproduced their best reported model, \emph{i.e.} KWLarge with width factor of $3$. 

The work of~\citet{singla2021householder} propose three methods to improve certifiable accuracies from SOC layers: a new HouseHolder activation function (HH),  last layer normalization (LLN), and certificate regularization (CR). The code associated with this paper is not open-sourced yet, so we just reported the results from their paper in ours results (Tables~\ref{table:c10-comp} and~\ref{table:c100-comp}) under the name SOC+. We were being able to implement the LLN method in all models. This method largely improve the result of all methods on CIFAR100, so we used it for all networks we compared on CIFAR100 (ours and concurrent approaches).


\begin{table*}[tb]
  \centering
  \sisetup{%
    table-align-uncertainty=true,
    separate-uncertainty=true,
    detect-weight=true,
    detect-inline-weight=math
  }
  \begin{tabular}
  {
    l
    S[table-format=2.2]
    S[table-format=2.2]
    S[table-format=2.2]
    S[table-format=2.2]
    S[table-format=2.2]
  }
  \toprule
    & \multicolumn{1}{c}{\textbf{Clean Accuracy}} & \multicolumn{3}{c}{\textbf{Provable Accuracy ($\varepsilon $)}} &  \multicolumn{1}{c}{\textbf{Time per epoch (s)}} 
    \\
    \cmidrule{3-5}
    & \multicolumn{1}{c}{ } & \multicolumn{1}{c}{36/255} & \multicolumn{1}{c}{72/255} &  \multicolumn{1}{c}{108/255} & \multicolumn{1}{c}{\textbf{}} 
    \\
  \midrule
    \textbf{CPL-S} & 75.6  & 62.3  & 46.9  & 32.2  & 21.9 \\
  \textbf{CPL-M} & 76.8  & 63.3  & 47.5  & 32.5  & 40.0 \\
  \textbf{CPL-L} & 77.7  & 63.9 & 48.1 & 32.9  & 93.4 \\
  \textbf{CPL-XL} & 78.5  & 64.4  & 48.0  & 33.0 & 163 \\
  \midrule
  \textbf{Cayley (KW3)} & 74.6  & 61.4  & 46.4  & 32.1  & 30.8\\
    \midrule

  \textbf{SOC-10} & 77.6  & 62.0  & 45.0  & 29.5  & 33.4 \\
  \textbf{SOC-20} & 78.0  & 62.7 & 46.0  & 30.3  &52.2\\
  \midrule
    \textbf{SOC+-10} & 76.2 &62.6 & 47.7 & 34.2& N/A\\
  \textbf{SOC+-20} & 76.3&62.6& 48.7& 36.0& N/A \\

  \bottomrule
  \end{tabular}%
  \caption{Results on the CIFAR10 dataset on standard and  provably certifiable accuracies for different values of perturbations $\varepsilon$ on CPL (ours), SOC and Cayley models. The average time per epoch in seconds is also reported in the last column. None of these networks uses Last Layer Normalization.}
  \label{table:c10-comp}%
\end{table*}%

\begin{table*}[tb]
  \centering
  \sisetup{%
    table-align-uncertainty=true,
    separate-uncertainty=true,
    detect-weight=true,
    detect-inline-weight=math
  }
  \begin{tabular}
  {
    l
    S[table-format=2.2]
    S[table-format=2.2]
    S[table-format=2.2]
    S[table-format=2.2]
    S[table-format=2.2]
  }
  \toprule
    & \multicolumn{1}{c}{\textbf{Clean Accuracy}} & \multicolumn{3}{c}{\textbf{Provable Accuracy ($\varepsilon $)}} &  \multicolumn{1}{c}{\textbf{Time per epoch (s)}} 
    \\
    \cmidrule{3-5}
    & \multicolumn{1}{c}{ } & \multicolumn{1}{c}{36/255} & \multicolumn{1}{c}{72/255} &  \multicolumn{1}{c}{108/255} & \multicolumn{1}{c}{\textbf{}} 
    \\
  \midrule
    \textbf{CPL-S} & 44.0  & 29.9  & 19.1  & 11.0  & 22.4 \\
  \textbf{CPL-M} & 45.6  & 31.1  & 19.3  & 11.3 & 40.7 \\
  \textbf{CPL-L} & 46.7  & 31.8 & 20.1  & 11.7  & 93.8 \\
  \textbf{CPL-XL} & 47.8  & 33.4  & 20.9  &  12.6  & 164 \\
  \midrule
  \textbf{Cayley (KW3)} & 43.3  & 29.2  & 18.8 & 11.0  & 31.3 \\
    \midrule

  \textbf{SOC-10} & 48.2  & 34.3  &22.7 & 14.0  & 33.8 \\
  \textbf{SOC-20} & 48.3  & 34.4 & 22.7  & 14.2 & 52.7 \\
  \midrule
    \textbf{SOC+-10} & 47.1& 34.5& 23.5& 15.7& N/A \\
  \textbf{SOC+-20} & 47.8 & 34.8 & 23.7 & 15.8 &  N/A\\

  \bottomrule
  \end{tabular}%
  \caption{Results on the CIFAR100 dataset on standard and  provably certifiable accuracies for different values of perturbations $\varepsilon$ on CPL (ours), SOC and Cayley models. The average time per epoch in seconds is also reported in the last column. All the reported networks use Last Layer Normalization.}
  \label{table:c100-comp}%
\end{table*}%


\subsection{Results}
\label{sec:results}


In this section, we present our results on adversarial robustness.
We provide results on provable $\ell_2$ robustness as well as empirical robustness on CIFAR10 and CIFAR100 datasets for all our models and the concurrent ones

\begin{figure}[h]
    \centering
    \begin{tabular}{cc}
    \includegraphics[width =0.48\textwidth]{sections/4_certification/images/cert_acc_eps_c10.pdf} & \includegraphics[width =0.48\textwidth]{sections/4_certification/images/cert_acc_eps_c100.pdf}
    \end{tabular}
    \caption{Certifiably robust accuracy in function of the perturbation $\varepsilon$ for our CPL networks and its concurrent approaches (SOC and Cayley models) on CIFAR10 and CIFAR100 datasets.}
    \label{fig:cert-acc}
\end{figure}

\paragraph{Certified Adversarial Robustness.} 
Results on CIFAR10 and CIFAR100 dataset are reported respectively in Tables~\ref{table:c10-comp} and~\ref{table:c100-comp}. We also plotted certified accuracy in function of $\varepsilon$ on Figure~\ref{fig:cert-acc}. On CIFAR10, our method outperforms the concurrent approaches in terms of standard and certified accuracies for every level of $\varepsilon$ except SOC+ that uses additional tricks we did not use. On CIFAR100, our method performs slightly under the SOC networks but better than Cayley networks. Overall, our methods reach competitive results with SOC and Cayley layers. 

Note that we observe a small gain using larger and deeper architectures for our models. This gain is less important as $\varepsilon$ increases but the gain is non negligible for standard accuracies. In term of training time, our small architecture (CPL-S) trains very fast compared to other methods, while larger ones are longer to train.


\begin{figure}[h]
    \centering
    \begin{tabular}{cc}
    \includegraphics[width =0.48\textwidth]{sections/4_certification/images/pgd_acc_eps_c10.pdf} & \includegraphics[width =0.48\textwidth]{sections/4_certification/images/pgd_acc_eps_c100.pdf}
    \end{tabular}
    \caption{Accuracy against PGD attack with 10 iterations in function of the perturbation $\varepsilon$ for our CPL networks and its concurrent approaches on CIFAR10 and CIFAR100 datasets.}
    \label{fig:pgd-acc}
\end{figure}
\paragraph{Empirical Adversarial Robustness.} We also reported in Figure~\ref{fig:pgd-acc} the accuracy of all the models against PGD $\ell_2$-attack~\citep{kurakin2016adversarial,madry2017towards} for various levels of $\epsilon$. We used $10$ iterations for this attack. We remark here that our methods brings a large gain of robust accuracy over all other methods. On CIFAR10 for $\varepsilon = 0.8$, the gain of CPL-S over SOC-10 approach is more than $10\%$. For CIFAR100, the gain is about $10\%$ too for $\varepsilon=0.6$. We remark that using larger architectures lead in a more substantial gain in empirical robustness. 

Our layers  only provide an upper bound on the Lipschitz constant, while orthonormal layers as Cayley and SOC are built to exactly preserve the norms. This might negatively influence the certified accuracy since the effective Lipschitz constant is smaller than the theoretical one, hence leading to suboptimal certificates. This might explain why our method performs so well of empirical robustness task.


\section{Conclusion}
In this paper, we presented a new generic method to build $1$-Lipschitz layers.
We leverage the continuous time dynamical system interpretation of Residual Networks and show that using convex potential flows naturally defines $1$-Lipschitz neural networks.
After proposing a parametrization based on Input Convex Neural Networks~\citep{amos2017input}, we  show that our models  reach competitive results in classification and robustness in comparison which other existing $1$-Lipschitz approaches.
We also experimentally show that our layers provide scalable approaches without further regularization tricks to train very deep architectures.

Exploiting the ResNet architecture for devising flows have been an important research topic.
For example, in the context of generative modeling, Invertible Neural Networks~\citep{behrmann2019invertible} and Normalizing Flows~\citep{rezende2015variational, verine2021expressivity} are both import research topic.
More recently, Sylvester Normalizing Flows~\citep{vdberg2018sylvester} or Convex Potential Flows~\citep{huang2021convex} have had similar ideas to this present work but for a very different setting and applications. In particular, they did not have interest in the contraction property of convex flows and the link with adversarial robustness have been under-exploited.


\paragraph{Further work.}
Propoisition~\ref{prop:continuous-lip} suggests to constraint the symmetric part of the Jacobian of $F_t$. We proposed to decompose $F_t$ as a sum of potential gradient and skew symmetric matrix. Finding other parametrizations is an open challenge.
Our models may not express all $1$-Lipschitz functions.
Knowing which functions can be approximated by our CPL layers is difficult even in the linear case (see Appendix~\ref{app:express}). This is  an important question that requires further investigation. 
One can also think of extending  our work by the study of  other dynamical systems. Recent architectures such as Hamiltonian Networks~\citep{greydanus2019hamiltonian} and Momentum Networks~\citep{sander2021momentum} exhibit interesting properties.
Finally, we hope to use similar approaches to build robust Recurrent Neural Networks~\citep{sherstinsky2020fundamentals} and Transformers~\citep{vaswani2017attention}.









% \section{Proofs}

% \subsection{Proof of Proposition~\ref{prop:continuous-lip}}
% \label{proof:continuous-lip}

% \begin{proof}
% Consider the time derivative of the square difference between the two flows $x_t$ and $z_t$ associated with the function $F_t$ and following the definition~\ref{def:flow}: 
% \begin{align*}
%   \frac{d}{dt} \lVert x_t-z_t\rVert_2^2 & = 2 \big\langle x_t-z_t,\frac{d}{dt}( x_t-z_t)\big\rangle\\
%     &=2 \big\langle x_t-z_t,F_{\theta_{t}}(x_{t})-F_{\theta_{t}}(z_{t})\big\rangle \\
%     &=  2 \big\langle x_t-z_t,\int_0^1\nabla_xF_{\theta_{t}}(x_{t}+s(z_t-z_t))(x_t-z_t)ds\big\rangle\textrm{, by Taylor-Lagrange formula}\\ 
%     &=  2 \int_0^1\big\langle x_t-z_t,\nabla_xF_{\theta_{t}}(x_{t}+s(z_t-z_t))(x_t-z_t)\big\rangle ds\\
%      &=  2 \int_0^1\big\langle x_t-z_t,S(\nabla_xF_{\theta_{t}}(x_{t}+s(z_t-z_t)))(x_t-z_t)\big\rangle ds
% \end{align*}
% In the last step, we used that for every skew-symmetric matrix $A$ and vector $x$, $\lVert x,Ax\rVert = 0$.
% Since $\mu_tI\preceq S(\nabla_xF_{\theta_{t}}(x_{t}+s(z_t-y_t)))\preceq  \lambda_tI$, we get
% \begin{align*}
%  2\mu_t \lVert x_t-z_t\rVert_2^2 \leq \frac{d}{dt} \lVert x_t-z_t\rVert_2^2 \leq 2\lambda_t \lVert x_t-z_t\rVert_2^2
% \end{align*}
% Then by Gronwall Lemma, we have
% \begin{align*}
%   \lVert x_0-y_0 \rVert e^{\int_0^t\mu_s ds}\leq \lVert x_t-y_t \rVert\leq \lVert x_0-y_0 \rVert e^{ \int_0^t\lambda_s ds}
% \end{align*}
% which concludes the proof.
% \end{proof}

% \subsection{Proof of Corollary~\ref{cor:conv-skew}}
% \label{proof:conv-skew}
% \begin{proof}

% For all $t,x$, we have $F_t(x) = -\nabla_x f_{t}(x)+A_t x$ ~ so~
% $\nabla_x F_t(x) = -\nabla_x^2 f_{t}(x)+A_t$. Then $S(\nabla_x F_t(x)) =-\nabla_x^2 f_{t}(x)$. Since $f$ is convex, we have $\nabla_x^2 f_{t}(x)\succeq 0$. So by application of Proposition~\ref{prop:continuous-lip}, we deduce $\lVert x_t-y_t \rVert\leq \lVert x_0-y_0 \rVert$ for all trajectories starting from $x_0$ and $y_0$.
% \end{proof}

% \subsection{Proof of Proposition~\ref{prop:discrete_convex_potentials}}
% \label{proof:discrete_convex_potentials}
% \begin{proof}
% With $c_t = \lVert x_t -z_t\rVert_2^2$, we can write:
% \begin{align*}
%    c_{t+\frac12} - c_t = &-2 h_t \big\langle x_t - z_t, \nabla_xF_{\theta_{t}}(x_t) - \nabla_xF_{\theta_{t}}(z_t)  \big\rangle+ h_t^2 \lVert \nabla_xF_{\theta_{t}}(z_t) - \nabla_xF_{\theta_{t}}(z_t)\rVert_2^2
% \end{align*}
% This equality allows us to derive the equivalence between  $c_{t+1} \leq c_t$ and: 
% \begin{align*}
%    \frac{h_t}{2}
%    \lVert  \nabla F_{\theta_{t}}(x_t) - \nabla F_{\theta_{t}}(z_t)\rVert_2^2
%    \leq
%    \langle x_t -z_t, \nabla F_{\theta_{t}}(z_t) - \nabla F_{\theta_{t}}(z_t) \rangle 
% \end{align*}
% Moreover, assuming that $F_{\theta_t}$ being  that:
% \begin{align*}
%    \frac{1}{L_t} &\lVert \nabla_xF_{\theta_{t}}(x_t) - \nabla_xF_{\theta_{t}}(z_t)\rVert_2^2 
%    \leq\big\langle x_t -z_t, \nabla_xF_{\theta_{t}}(x_t) - \nabla_xF_{\theta_{t}}(z_t) \big\rangle
% \end{align*}
% We can see with this last inequality that if we enforce  $h_t \leq \frac{2}{L_t}$, we get $c_{t+\frac12} \leq c_t$ which concludes the proof.
% \end{proof}


% \section{Additional Results}

% % \subsection{Functions whose gradient is skew-symmetric everywhere}
% % \label{sup:skew}
% Let $F:=(F_1,\dots,F_d):\RR^d\to\RR^d$ be a twice differentiable function such that $\nabla F(x)$ is skew-symmetric for all $x\in\RR^d$. Then we have for all $i,j,k$:
% \begin{align*}
%     \partial_i\partial_j F_k =  -\partial_i\partial_k F_j =-\partial_k\partial_i F_j = \partial_k\partial_j F_i = \partial_j\partial_k F_i = -\partial_j\partial_i F_k = -\partial_i\partial_j F_k
% \end{align*}
% So we have $\partial_i\partial_j F_k =0$ and then $F$ is linear: there exists a skew-symmetric matrix $A$ such that $F(x)=Ax$ 


% \subsection{Implicit discrete convex potential flows}
% \label{sup:implicit}

% Let us define the implicit update $x_{t+\frac12} = x_{t}-\nabla_xf_{t}(x_{t+\frac12})$. Let us remark that $x_{t+\frac12}$ is uniquely defined as:
% \begin{align*}
%  x_{t+\frac12} =\argminB_{x\in\RR^d} \frac 12\lVert x-x_t\rVert^2+ f_t(x)  
% \end{align*}
% We recognized here the proximal operator of $f_t$ that is uniquely defined since $f_t$ is convex. Moreover we have for two trajectories $x_t$ and $z_t$:
% \begin{align*}
%    \lVert x_t-z_t\rVert^2_2& =  \lVert x_{t+\frac12}-z_{t+\frac12} +\nabla_xf_{t}(x_{t+\frac12})-\nabla_xf_{t}(z_{t+\frac12}) \rVert^2_2 \\
%    &=  \lVert x_{t+\frac12}-z_{t+\frac12}\rVert^2 + 2\langle x_t-z_t,\nabla_xf_{t}(x_{t+\frac12})-\nabla_xf_{t}(z_{t+\frac12})\rangle +\lVert\nabla_xf_{t}(x_{t+\frac12})-\nabla_xf_{t}(z_{t+\frac12}) \rVert^2_2 \\
%    &\geq  \lVert x_{t+\frac12}-z_{t+\frac12}\rVert^2 
% \end{align*}
% where the last inequality is deduced from the convexity of $f_t$. So, without any further assumption on $f_t$, the discretized implicit convex potential flow is $1$-Lipschitz.

% To compute such a layer, one could solve the proximal operator strongly convex-minimization optimization problem. This strategy is not computationally efficient and not scalable. 




% \subsection{Expressivity of discretized convex potential flows}
% \label{app:express}
% Let us define $\mathcal{S}_1(\RR^{d\times d})$ the space of real symmetric matrices with singular values bounded by $1$. Let us also define $\mathcal{M}_1(\RR^{d\times d})$ the space of real matrices with singular values bounded by $1$ in absolute value.
% Let $\mathcal{P}(\RR^{d\times d})=\{A\in\RR^{d\times d}|\exists n\in \mathbb{N},S_1,\dots,S_n\in \mathcal{S}_1(\RR^d\times d)\text{ s.t. } A = S_1\dots S_n\}$. Then one can prove\footnote{A proof and justification of this result can be found here: \url{https://mathoverflow.net/questions/60174/factorization-of-a-real-matrix-into-hermitian-x-hermitian-is-it-stable}} that $\mathcal{P}(\RR^{d\times d}) \neq \mathcal{M}_1(\RR^{d\times d})$. Thus there exists $A\in\mathcal{M}_1(\RR^{d\times d})$ such that for all matrices $n$, for all matrices $S_1,\dots,S_n\in\mathcal{S}_1(\RR^{d\times d})$ such that $M\neq S_1,\dots,S_n$. 

% Applied to the expressivity of discretized convex potential flows, the previous result means that there exists a $1$-Lipschitz linear function that cannot be approximated as a discretized flow of any depth of convex linear $1$-smooth potential flows as in Proposition~\ref{prop:discrete_convex_potentials}. Indeed such a flow would write: $x\mapsto\prod_i(1-2S_i)x$ where $S_i$ are symmetric matrices whose eigenvalues are in $[0,1]$, in other words such transformations are exactly described by $x\mapsto Mx$  for some $M\in\mathcal{P}(\RR^{d\times d})$.




% \section{Additional experiments}
% \label{app:xp-supp}

% \subsection{Training stability: scaling up to $1000$ layers}

% \begin{figure}[h]
%     \centering
%     \includegraphics[width=0.5\textwidth]{sections/4_certification/images/final_cifar10_veryverydeep.pdf}
%     \caption{Standard test accuracy in function of the number of epochs (log-scale) for various depths for our neural networks ($100,300,500,700,1000$).}
%     \label{fig:verydeep}
% \end{figure}

% While the Residual Network architecture limits, by design, gradient vanishing issues, it still suffers from exploding gradients in many cases~\citep{hayou2021stable}.
% To prevent such scenarii, batch normalization layers~\citep{ioffe2015batch} are used in most Residual Networks to stabilize the training.

% Recently, several works~\citep{miyato2018spectral,farnia2018generalizable} have proposed to normalize the linear transformation of each layer by their spectral norm.
% Such a method would limit exploding gradients but would again suffer from gradient vanishing issues.
% Indeed, spectral normalization might be too restrictive: dividing by the spectral norm can make other singular values vanishingly
% small.
% While more computationally expensive (spectral normalization can be done with $1$ Power Method iteration), orthogonal projections prevent both exploding and vanishing issues. 

% On the contrary the architecture proposed in this paper has the advantage to naturally control the gradient norm of the output with respect to a given layer.
% Therefore, our architecture can get the best of both worlds: limiting exploding and vanishing issues while maintaining scalability. 
% To demonstrate the scalability of our approach, we experiment the ability to scale our architecture to very high depth (up to 1000 layers) without any additional normalization/regularization tricks, such as Dropout~\citep{srivastava2014dropout}, Batch Normalization~\citep{ioffe2015batch} or gradient clipping~\citep{pascanu2013difficulty}.
% With the work done by~\cite{xiao2018dynamical}, which leverage Dynamical Isometry and a Mean Field Theory to train a $10000$ layers neural network, we believe, to the best of our knowledge, to be the second to perform such training. 
% For sake of computation efficiency, we limit this experiment to architecture with $30$ feature maps.
% We report the accuracy in terms of epochs for our architecture in Figure~\ref{fig:verydeep} for a varying number of convolutional layers.
% It is worth noting that for the deepest networks, it may take a few epochs before the start of convergence.
% As \cite{xiao2018dynamical}, we remark there is no gain in using very deep architecture for this task.


% \subsection{Relaxing linear layers}

% \begin{center}
% \begin{tabular}{lrrr}
% \toprule
%   & \multicolumn{1}{c}{\textbf{h = 1.0}} & \multicolumn{1}{c}{\textbf{h = 0.1}} & \multicolumn{1}{c}{\textbf{h = 0.01}} \\
% \midrule
% \textbf{Clean} & 85.10 & 82.23 & 78.53 \\
% \textbf{PGD ($\varepsilon = 36/255$}) & 61.45 & 62.99 & 60.98 \\
% \bottomrule
% \end{tabular}%
% \end{center}
% The table above shows the result of the relaxed training of our StableBlock architecture, i.e. we fixed the step $h_t$ in the discretized convex potential flow of Proposition~\ref{prop:discrete_convex_potentials}.
% Increasing the constant $h$ allows for an important improvement  in the clean accuracy, but we loose in robust empirical accuracy.
% While computing the certified accuracy is not possible in this case due to the unknown value of the Lipschitz constant, we can still notice that the training of the network are still stable without normalization tricks, and offer a non-negligible level of robustness. 





% \subsection{Effect of Batch Size in Training}


% \begin{table*}[h]
%   \centering
%   \sisetup{%
%     table-align-uncertainty=true,
%     separate-uncertainty=true,
%     detect-weight=true,
%     detect-inline-weight=math
%   }
%   \begin{tabular}
%   {
%     l
%     S[table-format=2.2]
%     S[table-format=2.2]
%     S[table-format=2.2]
%     S[table-format=2.2]
%     S[table-format=2.2]
%     S[table-format=2.2]
%   }
%   \toprule
%   &\multicolumn{1}{c}{\textbf{Batch size }}& \multicolumn{1}{c}{\textbf{Clean Accuracy}} & \multicolumn{3}{c}{\textbf{Provable Accuracy ($\varepsilon $)}} &  \multicolumn{1}{c}{\textbf{Time per epoch (s)}} 
%     \\
%     \cmidrule{4-6}
%     & \multicolumn{1}{c}{ } & \multicolumn{1}{c}{ } &\multicolumn{1}{c}{36/255} & \multicolumn{1}{c}{72/255} &  \multicolumn{1}{c}{108/255} & \multicolumn{1}{c}{\textbf{}} 
%     \\
%   \midrule
%     \multirow{3}{*}{\textbf{CPL-S}} & 64 & 76.5& 62.9 & 47.3 & 32.0 & 48 \\
%                                     & 128 & 76.1 & 62.8 & 47.1 & 32.3  & 31 \\
%                                     & 256 & 75.6 & 62.3 & 46.9 & 32.2 & 22 \\
%     \midrule
%  \multirow{3}{*}{\textbf{CPL-M}} & 64 & 77.4 & 63.6 & 47.4 & 32.1  & 77 \\
%                                     & 128 & 77.2 &63.5 & 47.5 & 32.1 & 50 \\
%                                     & 256 & 76.8 & 63.2 & 47.4 & 32.4& 40 \\
%     \midrule

%  \multirow{3}{*}{\textbf{CPL-L}} & 64 & 78.4 & 64.2 & 47.8 & 32.2  & 162 \\
%                                     & 128 & 78.2 & 64.3 & 47.9 & 32.5 & 109 \\
%                                     & 256 & 77.6 & 63.9 & 48.1 & 32.7& 93 \\
%   \midrule
%  \multirow{3}{*}{\textbf{CPL-XL}} & 64 & 78.9 & 64.2 & 47.2 & 31.2  & 271 \\
%                                     & 128 & 78.9 & 64.2 & 47.5 & 31.8 & 198 \\
%                                     & 256 &78.5 & 64.4 & 47.8 & 32.4& 163 \\

%   \bottomrule
%   \end{tabular}%
%   \caption{Results on the CIFAR10 dataset on standard and  provably certifiable accuracies for different values of perturbations $\varepsilon$ on CPL (ours) models with various batch sizes. The average time per epoch in seconds is also reported in the last column. All the reported networks use Last Layer Normalization.}
%   \label{table:c10-comp-bs}%
% \end{table*}%



% \begin{table*}[h]
%   \centering
%   \sisetup{%
%     table-align-uncertainty=true,
%     separate-uncertainty=true,
%     detect-weight=true,
%     detect-inline-weight=math
%   }
%   \begin{tabular}
%   {
%     l
%     S[table-format=2.2]
%     S[table-format=2.2]
%     S[table-format=2.2]
%     S[table-format=2.2]
%     S[table-format=2.2]
%     S[table-format=2.2]
%   }
%   \toprule
%   &\multicolumn{1}{c}{\textbf{Batch size }}& \multicolumn{1}{c}{\textbf{Clean Acc.}} & \multicolumn{3}{c}{\textbf{Provable Acc. ($\varepsilon $)}} &  \multicolumn{1}{c}{\textbf{Time per epoch (s)}} 
%     \\
%     \cmidrule{4-6}
%     & \multicolumn{1}{c}{ } & \multicolumn{1}{c}{ } &\multicolumn{1}{c}{36/255} & \multicolumn{1}{c}{72/255} &  \multicolumn{1}{c}{108/255} & \multicolumn{1}{c}{\textbf{}} 
%     \\
%   \midrule
%   \multirow{3}{*}{\textbf{CPL-S}} & 64 & 45,6 & 30,8 & 19,3 & 11,2 & 47 \\
%                                     & 128 & 44,9 & 30,7 & 19,2 & 11,0 & 31\\
%                                     & 256 & 44,0 & 29,9 & 19,1 & 10,9 & 23\\
%   \midrule
%   \multirow{3}{*}{\textbf{CPL-M}} & 64 & 46.6 & 31,6 & 19,6 & 11,6 & 78 \\
%                                     & 128 & 46.3 & 31,1 & 19,7 & 11,5 & 55 \\
%                                     & 256 & 45.6 & 31,1 & 19,3 & 11,3 & 41 \\

%   \midrule

%   \multirow{3}{*}{\textbf{CPL-L}} & 64 & 48.1 & 32,7 & 20,3 & 11,7 & 163 \\ 
%                                     & 128 & 47,4 & 32,3 & 20,0 & 11,8 & 116 \\ 
%                                     & 256 & 46,8 & 31,8 & 20,1 & 11,7 & 95 \\ 
%   \midrule
%   \multirow{3}{*}{\textbf{CPL-XL}} & 64 & 49,0 & 33,7 & 21,1 & 12,0 & 293 \\
%                                     & 128 & 48,0 & 33,7 & 21,0 & 12,1 & 209 \\
%                                     & 256 &47,8 & 33,4 & 20,9 & 12,6 & 164 \\

%   \bottomrule
%   \end{tabular}%
%   \caption{Results on the CIFAR100 dataset on standard and  provably certifiable accuracies for different values of perturbations $\varepsilon$ on CPL (ours) models with various batch sizes. The average time per epoch in seconds is also reported in the last column. All the reported networks use Last Layer Normalization.}
%   \label{table:c100-comp-bs}%
% \end{table*}%

% In Tables~\ref{table:c10-comp-bs} and~\ref{table:c100-comp-bs}, we tried three different batch sizes (64, 128 and 256) for training our networks on CIFAR10 and CIFAR100 datasets, we remark a gain in standard accuracy in reducing the batch size for all settings. As the perturbation becomes larger, the gain in accuracy is reduced and can even in some cases we may loose some points in robustness.

% \subsection{Effect of the Margin Parameter}

% \begin{figure}[h]
%     \centering
%     \begin{tabular}{cc}
%     \includegraphics[width=0.49\textwidth]{sections/4_certification/images/cert_acc_margin_eps_c10.pdf}&\includegraphics[width=0.49\textwidth]{sections/4_certification/images/cert_acc_margin_eps_c100.pdf}
%     \end{tabular}
%     \caption{Certifiably robust accuracy in function of the perturbation $\varepsilon$ for our CPL-S  network with different margin parameters on CIFAR10 and CIFAR100 datasets.}
%     \label{fig:cert-acc-margin}
% \end{figure}


% \begin{figure}[h]
%     \centering
%     \begin{tabular}{cc}
%     \includegraphics[width=0.49\textwidth]{sections/4_certification/images/pgd_acc_margin_eps_c10.pdf}&\includegraphics[width=0.49\textwidth]{sections/4_certification/images/pgd_acc_margin_eps_c100.pdf}
%     \end{tabular}
%     \caption{Certifiably robust accuracy in function of the perturbation $\varepsilon$ for our CPL-S  network with different margin parameters on CIFAR10 and CIFAR100 datasets.}
%     \label{fig:pgd-acc-margin}
% \end{figure}

% In these experiments we varied the margin parameter in the margin loss in Figures~\ref{fig:cert-acc-margin} and~\ref{fig:pgd-acc-margin}. It clearly exhibits a tradeoff between standard and robust accuracy. When the margin is large, the standard accuracy is low, but the level of robustness remain high even for ``large'' perturbations. On the opposite, when the margin is small, we get a high standard accuracy but we are unable to keep a good robustness level as the perturbation increases. It is verified both on certified and empirical robustness.




\chapter{Conclusion}
\minitoc
\section{Summary of the Thesis}

In this thesis, we studied the problem of  classication in presernce of adversaries from different point of views for theoretical and practical finalities. We have tried to analyze the problem using both a high level
and a more precise analysis. We summarize our findings as follows.
\begin{tcolorbox}[colback=grund,colframe=rahmen,title=Summary of contributions]
\begin{enumerate}
    \item We provide a better understanding of the adversarial problem studying the nature of equilibria in this game. We proved the existence of mixed Nash equilibria for very general assumptions. We hope this research directions will lead to principled results that can be used in practice for better defending against adversarial examples.
    \item We studied and closed the problem of calibration in the adversarial binary-classification setting providing necessary and sufficient conditions. We paved a way to prove consistency results, and hope being able to conclude on consistency of shifted odd losses. It remains to find necessary and sufficient conditions for consistency.
    \item We derived a principled way based on dynamical system to build $1$-Lipschitz layers. Interestingly, we recovered some existing methods from the literature, but we were also able to build new interesting layers, namely the Convex Potential Layers. We hope this work would lead to study other possible dynamical systems and provide new provably robust neural networks.
\end{enumerate}
\end{tcolorbox}
Although this thesis proposed some solutions to the adversarial problem, we also opened many questions that would require further investigation. 
\section{Open Questions}



\subsection{Optimizing the Adversarial Attacks problem}

The optimization of the adversarial attacks problem is an open from multiple point of views.
The adversarial risk minimization problem writes
\begin{align*}
    \inf_{h\in\mathcal{H}} \mathbb{E}_{(x,y)\sim\PP}\left[\sup_{x'\in\XX\mid~d(x,x')\leq\varepsilon}\loss(h(x),y)\right]
\end{align*}
In classification, the end-objective is the accuracy, hence one need to optimize the $0/1$ loss. However, optimizing the $0/1$ loss is not computationally tractable. In the adversarial setting, the choice of a good surrogate loss $\loss$ to the $0/1$ loss  is a difficult question. In particular, we have shown that no convex losses can be a good surrogate in Chapter~\ref{chap:calibration}.   We paved a way that tends to show there might exist continuous and differentiable losses that are consistent with regards to the $0/1$ loss, but it is still an open problem.
\begin{tcolorbox}[colback=grund,colframe=rahmen]
    \begin{center}
        \emph{Does there exist a simple principled way to train the adversarial attacks problem for both the classifier and the attacker?}
    \end{center}
\end{tcolorbox}
Since no convex loss can be a good surrogate for the adversarial classification problem, the optimization of a suitable empirical risk  would be a  non-convex optimization problem which is misunderstood. The difficulty of this problem is also highlighted by the inner supremum which also non-convex. Then there is still a gap to bridge to understand the optimization of the adversarial problem. 
In Chapter~\ref{chap:game}, we proposed the following adversarial problem where the classifier and the attacker are 
\begin{align*}
    \inf_{\mu \in\mathcal{M}_1^+(\mathcal{H})}\sup_{\QQ\in\mathcal{A}_\varepsilon(\PP)} \mathbb{E}_{\mu\sim\mathcal{H},(x,y)\sim\QQ}\left[\loss(h(x),y)\right]
\end{align*}
This naturally leads to understand the adversarial problem as game between the attacker and the classifier with utility $\mathbb{E}_{\mu\sim\mathcal{H},(x,y)\sim\QQ}\left[L(h(x),y)\right]$. We show the existence of Nash equilibria for this game in Chapter~\ref{chap:game}. Although, we propose a way to learn the optimal mixtures of classifiers when their number is finite, the question of  computating equilibria has not been studied and would be a natural further step. On one hand, it would help building a robust classifier against every attacks in $\mathcal{A}_\varepsilon(\PP)$, and on the other hand, the  attacker that would be build would robust to change in the mixture of classifiers. This problem is a min-max optimization problem over the state of distributions, hence a difficult problem. Although the problem writes as a convex-concave problem over the space of distributions, the utility is not geodisically convex-concave in the Wassertein-2 space. Applying directly results on Wasserstein Gradient Flow is not possible. Deriving a tractable algorithm with convergence guarantees seems difficult. There have been some attempts by the Machine Learning community to understand find mixed Nash equilibri aby optimizing  using optimization over distribution techniques~\citep{hsieh2019finding,domingo2020mean} with applications to Generative Adversarial Networks for instance. Understanding and finding equilibria in games  in Machine Learning as the adversarial attacks problem and GANs is essential for the community to understand better these problems. 

% As proven in Chapter~\ref{chap:game} we can restate it as  




% Beyond equilibria questions, it raises multiple questions. 
\subsection{Understansing the Learning Theory in the Adversarial Setting}
The learning theory have focused on analysing what can be infered on the error outside the training set, often called generalization error. To analyse it, the risk is decomposed with bias-complexity. This bias complexity tradeoff has been question recently by double descent phenomenon~\citep{belkin2018overfitting,belkin2019reconciling}, suggesting that higher complexity models might lead to lower generalization errors. These recent findings underline the lack of understanding we have about generalization of Neural Networks.

Analysing generalization in the adversarial setting case is an underdeveloped question. There have been some works using Rademacher complexity~\citep{yin2019rademacher,awasthi2020adversarialx} to craft uniform convergence bounds. But, to our knowledge, very few works have focused on  understanding the bias-complexity tradeoff in the adversarial case.

\begin{tcolorbox}[colback=grund,colframe=rahmen]
    \begin{center}
    \emph{How does statistical generalization work in the adversarial attacks setting?}
    \end{center}
\end{tcolorbox}

This problem can be attacked from different angles. First, understanding the need or not of randomization for obtaining optimally robust classifiers is an important problem. From Chapter~\ref{chap:game} and~\citet{pydi2021many}, the answer of this question depends mostly on two things: the set of hypotheses $\mathcal{H}$ and the distribution $\PP$. If $\mathcal{H}$ is small and cannot be optimal for $\PP$, there might be an interest for randomization, while when it is  complex and sufficiently expressice, for instance the set of measurable functions~\citep{pydi2021many}, there is no need for randomization. 

However, choosing complex set of hypothesese might lead to overfitting, justifying the need of understanding generalization properties of randomized classifiers in the adversarial setting. While the question of uniform convergence bounds have been treated, generalization of randomized classifiers in the adversaial setting has only been partly tackled by~\citet{viallard2021pac} under the PAC-Bayes framework~\citep{guedj2019primer} . 

Beyond PAC-like bounds,  convergence rates of optimal classifiers in the adversarial setting like it was done by~\citet{fischer2020sobolev} in the case of kernel least squares regression in an important and not studied problem. Even the question of the choice of the norm for the convergence is difficult since the adversarial settigs involves points outside the support of the distribution. 



\subsection{Scaling Provably Robust Neural Networks}

In Chapter~\ref{chap:calibration}, we provide a general method to build provably Lipschitz layers. However, every single methods only lead to limited results on CIFAR10 dataset~\citep{cifar-10} with standard accuracies under $80\%$ and certifiable accuracies under $65\%$ for $\varepsilon=36/255$. The performances are far under the state-of-the-art on CIFAR10 standard classification task ($>95\%$). There is still a huge gap we need to bridge to have performant certifiably robust neural networks. Since we are unable to reach decent performances on simple datasets, the question of being robust on larger datasets as ImageNet~\citep{imagenet_cvpr09} is a bit anticipated.

\begin{tcolorbox}[colback=grund,colframe=rahmen]
    \begin{center}
        \emph{Is it possible to build non-vacuous certifiable neural networks on highly-dimensional large-scale datasets?}   
    \end{center}

\end{tcolorbox}

Building robust neural networks with deterministic non-vacuous guarantees is an active research area and one may w. Current methods that scale on ImageNets relies on non-deterministic bounds using for instance randomized smoothing~\citep{KolterRandomizedSmoothing,salman2019provably}. The advantages and the weaknesses of these methods rely in the same fact. While deterministic methods highly rely on the structure of the networks, randomized smoothing methods are agnostic to the structure of neeural Networks. One may hope using the structure of deep neural networks to get provable strategies. The question of robustness is also understudied for the recent Transformers~\citep{vaswani2017attention} neural networks whose basic element is an attention block:
\begin{align*}
    \text{Attention}(Q,K,V) = \text{softmax}\left(\frac{QK^T}{\sqrt{d_K}}\right)V
\end{align*}
where $d_K$ is the common dimension of $Q$ and $K$. The Transformers architectures are today state of the art both in NLP tasks~\citep{devlin2018bert} and computer vision tasks~\citep{dosovitskiy2020image}. Due to the recency of these approaches, their robustness have not been investigated. This question definitely worth more attention! 


Beyond, this question of scalability and robustness of architectures, one may ask the question of the feasability of such tasks. Enforcing Lipschitz constraints on the networks may hinder the networks performances. In complex datasets like ImageNet, it might not be possible to get simultaneously nice performances and non-vacuous certificates. Moreover, the proposed defenses often rely on a single norm, often the $\ell^2$. Designing networks that are ``universarlly'' robust for human perception  is an utopia, that we may never reach.  




% \subsection{Understanding Randomization in Adversarial Classification}

% \begin{itemize}
%     \item Statistical Bounds for Adversarial Robustness in the Case of Randomized Classifiers
%     \item Designing an Algorithm for computing Nash Equilibria in the General Case
% \end{itemize}


% \subsection{Loss Calibration General Results}

% \begin{itemize}
%     \item The non realisable case is difficult: showing either negative/positive general results
%     \item Further developing the margin loss analysis
% \end{itemize}
% \subsection{Exploiting the architecture of Neural Networks to get Guarantees}
% \begin{itemize}
%     \item Exploiting Helmoltz decomposition of flows
%     \item Exploiting other flows (Hamiltonian, Momentum, etc.)
% \end{itemize}








 
\appendix
% \begin{appendices}
\chapter{Additional Lemmas}
\section{Useful Lemmas for Chapter~\ref{chap:game}}

\begin{lemma}[Fubini's theorem]
\label{lem:fubini}
Let $l:\Theta\times(\mathcal{X}\times\mathcal{Y})\rightarrow [0,\infty)$ satisfying Assumption~\ref{ass:loss}. Then for all $\mu\in\mathcal{M}^1_+(\Theta)$, $\int l(\theta,\cdot)d\mu(\theta)$ is Borel measurable; for  $\QQ\in\mathcal{M}^1_+(\mathcal{X}\times\mathcal{Y})$, $\int l(\cdot,(x,y))d\QQ(x,y)$ is Borel measurable. Moreover: $\int l(\theta,(x,y))d\mu(\theta)d\QQ(x,y)=\int l(\theta,(x,y))d\QQ(x,y)d\mu(\theta)$
\end{lemma}

\begin{lemma}
\label{lem:usc1}
Let $l:\Theta\times(\mathcal{X}\times\mathcal{Y})\rightarrow [0,\infty)$ satisfying Assumption~\ref{ass:loss}.
Then for all $\mu\in\mathcal{M}^1_+(\Theta)$, $(x,y)\mapsto\int l(\theta,(x,y))d\mu(\theta)$ is upper semi-continuous and hence Borel measurable.  
\end{lemma}
\begin{proof}
Let $(x_n,y_n)_n$ be a sequence of $\mathcal{X}\times\mathcal{Y}$ converging to $(x,y)\in\mathcal{X}\times\mathcal{Y}$.  For all $\theta\in\Theta$, $M-l(\theta,\cdot)$ is non negative and lower semi-continuous. Then by Fatou's Lemma applied:
\begin{align*}
   \int M-l(\theta,(x,y))d\mu(\theta)&\leq\int \liminf_{n\to\infty}  M-l(\theta,(x_n,y_n))d\mu(\theta)\\
   &\leq  \liminf_{n\to\infty}  \int M-l(\theta,(x_n,y_n))d\mu(\theta) 
\end{align*}

Then we deduce that: $\int M- l(\theta,\cdot)d\mu(\theta)$ is lower semi-continuous and then $\int l(\theta,\cdot)d\mu(\theta)$ is upper-semi continuous.
\end{proof}


\begin{lemma}
\label{lem:usc2}

Let $l:\Theta\times(\mathcal{X}\times\mathcal{Y})\rightarrow [0,\infty)$ satisfying Assumption~\ref{ass:loss}
Then for all $\mu\in\mathcal{M}^1_+(\Theta)$, $\QQ\mapsto\int l(\theta,(x,y))d\mu(\theta)d\QQ(x,y)$ is upper semi-continuous for weak topology of measures. 
\end{lemma}
\begin{proof}
 $-\int l(\theta,\cdot)d\mu(\theta) $ is lower semi-continuous from Lemma~\ref{lem:usc1}. Then $M-\int l(\theta,\cdot)d\mu(\theta) $ is lower semi-continuous and non negative. Let denote $v$ this function. Let $(v_n)_n$ be a non-decreasing sequence of continuous bounded functions such that $v_n\to v$. Let $(\QQ_k)_k$ converging weakly towards $\QQ$. Then by monotone convergence:
 
 \begin{align*}
     \int vd\QQ = \lim_n \int v_nd\QQ =\lim_n \lim_k\int v_nd\QQ_k\leq \liminf_k \int vd\QQ_k
 \end{align*}
 Then $\QQ\mapsto\int vd\QQ$ is lower semi-continuous and then $\QQ\mapsto\int l(\theta,(x,y))d\mu(\theta)d\QQ(x,y)$ is upper semi-continuous for weak topology of measures. 
 \end{proof}



\begin{lemma}
\label{lem:measure-sup}
Let $l:\Theta\times(\mathcal{X}\times\mathcal{Y})\rightarrow [0,\infty)$ satisfying Assumption~\ref{ass:loss}.
Then for all $\mu\in\mathcal{M}^1_+(\Theta)$, $(x,y)\mapsto \sup_{(x',y'),d(x,x')\leq\varepsilon,y=y'} \int l(\theta,(x',y'))d\mu(\theta)$ is universally measurable (i.e. measurable for all Borel probability measures). And hence the adversarial risk is well defined. 
\end{lemma}
\begin{proof}
Let $\phi :(x,y)\mapsto \sup_{(x',y'),d(x,x')\leq\varepsilon,y=y'} \int l(\theta,(x',y'))d\mu(\theta)$. Then for $u\in\bar{\mathbb{R}}$:
\begin{align*}
\left\{\phi(x,y)>u\right\}=\text{Proj}_1\left\{((x,y),(x',y'))\mid\int l(\theta,(x',y'))d\mu(\theta)-c_\varepsilon((x,y),(x',y'))>u\right\}
\end{align*}
By Lemma~\ref{lem:usc2}: $((x,y),(x',y'))\mapsto \int l(\theta,(x',y'))d\mu(\theta)-c_\varepsilon((x,y),(x',y'))$ is upper-semicontinuous hence Borel measurable. So its level sets are Borel sets, and by~\citep[Proposition 7.39]{bertsekas2004stochastic}, the projection of a Borel set is analytic. And then $\left\{\phi(x,y)>u\right\}$ universally measurable thanks to~\citep[Corollary 7.42.1]{bertsekas2004stochastic}. We deduce that $\phi$ is universally measurable.
\end{proof}


\chapter{Adversarial Attacks on Linear Contextual Bandits}
\newcommand{\mtb}[1]{\textcolor{black}{#1}}
\newcommand{\otc}[1]{\textcolor{black}{#1}}
\newcommand{\changede}[1]{\textcolor{black}{#1}}
\newcommand{\changebr}[1]{\textcolor{black}{#1}}
\newcommand{\changelm}[1]{\textcolor{black}{#1}}
\newcommand{\changebrtwo}[1]{\textcolor{black}{#1}}
\newcommand{\changee}[1]{\textcolor{black}{#1}}
\newcommand{\cmmnt}[1]{\ignorespaces}

\chapter{Adversarial Attacks on Linear Contextual Bandits}
\label{paper:banditsattacks}
Contextual bandit algorithms are applied in a wide range of domains, from advertising to recommender systems, from clinical trials to education. In many of these domains, malicious agents may have incentives to {force a bandit algorithm into a desired behavior.} %attack the bandit algorithm to induce it to perform a desired behavior.
For instance, an unscrupulous ad publisher may try to increase their own revenue at the expense of the advertisers; a seller may want to increase the exposure of their products, or thwart a competitor's advertising campaign.
In this paper, we study several attack scenarios and show that a malicious agent can force a linear contextual bandit algorithm to pull any desired arm $T - o(T)$ times over a horizon of $T$ steps, while applying adversarial modifications to either rewards or contexts {with a cumulative cost} that only grow logarithmically as $O(\log T)$.
We also investigate the case when a malicious agent is interested in affecting the behavior of the bandit algorithm in a single context (e.g., a specific user). We first provide sufficient conditions for the feasibility of the attack and %we then propose 
an efficient algorithm to perform an attack. %test
{We empirically validate the proposed approaches in synthetic and real-world datasets.} %o  ur theoretical results on experiments performed on both synthetic and real-world datasets.





% !TEX root = main.tex
\section{Introduction}

% \todol{split the intro in two parts or make subtitles}

Recommender systems are at the heart of the business model of many industries like e-commerce or video streaming~\cite{davidson2010youtube,gomez2015netflix}. The two most common approaches for this task are based either on matrix factorization \cite{park2017comparative} or bandit algorithms \cite{li2010contextual}, which both
rely on a unaltered feedback loop between the recommender system and the user. In recent years, a fair amount of work has been dedicated to understanding how targeted perturbations in the feedback loop can fool a recommender system into recommending low quality items.

Following the line of research on adversarial attacks in supervised learning \cite{biggio2012poisoning,goodfellow2014explaining, jagielski2018manipulating, li2016data, liu2017robust}, attacks on recommender systems have been focused on filtering-based algorithms \cite{10.1145/3298689.3347031, mehta2008attack} and offline contextual bandits \cite{ma2018data}.
 % studying the effects of poisoning a dataset used by a learning algorithm. For recommender systems, those works are related to the study of attacks on Collaborative
The question of adversarial attacks for online bandit algorithms %(that is to say learning on the fly and not in batch)
has only \cmmnt{started being} \changee{been studied} quite recently \cite{jun2018adversarial, liu2019data, Immorlica2018AdversarialBW, guan2020robust}, and solely in the multi-armed stochastic setting.
Although the idea of online adversarial bandit algorithms is not new (see \expthree algorithm in \cite{auer2002finite}), the focus is different from what we are considering \changee{in this article}. Indeed, algorithms like \expthree or \expfour \cite{lattimore2018bandit} are designed to find optimal actions in hindsight in order to adapt to any rewards stream\cmmnt{ without any further assumptions}. 


 The opposition between adversarial and stochastic bandit settings has sparked interests in studying a middle ground.
 In \cite{bubeck2012best}, the learning algorithm has no knowledge of the type of feedback it receives \changee{(either stochastic or adversarial)}. In \cite{lykouris2018stochastic, li2019stochastic, gupta2019better, Lykouris2019CorruptionRE, kapoor2019corruption}, the rewards are assumed to be \changee{corrupted by adversarial rewards}\cmmnt{ but can be perturbed by some attacks}. The authors focus on \changee{building} algorithms able to find the optimal actions even in the presence of some non-random perturbations. \changee{This setting is different from what is studied in this article because} those perturbations are bounded and agnostic to \changee{arms pulled} by the learning algorithm, \changee{i.e., the adversary corrupt the rewards before the algorithm chooses an arm.}

% In addition, the opposition between the adversarial bandit setting and the stochastic setting %more classical one where the eward are assumed to be drawn from a stationary random distribution
% has sparked interests in studying a middle ground where the learning algorithm
% has no knowledge of the type of feedback it receives \citep[e.g.,][]{bubeck2012best} and %However, the recent growth in the use of bandit algorithms has
% % shown that the stochastic assumption is mostly true (up to some uncontrolled perturbations) in an overwhelming number of cases, the study of this setting where
% where rewards are assumed to be stochastic but can be perturbed by some attacks %has given birth to a very recent line of work
% \cite{li2019stochastic, gupta2019better, Lykouris2019CorruptionRE, kapoor2019corruption}. \changede{Those last works being focused} on constructing algorithms able to find the optimal actions even in the presence of some non-reward perturbations but bounded and agnostic to the choices of the learning algorithm.
% \todot{Previous sentence is not clear. You are referring to second work not ``best of both worlds'', right?}
In the broader Deep Reinforcement Learning (DRL) literature, the focus is placed on modifying the observations of different states to fool a DRL system at inference time
\cite{hussenot2019targeted, sunstealthy} \changee{or the rewards \cite{ma2019policy}.}
\paragraph{Contribution.}In this work, we first follow the research direction opened by \cite{jun2018adversarial} where the attacker has the objective of fooling a learning algorithm into taking a specific action as much as possible. \changee{For example} \cmmnt{Consider } in a news recommendation problem, as described in \cite{li2010contextual}, a bandit algorithm chooses between $K$ articles to recommend to a user, based on some information about them, \changee{called} context. We assume that an attacker sits between the user and the website, they can choose the reward (i.e., click or not) for the recommended article observed by the recommending algorithm. Their goal is to fool the bandit algorithm into recommending \changee{some articles} \cmmnt{\changelm{a particular or a set of target articles}} to most users. The contributions of our work can be summarized as follows:
\begin{itemize}
    \item We extend the work of~\cite{jun2018adversarial, liu2019data} to the contextual linear bandit setting showing how to perturb rewards for both stochastic and adversarial algorithms, forcing \textbf{any} bandit algorithms to pull a specific set of arms, $o(T)$ times for logarithmic cost for the attacker.
    \item We analyze, for the first time, the setting in which the attacker can only modify the context $x$ associated with the current user (the reward is not altered). The goal of the attacker is to fool the bandit algorithm into pulling arms of a target set for most users (\ie contexts) while minimizing the total norm of their attacks. We show that the widely known \linucb algorithm \cite{abbasi2011improved, lattimore2018bandit} is vulnerable to this new type of attack.
    \item We present a harder setting for the attacker, where the latter can only modify the context associated to a specific user. This situation may occur when a malicious agent has infected some computers with a Remote Access Trojan (RAT). The attacker can then modify the history of navigation of a specific user and, as a consequence, the information seen by the online recommender system.We show how the attacker can attack the two very common bandit algorithms \linucb and Linear Thompson Sampling (\lints) \cite{agrawal2013thompson,abeille2017linear} and, in certain cases, force them to pull a set of arms most of the time \changebr{when} a specific context (\ie user) is presented to the algorithm (\ie visits a website). 
\end{itemize}




% !TEX root = main.tex
\section{Preliminaries}\label{sec:preliminaries}
We consider the standard contextual linear bandit setting with $K\in \mathbb{N}$ arms. At each time $t$, the agent observes a context $x_{t}\in\mathbb{R}^{d}$, selects an action $a_{t}\in \llbracket 1, K\rrbracket$ and observes a reward: $r_{t,a_{t}} = \langle \theta_{a_{t}}, x_{t}\rangle + \eta_{a_{t}}^{t}$ where for each arm $a$, $\theta_{a}\in \mathbb{R}^{d}$ is a feature vector and $\eta_{a_{t}}^{t}$ is a conditionally independent zero-mean, $\sigma^{2}$-subgaussian noise.  \changee{The contexts are assumed to be sampled \textit{stochastically} except in App.~\ref{app:adversarial_rewards}.} \cmmnt{We also make the following assumptions on the contexts and parameter vectors.}
\begin{assump}\label{assumption1}
There exist $L>0$ and $\mathcal{D}\subset \mathbb{R}^{d}$, such that for all $t$, $x_{t}\in\mathcal{D}$ and, $\forall x\in\mathcal{D},\forall a\in\llbracket 1, K\rrbracket,~ \|x\|_{2} \leq L \text{ and } \langle \theta_{a}, x\rangle \in (0,1]$. In addition, we assume that there exists $S>0$ such that $\|\theta_{a}\|_{2}\leq S$ for all arms $a$.
\end{assump}
The agent minimizes the cumulative regret after $T$ steps $R_{T} = \sum_{t=1}^{T} \langle \theta_{a^{\star}_{t}}, x_{t}\rangle - \langle \theta_{a_{t}}, x_{t}\rangle$,
where $a_{t}^{\star} := \argmaxB_{a} \langle \theta_{a}, x_{t}\rangle$.
A bandit learning algorithm $\mathfrak{A}$ is said to be \emph{no-regret} when it satisfies $R_{T} = o(T)$, i.e., the average expected reward received by $\mathfrak{A}$ \changebr{converges} to the optimal one. Classical bandit algorithms (\eg \linucb and \lints) compute an estimate of the unknown parameters $\theta_a$ using past observations. Formally, for each arm $a \in [K]$ we define $S_a^t$ as the set of times up to $t-1$ (included) where the agent played arm $a$. Then, the estimated parameters are obtained through regularized least-squares regression as $\wh{\theta}_a^t = (X_{t,a} X_{t,a}^\top + \lambda I)^{-1} X_{t,a} Y_{t,a}$, where $\lambda > 0$, $X_{t,a} = (x_i)_{i \in S_a^t} \in \mathbb{R}^{d \times |S_a^t|}$ and $Y_{t,a} = (r_{i,a_i})_{i \in S_a^t} \in \mathbb{R}^{|S_a^t|}$.
Denote by $V_{t,a} = \lambda I + X_{t,a} X_{t,a}^\top$ the design matrix of the regularized least-square problem and by $\|x\|_{V} = \sqrt{x^\top V x}$ the weighted norm \wrt any positive matrix $V \in \mathbb{R}^{d \times d}$.
We define the confidence set:
\begin{equation}
\label{eq:confidence.intervals}
\mathcal{C}_{t,a} = \Big\{ \theta \in \mathbb{R}^d \,:\, \big\|\theta - \widehat{\theta}_{t,a} \big\|_{V_{t,a}} \leq \beta_{t,a} \Big\}
\end{equation}
% \begin{equation}
%         \label{eq:confidence.intervals}
%         \mathcal{C}_{t,a} = \left\{ \theta \in \mathbb{R}^d \,:\, \left\|\theta - \widehat{\theta}_{t,a} \right\|_{V_{t,a}} \leq \beta_{t,a} \right\}
% \end{equation}
 where 
%\begin{equation}
%\label{eq:beta_linucb}
        $\beta_{t,a} = \sigma\sqrt{d\log\big( (1 + L^2(1+|S_a^t|)/\lambda)/\delta \big)} + S\sqrt{\lambda},$
%\end{equation}
which guarantees that $\theta_{a}\in \mathcal{C}_{t,a}$, for all $t>0$, w.p.\ $1-\delta$.
%
This uncertainty is used to balance the exploration-exploitation trade-off either through optimism (\eg \linucb) or through randomization (\eg \lints). 



% \todot{add the confidence set property here}
% In this paper, we consider the case where a learning algorithm $\mathfrak{A}$ is attacked by an adversary that can modify different quantities in the learning process, like the rewards or the contexts observed by $\mathfrak{A}$. Similarly to \cite{liu2019data,jun2018adversarial} the goal of the attacker being to fool $\mathfrak{A}$ into pulling a target arm $a^{\dagger}$  $(T - o(T))$-times while maintaining the total number of modifications (i.e., cost) to be sub-linear. 



% !TEX root = main.tex
\section{Online Adversarial Attacks on Rewards}\label{sec:attacks_rewards}
The ultimate goal of a malicious agent is to force a bandit algorithm to perform a desired behavior. An attacker may simply want to induce the bandit algorithm to perform poorly\textemdash ruining the users' experience\textemdash or to force the algorithm to suggest a specific arm. The latter case is particularly interesting in advertising where a seller may want to increase the exposure of its product at the expense of the competitors. Note that the users' experience is also compromised by the latter attack since \changebr{ the suggestions they will receive will not be} tailored to their needs.
Similarly to~\cite{liu2019data,jun2018adversarial}, we focus on the latter objective, \ie to fool the bandit algorithm into pulling arms \changelm{ in $A^\dagger$, a set of target arms,} for $T-o(T)$ time steps (\emph{independently of the user}).

A way to obtain this behavior is to dynamically modify the reward in order to make the bandit algorithm believe that $a^\dagger$ is optimal, \changelm{for some $a^\dagger\in A^\dagger$.}
Clearly, the attacker has to pay a price in order to modify the perceived bandit problem and fool the algorithm.
If there is no restriction on when and how the attacker can alter the reward, the attacker can easily fool the algorithm.
However, this setting is not interesting since the attacker may pay a cost higher than the loss suffered by the attacked algorithm. An attack strategy is considered successful when the total cost of the attack is sublinear in $T$.
%\changebr{If there is no restriction on when and how they can attack, the attacker has complete control on the inputs of the bandit algorithm and can fool it easily. In practice, even though we restrict ourselves to attacking only the rewards or only the contexts,} we consider an attacker to be successful when the total cost of the attack is sublinear in $T$.%\todotout{explain better}

In this section, we show that under Assumption~\ref{assumption1}, there exists an attack algorithm that is successful against any bandit algorithm, stochastic or adverserial.

\textbf{Setting.} 
We assume that the attacker has the same knowledge as the bandit algorithm $\mathfrak{A}$ about the problem (\ie knows $\sigma$ and $L$). %\todotout{should know $L$ for $\beta$, right?}\todoeout{yes}
The attacker is assumed to be able to observe the context $x_t$, the arm $a_t$ pulled by $\mathfrak{A}$, and can modify the reward received by $\mathfrak{A}$.
When the attacker modifies the reward $r_{t, a_{t}}$ into $\wt{r}_{t, a_{t}}$ the \emph{instantaneous cost} of the attack is defined as $c_{t} :=\big| r_{t,a_{t}} - \wt{r}_{t, a_{t}}\big|$. The goal of the attacker is to fool algorithm $\mathfrak{A}$ such that \changelm{the arms in $A^\dagger$ are} pulled $T - o(T)$ times and $\sum_{t=1}^{T}  c_{t} = o(T)$. \changee{We also assume that the action for the arms in the target set is strictly positive for every context $x\in\mathcal{D}$. That is to say that $\Delta := \min_{x\in \mathcal{D}}\left\{ \langle x, \theta_{a_{\star}^{\dagger}(x)}\rangle - \max_{a\in A
^{\dagger}, a\neq a_{\star}^{\dagger}(x)} \langle x, \theta_{a}\rangle\right\}>0$ where $a_{\star}^{\dagger}(x) = \arg\max_{a\in A^{\dagger}} \langle x, \theta_{a}\rangle$ for every $x\in \mathcal{D}$}.


\textbf{Attack idea.} We leverage the idea presented in \cite{liu2019data} and \cite{jun2018adversarial} where the attacker lowers the reward of arms $a\notin A^{\dagger}$ so that algorithm $\mathfrak{A}$ learns that an arm of the target set is optimal for every context.
Since $\mathfrak{A}$ is assumed to be no-regret, the attacker only needs to modify the rewards $o(T)$ times to achieve this goal.
%As a consequence, the total cost of attacks is $o(T)$. 
%
Lowering the rewards has the effect of shifting
% Pushing down rewards consists in shifting  
the vectors $(\theta_{a})_{a\notin A^{\dagger}}$ to new vectors $(\theta'_{a})_{a\notin A^{\dagger}}$ such that for all arms $a\notin A^{\dagger}$ and all contexts $x\in\mathcal{D}$, there exists an arm $a^\dagger\in A^\dagger$  such that $\langle\theta'_{a}, x\rangle \leq \langle \theta_{a^{\dagger}}, x\rangle$. Since rewards are assumed to be bounded (see Asm.~\ref{assumption1}), this objective can be achieved by simply forcing the reward of non-target arms $a\notin A^\dagger$ to the minimum value.
Contextual ACE (see Fig.~\ref{alg:context_attack_protocol}) implements a soft version of this idea by leveraging the knowledge of the reward distribution.
At each round $t$, Contextual ACE modifies the reward perceived by $\mathfrak{A}$ as follows:
\vspace{-0.12cm}
 \begin{equation}
         \label{eq:perturbed.reward2}
 \widetilde{r}^{1}_{t,a_{t}} =\eta_{t}'\mathds{1}_{\{a_{t} \notin A^{\dagger}\}}+r_{t, a_t}\mathds{1}_{\{a_{t} \in A^{\dagger}\}} 
 \end{equation}
%  \begin{equation}
%          \label{eq:perturbed.reward2}
%  \widetilde{r}^{1}_{t,a_{t}} = \left\{\begin{matrix}
%  \eta_{t}'  & \text{if } a_{t} \neq a^{\dagger}\\ 
% 	 r_{t, a^{\dagger}} & \text{otherwise} 
% \end{matrix}\right.
%  \end{equation}

where $\eta_{t}'$ is a $\sigma$-subgaussian random variable generated by the attacker independently of all other random variables. Contextual ACE transforms the original problem into a \emph{stationary} bandit problem in which there is a targeted arm that is optimal for all contexts and all non targeted arms have expected reward of $0$. \changee{The following propostion shows that the cumulative cost of the attack is sublinear.}
%Despite this attack may seem expensive, the following proposition shows that its cumulative cost is sublinear.

% The approach used by the ACE algorithm \cite{liu2019data} effectively does this by constructing confidence intervals around the mean of each arm and feeding perturbed rewards to the bandit learner algorithm. However, the perturbed reward process seen by algorithm $\mathfrak{A}$ is non-stationary and in general there is no guarantee that an algorithm $\mathfrak{A}$ minimizing the regret in a stationary bandit problem keeps the same performance when the bandit problem is not stationary anymore. Nonetheless, transposing the idea of the ACE algorithm to our setting would give an attack of the following form, where at time $t$, Alg. $\mathfrak{A}$ pulls arm $a_{t}$ and receives rewards $\wt{r}^{1}_{t,a_{t}}$: 
% \begin{align*}
%         \wt{r}^{1}_{t, a_{t}} = \begin{cases}
%                 r_{t, a_{t}} + \max(-1, \min(0, C_{t, a_{t}})) & \textit{if } a_t \neq a^\dagger\\
%                 r_{t, a^{\dagger}} & \textit{otherwise}
%         \end{cases}
% \end{align*}
% with $C_{t,a_{t}} = (1 - \gamma)\min_{\theta \in \mathcal{C}_{t,a^{\dagger}}} \left\langle \theta, x_{t} \right\rangle - \max_{\theta\in\mathcal{C}_{t,a_t}} \left\langle \theta, x_{t}\right\rangle$.
% Note that $\mathcal{C}_{t,a}$ is defined as in Eq.~\ref{eq:confidence.intervals} using the \emph{non-perturbed} rewards, \ie $Y_{t,a} = (r_{i,a_i})_{i \in S_a^t}$.


%with $C_{t,a_{t}} = 0$ if $a_{t} = a^{\dagger}$ and $C_{t,a_{t}} = (1 - \gamma)\min_{\theta \in \mathcal{C}_{t,a^{\dagger}}} \left\langle \theta, x_{t} \right\rangle - \max_{\theta\in\mathcal{C}_{t,a_t}} \left\langle \theta, x_{t}\right\rangle$ where $\mathcal{C}_{t,a} = \big\{ \theta\in \mathbb{R}^{d} \mid ||\theta - \hat{\theta}_{a} ||_{\bar{V}_{a}(t)} \leq \beta_{a}(t)\big\}$ with $\bar{V}_{a,t} = \sum_{l, a_{l} = a} x_{l}x_{l}^{\intercal}$, $\beta_{a}(t)$ a real such that $\mathbb{P}\left( \theta_{a} \in \mathcal{C}_{t,a} \right) \geq 1 - \delta$ and $1 \geq \gamma \geq 0$ an attack parameter. For every time $t$, $\beta_{a}(t) = \sqrt{\lambda}S + \sigma\sqrt{d\log\left( (1 + N_{a}(t)L^{2}/\lambda)/\delta\right) }$ where $\lambda$ is regularization parameter and $N_{a}(t)$ is the number of times arm $a$ has been pulled before time $t$.
%\todot{Check if this version is reasonable.}

% Even though this approach works well in practice, we have no guarantees that all no-regret algorithms can adapt to a non-stationary bandit problem. This is why we analyze the following attack, defined as follows at time $t$: 
%  \begin{equation}
%          \label{eq:perturbed.reward2}
%  \widetilde{r}^{2}_{t,a_{t}} = \left\{\begin{matrix}
%  \eta_{t}'  & \text{if } a_{t} \neq a^{\dagger}\\ 
% 	 r_{t, a^{\dagger}} & \text{otherwise} 
% \end{matrix}\right.
%  \end{equation}
%   where $\eta_{t}'$ is a $\sigma$-subgaussian random variable generated by the attacker independently of all other random variables. With this new attack, Alg. $\mathfrak{A}$ ``sees'' the rewards from a bandit problem with $K$ arms, the optimal one being arm $a^{\dagger}$ with mean $\langle x, \theta_{a^{\dagger}}\rangle$ for all contexts $x$ and all the other arms appearing to have a reward of mean $0$ for all contexts. 

% \begin{algorithm}[t]
%   \caption{Contextual ACE}
%   \label{alg:attacker_rewards}
% \begin{algorithmic}
%   \FOR{$t=1,...,T$}
%   \STATE Alg. $\mathfrak{A}$ chooses arm $a_{t}$ based on context $x_{t}$
%   \STATE Environment generates reward: $r_{t,a_{t}} = \langle \theta_{a_{t}}, x_{t}\rangle + \eta_{t}$ with $\eta^{t}_{a_t}$ conditionally $\sigma^{2}$-subgaussian
%   \STATE Attacker observes reward $r_{t,a_{t}}$ and feeds the perturbed reward $\wt{r}^{1}_{t,a_{t}}$ (or $\wt{r}^{2}_{t,a_{t}}$) to $\mathfrak{A}$ 
%   \ENDFOR
% \end{algorithmic}
% \end{algorithm}

\begin{prop}\label{prop:reward_attack}
	For any $\delta\in(0, 1/K]$, when using Contextual ACE algorithm (Fig. ~\ref{alg:attacker_rewards}) with perturbed rewards $\wt{r}^{1}$, with probability at least $1-K\delta$, algorithm $\mathfrak{A}$ pulls \changelm{an arm in $A^{\dagger}$} for $T - o(T)$ time steps and the total cost of attacks is $o(T)$.
\end{prop}
The proof of this proposition is provided in App.~\ref{app:proof_prop_rewd_attack}. 
%It is based, on the fact that the rewards seen by the alg.~$\mathfrak{A}$ is similar to learning in a bandit environment where the arm $a^{\dagger}$ is optimal.
% \begin{proof}
% Let's consider the contextual bandit problem, $\mathcal{A}_{1}$ with $K$ arms with contexts $x\in \mathcal{D}$ such that the optimal arm has mean $\langle \theta_{a^{\dagger}}, x\rangle$ and every $K-1$ other arms has mean $0$. Then the regret of algorithm $\mathfrak{A}$ for this bandit problem is upper-bounded with probability at least $1 - \delta$ by a function $f_{\mathfrak{A}}(T)$ such that $f_{\mathfrak{A}}(T) = o(T)$. In addition, the reward process fed to alg. $\mathfrak{A}$ by the attacker is a stationary reward process with $\sigma^{2}$-subgaussian noise. Therefore, the number of times algorithm pulls an arm different from $a^{\dagger}$ is upper-bounded by $f_{\mathfrak{A}}(T)/\min_{x\in \mathcal{D}} \langle x, \theta_{a^{\dagger}}\rangle$. 
% In addition, the total cost of attack is upper-bounded by $\max_{a\in \llbracket 1, K\rrbracket} \max_{x\in \mathcal{D}} \langle x, \theta_{a}\rangle (T - N_{a^{\dagger}}(T))$ where $N_{a^{\dagger}}(T)$ is the number of times arm $a^{\dagger}$ has been pulled up to time $T$ but because of the previous argument $T - N_{a^{\dagger}}(T) \leq  f_{\mathfrak{A}}(T)/\min_{x\in \mathcal{D}} \langle x, \theta_{a^{\dagger}}\rangle$. 
% \end{proof}
While Prop.~\ref{prop:reward_attack} holds for any no-regret algorithm $\mathfrak{A}$, we can provide a more precise bound on the total cost by inspecting the algorithm.
% Prop.~\ref{prop:reward_attack} is true for every algorithm $\mathfrak{A}$ but the total number of pulls depends on the actual alg.~$\mathfrak{A}$. 
For example, we can show (see App.~\ref{app:algorithms}), that, with probability at least $1-K\delta$, \changebr{the number of times} \linucb~\cite{abbasi2011improved} pulls arms not in \changelm{$A^\dagger$ is at most $\sum_{j\notin A^{\dagger}} N_{j}(T) \leq  \frac{64K\sigma^{2}\lambda S^{2}}{\Delta^{2}}\Big( d\log\Big(\frac{\lambda + \frac{TL^{2}}{d}}{\delta^{2}}\Big) \Big)^{2}$} . 
% For example, when $\mathfrak{A}$ is \linucb from \cite{abbasi2011improved} (see also App~\ref{app:algorithms}), we have that with probability at least $1-K\delta$, arm $a^{\dagger}$ is not pulled at most:
% \begin{align*}
%     \sum_{j\neq a^{\dagger}} N_{j}(T) \leq  \frac{64K\sigma^{2}\lambda S^{2}}{\min_{x\in D} \langle \theta_{a^{\dagger}}, x\rangle^{2}}\Bigg( d\log\left(\frac{\lambda + \frac{TL^{2}}{d}}{\delta^{2}}\right) \Bigg)^{2}&
% \end{align*}
% where $\lambda$ is a regularization parameter. The total cost is bounded by the same upper-bound.
This directly translates \changebr{into} a bound on the total cost. % of the attacks.
% with a total cost of at most:
% \begin{align*}
% \frac{\min_{x\in D} \left\langle \theta_{a^{\dagger}}, x\right\rangle}{\left(\min_{x\in D} \left\langle \theta_{a^{\dagger}}, x\right\rangle\right)^{2}}64K\sigma^{2}\lambda S^{2}\Bigg( d\log\left(\frac{\lambda + \frac{TL^{2}}{d}}{\delta^{2}}\right)  \Bigg)^{2}&
% \end{align*}

\textbf{Comparison with ACE \cite{liu2019data}.} In the stochastic setting, the ACE algorithm~\cite{liu2019data} leverages a bound on the expected reward of each arm in order to modify the reward. 
However, the perturbed reward process seen by algorithm $\mathfrak{A}$ is non-stationary and in general there is no guarantee that an algorithm minimizing the regret in a stationary bandit problem keeps the same performance when the bandit problem is not stationary anymore. Nonetheless, transposing the idea of the ACE algorithm to our setting would give an attack of the following form, where at time $t$, Alg. $\mathfrak{A}$ pulls arm $a_{t}$ and receives rewards \changebr{$\wt{r}^{2}_{t,a_{t}}$}: 
\vspace{-0.2cm}
\begin{align*}
        \wt{r}^{2}_{t, a_{t}} =
                (r_{t, a_{t}} + \max(-1, \min(0, C_{t, a_{t}}))) \mathds{1}_{\{a_t \notin A^\dagger\}} + 
                r_{t, a_t} \mathds{1}_{\{a_t \in A^\dagger\}}
\end{align*}
with $C_{t,a_{t}} = (1 - \gamma)\min_{a^\dagger\in A^\dagger}\min_{\theta \in \mathcal{C}_{t,a^{\dagger}}} \left\langle \theta, x_{t} \right\rangle - \max_{\theta\in\mathcal{C}_{t,a_t}} \left\langle \theta, x_{t}\right\rangle$.
Note that $\mathcal{C}_{t,a}$ is defined as in Eq.~\ref{eq:confidence.intervals} using the \emph{non-perturbed} rewards, \ie $Y_{t,a} = (r_{i,a_i})_{i \in S_a^t}$. 

\textbf{Bounded Rewards.}  The bounded reward assumption is necessary in our analysis to prove a formal bound on the total cost of the attacks for \textit{any} no-regret bandit algorithm, otherwise we need more information about the attacked algorithm. In practice, the second attack on the rewards, $\wt{r}^{2}$, can be used in the case of unbounded rewards for any
algorithms. The difficulty for unbounded reward is that the attacker has to adapt to the environment reward but in order to do so the reward process observed by the bandit algorithm becomes non-stationary under the attack. Thus, there is no guarantee that an algorithm like \linucb will pull a target arm as the proof relies on the environment observed by the bandit algorithm being stationary. %To sum up, it is possible to construct an attack which does not assume bounded rewards but this comes at the price of a formal proof of the total cost for the attacker or knowing the bandit algorithm. 
We observe empirically that the total cost of attack is sublinear when using $\wt{r}^{2}$.
% In our setting, we make the assumption that the rewards are positive which is not necessary in the MAB setting of \cite{liu2019data}. This assumption comes from the linear structure of the rewards. Indeed the goal is to lower  rewards for every arm $a\not\in A^{\dagger}$ and every context $x\in\mathcal{D}$. But for a given vector $\theta$ and context $x\in\mathbb{R}^{d}$ \cmmnt{such that there exists a context $x\in \mathbb{R}^{d}$} such that, $\langle \theta_{a^{\dagger}} - \theta, x\rangle \geq 0$ for some $a^{\dagger} \in A^{\dagger}$, if $-x$ can be presented to the learning $\mathfrak{A}$ then it would learn not to pull $a^{\dagger}$ when presented with context $-x$. Hence it is not possible to find a parameter $\theta\neq 0$ dominated over the all space by some $\theta_{a^{\dagger}}$. Although for positive rewards, $\mathbf{0}^{d}$ is dominated by all parameters $(\theta_{a})_{a\in A^{\dagger}}$ under Assumption~\ref{assumption1}.
% To attack rewards between $[r_{\min}, r_{\max}]$, the attacker could use attacks similar to $\wt{r}_{t,a_{t}}^{2}$ (with $\gamma=0$ and small negative bias for arms not in $A^{\dagger}$), although there is no regret guarantee on the performance of this attack as the reward stream is non-stationary. Hence, the learning algorithm is not guaranteed to converge toward an environment where $A^{\dagger}$ contains an optimal arm for every context. In Sec.~\ref{sec:experiments}, we show that in practice this is not an issue as the attacker exhibits a logarithmic cumulative cost.

\cite{jun2018adversarial} does not assume that rewards are bounded but focus on attacking algorithms in the stochastic multi-armed setting. That is to say they study attacks only designed for $\varepsilon$-greedy and \ucb while we provide an efficient attack for any algorithms in the linear contextual case. We can extend their work, and thus remove the bounded reward assumption, in the linear contextual case by using the following attack, designed only for \linucb:
%If the bandit algorithm is \linucb then using the following attack:
\begin{align}
    \wt{r}^{3}_{t, a_{t}} = \left(r_{t, a_{t}} + \min_{a^\dagger\in A^\dagger}\min_{\theta \in \mathcal{C}_{t,a^{\dagger}}} \left\langle \theta, x_{t} \right\rangle - \max_{\theta\in\mathcal{C}_{t,a_t}} \left\langle \theta, x_{t}\right\rangle\right) \mathds{1}_{\{a_t \notin A^\dagger\}} + r_{t, a_t} \mathds{1}_{\{a_t \in A^\dagger\}}
\end{align}
with $C_{t,a}$ defined as in Eq.~\eqref{eq:confidence.intervals}. Although, the attack $\wt{r}^{3}$ is not stationary, it is possible to prove that the total cost of attack is $\mathcal{O}(\log(T))$ because we know that the attacked bandit algorithm is \linucb. 
% The attacker can fool \linucb algorithm into pulling arms $A^{\dagger}$ $o(T)$ times while keeping the total cost logarithmic in $T$.

\textbf{Constrained Attack.}
When the attacker has a constraint on the instantaneous cost of \changebr{the} attack, using the perturbed reward $\widetilde{r}^{1}$ may not be possible as the cost of the attack at time $t$ is not decreasing over time. Using the perturbed reward $\widetilde{r}^{2}$ offers a more flexible type of attack with more control on the instantaneous cost thanks to the parameter $\gamma$. \changede{But it still suffers from a minimal cost of attack from lowering rewards of arms not in $A^{\dagger}$.}%there is still a minimum of perturbations to apply. % I think the previous version of this sentence sounded bad. I'm not sure we even need this sentence so feel free to comment.

\textbf{Defense mechanism.}
The attack based on reward $\wt{r}_1$ is hardly detectable without prior knownledge about the problem.
In fact, the reward process associated to $\wt{r}_1$ is stationary and compatible with the assumption about the true reward (\eg subgaussian). While having very low rewards is reasonable in advertising, it can make the attack easily detectable in some other problems.
On the other hand, the fact that $\wt{r}_2$ is a non-stationary process makes this attack easier to detect.
%The learner cannot detect the attack solely based on the observed rewards and contexts. \changede{Although, because the perturbed reward process $\wt{r}^{1}$ is non-stationary, the attack could be easier to detect \changebr{by} monitoring the distribution of rewards.} However, 
When some data are already available on \changebr{each arm}, the learner can monitor the difference between the average reward\changebr{s} per action \changebr{computed on new and old data.}%based on previously collected data and the one from the online rewards. 
%\todobr{I'm not that happy about my new sentence here but I think the previous one was really unclear (it can still be seen in comments)}
%\todot{Maybe stress that $r^2$ is non-stationary while $r^1$ is stationary and compatible with the assumption about the true reward (subgaussian). Is it correct to say that $r^1$ is simpler to detect?}
% In a cold start problem, the learner will not be aware of the bias induced by the attacks.  Since this perturbation is ``small", it would be difficult for the learner to detect the attack. One possible way to detect tampering is for the learner to observe its average reward decrease over time. But in the case of a warm start, the learner has a prior estimation of the reward distribution for each arm given an input context. Hence, a bias on the real reward may be detected, i.e. the learner observes an ``extreme" reward  w.r.t. the one he expected. 

% \begin{remark}
%         It is possible to extend this attack to multiple target arms $a^\dagger \in A^\dagger$.
%         % This attack can be extended to the case of multiple targets arms. Let $A^\dagger$ be the set of target arms.
%         Similarly to~\eqref{eq:perturbed.reward2}, we can set $\widetilde{r}^1_{t,a_t} = \eta'_t\,$ when $a_t \notin A^\dagger$.
	%When there are several target arms, we can still apply the same type of attacks. One only needs to replace the reward $r_{t,a_t}$ if $a_t$ is not in the set of target arms.% Otherwise we pick a random arm inside this target set, instead of $a^{\dagger}$, in the attack.}
% \end{remark}



% !TEX root = main.tex
\section{Online Adversarial Attacks on Contexts}
\label{sec:attack_all_context}
In this section, we consider the attacker to be able to alter the context $x_t$ perceived by the algorithm rather than the reward. The attacker is now restricted to change the type of users presented to the learning algorithm $\mathfrak{A}$, hence changing its perception of the environment. We show that under the assumption that the attacker knows a lower-bound to the reward of the target set, it is possible to fool \linucb.

\textbf{Setting.} As in Sec.~\ref{sec:attacks_rewards}, we consider the attacker to have the same knowledge about the problem as $\mathfrak{A}$.
The main difference with \otc{the} previous setting is that the attacker \changee{attacks} before the algorithm. \changee{We adopt a \textit{white-box} \cite{goodfellow2014explaining} setting attacking \linucb.}
%It means that the attacker does not know the arm that would have been chosen by $\mathfrak{A}$ when presented with the true context $x_t$.Therefore, we need to have knowledge about the way the algorithm $\mathfrak{A}$ behaves. We focus on \linucb and we assume the attacker knows the parameter\changebr{s} of the algorithm. This is known as \changebr{a \emph{white-box} setting} in the \changebr{adversarial attacks} literature~\cite{goodfellow2014explaining}.
The goal of the attacker is unchanged: \changebr{they aim at forcing} the algorithm to pull \changelm{arms in $A^\dagger$ for} $T -o(T)$ \changebr{time steps} while paying a sublinear total cost. 
We denote by $\widetilde{x}_t$ the context after the attack and by $c_t = \|x_t - \widetilde{x}_t\|_2$ the instantaneous cost.

\textbf{Difference between attacks on contexts and rewards.} Perturbing contexts is fundamentally different from perturbing the rewards. The attacker only modifies the context that is shown to the bandit algorithm. The true context, which is used to compute the reward, remains unchanged. In other words, the attacker cannot modify the reward observed by the bandit algorithm. Instead, the attack algorithm described in this  section fools the bandit algorithm by making the rewards appear small relative to the contexts and requires more assumptions on the bandit algorithm than in Sec.~\ref{sec:attacks_rewards}.
% Finally, we assume that the attacker knows a positive lower bound of the expected reward as follows.
% \begin{assumption}\label{assumption2}
% 	For all $x\in \mathcal{D}$, \changelm{there exists $a^\dagger\in A^\dagger$}, such that $0 <\nu \leq \left\langle x, \theta_{a^{\dagger}} \right\rangle$.
% \end{assumption}
%\todot{we use a different notation for the cost compared to previous section BR: fixed it (use $c_t$ everywhere)} 
%In this second setting, the attacker can decide to modify the context $x_t$ before it is observed by algorithm $\mathfrak{A}$ but the reward received by alg.~$\mathfrak{A}$ is the same as if the context was not modified. The goal of the attacker is still to have a cumulative cost $o(T)$ and to fool the algorithm $\mathfrak{A}$ to choose the target arm $a^\dagger$, $T - o(T)$ times. 
%%\todol{the two following paragraphs need to be merged}
%We denote by $\wt{x_t}$ the context after the attack and we define the instantaneous cost of the attack as $c_t = ||x_t - \wt{x_t}||_2$. In this setting, the attacker does not have access to the arm chosen by algorithm $\mathfrak{A}$ when presented with context $x_{t}$. Therefore, they cannot adapt to the algorithm effectively. Hence, we focus on \linucb and assume that the attacker has access to the parameters of the algorithm\otc{: }it is a \emph{white-box} setting, as in  \cite{goodfellow2014explaining}. %.\todot{Put a reference to mention that it is a standard case in supervised learning}
%\otc{ Finally, we assume that the attacker knows a lower bound on expected rewards as follows:}
% \todo{Maybe reference these two assumptions, namely in latex terms assumption1 and assumption2}

%% \begin{assumption}\label{assumption2}
%% 	For all $x\in \mathcal{D}$, $0 <\nu \leq \left\langle x, \theta_{a^{\dagger}} \right\rangle$.
%% \end{assumption}
% \paragraph{Cost of attacking contexts:}
% In this second setting, the attacker can modify the context $x_{t}$ into $\wt{x}_{t}$ which is then presented to the algorithm $\mathfrak{A}$. The instantaneous cost of attack for a context is defined as $c^{x}_{t} := || x_{t} - \wt{x_{t}} ||_{2}$.
\textbf{Attack Idea.} The idea \changee{of} the attack in this setting is similar to the attack \changee{of} Sec.~\ref{sec:attacks_rewards}. The attacker builds a bandit problem where arm an $a^{\dagger}\in A^{\dagger}$ is optimal for all contexts by lowering the perceived value of all other arms not in $A^{\dagger}$. 
The attacker cannot modify the reward but, thanks to the linear reward assumption, they can scale the contexts to decrease the predicted rewards in the original context. 

At time $t$, the attacker receives the context $x_t$ and computes the attack. 
Thanks to the white-box setting, it computes the arm $a_{t}$ that algorithm $\mathfrak{A}$ would pull if presented with context $x_{t}$. If \changelm{$a_{t} \notin A^{\dagger}$} then the attacker changes the context to $\wt{x}_{t} =  \alpha_{a_t} x_{t}$ with $\alpha_{a_t} > \max_{x \in \mathcal{D}}\changelm{\min_{a^\dagger\in A^\dagger}} \langle \theta_{a_t}, x \rangle/\langle \theta_{a^{\dagger}}, x \rangle$.\changee{This factor is chosen such that for a ridge regression computed on the dataset $(\alpha x_{i}, \langle \theta, x_{i} \rangle)_{i}$ outputs a parameter close to $\theta/\alpha$ therefore the attacker needs to choose $\alpha$ such that for every context $x\in\mathcal{D}$, $\langle x, \theta/\alpha\rangle \leq \max_{a^{\dagger}\in A^{\dagger}} \langle x, \theta_{a
^{\dagger}}, x \rangle$.}
%
In other words, the attacker performs a dilation of the incoming context every time algorithm $\mathfrak{A}$ does not pull \changelm{an arm in $A^{\dagger}$}. The fact that the decision rule used by \linucb is invariant by dilation guarantees that the attacker will not inadvertently lower the perceived rewards for \changelm{arms in $A^{\dagger}$}.
%This idea of re-scaling is motivated by the fact that the decision rule used by \linucb is invariant by dilatation.
Because the rewards are assumed to be linear, presenting a large context $\alpha x$ and receiving the reward associated with the normal context $x$ will skew the estimated rewards of \linucb. The attack protocol is summarized in Fig.~\ref{alg:context_attack_protocol}. 


\changee{In order to compute the parameter $\alpha$ used in the attack, we make the following assumption concerning the performance of the arms in the target set:}
\begin{assump}\label{assumption2}
	For all $x\in \mathcal{D}$, \changelm{there exists $a^\dagger\in A^\dagger$}, such that $0 <\nu \leq \left\langle x, \theta_{a^{\dagger}} \right\rangle$ and $\nu$ is known to the attacker.
\end{assump}

\changee{\textbf{Knowing $\nu$.} For advertising and recommendation systems, knowing $\nu$ is not problematic. Indeed in those cases, the reward is the probability of impression of the ad ($r \in [0,1]$). The attacker has the freedom to choose one of multiple target arms with strictly positive click probability in every context. This freedom is an important aspect for the attacker since it allows the attacker to cherry pick the target ad(s). In particular, the attacker can estimate $\nu$ based on data from previous campaigns (only for the target ad). For instance, a company could have run many ad campaigns for one of their products and try to get the defender’s system to advertise it.}


An issue is that the norm of the attacked context can be greater that the upper bound $L$ of Assumption~\ref{assumption1}. To prevent this issue, we choose a context-dependent multiplicative constant $\alpha(x) = \min\{ 2/\nu, L/\|x\|_{2}\}$ which amounts to clip the norm of the attacked context to $L$. In Sec.~\ref{sec:experiments}, we show that this attack is effective for different size of target arms sets. We also show that in the case of contexts such that $\|x\|_{2} \leq \nu L/2$ that the cost of attacks is logarithmic in the horizon $T$.


\begin{figure}[t]
\begin{minipage}{0.45\linewidth}
%\vspace{-0.3in}
\bookboxx{
        %\textbf{Input:} attack parameter: $\alpha$ \\
        \noindent \textbf{For} time $t=1, 2, ..., T$ \textbf{do}
        \begin{enumerate}[leftmargin=4mm,itemsep=0mm]
                \item Alg. $\mathfrak{A}$ chooses arm $a_{t}$ based on context $x_{t}$
                \item Environment generates reward: $r_{t,a_{t}} = \langle \theta_{a_{t}}, x_{t}\rangle + \eta_{t}$ with $\eta^{t}_{a_t}$ conditionally $\sigma^{2}$-subgaussian
                \item Attacker observes reward $r_{t,a_{t}}$ and feeds the perturbed reward $\wt{r}^{1}_{t,a_{t}}$ (or $\wt{r}^{2}_{t,a_{t}}$) to $\mathfrak{A}$ 
        \end{enumerate}
}
\vspace{-0.1in}
\caption{\small Contextual ACE algorithm}
\label{alg:attacker_rewards}
\end{minipage}\hfill
\begin{minipage}{0.52\linewidth}
%\vspace{-0.3in}
\bookboxx{
        \textbf{Input:} attack parameter: $\alpha$ \\
        \noindent \textbf{For} time $t=1, 2, ..., T$ \textbf{do}
        \begin{enumerate}[leftmargin=4mm,itemsep=0mm]
                \item Attacker observes the context $x_{t}$, computes potential arm $a_{t}'$ and \otc{sets} $\wt{x}_{t} = x_{t} + (\alpha(x_{t}) -1 )x_{t}~\mathds{1}_{\{ a_{t}' \notin  A^{\dagger}\}}$
                \item Alg. $\mathfrak{A}$ chooses arm $a_{t}$ based on context $\wt{x}_{t}$
                \item Environment generates reward: $r_{t,a_{t}} = \langle \theta_{a_{t}}, x_{t}\rangle + \eta_{t}$ with $\eta_{t}$ conditionally $\sigma^{2}$-subgaussian
                \item Alg. $\mathfrak{A}$ observes reward $r_{t,a_{t}}$
        \end{enumerate}
}
\vspace{-0.1in}
\caption{\small ConicAttack algorithm.}
\label{alg:context_attack_protocol}
\end{minipage}
\vspace{-0.15in}
\end{figure}
 
% Thus if we have the bandit problem of Fig.\ref{fig:context_positive} then the attacker can not make the alg. $\mathfrak{A}$ learns an environment where $a^{\dagger}$ is optimal for all contexts.


% \begin{minipage}{0.45\linewidth}
% \centering
% \includegraphics[width=0.5\linewidth]{images/positive_context.pdf}
% %\caption{Example of bandit problem where the target arm can not be learned as optimal all the time}
% \end{minipage}\hfill
% \begin{minipage}{0.45\linewidth}
% \end{minipage}

\begin{prop}
\label{prop:cost_attack_all_ctx}
	Using the attack described in Fig.~\ref{alg:context_attack_protocol} \changee{and assuming that $\|x\|_{2}\leq \nu L/2$ for all contexts $x\in\mathcal{D}$}, for any $\delta\in (0,1/K]$, with probability at least $1 - K\delta$, the number of times \linucb does not pull \changebrtwo{an} arm \changelm{in $A^{\dagger}$} \otc{before time $T$} is at most     
        \begin{align*}
                \sum_{j\notin A^{\dagger}} N_{j}(T) \leq 32K^{2}\left( \frac{\lambda}{\alpha^{2}} + \sigma^{2}d\log\left(\frac{\lambda d + TL^2\alpha^{2}}{d\lambda\delta}\right) \right)^{3}
        \end{align*}
% \begin{align*}
%     \sum_{j\neq a^{\dagger}} N_{j}(T) \leq 32K^{2}\left( \frac{\lambda}{\alpha^{2}} + \sigma^{2}d\log\left(\frac{\lambda d + TL^2\alpha^{2}}{d\lambda\delta}\right) \right)^{3}
% \end{align*}
	with $N_{j}(T)$ the number of times arm $j$ has been pulled \otc{during the first }$T$ steps, 
	%$|| \theta_{a}|| \leq S$ for all arms $a$, $\lambda$ the regularization parameter of \linucb and for all $x\in \mathcal{D}$, $||x||_{2}\leq L$.
The total cost for the attacker is bounded by: $   \sum_{t=1}^{T} c_{t} \leq \frac{64K^{2}}{\nu}\left( \frac{\lambda}{\alpha^{2}} + \sigma^{2}d\log\left(\frac{\lambda d + TL^2\alpha^{2}}{d\lambda\delta}\right) \right)^{3}$ with $\alpha = 2/\nu$. 
% \begin{align*}
%     \sum_{t=1}^{T} c_{t} \leq \frac{64K^{2}}{\nu}\left( \frac{\lambda}{\alpha^{2}} + \sigma^{2}d\log\left(\frac{\lambda d + TL^2\alpha^{2}}{d\lambda\delta}\right) \right)^{3}
% \end{align*}
\end{prop}

The proof of Proposition \ref{prop:cost_attack_all_ctx} (see App.~\ref{app:proof_attack_all_ctx}) assumes that the attacker can attack at any time step, and that \changebr{they} can know in advance which arm will be pulled by Alg. $\mathfrak{A}$ in a given context. Thus it is not applicable to random exploration algorithms like \lints \cite{agrawal2013thompson} and \epsgreedy.  We also observed empirically that thowe two randomized algorithms are more robust to attacks (see Sec.~\ref{sec:experiments}) than \linucb.

\textbf{Norm Clipping.} Clipping the norm of the attacked contexts is not beneficial for the attacker. Indeed, this means that an attacked context was violating the assumption (used by the bandit algorithm) that contexts are bounded by $L$. The attack could then be easily detectable and may succeed only because it is breaking an underlying assumption used by the bandit algorithm. Prop.~\ref{prop:cost_attack_all_ctx} provides a theoretical grounding for the proposed attack when contexts are bounded by $\nu L/2$ and not only $L$.
Although, we can not prove a bound on the cumulative cost of attacks in general, we show in Sec.~\ref{sec:experiments} that attacks are still successful for multiple datasets where contexts are not bounded by $\nu L/2$. 
%This difficulty for random exploration algorithms is reflected in our experiments. See Section \ref{sec:experiments} for further details.
%\todot{We have also observed empirically that randomized algorithms are more robust to attacks (see Sec.~\ref{sec:experiments}).}

% \begin{remark}
%         If the attacker wants alg. $\mathfrak{A}$ to pull any arm in a set of target arms $A^{\dagger}$, the same type of attack can still be used with $\nu$ such that $0<\nu \leq \max_{a \in A^{\dagger}} \langle x, \theta_{a}\rangle$ for all $x \in \mathcal{D}$. %\todot{min? BR: No it would be a min on the contexts but it is a max on the target arms} 
%         Then, the context is multiplied by $\alpha = 2 / \nu$ when alg. $\mathfrak{A}$ is going to pull an arm not in $A^{\dagger}$.
% \end{remark}

% \todot{can we say something on estimating $\nu$. For example by constructing an estimate of $\theta_{a^\dagger}$ and the associated confidence interval?}
% \todoe{We could use an estimate of $\nu$ but we do not have any guarantees on the number of target pulls}

% \begin{proof}
  
% Let $a_{t}$ be the arm pulled by \linucb at time $t$. For each arms $a$, let $\wt{\theta}_a(t)$ be the result of the linear regression with the attacked context and $\hat{\theta}_{a}(t, \lambda/\alpha^{2})$ the one with the unattacked context and a regularization of $\frac{\lambda}{\alpha^{2}}$. At any time step $t$, we can write, for all $a\neq a^\dagger$:

% \begin{align*}
% \wt{\theta}_a(t) &=  \left(\lambda I_d + \sum_{l=0, a_{l} = a}^{t} \alpha^{2} x_l x_l^{\intercal}\right)^{-1} \sum_{k=0, a_{k} = a}^{t} r_k \alpha x_{k}\\
%   & =\frac{1}{\alpha} \left(\frac{\lambda}{\alpha^2} I_d + \sum_{k=0, a_{k} = a}^t x_k x_k^{\intercal}\right)^{-1} \sum_{k=0, a_{k} = a}^t r_k x_k \\
%     &= \frac{\hat{\theta}_{a}(t,\lambda/\alpha^{2})}{\alpha}
% \end{align*}
% We also note that, since the contexts are not modified for arm $a^\dagger$: $\wt{\theta}_{a^\dagger}(t)=\hat{\theta}_{a^\dagger}(t,\lambda)$.

% In addition, for any context $x$ and arm $a\neq a^\dagger$, the exploration term used by \linucb becomes:
% \begin{align}
%     ||x||_{\wt{V}_{a,t}^{-1}}&= \frac{1}{\alpha} ||x||_{\hat{V}_{a,t}^{-1}}
% \end{align}
% where $\wt{V}_{a,t} = \lambda I_d + \sum_{l=0, a_{l} = a}^{t} \alpha^{2} x_l x_l^{\intercal}$ and $\hat{V}_{a,t}^{-1} =\lambda/ \alpha^2 I_d + \sum_{k=0, a_{k} = a}^t x_k x_k^{\intercal}$. For a time $t$, if presented with context $x_{t}$ \linucb pulls arm $a_{t} \neq a^{\dagger}$, we have:
% \begin{align*}
% \alpha\left(\left\langle \hat{\theta}_{a^\dagger}(t), x_{t} \right\rangle +\beta_{a^\dagger}(t)||x_t||_{V_{a^\dagger,t}^{-1}}\right)\\\leq \left\langle \hat{\theta}_{a_{t}}(t, \lambda/\alpha^{2}), x_{t} \right\rangle +  \beta_{a_{t}}(t)||x_{t}||_{\hat{V}_{a_{t},t}^{-1}} 
% \end{align*}

% % Assuming that the confidence set for $a^{\dagger}$ holds, we have:
% % \begin{align*}
% % \alpha \left\langle \theta_{a^\dagger}(t), x_{t} \right\rangle \leq \left\langle \hat{\theta}_{a_{t}}(t, \lambda/\alpha^{2}), x_{t} \right\rangle +  \beta_{a_{t}}(t)||x_{t}||_{\hat{V}_{a_{t},t}^{-1}} 
% % \end{align*}


% As $\alpha = \frac2\nu\geq\frac{2}{\left\langle \theta_{a^\dagger}, x_{t} \right\rangle}$, we deduce that on the event that the confidence sets hold: 
% \begin{align*}
%     2&\leq\left\langle \hat{\theta}_{a_{t}}(t, \lambda/\alpha^{2}), x_{t} \right\rangle +  \beta_{a_{t}}(t)||x_{t}||_{\hat{V}_{a_{t},t}^{-1}}\\
%     &\leq \langle\theta_{a_{t}}, x_{t}\rangle+2\beta_{a_{t}}(t)||x_{t}||_{\hat{V}_{a_{t},t}^{-1}}
% \end{align*}
% And, then:

% \begin{align*}
%   1 \leq 2 - \langle\theta_{a_{t}}, x_{t}\rangle \leq 2\beta_{a_{t}}(t)||x_{t}||_{\hat{V}_{a_{t},t}^{-1}} 
% \end{align*}
% Therefore,
% \begin{align*}
%     \sum_{t=1}^{T} \mathds{1}_{\{a_{t}\neq a^{\dagger}\}} \leq \sum_{t=1}^{T} \min(2\beta_{a_{t}}(t)||x_{t}||_{\hat{V}_{a_{t},t}^{-1}},1)\mathds{1}_{\{a_{t} \neq a^{\dagger}\}}& \\
%     \leq \sum_{j\neq a^{\dagger}} 2\beta_{j}(T)\sqrt{\sum_{t=1}^{T}\mathds{1}_{\{a_{t}=j\}}\sum_{t=1, a_{t}=j}^{T} \min(1, ||x_{t}||^{2}_{\hat{V}_{j,t}^{-1}})}&
%   \end{align*}
%   But using Lemma $11$ from \cite{abbasi2011improved} and the bound on the $\beta_{j}(T)$ for all arm $j$, we have with Jensen inequality:
%   \begin{align*}
%     \sum_{t=1}^{T} \mathds{1}_{\{a_{t}\neq a^{\dagger}\}} \leq 4\sqrt{K\sum_{t=1}^{T} \mathds{1}_{\{a_{t}\neq a^{\dagger}\}}d\log\left( \lambda/\alpha^{2} + TL/d\right)}&\\
%     \Big( \sqrt{\lambda/\alpha^{2}} S + \sigma\sqrt{2\log(1/\delta) + d\log(1 + TL\alpha^{2}/(\lambda d)}\Big)&
% \end{align*}
% Hence the result.
% \end{proof}
% \begin{prop}
% \label{prop:cost_attack_all_ctx}
% As in proposition \ref{prop:cost_attack_reward},
% for \linucb, the number of times the arm $a^{\dagger}$ is pulled is at least:
% \begin{align*}
%     T - \frac{256\sigma^{2}\lambda S^{2}}{\left(\min_{x\in D} \left\langle \theta_{a^{\dagger}}, x\right\rangle\right)^{2}}\Bigg( d\log\left(\lambda + \frac{TL^{2}}{d}\right) + &\\ 2\log\left(\frac{1}{\delta}\right) \Bigg)^{2}&
% \end{align*}

% %As the attacker only attacks when the arm $a^{\dagger}$ is not pulled and the contexts are bounded by 1, 
% The total cost of the attack is bounded by :
% \begin{align*}
%     \frac{256\sigma^{2}\lambda S^{2}\delta}{\left(\min_{x\in D} \left\langle \theta_{a^{\dagger}}, x\right\rangle\right)^{2}}\Bigg( d\log\left(\lambda + \frac{TL^{2}}{d}\right) + &\\ 2\log\left(\frac{1}{\delta}\right) \Bigg)^{2}&
% \end{align*}
% \end{prop}
% \paragraph{Context attacks for adversarial contextual bandits:}
% This type of attack can not work for general contextual bandit algorithm as it is relies heavily on the fact that algorithms like \linucb use linear models to estimate the bandit environment. That is not the case anymore for more general algorithm like Exp$4$. 

%However, our experiments show that this attack works well in practice when the attacker can only modify the contexts for a fraction of the time steps or when the attacked algorithm is \epsgreedy. 

% \begin{remark}\label{rk:detection}
% Similarly to the previous section, in a cold start setting the learner is unlikely to see any bias induced by the attack. But, in this case, a simple yet effective defense would be to filtrate on the contexts that have a ``too big" norm, i.e. remove outliers from the datasets in terms of norm. In that sense, it  might constitute a defense against this type of  attack.
% \todoeout{This does not work}
% \end{remark}

% !TEX root = main.tex

\vspace{-.1in}
\section{Offline attacks on a Single Context}\label{sec:attack_one_context}
Previous sections focused on the man-in-the-middle (MITM) attack either on reward or context.
{The} MITM attack allows the attacker to arbitrarily change the information observed by the recommender system at each round.
{This attack may be hardly feasible in practice, since the exchange channels are generally protected by authentication and cryptographic systems.} %This scenario may be not possible in many applications where authentication methods (e.g., cryptographic systems) are used to secure the exchange of messages.
In this section, we consider the scenario where the attacker has control over a single user $u$.
As an example, consider the case where the device of the user is infected by a malware (e.g., Trojan horse), giving full control of the system to the malicious agent.
% \todot{This example is not more related to attacks on the reward? why do we not consider the reward?}
% \todoe{At first, the attacker was not suppose to be able to control the clicks of the user otherwise it would be easily detectable}
The attacker can thus modify the context of the specific user (e.g., by altering the cookies) that is perceived by the recommender system. %\footnote{MAYBE THIS INSTEAD OF RED\todot{We believe that changes to the context (\eg cookies) are more subtle and less easily detectable than changes to the reward (\eg signal). Moreover, if the reward is a purchase, this cannot be easily altered by taking control of the user's device.}} 
We believe that changes to the context (\eg cookies) are more subtle and less easily detectable than changes to the reward (\eg click). Moreover, if the reward is a purchase, it cannot be altered easily by taking control of the user's device.
% \changebr{\otc{Still, we consider that the attacker cannot modify} the rewards, since it would imply controlling the clicks of the user and \otc{would }be easily detectable.} 
Clearly, the impact of the attacker on the overall performance of the recommender system depends on the frequency of the specific user, that is out of the attacker's control. It may be thus difficult to obtain guarantees on the cumulative regret of algorithm $\mathfrak{A}$.
For this reason, we mainly focus on the study of the feasibility of the attack.

The attacker targets a specific user (i.e., the infected user) associated to a context $x^\dagger$.
Similarly to Sec.~\ref{sec:attack_all_context}, the objective of the attacker is to find the minimal change to the context presented to the recommender system $\mathfrak{A}$ such that $\mathfrak{A}$ {selects an arm in $A^\dagger$}.
$\mathfrak{A}$ observes a modified context $\wt{x}$ instead of $x^\dagger$. After selecting an arm $a_t$, $\mathfrak{A}$ observes the true noisy reward $r_{t,a_t} = \langle \theta_{a_t}, x^{\dagger}\rangle + \eta^t_{a_t}$.
We still study a white-box setting: the attacker can access all the \changebr{parameters of} $\mathfrak{A}$.

In this section, we show under which condition it is possible for an attacker to fool both an optimistic and posterior sampling algorithm.



%% \otc{The two previous sections focused} on attacks where the attacker could modify all rewards or all contexts. However, in a more practical setting, this assumption is a bit strong if, for example, a large number of contexts are available.
%% %as it assumes that the attacker can modify the reward/context associated with every user.
%% In this section, we consider a weaker attack where one user is infected by a Trojan, giving the attacker the possibility to modify the context associated with this particular user. That way, the attacker is weaker than what is previously assumed in the literature.
%% %But also more realistic as modifying the rewards (resp. context) associated with each arm (resp. contexts) means that an attacker can control the information pipeline between all users and a recommender system.
%%
%% Because the attacker \otc{now depends} on the visiting times of a specific user, they can not have a long-term impact on \otc{A}lgorithm $\mathfrak{A}$. Now, the attacker targets a specific user $u^\dagger$  (the infected user) associated with a target context $x^{\dagger}$, the goal of the attacker is to modify the context $x^{\dagger}$ presented to a bandit algorithm such that a target arm $a^{\dagger}$ is presented to the user $u^{\dagger}$ as \otc{often} as possible. \otc{While the bandit algorithm observes a modified context}, the expected reward observed by algorithm $\mathfrak{A}$ \otc{is still} $\langle \theta_{a^{\dagger}}, x^{\dagger}\rangle$. Although, we do not enforce any \otc{constraint} on the cumulative cost of an attack, as the attack is localized on the user side. The attacker still aims to have a minimal instantaneous cost for an attack. Furthermore, we study a white-box setting where the attacker has access to all the statistics used by alg. $\mathfrak{A}$.
%%
%% %This algorithm is also assumed to be either $\varepsilon$-greedy or \linucb \cite{abbasi2011improved}. We also investigate this type of attack for random exploration based algorithms like \lints \cite{agrawal2013thompson}.


\subsection{Optimistic Algorithm: \linucb}
\label{sec:optimistic_algorithms}
% The objective of the attacker is to force \linucb to pull arm $a^\dagger$ once presented with context $x^\dagger$.
% Since \linucb pulls the optimistic arm, this means to find the perturbation to the context $x^\dagger$ that makes $a^\dagger$ the most optimistic arm.
We consider the \linucb algorithm which chooses the arm to pull by maximizing an upper-confidence bound on the expected reward.
For each arm $a$ and context $x$, the UCB value is given by $\max_{\theta \in \mathcal{C}_{t,a}}  \langle x, \theta \rangle = \langle x, \hat{\theta}_{a}^t \rangle + \beta_{t,a} \|x \|_{\wt{V}_{t,a}^{-1}}$ (see Sec.~\ref{sec:preliminaries}).
% At any time $t$, the score of a context-arm pair $(x_t, a)$ is defined as $\phi(x_t, a) = \langle \hat{\theta}_{a}^t, x_t  \rangle + \beta_a(t)\|x_t \|_{\wt{V}_a^{-1}(t)}$ where we leverage the least-squares regression for the estimate of $\hat{\theta}_{a}^t$ (see Sec.~\ref{sec:preliminaries}).
%
The objective of the attacker is to force \linucb to pull \changebrtwo{an arm in $A^\dagger$} once presented with context $x^\dagger$.
This means to find a perturbation of context $x^\dagger$ that makes \changebrtwo{any arm in $A^\dagger$} the most optimistic arm.
Clearly, we would like to keep the perturbation as small as possible to reduce the cost for the attacker and the probability of being detected. Formally, the attacker needs to solve the following \emph{non-convex} optimization problem:
\begin{equation}\label{eq:attack_one_user}
\begin{aligned}
\min_{y\in \mathbb{R}^{d}} \quad & \|y\|_2 \quad \quad \text{s.t }\quad &  \changelm{\max_{a\notin A^\dagger}}\max_{\theta \in \wt{\mathcal{C}}_{t,a}}  \langle x^\dagger + y, \theta \rangle + \xi \leq \changebrtwo{\max_{a^\dagger \in A^\dagger}} \max_{\theta \in \widetilde{\mathcal{C}}_{t,a^\dagger}}  \langle x^\dagger + y, \theta \rangle \\
%\min_{y\in \mathbb{R}^{d}} \quad & \|y\|_2 \\
%        \text{s.t }\quad &  \max_{\theta \in \wt{\mathcal{C}}_{t,a}}  \langle x^\dagger + y, \theta \rangle + \xi \leq \max_{\theta \in \widetilde{\mathcal{C}}_{t,a^\dagger}}  \langle x^\dagger + y, \theta \rangle \\
\end{aligned}
\end{equation}
where $\xi>0$ is a parameter of the attacker and $\wt{\mathcal{C}}_{t,a} := \big\{\theta \mid \|\theta - \hat{\theta}_{a}^t\|_{\wt{V}_{t,a}} \leq \beta_{t,a} \big\}$ is the confidence set constructed by \linucb. We use the notation $\wt{\mathcal{C}}, \widetilde{V}$ to stress the fact that \linucb observes only the modified context.
%
% \todot{Fix notation for the estimated parameter, $t$ is missing}
%At time $t$ when context $x^\dagger$ is presented, the attacker needs to modify it so that $\mathfrak{A}$ decides to pull arm $a^{\dagger}$. The attacker needs to solve the following optimization problem:
%
%\begin{equation}\label{eq:attack_one_user}
%\begin{aligned}
%\min_{y\in \mathbb{R}^{d}} \quad & ||y|| \\
% \text{s.t }\quad &  \forall a\neq a^{\dagger}, \phi(x_t, a) + \xi \leq \phi(x_t, a^{\dagger}) \\
%\end{aligned}
%\end{equation}
%
% \begin{equation}\label{eq:attack_one_user}
% \begin{aligned}
% \min_{y\in \mathbb{R}^{d}} \quad & ||y|| \\
%  \text{s.t }\quad &  \forall a\neq a^{\dagger}, \langle \hat{\theta}_{a}^t - \hat{\theta}_{a^\dagger}^t, x^{\dagger}+y \rangle + \beta_{a}(t)||x^{\dagger}+y||_{\wt{V}_{a}^{-1}(t)} \\
% 	&+ \xi\leq \beta_{a^{\dagger}}(t)||x^{\dagger}+y||_{\wt{V}_{a^\dagger}^{-1}(t)}
% \end{aligned}
% \end{equation}
%where $\wt{V}_{a}(t)$ the \say{perturbed} design matrix of arm $a$ (\say{perturbed} because of the attacks on the context $x^{\dagger}$), $\beta_{a}(t)$ (as described in App.~\ref{app:algorithms}) are exploration bonus and $\xi > 0$ is a parameter of the attacker.
% For \linucb, $\beta_{a}(t) = \sqrt{\lambda}S + \sigma\sqrt{d\log\left( (1 + N_{a}(t)L^{2}/\lambda)/\delta\right) } $ and $\beta_{a}(t) = 0$ for $\varepsilon$-greedy with for all arm $a$, $||\theta_{a}||\leq S$, for all (unattacked) context $||x||_{2}\leq L$ and $\lambda>0$ a regularization parameter.
% Problem \eqref{eq:attack_one_user} is not convex. 
In contrast to Sec.~\ref{sec:attacks_rewards} \changebr{and}~\ref{sec:attack_all_context}, the attacker may not be able to force the algorithm to pull \changebrtwo{any of the target arms in $A^\dagger$}. In other words, Problem~\ref{eq:attack_one_user} may not be feasible. 
% Finally, we need to investigate under which condition those two problems are feasible. This is the object of the following theorem.
However, we are able to characterize the feasibility of~\eqref{eq:attack_one_user}.
%
\begin{thm}\label{thm:feasibility_attack_one_user}
        %For $\xi>0$, 
        Problem \eqref{eq:attack_one_user} is feasible at time $t$ \emph{iff.} 
        \begin{equation}\label{eq:feasibilty_condition}
        \exists \theta \in \changelm{\cup_{a^\dagger\in A^{\dagger}}}\wt{\mathcal{C}}_{t, a^{\dagger}}, ~ \theta\not\in \text{Conv}\Big( \cup_{\changelm{a\notin A^{\dagger}}} \wt{\mathcal{C}}_{t,a}\Big)
        \end{equation}
% \begin{align}\label{eq:feasibilty_condition}
%         \exists \theta \in \wt{\mathcal{C}}_{t, a^{\dagger}}, \qquad \theta\not\in \text{Conv}\left( \bigcup_{a\neq a^{\dagger}} \wt{\mathcal{C}}_{t,a}\right)
% \end{align}
%	where for every arm $a$,  $\mathcal{C}_{t,a} := \big\{\theta \mid ||\theta - \hat{\theta}_{a}(t)||_{\wt{V}_{a,t}} \leq \beta_{a}(t) \big\}$, $\wt{V}_{a,t}$ is the design matrix of \linucb at time $t$ and $\hat{\theta}_{a}(t)$ is the least squares estimate for arm $a$ built by \linucb.
\end{thm}

The condition given by Theorem \ref{thm:feasibility_attack_one_user} says that this attack can be done when there exists a vector $x$ for which \changebrtwo{an arm in $A^{\dagger}$} is assumed to be optimal according to \linucb. The condition mainly stems from the fact that optimizing a linear product on a convex compact set will reach its maximum on the edge of this set. In our case this set is the convex hull of the confidence ellipsoids of \linucb. Although it is possible to use \otc{an} optimization algorithm for this class of non-convex problems\textemdash \eg DC programming~\cite{tuy1995dc}\textemdash they are still slow compared to convex algorithms. Therefore, we present a simple convex relaxation of the previous problem \changebrtwo{for a single target arm $a^\dagger \in A^\dagger$} that still enjoys some empirical performance compared to Problem \eqref{eq:attack_one_user}. \changebrtwo{The final attack can then be computed as the minimum of the attacks obtained for each $a^\dagger \in A^\dagger$.} The relaxed problem is the following \changebrtwo{for each $a^\dagger\in A^\dagger$}:
\begin{equation}\label{eq:relaxed_attack_one_user}
\begin{aligned}
\min_{y\in \mathbb{R}^{d}} \quad & \| y \|_2 \quad \quad \text{s.t }\quad &  \max_{a\neq a^{\dagger}, \changee{a\not\in A^{\dagger}}} \max_{\theta \in \mathcal{C}_{t,a}} \langle x^{\dagger} + y, \theta - \hat{\theta}_{a^{\dagger}}^t \rangle \leq -\xi
%\min_{y\in \mathbb{R}^{d}} \quad & \| y \|_2 \\
%	\text{s.t }\quad &  \max_{a\neq a^{\dagger}} \max_{\theta \in \mathcal{C}_{t,a}} \langle x^{\dagger} + y, \theta - \hat{\theta}_{a^{\dagger}}^t \rangle \leq -\xi
\end{aligned}
\end{equation}
Since the RHS of the constraint in Problem \eqref{eq:attack_one_user} can be written as $\max_{\theta\in\mathcal{C}_{t,a^{\dagger}}} \langle \theta, x^{\dagger} + y \rangle$ for any $y$, the relaxation here consists in using $\langle\theta, x^{\dagger}+y\rangle$ as a lower-bound to this maximum for any $\theta\in\mathcal{C}_{t,a^{\dagger}}$. 

% \begin{proof}
% The proof of Theorem \ref{thm:feasibility_attack_one_user} is decomposed in two parts.

% First, let's assume that Equation \eqref{eq:feasibilty_condition} is satisfied. Then let $\theta \in \mathcal{C}_{t,a^{\dagger}}\setminus \text{Conv}\left( \bigcup_{a\neq a^{\dagger}} \mathcal{C}_{t,a}\right) $, then by the theorem of separation of convex sets applied to $\mathcal{C}_{t,a^{\dagger}}$ and $\{ \theta \}$. There exists a vector $v$ and $c_{1}< c_{2}$ such that for all $y \in \text{Conv}\left( \bigcup_{a\neq a^{\dagger}} \mathcal{C}_{t,a}\right)$:
% \begin{align*}
% \left\langle y, v\right\rangle \leq c_{1} < c_{2} \leq \left\langle \theta,v\right\rangle
% \end{align*}.
% Hence, for $\xi>0$ we have that for $\wt{v} = \frac{\xi}{c_{2}-c_{1}} v$ that:
% \begin{align*}
%     \left\langle y, \wt{v}\right\rangle + \xi \leq \left\langle \theta, \wt{v} \right\rangle
% \end{align*}

% Now let's assume that an attack is feasible then there exists a vector $y$ such that:
% \begin{align*}
%     \max_{\theta\in \mathcal{C}_{t,a^{\dagger}}} \left\langle y, \theta\right\rangle > c_{1} := \max_{a\neq a^{\dagger}} \max_{\theta\in \mathcal{C}_{t,a}} \left\langle y, \theta\right\rangle
% \end{align*}
% Let's reason by contradiction and assume that $\mathcal{C}_{t,a^{\dagger}} \subset \text{Conv}\left( \bigcup_{a\neq a^{\dagger}} \mathcal{C}_{t,a}\right)$ and consider $\theta\in \mathcal{C}_{t,a^{\dagger}}$. There exists $n\in\mathbb{N}^{\dagger}$, $\lambda_{1},\cdots, \lambda_{n}\geq 0$ and $\theta_{1}, \cdots, \theta_{n}\in \bigcup_{a\neq a^{\dagger}} \mathcal{C}_{t,a}$ such that $\theta = \sum_{i=1}^{n} \lambda_{i}\theta_{i}$ and $\sum_{i=1}^{n} \lambda_{i} = 1$. Thus
% \begin{align*}
%     \left\langle y, \theta\right\rangle = \sum_{i} \lambda_{i} \left\langle y, \theta_{i} \right\rangle \leq c_{1}\sum_{i=1}^{n} \lambda_{i} = c_{1}
% \end{align*}
% But because the problem is feasible we have that $c_{1}<  \max_{\theta\in \mathcal{C}_{t,a^{\dagger}}} \left\langle y, \theta\right\rangle$. We have a contradiction.
% \end{proof}
% \begin{remark}
% For \epsgreedy, but with $\mathcal{C}_{t,a} = \{\hat{\theta}_{a}(t)\}$ for all arm $a$.
% \end{remark}
For the relaxed Problem \eqref{eq:relaxed_attack_one_user}, the same type of reasoning as for Problem \eqref{eq:attack_one_user} gives that Problem \eqref{eq:relaxed_attack_one_user} is feasible if and only if $     \hat{\theta}_{a^{\dagger}}(t)\not\in \text{Conv}\left( \bigcup_{a\neq a^{\dagger}, \changee{a\not\in A^{\dagger}}} \mathcal{C}_{t,a}\right)$. 
% \begin{equation*}
%     \hat{\theta}_{a^{\dagger}}(t)\not\in \text{Conv}\left( \bigcup_{a\neq a^{\dagger}} \mathcal{C}_{t,a}\right)
% \end{equation*}
% Proposition \ref{thm:feasibility_attack_one_user} shows that it may not be possible to attack \linucb (or similar algorithms) if the bandit problem exhibit some special structure. For example, if $\theta_{a^{\dagger}} \in \text{Conv}\left( \theta_{a}, a\neq a^{\dagger} \right)$, it is not possible for the attacker to  solve either of problems \ref{eq:relaxed_attack_one_user} and \ref{eq:attack_one_user}.

% \begin{remark}
% When a set of target arms is available, the feasibility condition is the same except that the attacker cares about the union of the confidence ellipsoids for each arm in the set of target arms.
% \end{remark}

If Condition \eqref{eq:feasibilty_condition} is not met, \changebrtwo{no arm $a^{\dagger} \in A^\dagger$} can be pulled by \linucb. Indeed, the proof of Theorem \ref{thm:feasibility_attack_one_user} shows that the upper-confidence of \changebrtwo{every arm in $A^{\dagger}$} is always dominated by another arm for any context. In other words, if \changebrtwo{any arm in $A^{\dagger}$} is optimal for some contexts then the condition is satisfied a linear number of times for \linucb (for formal proof of this fact see App.~\ref{app:condition_linear}).


 \subsection{Random Exploration Algorithm: \lints}

The previous subsection focused on \linucb, however we can obtain similar guarantees for  algorithms with random exploration such as \lints. In this case, it is not possible to guarantee that a specific arm will be pulled for a given context because of the randomness in the arm selection process. The objective is to guarantee that \changebrtwo{an arm from $A^{\dagger}$} is pulled with probability at least $1-\delta$.
% \begin{algorithm}[tb]
%   \caption{Linear Thompson Sampling with Gaussian prior}
%   \label{alg:linTS}
% \begin{algorithmic}
%   \STATE {\bfseries Input:} regularization  $\lambda$, number of arms $K$, number of rounds $T$, variance $\nu$
%   \STATE Initialize for all arm $a$, $\bar{V}_{a}^{-1}(t) = \lambda I_{d}$ and $\hat{\theta}_{a}(t) = 0$
%   \FOR{$t=1,..., T$}
%   \STATE Observe context $x_{t}$
%   \STATE Draw $\wt{\theta}_{a}\sim\mathcal{N}(\hat{\theta}_{a}(t), \nu^{2}\bar{V}_{a}^{-1}(t))$
%   \STATE Pull arm $a_{t} = \argmax_{a\in \llbracket 1, K\rrbracket} \left\langle \wt{\theta}_{a}, x_{t}\right\rangle$
%   \STATE Observe reward $r_{t}$ and update parameters $\hat{\theta}_{a}(t)$ and $\bar{V}_{a}^{-1}(t)$
%   \ENDFOR
% \end{algorithmic}
% \end{algorithm}
Similarly to the previous subsection, the problem of the attacker can be written as: %$ \min_{\delta\in \mathbb{R}^{d}} \quad & || y ||$
\begin{equation}\label{eq:TS_attack_one_user}
\begin{aligned}
\min_{y\in \mathbb{R}^{d}} \quad & \| y \| \quad \quad \text{s.t }\quad &  \mathbb{P}\left( \exists {a^\dagger\in A^\dagger},~\forall \changebrtwo{a\not\in A^{\dagger},~} \langle x^{\dagger} + y, \wt{\theta}_{a} - \wt{\theta}_{a^{\dagger}} \rangle \leq - \xi\right) \geq 1 - \delta
%\min_{y\in \mathbb{R}^{d}} \quad & \| y \| \\
% \text{s.t }\quad &  \mathbb{P}\left( \forall a\neq a^{\dagger}, \langle x^{\dagger} + y, \wt{\theta}_{a} - \wt{\theta}_{a^{\dagger}} \rangle \leq - \xi\right)
% \geq 1 - \delta
\end{aligned}
\end{equation}
% \changebrtwo{TODO: Not 100\% sure about this equation. Should we write the relaxation to a single arm directly ?}\changelm{I am OK perso} 
% \changee{Try this instead:
% \begin{equation}
% \begin{aligned}
% \min_{y\in \mathbb{R}^{d}} \quad & \| y \| \quad \quad \text{s.t }\quad &  \max_{a^\dagger\in A^\dagger}\mathbb{P}\left( \forall a\not\in A^{\dagger}, \langle x^{\dagger} + y, \wt{\theta}_{a} - \wt{\theta}_{a^{\dagger}} \rangle \leq - \xi\right) \geq 1 - \delta
% %\min_{y\in \mathbb{R}^{d}} \quad & \| y \| \\
% % \text{s.t }\quad &  \mathbb{P}\left( \forall a\neq a^{\dagger}, \langle x^{\dagger} + y, \wt{\theta}_{a} - \wt{\theta}_{a^{\dagger}} \rangle \leq - \xi\right)
% % \geq 1 - \delta
% \end{aligned}
% \end{equation}}

where the $\wt{\theta}_{a}$ for different arms $a$ are independently drawn from a normal distribution with mean $\hat{\theta}_{a}(t)$ and covariance matrix $\upsilon^{2}\bar{V}_{a}^{-1}(t)$ with $\upsilon = \sigma\sqrt{9d\ln(T/\delta)}$. Solving this problem is not easy and in general not possible, \changebrtwo{even for a single arm}. For a given $x$ and arm $a$, the random variable $\langle x, \wt{\theta}_{a}\rangle$ is normally distributed with mean $\mu_{a}(x) := \langle \hat{\theta}_{a}(t), x\rangle$ and variance $\sigma_{a}^{2}(x) := \nu^{2}||x||_{\bar{V}_{a}^{-1}(t)}^{2}$. We can then write $\langle x,\wt{\theta}_{a}\rangle = \mu_{a}(x) + \sigma_{a}(x)Z_{a}$ with $(Z_{a})_{a}\sim\mathcal{N}(0, I_{K})$. For \changelm{the sake of} clarity, we drop the variable $x$ when writing $\mu_{a}(x)$ and $\sigma_{a}(x)$. 


Let's imagine (just for this paragraph) that $A^{\dagger} = \{ a^{\dagger}\}$, then the constraint in Problem \eqref{eq:TS_attack_one_user} becomes $\changebrtwo{\left[1-\mathbb{E}_{Z_{a^{\dagger}}}\left( \Pi_{\changebrtwo{a\not\in A^{\dagger}}} \Phi\left( \frac{\sigma_{a^\dagger}Z_{a^{\dagger}}+\mu_{a^\dagger} - \mu_{a}}{\sigma_{a}}\right)\right)\right]\leq\delta}$ \cmmnt{ \todoeout{In fact, we can not wirte this because the events for $a^{\dagger}\in A^{\dagger}$ are not independent}\todol{yes true :/ }
\begin{align*}
\mathbb{E}_{Z_{a^{\dagger}}}\left( \Pi_{a\neq a^{\dagger}} \Phi\left( \frac{\sigma_{a^\dagger}Z_{a^{\dagger}}+\mu_{a^\dagger} - \mu_{a}}{\sigma_{a}}\right)\right)
 \geq 1 - \delta
\end{align*}}
 where $\Phi$ is the cumulative distribution function of a normally distributed Gaussian random variable. Unfortunately, computing exactly this expectation is an open problem.%, we thus have to use approximations to derive a sufficient condition for feasibility of the attack for \lints. 
 
In the more general case where $|A^{\dagger}|\geq 1$, rewriting the constraints of Problem~\eqref{eq:TS_attack_one_user} is not possible. Following the idea of \cite{liu2019data}, \changebrtwo{for every single target arm $a^\dagger\in A^\dagger$}, a possible relaxation of the constraint in Problem \eqref{eq:TS_attack_one_user} is, \changee{to ensure that there exists an arm $a^{\dagger}\in A^{\dagger}$ such that} for every arm $a\not\in A^\dagger$,    $ 1 - \Phi\left( (\mu_{a^{\dagger}} - \mu_{a} - \xi)/(\sqrt{\sigma_{a}^{2} + \sigma_{a^{\dagger}}^{2}})\right) \leq \frac{\delta}{\changelm{K-|A^\dagger|}}$, \changelm{where $|A
^\dagger|$ is the cardinal of  $A^\dagger$}.
% \begin{equation*}
%     1 - \Phi\left( \frac{\mu_{a^{\dagger}} - \mu_{a} - \xi}{\sqrt{\sigma_{a}^{2} + \sigma_{a^{\dagger}}^{2}}}\right) \leq \frac{\delta}{K-1}
% \end{equation*}
Thus the relaxed version of the attack on \lints \changebrtwo{for a single arm $a^\dagger$} is:
\begin{align}
\min_{y\in \mathbb{R}^{d}} \| y \|  
 \quad \text{s.t.} \quad  \forall \changebrtwo{a\not\in A^{\dagger}},\langle x^{\dagger}+y, \hat{\theta}_{a^{\dagger}} - \hat{\theta}_{a}\rangle - \xi  \geq \nu\Phi^{-1}\left(1 - \tfrac{\delta}{\changelm{K-|A^\dagger|}}\right)\big\| x^{\dagger} + y \big\|_{\bar{V}_{a}^{-1} + \bar{V}_{a^{\dagger}}^{-1}} \label{eq:relaxed_TS_attack_one_user}
\end{align} 
Problem \eqref{eq:relaxed_TS_attack_one_user} is similar to Problem \eqref{eq:relaxed_attack_one_user} as the constraint is also a Second Order Cone Program but with different parameters (see App.~\ref{app:one_context_ts_linucb}). \changebrtwo{As in section \ref{sec:optimistic_algorithms}, we compute the final attack as the minimum of the attacks computed for each arm in $A^\dagger$.}



\vspace{-.1in}
\section{Experiments}\label{sec:experiments}
In this section, we conduct experiments on the attacks on contextual bandit problems with simulated data and two real-word datasets: MovieLens25M \cite{harper2015movielens} and Jester \cite{goldberg2001eigentaste}. The synthetic dataset and the data preprocessing step are presented in App.~\ref{app:experiments_setup}.
\vspace{-0.05in}
\subsection{Attacks on Rewards}
\vspace{-0.02in}
We study the impact of the reward attack for $4$ contextual algorithms: \linucb, \lints, \epsgreedy and \expfour. As parameters, \changebr{we use $L=1$ for the maximal norm of the contexts}, $\delta = 0.01$, $\upsilon = \sigma\sqrt{d\ln(t/\delta))/2}$, $\varepsilon_{t} = 1/\sqrt{t}$ at each time step $t$ and $\lambda = 0.1$. \changee{We choose only a \textit{unique target arm} $a^{\dagger}$.} For \expfour, we use $N = 10$ experts with $N-2$ experts returning a random arm at each time, one expert choosing arm $a^{\dagger}$ every time and one expert returning the optimal arm for every context. With this set of experts the regret of bandits with expert advice is the same as in the contextual case. To test the performance of each algorithm, we generate $40$ random contextual bandit problems and run each algorithm for $T = 10^{6}$ steps on each. We report the average cost and regret for each of the $40$ problems.  
% \begin{figure}[t]
% \centering
% % \includegraphics[width=0.45\textwidth]{images/rewards_attacks_simulations/gen_synth_regret.pdf}
% \includegraphics[width=.4\textwidth]{images/rewards_attacks_simulations/gen_synth_cost.pdf}
% \caption{Total cost of attacks for synthetic data with $\gamma = 0.22$.}
% \label{fig:synthetic_reward_stochastic}
% \end{figure}
Figure \ref{fig:costs_plot} (Top) shows the attacked algorithms using the attacked reward $\wt{r}^{1}$ (reported as ``stationary CACE'') and the rewards $\wt{r}^{2}$ (reported as CACE).


These experiments show that, even though the reward process is non-stationary,  usual stochastic algorithms like \linucb can still adapt to it and pull the optimal arm for this reward process (which is arm $a^{\dagger}$). The true regret of the attacked algorithms is linear as $a^{\dagger}$ is not optimal for all contexts. %At $T = 10^{6}$.\todol{There misses sth here?}
In the synthetic case, for the algorithms attacked with the rewards $\wt{r}^{2}$, over 1M iterations and $\gamma = 0.22$, the target arm is drawn more than $99.4\%$ of the time on average for every algorithm and more than $97.8\%$ of the time for the stationary attack $\wt{r}^1$ (see Table~\ref{table:number_of_draws} in App.~\ref{app:additional_fig_rwds}). The dataset-based environments (see Figure~\ref{fig:costs_plot} (Left)) exhibit the same behavior: the target arm is pulled more than $94.0\%$ of the time on average for all our attacks on Jester and MovieLens and more than $77.0\%$ of the time in the worst case (for \lints attacked with the stationary rewards) (see Table~\ref{table:number_of_draws}).
% \begin{figure}[h]
% \centering
% \includegraphics[width=0.4\textwidth]{images/rewards_attacks_jester/gen_jester_cost.pdf}
% \includegraphics[width=.4\textwidth]{images/rewards_attacks_movielens/gen_movielens_cost.pdf}
% \caption{Total cost of attacks on Jester (Top) and MovieLens (Bottom) with $\gamma = 0.5$.}
% \label{fig:dataset_reward_attacks}
% \end{figure}
% \begin{figure}[t]
% \centering
% \begin{tabular}{c c}
% Jester \vspace{-0.3cm} & MovieLens\\
% \includegraphics[width=.46\columnwidth]{images/costs_attacks_rewards_no_legend/gen_jester_cost.pdf}& 
% \includegraphics[width=.45\columnwidth]{images/costs_attacks_rewards_no_legend/gen_movielens_cost.pdf}\\
%  & Synthetic\\
% \includegraphics[width=.43\columnwidth]{images/costs_attacks_rewards_no_legend/legend_rewards_attacks.pdf}
% &
% 	\includegraphics[width=0.49\columnwidth]{images/costs_attacks_rewards_no_legend/gen_synth_cost.pdf}
% \end{tabular}%\\
% %	~\vspace*{-0.4cm}~\\
% \caption{Total cost of the attacks on the rewards on the synthetic and dataset-based environments with $\gamma =0.5$ for Jester and MovieLens and $\gamma=0.22$ for the synthetic dataset.\vspace{-0.8cm}}%. The cumulative cost of the attacks is sublinear and most attacks occur during the first few iterations.}
% \label{fig:costs_reward_attacks}
% \end{figure}

%\todol{Confidence sets on MovieLense?} they seem to be there, just very small
\vspace{-0.05in}
\subsection{Attacks on Contexts}\label{subsec:exp_attack_all_context}
\vspace{-0.02in}
\changee{We now illustrate the effectiveness of the attack in Alg.~\ref{alg:context_attack_protocol}.  We study the behavior of attacked \linucb, \lints, \epsgreedy with different size of target arms set ($|A^{\dagger}|/K\in \{ 0.3, 0.6, 0.9\}$ with $K$ the total number of arms).} We test the performance of \linucb with the same parameters as in the previous experiments. Yet since the variance is much smaller in this case, we generate a random problem and run $20$ simulations for each algorithm. \changee{The target arms are chosen randomly and we use the exact lower-bound on the reward of those arms to compute $\nu$.\cmmnt{In addition, to measure the robustness of random exploration algorithm like \lints and \epsgreedy we use an unbounded attack on the context}}



% We now illustrate the setting %where the attacker is allowed to attack all the contexts. 
% of Sec.~\ref{sec:attack_all_context}. We test the performance of \linucb, \lints and \epsgreedy with the same parameters as in the previous experiments. Yet since the variance is much smaller in this case, we generate a random problem and run $20$ simulations for each algorithm \changebr{and each attack type}. The target arm is chosen to minimize the average expected reward over all contexts and we use the exact lower-bound on the reward for this target arm as $\nu$. In addition, we also test the performance of an attack where the contexts are multiplied by $5$ compared to the attack in Sec.~\ref{sec:attack_all_context} but \changebr{where the attacker is only allowed to attack } $20\%$ of the time. The rest of the time the attacker does 
% not modify the context. We \changebr{call} this attack as CC$20$.
\begin{table}[h]
\begin{center}
% \scalebox{0.8}{
% \begin{tabular}{lccc}
% \toprule
% {} & Synthetic &  Jester & Movilens \\
% \midrule
% \linucb          &      2.38\% &   1.47\% &    2.24\% \\
% CC \linucb    &     99.99\% &  99.53\% &   99.74\% \\
% % CC$20$ \linucb       &     99.96\% &  99.24\% &   99.55\% \\
% \epsgreedy      &      0.26\% &   0.58\% &    0.30\% \\
% CC \epsgreedy &     99.98\% &  99.83\% &   99.90\% \\
% % CC$20$ \epsgreedy    &     99.97\% &  99.30\% &   99.65\% \\
% \lints           &      3.27\% &   1.27\% &    1.29\% \\
% CC \lints      &      9.08\% &   7.24\% &   99.13\% \\
% % CC$20$ \lints          &     32.22\% &  43.78\% &   95.78\% \\
% \bottomrule
% \end{tabular}
% }
\caption{Percentage of iterations for which the algorithm pulled an arm in the target set $A^{\dagger}$ (with a target set size of $0.3K$ arms) (\textbf{Left}) Online attacks using ContextualConic ($CC$) algorithm. Percentages are averaged over 20 runs of 1M iterations. (\textbf{Right}) Offline attacks with exact (Full) and Relaxed optimization problem. Percentages are averaged over 40 runs of 1M iterations. \label{table:success_rate}\vspace{-0.2cm}}
\scalebox{0.8}{
\begin{tabular}{lccc}
\toprule
{} & Synthetic &  Jester & Movilens \\
\midrule
\linucb          &      28.91\% &   26.59\% &    31.13\% \\
CC LinUCB     &     98.55\% &  98.36\% &   99.61\% \\
% CC$20$ \linucb       &     99.96\% &  99.24\% &   99.55\% \\
\epsgreedy      &     25.7\% &   25.85\% &    31.78\% \\
CC \epsgreedy &     89.71\% &  99.85\% &   99.92\% \\
% CC$20$ \epsgreedy    &     99.97\% &  99.30\% &   99.65\% \\
\lints           &      27.2\% &   26.10\% &    33.24\% \\
CC \lints      &      30.93\% &  97.26\% &   98.82\% \\
% CC$20$ \lints          &     32.22\% &  43.78\% &   95.78\% \\
\bottomrule
\end{tabular}
}
\scalebox{0.7}{
\begin{tabular}{lccc}
\toprule
{} & Synthetic &  Jester & MovieLens \\
\midrule
\linucb             &      $0.07\%$ &   $0.01\%$ &    $0.39\%$ \\
\linucb Relaxed     &     $13.76\%$ &  $97.81\%$ &    $4.09\%$ \\
\linucb Full        &     $88.30\%$ &  $99.98\%$ &   $99.99\%$ \\
\epsgreedy         &      $0.01\%$ &   $0.00\%$ &    $0.03\%$ \\
\epsgreedy Full &     $99.98\%$ &  $99.95\%$ &   $99.97\%$ \\
\lints              &      $0.02\%$ &   $0.01\%$ &    $0.05\%$ \\
\lints Relaxed      &     $18.21\%$ &  $80.48\%$ &    $5.56\%$ \\
\bottomrule
\end{tabular}
}

% \caption{Percentage of iterations for which the algorithm pulled the target arm $a^{\dagger}$  (On left) Percentage of iterations for which the algorithm pulled the target arm $a^{\dagger}$ for each type of attack, averaged on 20 runs of 1M iterations. In the $CC$ version, the contexts are modified using the ContextualConic attack (Sec.~\ref{sec:attack_all_context}). (On right) Percentage of times  an algorithm pulled the target arm $a^{\dagger}$ when context $x^{\star}$ was drawn, averaged on 40 runs of 1M iterations. When applicable, the Relaxed version corresponds to solving a relaxed convex version of the problem while the Full attack corresponds to solving the exact optimization problem.}
 %In the $CC 20$ version, the attacker can only modify the contexts for 20\% of the iterations. In this case, we multiply the context by $5\alpha$ instead of $\alpha$ when the arm the algorithm is most likely to draw is not $a^{\dagger}$.}
\end{center}
% \vspace{-.03in}
\end{table}
% \begin{figure}[t]
% \centering
%  % \includegraphics[width=0.45\textwidth]{images/all_context_3_alg_sim/gen_ratio.pdf}
% \includegraphics[width=.4\textwidth]{images/all_context_3_alg_sim/gen_cost.pdf}
% \caption{Total cost of ContextualConic attacks on synthetic environment.}
% \label{fig:synthetic_all_context}
% \end{figure}

Table~\ref{table:success_rate} (Left) shows the percentage of times \changee{an arm in $A^{\dagger}$, for $|A^{\dagger}| = 0.3K$}, has been selected by the attacked algorithm\changebr{. We} see that, as expected, CC \linucb reaches a ratio of almost $1$, meaning the target arms are indeed pulled a linear number of times. A more surprising result (at least not covered by the theory) is that \epsgreedy exhibits the same behavior. \changebr{Similarly to \lints, \epsgreedy exhibits some randomness in the action selection process. It can cause an arm $a^{\dagger}\in A^{\dagger}$ to be chosen when the context is attacked and interfere with the principle of the attack.} We suspect that is what happens for \lints. Fig.~\ref{fig:costs_plot} (Bottom) shows the total cost of the attacks for the attacked algorithms \cmmnt{(except for \lints, for which the cost is linear)}. % Although only CC is covered by the theory, %the CC20 attack reaches almost the same success rate \changebr{as CC for \linucb and \epsgreedy}. %when the attacker is only allowed to attack for some (20\%) of the time steps and we increase the attack norm accordingly. Moreover, the cumulative norm of this attack is still sublinear. The success rate of this attack is even better than that of CC \lints.
\changee{Despite the fact that the estimate of $\theta_{a^\dagger}$ can be polluted by attacked samples, it seems that \lints can still pick up $a^{\dagger}$ as being optimal for some dataset like MovieLens and Jester but not on the simulated dataset.}

\begin{figure}
    \centering
    \begin{minipage}{0.25\linewidth}
        \centering
        \includegraphics[width=0.85\linewidth]{sections/appendix/nips2020-bandits/images/attack_reward/simulations/legend.pdf}
        %\caption{Lorem ipsum}
    \end{minipage}\hfill
    \begin{minipage}{0.25\linewidth}
    \centering
    \includegraphics[width=0.95\linewidth]{sections/appendix/nips2020-bandits/images/attack_reward/simulations/cost_attack_reward_simulations.pdf}
    % \caption{Synthetic}
    \end{minipage}\hfill
    \begin{minipage}{0.25\linewidth}
    \centering
    \includegraphics[width=0.85\linewidth]{sections/appendix/nips2020-bandits/images/attack_reward/jester/cost_attack_reward_jester.pdf}
    % \caption{Jester}
    \end{minipage}\hfill
    \begin{minipage}{0.25\linewidth}
    \centering
    \includegraphics[width=0.85\linewidth]{sections/appendix/nips2020-bandits/images/attack_reward/movielens/cost_reward_attack_movielens.pdf}
    % \caption{MovieLens}
    \end{minipage}\\
    \begin{minipage}{0.25\linewidth}
        \centering
        \includegraphics[width=0.85\linewidth]{sections/appendix/nips2020-bandits/images/regret_cost_attacks_context/simulations/legend.pdf}
        %\caption{Lorem ipsum}
    \end{minipage}\hfill
    \begin{minipage}{0.25\linewidth}
    \centering
    \includegraphics[width=0.95\linewidth]{sections/appendix/nips2020-bandits/images/regret_cost_attacks_context/simulations/cost_context_attack_simulations.pdf}
    % \caption{Synthetic}
    \end{minipage}\hfill
    \begin{minipage}{0.25\linewidth}
    \centering
    \includegraphics[width=0.85\linewidth]{sections/appendix/nips2020-bandits/images/regret_cost_attacks_context/jester/cost_context_attack_jester.pdf}
    % \caption{Jester}
    \end{minipage}\hfill
    \begin{minipage}{0.25\linewidth}
    \centering
    \includegraphics[width=0.85\linewidth]{sections/appendix/nips2020-bandits/images/regret_cost_attacks_context/movielens/cost_context_attack_movielens.pdf}
    % \caption{MovieLens}
    \end{minipage}
    \caption{Total cost of attacks on rewards for the synthetic (Left, $\gamma=0.22$), Jester (Center, $\gamma=0.5$) and MovieLens (Right, $\gamma=0.5$) environments. Bottom, total cost of ContextualConic attacks on the synthetic (Left), Jester (Center) and MovieLens (Right) environments.\label{fig:costs_plot}
}%. The cumulative cost of the attacks is sublinear and most attacks occur during the first few iterations.}
\end{figure}

% \begin{figure}[t]
% \centering
% \begin{tabular}{cccc}
% % Legend & Synthetic & Jester & MovieLens\\
% \includegraphics[width=.22\columnwidth]{images/costs_attacks_rewards_no_legend/legend_rewards_attacks.pdf}&
% \includegraphics[width=.22\columnwidth]{images/costs_attacks_rewards_no_legend/gen_synth_cost.pdf}& 
% \includegraphics[width=.22\columnwidth]{images/costs_attacks_rewards_no_legend/gen_jester_cost.pdf}& 
% \includegraphics[width=.22\columnwidth]{images/costs_attacks_rewards_no_legend/gen_movielens_cost.pdf}\\
% %  Legend & Synthetic & Jester & MovieLens\\
% \includegraphics[width=.22\columnwidth]{images/attacks_all_contexts_no_legend/legend_all_contexts.pdf}&
% \includegraphics[width=.22\columnwidth]{images/attacks_all_contexts_no_legend/gen_cost.pdf}&
% \includegraphics[width=.22\columnwidth]{images/attacks_all_contexts_no_legend/gen_jester_cost.pdf}&
% \includegraphics[width=.22\columnwidth]{images/attacks_all_contexts_no_legend/gen_movielens_cost.pdf}\\
% \vspace{-0.1in}


% \end{tabular}%\\
% 	%~\vspace*{-0.4cm}~\\
% \caption{On the left, total cost of the attacks on the rewards on the synthetic and dataset-based environments with $\gamma =0.5$ for Jester and MovieLens and $\gamma=0.22$ for the synthetic dataset. On the right, total cost of ContextualConic attacks on the synthetic and dataset-based environments.\vspace{-0.5cm}}%. The cumulative cost of the attacks is sublinear and most attacks occur during the first few iterations.}
% \label{fig:costs_plot}
% % \vspace{-0.1in}
% \end{figure}
% \begin{figure}[t]
% \centering
% \includegraphics[width=0.4\textwidth]{images/all_contexts_3_algs_jester/gen_jester_cost.pdf}
% \includegraphics[width=.4\textwidth]{images/all_contexts_3_alg_movielens/gen_movielens_cost.pdf}
% \caption{Total cost of ContextualConic attacks on dataset-based environments (Jester above, Movielens below).}%. The cumulative cost of the attacks is sublinear and most attacks occur during the first few iterations.}
% \label{fig:datasets_all_context}
% \end{figure}
\vspace{-0.2cm}
\subsection{Offline attacks on a Single Context}
\vspace{-0.02in}
 %We study the setting described in section \ref{sec:attack_one_context}, where the attacker can only modify a single context. 
 We now move to the setting described in Sec.~\ref{sec:attack_one_context} and test the same algorithms as in Sec.~\ref{subsec:exp_attack_all_context}. We run 40 simulations for each algorithm \changebr{and each attack type}. The target context $x^{\dagger}$ is chosen randomly and the target arm as the arm minimizing the expected reward for $x^{\dagger}$. %the performance of \linucb, \epsgreedy and \lints on a synthetic dataset and two datasets based on real data - Jester and MovieLens. The datasets are preprocessed in the same way as for the other experiments and we use the same parameters for all the algorithms. We run 40 simulations for each algorithm. 
 %In this setting, we choose a target context randomly and define the target arm as the arm minimizing the reward in this context. 
 The attacker is only able to modify the incoming context for the target context (which corresponds to the context of one user) and the incoming contexts are sampled uniformly from the set of all possible contexts (of size $100$). %. Although our attacks do not require it, we limit the number of possible contexts to $100$ in order to be able to observe significant differences between the attacked and unattacked algorithms.
%  \begin{table}[t]
% \begin{center}
% \begin{tabular}{lccc}
% \toprule
% {} & Synthetic &  Jester & MovieLens \\
% \midrule
% \linucb             &      $0.07\%$ &   $0.01\%$ &    $0.39\%$ \\
% \linucb Relaxed     &     $13.76\%$ &  $97.81\%$ &    $4.09\%$ \\
% \linucb Full        &     $88.30\%$ &  $99.98\%$ &   $99.99\%$ \\
% \epsgreedy         &      $0.01\%$ &   $0.00\%$ &    $0.03\%$ \\
% \epsgreedy Full &     $99.98\%$ &  $99.95\%$ &   $99.97\%$ \\
% \lints              &      $0.02\%$ &   $0.01\%$ &    $0.05\%$ \\
% \lints Relaxed      &     $18.21\%$ &  $80.48\%$ &    $5.56\%$ \\
% \bottomrule
% \end{tabular}
% \caption{Percentage of times  an algorithm pulled the target arm $a^{\dagger}$ when context $x^{\star}$ was drawn, averaged on 40 runs of 1M iterations. When applicable, the Relaxed version corresponds to solving a relaxed convex version of the problem while the Full attack corresponds to solving the exact optimization problem.
% \label{table:success_rates_one_ctx}
% \vspace{-0.3cm}
% }
% \end{center}
% \end{table}
 Table~\ref{table:success_rate} (Right) shows the percentage of success for \changebr{each attack}. We observe that the non-relaxed attacks on \epsgreedy and \linucb work well across all datasets. %is also very successful on datasets based on real data, and quite successful on synthetic datasets. 
 However, the relaxed attack %solving the problems with convex relaxations of the constraints 
 for \linucb and \lints are not as successful, on the synthetic dataset and MovieLens25M. 
 The Jester dataset seems to be particularly suited to this type of attacks because the true feature vectors are well separated from the convex hull formed by the feature vectors of the other arms: only $5$\% of Jester's feature vectors are within the convex hull of the others versus $8\%$ for MovieLens and $20\%$ for the synthetic dataset.
 %\todo{Can we find a better explanation? The current one doesn't explain why the success rate is better in the synthetic dataset than in MovieLens (probably because there are less arms)}
 As expected, the cost of the attacks is linear on all the datasets (see Figure \ref{fig:cost_attack_one_ctx} in App.~\ref{app:additional_fig_one_ctx}). The cost is also lower for the non-relaxed than for the relaxed version of the attack on \linucb. Unsurprisingly, the cost of the attacks on \lints is the highest %higher than for the other algorithms 
due to the need to guarantee that %the arm 
$a^{\dagger}$ will be chosen with high probability (95\% in our experiments).

% \subsection{Size of $A^\dagger$}
% \begin{figure}
%     \centering
%     \begin{minipage}{0.3\linewidth}
%         \centering
%         \includegraphics[width=\linewidth]{images/target_sets_simulations/attacks_norms_set_targets_all_context_linucb_sets_targets_simulations.pdf}
%         %\caption{Lorem ipsum}
%     \end{minipage}\hfill
%     \begin{minipage}{0.3\linewidth}
%     \centering
%     \includegraphics[width=\linewidth]{images/target_sets_jester/attacks_norms_allall_context_linucb_sets_targets_jester.pdf}
%     % \caption{Synthetic}
%     \end{minipage}\hfill
%     \begin{minipage}{0.3\linewidth}
%     \centering
%     \includegraphics[width=\linewidth]{images/target_sets_movielens/attacks_norms_all_context_linucb_sets_targets_movielens.pdf}
%     % \caption{Jester}
%     \end{minipage}\hfill\\
%         \begin{minipage}{0.3\linewidth}
%         \centering
%         \includegraphics[width=\linewidth]{images/target_sets_simulations/regret_all_context_linucb_sets_targets_simulations.pdf}
%         %\caption{Lorem ipsum}
%     \end{minipage}
%     \begin{minipage}{0.3\linewidth}
%     \centering
%     \includegraphics[width=\linewidth]{images/target_sets_jester/regret_all_context_linucb_sets_targets_jester.pdf}
%     % \caption{Synthetic}
%     \end{minipage}
%     \begin{minipage}{0.3\linewidth}
%     \centering
%     \includegraphics[width=\linewidth]{images/target_sets_movielens/regret_all_context_linucb_sets_targets_movielens.pdf}
%     % \caption{Jester}
    
%     % \caption{MovieLens}
%     \end{minipage}
%     \caption{From left to right, effect of the size of $A^\dagger$ on the regret and the norm of the attacks.}
% \label{fig:costs_plot}
% \end{figure}
% \begin{figure}[h]
%     \centering
%     \includegraphics[width=0.45\textwidth]{images/one_context_3_alg_sim/attacks_norms_allone_context_3_alg_sim_remade.pdf}
%     \includegraphics[width=0.45\textwidth]{images/one_context_3_alg_jester/attacks_norms_allone_context_3_alg_jester_remade.pdf}
%     \includegraphics[width=0.45\textwidth]{images/one_context_3_alg_movielens/attacks_norms_allone_context_3_alg_movielens_remade.pdf}
%     \caption{Total cost of the attacks for the attacks one one context on our synthetic dataset, Jester and MovieLens. As expected, the total cost is linear.}
%     \label{fig:cost_attack_one_ctx}
% \end{figure}
% \include{images/long_term_attacks_graphs/attacks_norms_jester_not_all_context.tex}
% \begin{figure}[H]
% \includegraphics[width=.4\textwidth]{images/all_contexts_3_algs_jester/regret_all_contexts_3_alg_jester.png}
% \includegraphics[width=.4\textwidth]{images/all_contexts_3_algs_jester/attacks_norms_all_contexts_3_alg_jester.png}
% \includegraphics[width=.4\textwidth]{images/all_contexts_3_algs_jester/ratio_success_all_contexts_3_alg_jester.png}
% \includegraphics[width=.4\textwidth]{images/all_contexts_3_algs_jester/time_target_arm_chosen_all_contexts_3_alg_jester.png}
%   \caption{Jester experiment: from left to right, the regret, sum of the norms of the attacks, and percentage of successful attacks for each time step t for experiments conducted on the jester dataset. In blue, the attack described in section \ref{sec:attack_all_context}. In green, a version where attacks can only be performed on 20\% of the incoming contexts and the norm of the attack is multiplied by 5 to compensate. The regret for the attacked algorithms (in blue and green) is linear and the ratio of successful attacks quickly jumps close to 1. The cost of the attacks is only logarithmic (middle graph). The cost of the attacks is slightly smaller when their norms is greater (for the green curve). 95\% confidence intervals are represented on the curves, although they are often not visible. We removed the graphs of the attack norm for attacked \lints since they were linear and flattened the rest of the curves.\label{fig:jester}}
% \end{figure}
% \begin{figure}[H]
% \includegraphics[width=.4\textwidth]{images/all_contexts_3_alg_movielens/regret_all_context_3_alg_movielens.pdf}
% \includegraphics[width=.4\textwidth]{images/all_contexts_3_alg_movielens/attacks_norms_allall_context_3_alg_movielens.pdf}
% \includegraphics[width=.4\textwidth]{images/all_contexts_3_alg_movielens/ratio_success_all_context_3_alg_movielens.pdf}
% \includegraphics[width=.4\textwidth]{images/all_contexts_3_alg_movielens/a_star_chosenall_context_3_alg_movielens.pdf}
%   \caption{MovieLens experiment: from left to right, the regret, sum of the norms of the attacks, and percentage of successful attacks for each time step t for experiments conducted on the movielens dataset.\label{fig:movielens}}
% \end{figure}
% \begin{figure}[H]
%     \includegraphics[width=.4\textwidth]{images/all_context_3_alg_sim/regret_all_contexts_3_alg_sim.png}
%     \includegraphics[width=.4\textwidth]{images/all_context_3_alg_sim/attacks_norms_all_contexts_3_alg_sim.png}
%     \includegraphics[width=.4\textwidth]{images/all_context_3_alg_sim/ratio_success_all_contexts_3_alg_sim.png}
%         \includegraphics[width=.4\textwidth]{images/all_context_3_alg_sim/time_target_arm_chosen_all_contexts_3_alg_sim.png}
%   \caption{Experiments on synthetic dataset: from left to right, the regret, sum of the norms of the attacks, and percentage of successful attacks for each time step t for experiments conducted on the movielens dataset.\label{fig:simulations}}
%   \end{figure}
%   \begin{figure}[t]
%     % \includegraphics[width=.4\textwidth]{images/one_context_3_alg_sim/regret_one_context_3_alg_sim.pdf}
%     \includegraphics[width=.4\textwidth]{images/one_context_3_alg_sim/attacks_norms_allone_context_3_alg_sim.pdf}
%     \includegraphics[width=.4\textwidth]{images/one_context_3_alg_sim/a_star_chosenone_context_3_alg_sim.pdf}
%     % \includegraphics[width=.4\textwidth]{images/one_context_3_alg_sim/a_star_chosenone_context_3_alg_sim.pdf}
%         \includegraphics[width=.4\textwidth]{images/one_context_3_alg_sim/ratio_one_ctx_one_context_3_alg_sim.pdf}
%   \caption{Experiments on synthetic dataset for attacks one one context. We notice that epsilon greedy is relatively easier to attack, especially at the beginning (the ratio of successful attacks is higher faster) and the cost of the attacks is also comparatively lower. The attack cost is similar for \linucb and \lints but the attacks on \linucb are successful much more often. \label{fig:sim_one_ctx}}
%   \end{figure}
%     \begin{figure}[t]
%     \includegraphics[width=.4\textwidth]{images/one_context_3_alg_movielens/regret_one_context_3_alg_movielens.pdf}
%     \includegraphics[width=.4\textwidth]{images/one_context_3_alg_movielens/attacks_norms_allone_context_3_alg_movielens.pdf}
%     \includegraphics[width=.4\textwidth]{images/one_context_3_alg_movielens/a_star_chosenone_context_3_alg_movielens.pdf}
%     % \includegraphics[width=.4\textwidth]{images/one_context_3_alg_sim/a_star_chosenone_context_3_alg_sim.pdf}
%         \includegraphics[width=.4\textwidth]{images/one_context_3_alg_movielens/ratio_one_ctx_one_context_3_alg_movielens.pdf}
%   \caption{Experiments on movielens dataset for attacks one one context. The short-term attacks are successful for epsilon-greedy and when solving the non-relaxed version of the problem for \linucb (the success rate is close to 100\%). However, it fails most of the time for Linear Thomson Sampling and when solving the relaxed problem with \linucb: while the number of times the arm $a^{\dagger}$ is pulled in context $a^{\dagger}$ is way above that of the unattacked version, the success rate is still below 10\%. \label{fig:movielens_one_ctx}}  
%   \end{figure}
%     \begin{figure}[t]
%     \includegraphics[width=.4\textwidth]{images/one_context_3_alg_jester/regret_one_context_3_alg_jester.pdf}
%     \includegraphics[width=.4\textwidth]{images/one_context_3_alg_jester/attacks_norms_allone_context_3_alg_jester.pdf}
%     % \includegraphics[width=.4\textwidth]{images/one_context_3_alg_movielens/a_star_chosenone_context_3_alg_sim_movielens.pdf}
%     % \includegraphics[width=.4\textwidth]{images/one_context_3_alg_sim/a_star_chosenone_context_3_alg_sim.pdf}
%     \includegraphics[width=.4\textwidth]{images/one_context_3_alg_jester/ratio_one_ctx_one_context_3_alg_jester.pdf}
%   \caption{Experiments on jester dataset for attacks one one context. On the jester dataset, all the types of attacks are reasonably successful. As for movielens (Fig. \ref{fig:movielens_one_ctx}), the attacked we obtained by solving the unrelaxed problems for epsilon-greedy and \linucb are almost always successful. Here, the relaxed attacks are also successful more than 75\% of the time for both Linear Thompson Sampling and \linucb. \label{fig:jester_one_ctx}}
%   \end{figure}
%      \begin{figure}[t]
%     \includegraphics[width=.4\textwidth]{images/one_context_3_alg_sim/regret_one_context_3_alg_sim.pdf}
%     \includegraphics[width=.4\textwidth]{images/one_context_3_alg_sim/attacks_norms_allone_context_3_alg_sim.pdf}
%     \includegraphics[width=.4\textwidth]{images/one_context_3_alg_sim/a_star_chosenone_context_3_alg_sim.pdf}
%     % \includegraphics[width=.4\textwidth]{images/one_context_3_alg_sim/a_star_chosenone_context_3_alg_sim.pdf}
%         \includegraphics[width=.4\textwidth]{images/one_context_3_alg_sim/ratio_one_ctx_one_context_3_alg_sim.pdf}
%   \caption{Experiments on a simulated dataset for one context \label{fig:sim_one_ctx}}  
%   \end{figure}
%       \begin{figure}[t]
%     \includegraphics[width=.4\textwidth]{images/rewards_attacks_jester/avg_regret.png}
%     \includegraphics[width=.4\textwidth]{images/rewards_attacks_jester/avg_cost.png}
%     \includegraphics[width=.4\textwidth]{images/rewards_attacks_jester/avg_draws.png}
%   \caption{Reward attacks jester \label{fig:jester_reward_attacks}}
%   \end{figure}
%         \begin{figure}[t]
%     \includegraphics[width=.4\textwidth]{images/rewards_attacks_movielens/avg_regret.png}
%     \includegraphics[width=.4\textwidth]{images/rewards_attacks_movielens/avg_cost.png}
%     % \includegraphics[width=.4\textwidth]{images/one_context_3_alg_movielens/a_star_chosenone_context_3_alg_sim_movielens.pdf}
%     % \includegraphics[width=.4\textwidth]{images/one_context_3_alg_sim/a_star_chosenone_context_3_alg_sim.pdf}
%     \includegraphics[width=.4\textwidth]{images/rewards_attacks_movielens/avg_draws.png}
%   \caption{Reward attacks movielens \label{fig:movielens_reward_attacks}}
%   \end{figure}
%       \begin{figure}[t]
%     \includegraphics[width=.4\textwidth]{images/rewards_attacks_simulations/avg_regret.png}
%     \includegraphics[width=.4\textwidth]{images/rewards_attacks_simulations/avg_cost.png}
%     % \includegraphics[width=.4\textwidth]{images/one_context_3_alg_movielens/a_star_chosenone_context_3_alg_sim_movielens.pdf}
%     % \includegraphics[width=.4\textwidth]{images/one_context_3_alg_sim/a_star_chosenone_context_3_alg_sim.pdf}
%     \includegraphics[width=.4\textwidth]{images/rewards_attacks_simulations/avg_draws.png}
%   \caption{Reward attacks simulations \label{fig:sim_reward_attacks}}
%   \end{figure}

\section{Conclusion}
We presented several settings for online attacks on contextual bandits.
We showed that an attacker can force any contextual bandit algorithm to almost always pull an arbitrary target arm $a^{\dagger}$ with only sublinear modifications of the rewards. When the attacker can only modify the contexts, we prove that \linucb can still be attacked and made to almost always pull an arm in $A^{\dagger}$ by adding sublinear perturbations to the contexts. 
When the attacker can only attack a single context, we derive a feasibility condition for the attacks and we introduce a method to compute some attacks of small instantaneous cost for \linucb, \epsgreedy and \lints.
To the best of our knowledge, this paper is the first to describe effective attacks on the contexts of contextual bandit algorithms. Our numerical experiments, conducted on both synthetic and real-world data, validate our results and show that the attacks on all contexts are actually effective on several algorithms and with more permissible settings. 





% !TEX root = main.tex

\section{Proofs}
In this appendix, we present the proofs of different theoretical results presented in the paper.
\subsection{Proof of Proposition \ref{prop:reward_attack}}\label{app:proof_prop_rewd_attack}

\begin{prop*}
	For any $\delta\in(0, 1/K]$, when using Contextual ACE algorithm (Alg. ~\ref{alg:attacker_rewards}) with perturbed rewards $\tilde{r}^{1}$, with probability at least $1-K\delta$, algorithm $\mathfrak{A}$ pulls \changelm{an arm in $A^{\dagger}$} for $T - o(T)$ time steps and the total cost of attacks is $o(T)$.
\end{prop*}

\begin{proof}
	Let us consider the contextual bandit problem $\mathcal{A}_{1}$, with $K$ arms with contexts $x\in \mathcal{D}$ such that every arm in \changelm{$a^\dagger\in A^\dagger$ has mean reward $\langle \theta_{a^{\dagger}}, x\rangle$} and all other arms has mean $0$. Then the regret of algorithm $\mathfrak{A}$ for this bandit problem is upper-bounded with probability at least $1 - \delta$ by a function $f_{\mathfrak{A}}(T)$ such that $f_{\mathfrak{A}}(T) = o(T)$. In addition, the reward process fed to Alg. $\mathfrak{A}$ by the attacker is a stationary reward process with $\sigma^{2}$-subgaussian noise. Therefore, the number of times algorithm $\mathfrak{A}$ pulls an arm \changelm{not in $A^{\dagger}$ is upper-bounded by \changee{$f_{\mathfrak{A}}(T)/\min_{x\in \mathcal{D}} \Delta(x)$ where for every context $x\in\mathcal{D}$, let $a
^{\dagger}_{\star}(x) := \arg\max_{a\in A^{\dagger}} \langle x, \theta_{a}\rangle$ and $\Delta(x) = \langle x, \theta_{a^{\dagger}_{\star}(x)}\rangle - \max_{a\in A^{\dagger}, a\neq a_{\star}^{\dagger}(x)} \langle x, \theta_{a}\rangle$}}. 

\changelm{In addition, the total cost of the attack is upper-bounded by $\max_{a\in \llbracket 1, K\rrbracket} \max_{x\in \mathcal{D}} |\langle x, \theta_{a}\rangle| (T - N_{A^{\dagger}}(T))$ where $N_{A^{\dagger}}(T)$ is the number of times an arm  in $A^{\dagger}$ has been pulled up to time $T$. Thanks to the previous argument, $T - N_{A^{\dagger}}(T) \leq  f_{\mathfrak{A}}(T)/\min_{x\in \mathcal{D}}\Delta(x)$.}
\end{proof}
% \subsection{Proof of Proposition \ref{prop:cost_attack_all_ctx}:}
% \begin{proof}
% The bound of the number of times the arm $a^*$ is the same as for proposition \ref{prop:reward_attack}.
% As the attacker only attacks when the arm $a^{\star}$ is not pulled and the contexts are bounded by 1, the total cost of each attack is upper bounded by $\delta$, hence the bound on the total cost of the attacks.

% \end{proof}

\subsection{Proof of Proposition \ref{prop:cost_attack_all_ctx}}\label{app:proof_attack_all_ctx}

\begin{prop*}
Using the attack described in Alg.~\ref{alg:context_attack_protocol}, for any $\delta\in (0, 1/K]$, with probability at least $1 - K\delta$, the number of times \linucb does not pull \changelm{an arm in $A^{\dagger}$} is at most:
\begin{align*}
    \sum_{j\changelm{\notin A^{\dagger}}} N_{j}(T) \leq 32K^{2}\left( \frac{\lambda}{\alpha^{2}} + \sigma^{2}d\log\left(\frac{\lambda d + TL^2\alpha^{2}}{d\lambda\delta}\right) \right)^{3}
\end{align*}
with $N_{j}(T)$ the number of times arm $j$ has been pulled after $T$ steps, $|| \theta_{a}|| \leq S$ for all arms $a$, $\lambda$ the regularization parameter of \linucb and for all $x\in \mathcal{D}$, $||x||_{2}\leq L$. The total cost for the attacker is bounded by:
\begin{align*}
    \sum_{t=1}^{T} c_{t} \leq \frac{64K^{2}}{\nu}\left( \frac{\lambda}{\alpha^{2}} + \sigma^{2}d\log\left(\frac{\lambda d + TL^2\alpha^{2}}{d\lambda\delta}\right) \right)^{3}
\end{align*}
\end{prop*}

\begin{proof}
Let $a_{t}$ be the arm pulled by \linucb at time $t$. For each arms $a$, let $\tilde{\theta}_a(t)$ be the result of the linear regression with the attacked context and $\hat{\theta}_{a}(t, \lambda/\alpha^{2})$ the one with the unattacked context and a regularization of $\frac{\lambda}{\alpha^{2}}$. At any time step $t$, we can write, for all $a\changebrtwo{\not\in A^\dagger}$:

\begin{align*}
\tilde{\theta}_a(t) &=  \left(\lambda I_d + \sum_{l=0, a_{l} = a}^{t} \alpha^{2} x_l x_l^{\intercal}\right)^{-1} \sum_{k=0, a_{k} = a}^{t} r_k \alpha x_{k} \\
&= \frac{1}{\alpha} \left(\frac{\lambda}{\alpha^2} I_d + \sum_{k=0, a_{k} = a}^t x_k x_k^{\intercal}\right)^{-1} \sum_{k=0, a_{k} = a}^t r_k x_k \\
&= \frac{\hat{\theta}_{a}(t,\lambda/\alpha^{2})}{\alpha}
\end{align*}
\changelm{We also note that, since the contexts are not modified for arms in  $a^\dagger\in A^\dagger$: $\tilde{\theta}_{a^\dagger}(t)=\hat{\theta}_{a^\dagger}(t,\lambda)$. In addition, for any context $x$ and arm $a\notin A^\dagger$, the exploration term used by \linucb becomes:}
\begin{align}
    ||x||_{\tilde{V}_{a,t}^{-1}}&= \frac{1}{\alpha} ||x||_{\hat{V}_{a,t}^{-1}}
\end{align}
where $\tilde{V}_{a,t} = \lambda I_d + \sum_{l=0, a_{l} = a}^{t} \alpha^{2} x_l x_l^{\intercal}$ and $\hat{V}_{a,t}^{-1} =\lambda/ \alpha^2 I_d + \sum_{k=0, a_{k} = a}^t x_k x_k^{\intercal}$. For a time $t$, if presented with context $x_{t}$ \linucb pulls arm \changelm{$a_{t} \notin A^{\dagger}$,} we have:
\begin{align*}
\alpha\left(\left\langle \hat{\theta}_{a^\dagger}(t), x_{t} \right\rangle +\beta_{a^\dagger}(t)||x_t||_{V_{a^\dagger,t}^{-1}}\right)\leq \left\langle \hat{\theta}_{a_{t}}(t, \lambda/\alpha^{2}), x_{t} \right\rangle +  \beta_{a_{t}}(t)||x_{t}||_{\hat{V}_{a_{t},t}^{-1}} 
\end{align*}

As \changelm{$\alpha = \frac2\nu\geq\min_{a^\dagger\in A^\dagger}\frac{2}{\left\langle \theta_{a^\dagger}, x_{t} \right\rangle}$}, we deduce that on the event that the confidence sets (Theorem $2$ in \cite{abbasi2011improved}) hold for arm $a^{\star}$: 
\begin{align*}
    2&\leq\left\langle \hat{\theta}_{a_{t}}(t, \lambda/\alpha^{2}), x_{t} \right\rangle +  \beta_{a_{t}}(t)||x_{t}||_{\hat{V}_{a_{t},t}^{-1}}\leq \langle\theta_{a_{t}}, x_{t}\rangle+2\beta_{a_{t}}(t)||x_{t}||_{\hat{V}_{a_{t},t}^{-1}}
\end{align*}
Thus, $1 \leq 2 - \langle\theta_{a_{t}}, x_{t}\rangle \leq 2\beta_{a_{t}}(t)||x_{t}||_{\hat{V}_{a_{t},t}^{-1}}$. Therefore,
\begin{align*}
    \sum_{t=1}^{T} \mathds{1}_{\{a_{t}\notin A^{\dagger}\}} &\leq \sum_{t=1}^{T} \min(2\beta_{a_{t}}(t)||x_{t}||_{\hat{V}_{a_{t},t}^{-1}},1)\mathds{1}_{\{a_{t} \notin A^{\dagger}\}}\\
    &\leq \sum_{j\notin A^{\dagger}} 2\beta_{j}(T)\sqrt{\sum_{t=1}^{T}\mathds{1}_{\{a_{t}=j\}}\sum_{t=1, a_{t}=j}^{T} \min(1, ||x_{t}||^{2}_{\hat{V}_{j,t}^{-1}})}&
  \end{align*}
  But using Lemma $11$ from \cite{abbasi2011improved} and the bound on the $\beta_{j}(T)$ for all arms $j$, we have with Jensen inequality:
  \begin{align*}
    \sum_{t=1}^{T} \mathds{1}_{\{a_{t}\notin A^{\dagger}\}} \leq &4\sqrt{K\sum_{t=1}^{T} \mathds{1}_{\{a_{t}\notin A^{\dagger}\}}d\log\left(1 + \frac{\alpha^2TL^2}{\lambda d}\right)}\\
    &\times\Big( \sqrt{\lambda/\alpha^{2}} S + \sigma\sqrt{2\log(1/\delta) + d\log(1 + \frac{\alpha^2TL^2}{\lambda d})}\Big)
\end{align*}
\end{proof}



\subsection{Proof of Theorem \ref{thm:feasibility_attack_one_user}}\label{app:feasibility_attack_one_user}

\begin{thm*}
For any $\xi>0$, Problem \eqref{eq:attack_one_user} is feasible if and only if:
\begin{align}\label{eq:feasibilty_condition_bis}
\exists \theta \in  \changelm{\bigcup_{a^\dagger\in A^{\dagger}}}\mathcal{C}_{t, a^{\dagger}}, \qquad \theta\not\in \text{Conv}\left( \bigcup_{a\notin A^{\dagger}} \mathcal{C}_{t,a}\right)
\end{align}
	where for every arm $a$,  $\mathcal{C}_{t,a} := \big\{\theta \mid ||\theta - \hat{\theta}_{a}(t)||_{\tilde{V}_{a,t}} \leq \beta_{a}(t) \big\}$ with $\hat{\theta}_{a}(t)$ the least squares estimate for arm $a$ built by \linucb and 
	$$\tilde{V}_{a,t} = \lambda I_{d} + \sum_{l=1, x_{l}\neq x^{\dagger}}^{t} \mathds{1}_{\{a_{l} = a\}}x_{l}x_{l}^{\intercal} + \sum_{l=1, x_{l}= x^{\dagger}}^{t} \mathds{1}_{\{a_{l} = a\}}\tilde{x}_{l}\tilde{x}_{l}^{\intercal} $$ 
	the design matrix of \linucb at time $t$ for all arms $a$ (where $\tilde{x}_{l}$ is the modified context)
\end{thm*}

\begin{proof}
The proof of Theorem \ref{thm:feasibility_attack_one_user} is decomposed in two parts. 

First, let us assume that Equation \eqref{eq:feasibilty_condition_bis} is satisfied. Then, \changebrtwo{let us define $a^\dagger \in A^\dagger$ such that} $\theta \in \mathcal{C}_{t,a^{\dagger}}\setminus \text{Conv}\left( \bigcup_{a\notin A^{\dagger}} \mathcal{C}_{t,a}\right) $, then by the theorem of separation of convex sets applied to $\mathcal{C}_{t,a^{\dagger}}$ and $\{ \theta \}$. There exists a vector $v$ and $c_{1}< c_{2}$ such that for all $y \in \text{Conv}\left( \bigcup_{a\neq a^{\dagger}} \mathcal{C}_{t,a}\right)$:
\begin{align*}
\left\langle y, v\right\rangle \leq c_{1} < c_{2} \leq \left\langle \theta,v\right\rangle.
\end{align*}
Hence, for $\xi>0$ we have that for $\tilde{v} = \frac{\xi}{c_{2}-c_{1}} v$ that:
\begin{align*}
    \left\langle y, \tilde{v}\right\rangle + \xi \leq \left\langle \theta, \tilde{v} \right\rangle
\end{align*}
So the problem is feasible.

Secondly, let us assume that an attack is feasible. Then there exists a vector $y$ such that:
\begin{align}
    \changebrtwo{\max_{a^\dagger \in A^\dagger}}\max_{\theta\in \mathcal{C}_{t,a^{\dagger}}} \left\langle y, \theta\right\rangle > c_{1} := \max_{a\notin A^{\dagger}} \max_{\theta\in \mathcal{C}_{t,a}} \left\langle y, \theta\right\rangle
    \label{eq:feasible_in_proof}
\end{align}
% \changebrtwo{We can define $a^\dagger = \argmax_{a\neq a^{\dagger}} \max_{\theta\in \mathcal{C}_{t,a}}$, which verifies:}
%     \begin{align*}
%     \max_{\theta\in \mathcal{C}_{t,a^{\dagger}}} \left\langle y, \theta\right\rangle > c_{1} := \max_{a\neq a^{\dagger}} \max_{\theta\in \mathcal{C}_{t,a}} \left\langle y, \theta\right\rangle
%     \end{align*}
\changelm{
	Let us reason by contradiction. We assume that $ \bigcup_{a\in A^{\dagger}}\mathcal{C}_{t,a^{\dagger}} \subset \text{Conv}\left( \bigcup_{a\notin A^{\dagger}} \mathcal{C}_{t,a}\right)$ and consider 
	\begin{align*}
	    \theta^*\in\bigcup_{a\in A^{\dagger}}\mathcal{C}_{t,a^{\dagger}}\text{ such that } \left\langle y, \theta^*\right\rangle=\max_{a^\dagger \in A^\dagger}\max_{\theta\in \mathcal{C}_{t,a^{\dagger}}} \left\langle y, \theta\right\rangle
	\end{align*}
	As we assumed $ \bigcup_{a\in A^{\dagger}}\mathcal{C}_{t,a^{\dagger}} \subset \text{Conv}\left( \bigcup_{a\notin A^{\dagger}} \mathcal{C}_{t,a}\right)$, there exists $n\in\mathbb{N}^{\star}$, $\lambda_{1},\cdots, \lambda_{n}\geq 0$ and $\theta_{1}, \cdots, \theta_{n}\in \bigcup_{a\notin A^{\dagger}} \mathcal{C}_{t,a}$ \text{such that}
	\begin{align*}
	    \theta^* = \sum_{i=1}^{n} \lambda_{i}\theta_{i}\text{ and } \sum_{i=1}^{n} \lambda_{i} = 1
	\end{align*}
	Thus
\begin{align}
    \left\langle y, \theta^*\right\rangle = \sum_{i} \lambda_{i} \left\langle y, \theta_{i} \right\rangle \leq c_{1}\sum_{i=1}^{n} \lambda_{i} = c_{1}\label{cdas}
\end{align}
\changebrtwo{We assumed that the problem is feasible, so $c_{1}<  
\left\langle y, \theta^*\right\rangle$ according to Eq.~\ref{eq:feasible_in_proof}. It} contradicts Eq. \ref{cdas}.
}
\end{proof}


\subsection{Condition of Sec.~\ref{sec:attack_one_context}}\label{app:condition_linear}
\begin{figure}[ht]
\centering
    \includegraphics[width=0.5\linewidth]{sections/appendix/nips2020-bandits/images/condition_feasibility_acceptance.pdf}
	\caption{Illustrative example of condition \eqref{eq:feasibilty_condition}. The target arm is arm $3$ or $5$ and the dashed black line is the convex hull of the other confidence sets. The ellipsoids are the confidence sets $\mathcal{C}_{t,a}$ for each arm $a$. If we consider only arms $\{1,2,4,5\}$, and we use $5$ as the target arm, the condition \eqref{eq:feasibilty_condition} is satisfied as there is a $\theta$ outside the convex hull of the other confidence sets. On the other hand, if we consider arms $\{1,2,3,4\}$ and we use $3$ as the target arm, the condition is not satisfied anymore. \label{fig:feasibility_condition}}
\end{figure}

\changebr{Let us assume that \changebrtwo{there is an arm in $a^\dagger\in A^\dagger$ which is} optimal for some contexts. More formally, there exists a subspace $V\subset \mathcal{D}$ such that:} 
\begin{equation*}    
\forall x\in V, \exists a^{\dagger}_{\star}(x)\in A^\dagger, \forall a\in \llbracket 1, K\rrbracket\setminus\{a^{\dagger}_{\star}(x)\} \qquad \langle x, \theta_{a^{\dagger}_{\star}(x)}\rangle > \left\langle x, \theta_{a}\right\rangle.
\end{equation*}
\changebr{We also assume that} the distribution of the contexts is such that, for all $t$, $\mu := \mathbb{P}\left(x_{t}\in V\right) >0$.
Then\changebr{,} the regret is lower-bounded in expectation by:
\begin{align*}
    \mathbb{E}(R_{T}) &= \mathbb{E}\left(\sum_{t=1}^{T} \mathds{1}_{\{x_{t}\in V\}}\big( \left\langle x_{t}, \theta_{a^{\dagger}_{\star}(x_{t})} - \theta_{a_{t}}\right\rangle\big)\right) \geq \mu m(T) \min_{x\in V} \max_{a\neq a^{\dagger}_\star(x)} \langle \theta_{a^{\dagger}_{\star}(x)} - \theta_{a}, x\rangle
\end{align*}
where $m(T)$ is the expected number of times $t\leq T$ such that condition \eqref{eq:feasibilty_condition} is not met. \changebr{\linucb guarantees that}  $\mathbb{E}(R_{T}) \leq \mathcal{O}(\sqrt{T})$ for every $T$. Hence, $m(T) \leq \mathcal{O}\left(\frac{\sqrt{T}}{\mu\min_{x\in V}\max_{a\neq a^{\dagger}_\star(x)} \langle \theta_{a^{\dagger}_{\star}(x)} - \theta_{a}, x\rangle}\right)$. This means that, in an unattacked problem, condition \eqref{eq:feasibilty_condition} is met $T - \mathcal{O}(\sqrt{T})$ times. On the other hand, when the algorithm is attacked the regret of \linucb is not sub-linear as the confidence bound for the target arm is not valid anymore. Hence we cannot provide the same type of guarantees for the attacked problem.



% !TEX root = main.tex
\section{Experiments}

\subsection{Datasets and preprocessing}\label{app:experiments_setup}
% \todoe{This section also explain how we preprocessed the real-wolrd datasets too.}

We present here the datasets used in the article and how we preprocess them for numerical experiments conducted in Section \ref{sec:experiments}.

We consider two types of experiments, one on synthetic data with a contextual MAB problems with $K = 10$ arms such that for every arm $a$, $\theta_{a}$ is drawn from a folded normal distribution in dimension $d = 30$. We also use a finite number of contexts ($10$), each of them is drawn from a folded normal distribution projected on the unit circle multiplied by a uniform radius variable (i.i.d. across all contexts). Finally, we scale the expected rewards in $(0,1]$ and the noise is drawn from a centered Gaussian distribution $\mathcal{N}(0, 0.01)$. 
% The target arm is chosen at random among all arms. %I removed it because this is not always true 

The second type of experiments is conducted in the real-world datasets Jester \cite{goldberg2001eigentaste} and MovieLens25M \cite{harper2015movielens}. Jester consists of joke ratings on a continuous scale from $-10$ to $10$ for $100$ jokes from a total of $73421$ users. We use the features extracted via a low-rank matrix factorization ($d = 35$) to represent the actions (i.e., the jokes). We consider a complete subset of $40$ jokes and $19181$ users . Each user  rates all the $40$ jokes. At each time, a user is randomly selected from the $19181$ users and mean rewards are normalized in $[0, 1]$. The reward noise is $\mathcal{N} (0, 0.01)$. The second dataset we use is MovieLens25M. It contains $25000095$ ratings created by $162541$ users on $62423$ movies. We perform a low-rank matrix factorization to compute users features and movies features. We keep only movies with at least $1000$ ratings, which leave us with $162539$ users and $3794$ movies. At each time step, we present a random user, and the reward is the scalar product between the user feature and the recommend movie feature. All rewards are scaled to lie in $[0,1]$ and a Gaussian noise $\mathcal{N}(0, 0.01)$ is added to the rewards. 


\subsection{Attacks on Rewards}\label{app:additional_fig_rwds}
In this appendix, we present empirical evolution of the total cost and the number of draws for a unique target arm as a function of the attack parameter $\gamma$ for the Contextual ACE attack with perturbed rewards $\tilde{r}^{2}$ on generated data.

\begin{figure}[htbp]
    \centering
    \subfigure[Total cost]{\includegraphics[width=0.35\textwidth]{images/attack_reward/simulations/cost_epsilon.pdf}}
    \subfigure[Number of draws]{\includegraphics[width=0.3\textwidth]{images/attack_reward/simulations/draws_epsilon.pdf}}
    \caption{Total cost of attacks and number of draws of the target arm at $T = 10^{6}$ as a function of $\gamma$ on synthetic data}
    \label{fig:synth_cost_draws_gamma}
\end{figure}

Fig.~\ref{fig:synth_cost_draws_gamma} (left) shows that the total cost of attacks seems to be quite invariant w.r.t.  $\gamma$ except when $\gamma \rightarrow 0$ because the difference between the target arm and the other becomes negligible. This is also depicted by the total number of draws (Fig.~\ref{fig:synth_cost_draws_gamma}, Right) as the number of draws plummets when $\gamma \rightarrow 0$.

\begin{table}
\begin{center}
	\caption{\label{table:number_of_draws}Number of draws of the target arm $a^{\dagger}$ at $T=10^{6}$, for the synthetic data, $\gamma = 0.22$ for the Contextual ACE algorithm and for the Jester and MovieLens datasets $\gamma = 0.5$\otc{.}}
\begin{tabular}{lccc}
\toprule
{} & Synthetic &  Jester & Movilens \\
\midrule
\linucb          &      $86, 731.6$ &  $23, 548.16$ &    $25, 017.31$ \\
CACE \linucb     &     $996, 238.6$ &  $921, 083.69$ &   $944, 721.28$ \\
Stationary CACE \linucb &     $995, 578.88$ & $862, 095.67$ &   $931, 531.6$ \\
\epsgreedy       &     $111, 380.44$ & $21, 911.54$ &    $3, 165.81$    \\
CACE \epsgreedy  &    $999, 812.92$ &  $999, 755.72$ &   $999, 776.82$ \\
Stationary CACE \epsgreedy &     $999, 806.32$ &  $999, 615.98$ &   $999, 316.76$ \\
\lints           &      $91, 664.8$ &  $23, 398.3$ &    $30, 189.84$ \\
CACE \lints      &      $998, 997.04$ &   $976, 708.9$ &   $990, 250.67$ \\
Stationary CACE \lints &     $977, 850.96$ & $784, 715.62$ &   $845, 512.98$ \\
\expfour         &     $93, 860.4$ &  $29, 147.01$ &    $17, 985.78$ \\
CACE \expfour    &    $992, 793.36$ &   $989, 214.36$ &    $936, 230.4$ \\
Stationary CACE \expfour &     $993, 673.24$ &  $988, 463.56$ &   $934, 304.23$ \\
\bottomrule
\end{tabular}

\end{center}
\end{table}

\subsection{Attacks on all Contexts}\label{app:additional_fig_all_ctx}

% \begin{figure}[h]
%     \centering
%     \subfigure[Synthetic data]{\includegraphics[width=0.33\textwidth]{images/all_context_3_alg_sim/gen_cost_all.pdf}}\hfill
%     \subfigure[Jester Dataset]{\includegraphics[width=0.23\textwidth]{images/all_contexts_3_algs_jester/gen_jester_cost_all.pdf}}\hfill
%     \subfigure[MovieLens Dataset]{\includegraphics[width=0.23\textwidth]{images/all_contexts_3_alg_movielens/gen_movielens_cost_all.pdf}}
%     \caption{Total cost of the attacks for the attack of Sec.~\ref{sec:attack_all_context} on our synthetic dataset, Jester and MovieLens}
%     \label{fig:cost_all_algs_attack_all_ctx}
% \end{figure}

% Fig.~\ref{fig:cost_all_algs_attack_all_ctx} shows the total cost for all the attacks (that is to say including CC \lints and CC$20$ \lints compared to Fig~\ref{fig:costs_plot} (Right)). This figure shows that even though the total cost of attacks is linear for the synthetic and Jester dataset, it seems that for MovieLens the attacker achieves their goal with a logarithmic total. Therefore, despite the fact that the estimate of $\theta_{a^\dagger}$ can be polluted by attacked samples, it seems that \lints can still pick up $a^{\dagger}$ as being optimal for this particular instance.

\begin{figure}[h]
   \begin{minipage}{0.25\linewidth}
        \centering
        \includegraphics[width=0.85\linewidth]{images/regret_cost_attacks_context/simulations/legend.pdf}
        %\caption{Lorem ipsum}
    \end{minipage}\hfill
    \begin{minipage}{0.25\linewidth}
    \centering
    \includegraphics[width=0.95\linewidth]{images/regret_cost_attacks_context/simulations/standalone.pdf}
    % \caption{Synthetic}
    \end{minipage}\hfill
    \begin{minipage}{0.25\linewidth}
    \centering
    \includegraphics[width=0.85\linewidth]{images/regret_cost_attacks_context/jester/standalone.pdf}
    % \caption{Jester}
    \end{minipage}\hfill
    \begin{minipage}{0.25\linewidth}
    \centering
    \includegraphics[width=0.85\linewidth]{images/regret_cost_attacks_context/movielens/standalone.pdf}
    % \caption{MovieLens}
    \end{minipage}
    \label{fig:regret_all_algs_attack_all_ctx}
\end{figure}

Fig.~\ref{fig:regret_all_algs_attack_all_ctx} shows the regret for all the attacks. This figure shows that even though the total cost of attacks is linear for algorithms like \lints in the synthetic dataset, the regret is linear. More generally, we observe that the regret is linear for all attacked algorithms on all datasets.

\subsection{Attack on a single context}\label{app:additional_fig_one_ctx}

The attacks are computed by solving the optimization problems \ref{eq:attack_one_user} and \ref{eq:relaxed_attack_one_user} (Sec.~\ref{sec:attack_one_context}). We choose the libraries according to their efficiency for each problem we need to solve. For Problem \eqref{eq:relaxed_attack_one_user} and Problem \eqref{eq:relaxed_TS_attack_one_user} we use \textsc{cvxpy}  \cite{cvxpy_rewriting} and the \textsc{ECOS} solver. %to solve the convex relaxations of the optimization problems for \linucb and \lints (equations \ref{eq:relaxed_attack_one_user}, \ref{eq:relaxed_TS_attack_one_user})
For Problem \eqref{eq:attack_one_user} we use the \textsc{SLSQP} method from the Scipy optimize library \cite{scipy} to solve the full \linucb problem (Equation \ref{eq:attack_one_user}) and \textsc{quadprog} to solve the quadratic problem to attack \epsgreedy.


\begin{figure}[htbp]
    \centering
    \subfigure[Synthetic data]{\includegraphics[width=0.33\textwidth]{images/one_context/simulations/standalone.pdf}}\hfill
    \subfigure[Jester Dataset]{\includegraphics[width=0.33\textwidth]{images/one_context/jester/standalone.pdf}}\hfill
    \subfigure[MovieLens Dataset]{\includegraphics[width=0.33\textwidth]{images/one_context/movielens/standalone.pdf}}
    \caption{Total cost of the attacks for the attacks one one context on our synthetic dataset, Jester and MovieLens. As expected, the total cost is linear.}
    \label{fig:cost_attack_one_ctx}
\end{figure}
% !TEX root = main.tex
\section{Problem \eqref{eq:relaxed_TS_attack_one_user} as a Second Order Cone (SOC) Program}\label{app:one_context_ts_linucb}
Problem \eqref{eq:relaxed_attack_one_user} and Problem \eqref{eq:relaxed_TS_attack_one_user} are both SOC programs. We can see the similarities between both problems as follows. Let us define for every arm \changee{$a\not\in A^{\dagger}$}, the ellipsoid:

$$\mathcal{C}_{t,a}^{'} := \Big\{ y \in \mathbb{R}^{d} \mid || y - \hat{\theta}_{a}(t)||_{A_{a}^{-1}(t)} \leq \upsilon\Phi^{-1}\left( 1 - \frac{\delta}{K-|A^{\dagger}|}\right)\Big\}$$

with $A_{a}(t) = \tilde{V}_{a}^{-1}(t) + \tilde{V}_{a^{\dagger}}^{-1}(t)$ with $\tilde{V}_{a}(t)$ and $\tilde{V}_{a^{\dagger}}(t)$ the design matrix built by \lints and $\hat{\theta}_{a}(t)$ the least squares estimate of $\theta_a$ at time $t$. Therefore for an arm $a$, the constraint in Problem \eqref{eq:relaxed_TS_attack_one_user} can be written for any $y\in\mathbb{R}^{d}$ and some arm $a^{\dagger}\in A^{\dagger}$ as:
\begin{align*}
    \left\langle x^{\star}+y, \hat{\theta}_{a^{\dagger}}(t)\right\rangle - \xi \geq \max_{z\in \mathcal{C}_{t,a}^{'}} \left\langle z, x^{\star} + y\right\rangle
\end{align*}
Indeed for any $x\in \mathbb{R}^{d}$,
\begin{align*}
    \max_{y\in \mathcal{C}_{t,a}^{'}} \left\langle y,x\right\rangle &= \left\langle x, \hat{\theta}_{a}(t)\right\rangle + \upsilon\Phi^{-1}\left( 1 - \frac{\delta}{K-|A^{\dagger}|}\right)\times\max_{ ||A_{a}^{-1/2}(t)u||_{2}\leq 1} \left\langle u, x\right\rangle\\
    &= \left\langle x, \hat{\theta}_{a}(t)\right\rangle + \upsilon\Phi^{-1}\left( 1 - \frac{\delta}{K-|A^{\dagger}|}\right)\max_{ ||z||_{2}\leq 1} \left\langle z, A_{a}^{1/2}(t)x\right\rangle \\
    &= \left\langle x, \hat{\theta}_{a}(t)\right\rangle + \upsilon\Phi^{-1}\left( 1 - \frac{\delta}{K-|A^{\dagger}|}\right)\lVert A_{a}^{1/2}(t)x\rVert_{2}
\end{align*}
Thus, the constraint is feasible if and only if:
\begin{align*}
    \hat{\theta}_{a^{\dagger}}(t) \not\in \text{Conv}\left( \bigcup_{a\not\in A^{\dagger}}  \mathcal{C}_{t,a}^{'}\right)
\end{align*}

\section{Appendix: Attacks on Adversarial Bandits}\label{app:adversarial_rewards}
In the previous sections, we studied algorithms with sublinear regret $R_T$, \ie mainly bandit algorithms designed for stochastic stationary environments. Adversarial algorithms like \expfour do not provably \otc{enjoy} a \changee{sublinear \textbf{stochastic} regret $R_{T}$ (as defined in the introduction) \footnote{\expfour enjoys a sublinear hindsight regret though. Showing a sublinear upper-bound for the stochastic regret of \expfour is still an open problem (see Section $29.1$ in \cite{lattimore2018bandit})}}. In addition, because this type of algorithms are, by design, robust to non-stationary environments, one could expect them to induce a linear cost on the attacker. In this section, we show that this is not the case for most contextual adversarial algorithms. Contextual adversarial algorithms are studied through the reduction to the bandit with expert advice problem. This is a bandit problem with $K$ arms where at every step, $N$ experts suggest a probability distribution over the arms. The goal of the algorithm is to learn which expert gets the best expected reward in hindsight after $T$ steps. The regret in this type of problem is defined as $R_{T}^{\text{exp}} = \mathbb{E}\left( \max_{m\in \llbracket 1, N \rrbracket}\sum_{t=1}^{T} \sum_{j=1}^{K} E_{m,j}^{(t)}r_{t,j} - r_{t,a_{t}}\right)$
% \begin{equation*}
% R_{T}^{\text{exp}} = \mathbb{E}\left( \max_{m\in \llbracket 1, N \rrbracket}\sum_{t=1}^{T} \sum_{j=1}^{K} E_{m,j}^{(t)}r_{t,j} - r_{t,a_{t}}\right)
% \end{equation*} 
where $E_{m,j}^{(t)}$ is the probability of selecting arm $j$ for expert $m$. In the case of contextual adversarial bandit\changebr{s}, the experts first observe the context $x_{t}$ before recommending an expert $m$. 
% In the setting studied in \otc{the present paper, we assume that the rewards are linear. However, defining  a no-regret, similar to the stochastic regret, algorithm for adversarial bandit with this assumption is still an open problem\cite{lattimore2018bandit}.}
% \todobr{I'm not sure I understand this sentence. Does it mean that getting a sublinear regret for linear adversarial bandits is an open problem ? 
% Should we say: In our setting, we assume that the rewards are linear. Finding an sublinear-regret algorithm for adversarial bandits in this setting is still an open problem \cite{lattimore2018bandit}.}
Assuming the current setting with linear rewards, we can show that if an algorithm $\mathfrak{A}$, like \expfour, enjoys a sublinear regret $R_{T}^{\text{exp}}$, then, using the Contextual ACE attack with either $\tilde{r}^{1}$ or $\tilde{r}^{2}$, the attacker can fool the algorithm into pulling arm $a^{\dagger}$ a linear number of times under some \otc{mild} assumptions. However, attacking contexts for this type of algorithm is difficult because, even though the rewards are linear, the experts are not assumed to use a specific model for selecting an action.

% However, for contextual adversarial algorithms like \expfour the regret is not defined in the same way. But through the reduction to bandit with expert advice. Where 
% Proposition \ref{prop:reward_attack} assumes that the regret $R_{T}$ of algorithm $\mathfrak{A}$ is of order $o(T)$ which excludes adversarial algorithms in like \expfour. However, it does not mean that adversarial algorithms are robust to adversarial attacks similar to Contextual ACE. Indeed, for contextual adversarial algorithms the regret is defined through a reduction to bandit with expert advice where at each step a context is presented to a set of $N$ policies and the bandit algorithm has to find the best policy in hindsight after $T$ steps. In that case, the regret is defined as:

% By assuming that to be in the same setting as previously, that is to say for all arms $a$, the reward for arm $a$ is $\langle \theta_{a}, x_{t}\rangle + \eta_{t}$, we can show a similar result concerning the number of times arm $a^{\dagger}$ is pulled as in proposition \ref{prop:rewards_attack} using Contextual ACE with either $\tilde{r}^{1}$ or $\tilde{r}^{2}$ as adversarial algorithms can adapt to no-stationary reward processes.
%\todol{Explain the condition on the expert}
\begin{prop}\label{prop:rwd_attack_adv}
	Suppose an adversarial algorithm $\mathfrak{A}$ satisfies a regret $R_{T}^{\exp}$ of order $o(T)$ for any bandit problem and that there exists an expert $m^{\star}$ such that $ T - \sum_{t=1}^{T} \mathbb{E}\left(E^{(t)}_{m^{\star}, a_{t,\star}^{\dagger}}\right) = o(T)$ with $a_{t,\star}^{\dagger}$ the optimal arim in $A^{\dagger}$ at time $t$. Then attacking alg. $\mathfrak{A}$ with Contextual ACE leads to pulling arm $a^{\dagger}$, $T-o(T)$ of times in expectation with a total cost of $o(T)$ for the attacker.
\end{prop}

\begin{proof}
Similarly to the proof of Proposition \ref{prop:reward_attack}, let's define the bandit with expert advice problem, $\mathcal{A}_{i}$, such that at each time $t$ the reward vector is $(\tilde{r}^{i}_{t,a})_{a}$ (with $i\in\{1, 2\}$). The regret of this algorithm is: $\Tilde{R}_{T}^{i,\text{exp}} = \mathbb{E}\left( \max_{m\in \llbracket 1, N \rrbracket}\sum_{t=1}^{T} E_{m}^{(t)}\Tilde{r}^i_{t} - \Tilde{r}^i_{t,a_{t}}\right)\in o(T)$. The regret of the learner is: $\mathbb{E}\left( \max_{m\in \llbracket 1, N \rrbracket}\sum_{t=1}^{T} E_{m}^{(t)}r_{t} - r_{t,a_{t}}\right)$ where $a_t$ are the actions taken by algorithm $\mathcal{A}_i$ to minimize $\Tilde{R}_{T}^{i,\text{exp}}$. Then we have:
\begin{align*}
    \Tilde{R}_{T}^{i,\text{exp}} \geq \mathbb{E}\left(\sum_{t=1}^{T}\sum_{j=1}^{K} (E_{m^{\star}, j}^{(t)} - \mathds{1}_{\{\changee{j = a_{t,\star}^{\dagger}}\}})\tilde{r}_{t,j}^{i} + \sum_{t=1}^{T} \tilde{r}^{i}_{t, a^{\dagger}_{t,\star}} - \tilde{r}^{i}_{t,a_{t}} \right)
\end{align*}
Therefore, 
\begin{align*}
  \mathbb{E}\left(\sum_{t=1}^{T} \tilde{r}^{i}_{t, a^{\dagger}_{t, \star}} - \tilde{r}^{i}_{t,a_{t}} \right) &\leq \Tilde{R}_{T}^{i,\text{exp}} + \mathbb{E}\left(\sum_{t=1}^{T}\sum_{j=1}^{K} (\mathds{1}_{\{\changee{j = a_{t,\star}^{\dagger}}\}} - E_{m^{\star}, j}^{(t)})\tilde{r}_{t,j}^{i}\right) \\
  &\leq \Tilde{R}_{T}^{i,\text{exp}} + \mathbb{E}\left(\sum_{t=1}^{T}(1 - E_{m^{\star}, a^{\dagger}_{t,\star}}^{(t)})\tilde{r}_{t,j}^{i}\right) \\
  &\leq \Tilde{R}_{T}^{i,\text{exp}} + \mathbb{E}\left(\sum_{t=1}^{T}(1 - E_{m^{\star}, a^{\dagger}_{t,\star}}^{(t)})\right)
\end{align*}

For strategy $i=1$, we have:
\begin{align*}
    \mathbb{E}\left(\sum_{t=1}^{T} \tilde{r}^{1}_{t, a^{\dagger}_{t, \star}} - \tilde{r}^{1}_{t,a_{t}} \right) &=
    \changee{\sum_{t=1}^{T} \mathbb{E}\left(r_{t,a_{t,\star}^{\dagger}} - \mathds{1}_{\{ a_{t}\in A^{\dagger}\}}\right)\geq \left(T-\mathbb{E}\left(\sum_{t=1}^{T} \mathds{1}_{\{ a_{t} = a_{t,\star}^{\dagger}\}}\right)\right)\Delta}
\end{align*}
\changee{where $\Delta := \min_{x\in \mathcal{D}}\left\{ \langle \theta_{a^\dagger(x)}, x\rangle - \max_{a\in A^{\dagger},a\neq a^{\dagger}(x)} \langle \theta_{a'}, x\rangle\right\}$ with $a^{\dagger}(x) := \arg\max_{a\in A^{\dagger}} \langle \theta_{a}, x\rangle$}. Then, as $\Tilde{R}_{T}^{1,\text{exp}}\in o(T)$ and $\mathbb{E}\left(\sum_{t=1}^{T}(1 - E_{m^{\star}, a^{\dagger}_{t, \star}}^{(t)})\right)\in o(T)$, we deduce that
\begin{align*}
    \mathbb{E}(\sum_{t} \mathds{1}_{\{ a_{t} = a_{t,\star}
^{\dagger}\}}) = T-o(T)\quad.
\end{align*}

% For this strategy the cost is therefore bounded by: 
% \begin{align*}
% \mathbb{E}\left( \sum_{t=1}^{T} c_{t}\right) &\leq \mathbb{E}\left(\sum_{t=1}^{T} \mathds{1}_{\{a_{t}\neq a^{\dagger}\}} + |\eta_{a_{t}, t}|+ |\eta_{a_{t},t}'|\right) \leq (1 + 2\sigma) \mathbb{E}\left( N_{a^{\dagger}}(T)\right)
% \end{align*}

For strategy $i=2$, and $\delta>0$, let us denote by $E_{\delta}$ the event that all confidence intervals hold with probability $1 - \delta$. But on the event $E_{\delta}$, for a time $t$ where $a_{t}\neq a^{\dagger}_{t,\star}$ and such that $-1\leq C_{t,a_{t}} \leq 0$:
\begin{align*}
\tilde{r}^{2}_{t,a_{t}} = r_{t, a_{t}} + C_{t,a_{t}} &= (1 - \gamma)\min_{a^{\dagger}\in A^{\dagger}} \min_{\theta\in \mathcal{C}_{t,a^{\dagger}}} \langle \theta, x_{t}\rangle + \eta_{a_{t},t} + \langle\theta_{a}, x_{t}\rangle - \max_{\theta\in \mathcal{C}_{t,a_{t}}} \langle \theta, x_{t}\rangle \\
&\leq (1 - \gamma) \langle \theta_{a^{\dagger}_{t,\star}}, x_{t}\rangle + \eta_{a_{t},t}
\end{align*}
when $C_{t,a_{t}} >0$ (still on the event $E_{\delta}$):
\begin{align*}
\tilde{r}^{2}_{t,a_{t}} = r_{t,a_{t}} \leq (1 - \gamma) \langle \theta_{a^{\dagger}_{t,\star}}, x_{t}\rangle + \eta_{a_{t},t}
\end{align*}
because $C_{t,a_{t}}>0$ means that $(1 - \gamma) \langle \theta_{a^{\dagger}_{t,\star}}, x_{t}\rangle \geq (1 - \gamma)\min_{a^{\dagger}\in A^{\dagger}}\min_{\theta\in \mathcal{C}_{t,a^{\dagger}}} \langle \theta, x_{t}\rangle \geq \max_{\theta\in \mathcal{C}_{t,a_{t}}} \langle \theta, x_{t}\rangle \geq \langle \theta_{a}, x_{t}\rangle$. But finally, when $C_{t,a_{t}} \leq -1$, $\tilde{r}^{2}_{t,a_{t}} = r_{t,a_{t}} -1 \leq \eta_{a_{t},t} \leq (1- \gamma)\langle \theta_{a^{\dagger}_{t,\star}}, x_{t}\rangle + \eta_{a_{t},t}$.  But on the complementary event $E_{\delta}^{c}$,  $ \tilde{r}^{2}_{t,a_{t}} \leq r_{t,a_t}$. Thus, given that the expected reward is assumed to be bounded in $(0,1]$ (Assumption~\ref{assumption1}):
\begin{align*}
    \mathbb{E}\left(\sum_{t=1}^{T} \tilde{r}^{2}_{t, a^{\dagger}_{t,\star}} - \tilde{r}^{2}_{t,a_{t}} \right) & =  \mathbb{E}\left(\sum_{t=1}^{T} (r_{t, a^{\dagger}} - \tilde{r}^{2}_{t,a_{t}})\mathds{1}_{\{a_{t}\neq a^{\dagger}_{t,\star}\}} \right) \\
    &\geq \mathbb{E}\left(\sum_{t=1}^{T} \min\{\gamma\min_{x\in\mathcal{D}} \langle x, \theta_{a^{\dagger}_{t,\star}}\rangle, \Delta\} \mathds{1}_{\{a_{t}\neq a^{\dagger}_{t,\star}\}}\mathds{1}_{\{E_{\delta}\}}\right)-T\delta
\end{align*}
Finally, putting everything together we have:
\begin{align*}
    \mathbb{E}&\left(\sum_{t=1}^{T} \gamma\min_{x\in\mathcal{D}} \langle x, \theta_{a^{\dagger}_{t,\star}}\rangle \mathds{1}_{\{a_{t}\neq a^{\dagger}_{t,\star}\}}\right) \leq &\Tilde{R}_{T}^{2,\text{exp}} + \mathbb{E}\left(\sum_{t=1}^{T}(1 - E_{m^{\star}, a^{\dagger}_{t,\star}}^{(t)})\right)\\
     &+ \delta T \left(\min\{\gamma\min_{a^{\dagger}\in A^{\dagger}}\min_{x\in\mathcal{D}} \langle x, \theta_{a^{\dagger}}\rangle, \Delta\} +1\right)
\end{align*}
Hence, because $\Tilde{R}_{T}^{1,\text{exp}} = o(T)$ and $\mathbb{E}\left(\sum_{t=1}^{T}(1 - E_{m^{\star}, a^{\dagger}}^{(t)})\right) = o(T)$ we have that for $\delta \leq 1/T$, the expected number of pulls of the optimal arm in $A^{\dagger}$ is of order $o(T)$. In addition, the cost for the attacker is bounded by: 
\begin{align*}
\mathbb{E}\left(\sum_{t=1}^{T} c_{t}\right) &= \mathbb{E}\left(\sum_{t=1}^{T} \mathds{1}_{\{a_{t}\neq a^{\dagger}_{t,\star}\}} \big|\max(-1, \min(C_{t, a_{t}},0))\big| \right)\leq  \mathbb{E}\left( \sum_{t=1}^{T} \mathds{1}_{\{a_{t}\neq a^{\dagger}_{t,\star}\}}\right)
\end{align*}
\end{proof}

The proof is similar to the one of Prop.~\ref{prop:reward_attack}. The condition on the expert in Prop.~\ref{prop:rwd_attack_adv} means that there exists an expert which believes an arm $a^{\dagger}\in A^{\dagger}$ is optimal most of the time. The adversarial algorithm will then learn that this expert is optimal. %The rewards presented to the algorithm are build in such a way that the latter learns that the expert pulling arm $a^{\dagger}$ the majority of times being optimal.% \begin{proof}
% Similarly to the proof of Proposition \ref{prop:reward_attack}, let's define the bandit with expert advice problem, $\mathcal{A}_{i}$, such that at each time $t$ the reward vector is $(\tilde{r}^{i}_{t,a})_{a}$ (with $i\in\{1, 2\}$). The regret of this algorithm is: $\Tilde{R}_{T}^{i,\text{exp}} = \mathbb{E}\left( \max_{m\in \llbracket 1, N \rrbracket}\sum_{t=1}^{T} E_{m}^{(t)}\Tilde{r}^i_{t} - \Tilde{r}^i_{t,a_{t}}\right)\in o(T)$. The regret of the learner is: $\mathbb{E}\left( \max_{m\in \llbracket 1, N \rrbracket}\sum_{t=1}^{T} E_{m}^{(t)}r_{t} - r_{t,a_{t}}\right)$ where $a_t$ are the actions taken by algorithm $\mathcal{A}_i$ to minimize $\Tilde{R}_{T}^{i,\text{exp}}$. Then we have:
% \begin{align*}
%     \Tilde{R}_{T}^{i,\text{exp}} \geq \mathbb{E}\left(\sum_{t=1}^{T}\sum_{j=1}^{K} (E_{m^{\star}, j}^{(t)} - \mathds{1}_{\{j\neq a^{\dagger}\}})\tilde{r}_{t,j}^{i} + \sum_{t=1}^{T} \tilde{r}^{i}_{t, a^{\dagger}} - \tilde{r}^{i}_{t,a_{t}} \right)
% \end{align*}
% Therefore, 
% \begin{align*}
%   \mathbb{E}\left(\sum_{t=1}^{T} \tilde{r}^{i}_{t, a^{\dagger}} - \tilde{r}^{i}_{t,a_{t}} \right) &\leq \Tilde{R}_{T}^{i,\text{exp}} + \mathbb{E}\left(\sum_{t=1}^{T}\sum_{j=1}^{K} (\mathds{1}_{\{j\neq a^{\dagger}\}} - E_{m^{\star}, j}^{(t)})\tilde{r}_{t,j}^{i}\right) \\
%   &\leq \Tilde{R}_{T}^{i,\text{exp}} + \mathbb{E}\left(\sum_{t=1}^{T}(1 - E_{m^{\star}, a^{\dagger}}^{(t)})\tilde{r}_{t,j}^{i}\right) \\
%   &\leq \Tilde{R}_{T}^{i,\text{exp}} + \mathbb{E}\left(\sum_{t=1}^{T}(1 - E_{m^{\star}, a^{\dagger}}^{(t)})\right)
% \end{align*}
% For strategy $i=1$, and $\delta>0$, let denote $E_{\delta}$ the event that all confidence intervals holds with probability $1 - \delta$. But on the event $E_{\delta}$, for a time $t$ where $a_{t}\neq a^{\dagger}$ and such that $-1\leq C_{t,a_{t}} \leq 0$:
% \begin{align*}
% \tilde{r}^{1}_{t,a_{t}} = r_{t, a_{t}} + C_{t,a_{t}} &= (1 - \gamma) \min_{\theta\in \mathcal{C}_{t,,a^{\dagger}}} \langle \theta, x_{t}\rangle + \eta_{a_{t},t} + \langle\theta_{a}, x_{t}\rangle - \max_{\theta\in \mathcal{C}_{t,a_{t}}} \langle \theta, x_{t}\rangle \\
% &\leq (1 - \gamma) \langle \theta_{a^{\dagger}}, x_{t}\rangle + \eta_{a_{t},t}
% \end{align*}
% when $C_{t,a_{t}} >0$ (still on the event $E_{\delta}$):
% \begin{align*}
% \tilde{r}^{1}_{t,a_{t}} = r_{t,a_{t}} \leq (1 - \gamma) \langle \theta_{a^{\dagger}}, x_{t}\rangle + \eta_{a_{t},t}
% \end{align*}
% because $C_{t,a_{t}}>0$ means that $(1 - \gamma) \langle \theta_{a^{\dagger}}, x_{t}\rangle \geq (1 - \gamma)\min_{\theta\in \mathcal{C}_{t,,a^{\dagger}}} \langle \theta, x_{t}\rangle \geq \max_{\theta\in \mathcal{C}_{t,a_{t}}} \langle \theta, x_{t}\rangle \geq \langle \theta_{a}, x_{t}\rangle$. But finally, when $C_{t,a_{t}} \leq -1$, $\tilde{r}^{1}_{t,a_{t}} = r_{t,a_{t}} -1 \leq \eta_{a_{t},t} \leq (1- \gamma)\langle \theta_{a^{\dagger}}, x_{t}\rangle + \eta_{a_{t},t}  $.  But on the complementary event $E_{\delta}^{c}$,  $ \tilde{r}^{1}_{t,a_{t}} \leq r_{t,a_t}$. Thus we have because the expected reward are assumed to be bounded in $(0,1]$:
% \begin{align*}
%     \mathbb{E}\left(\sum_{t=1}^{T} \tilde{r}^{1}_{t, a^{\dagger}} - \tilde{r}^{1}_{t,a_{t}} \right)  =  \mathbb{E}\left(\sum_{t=1}^{T} (r_{t, a^{\dagger}} - \tilde{r}^{1}_{t,a_{t}})\mathds{1}_{\{a_{t}\neq a^{\dagger}\}} \right) &\\
%     \geq \mathbb{E}\left(\sum_{t=1}^{T} \gamma\min_{x\in\mathcal{D}} \langle x, \theta_{a^{\dagger}}\rangle \mathds{1}_{\{a_{t}\neq a^{\dagger}\}}\mathds{1}_{\{E_{\delta}\}}\right)-T\delta
% \end{align*}
% Finally, putting everything together we have:
% \begin{align*}
%     \mathbb{E}&\left(\sum_{t=1}^{T} \gamma\min_{x\in\mathcal{D}} \langle x, \theta_{a^{\dagger}}\rangle \mathds{1}_{\{a_{t}\neq a^{\dagger}\}}\right) \leq \Tilde{R}_{T}^{1,\text{exp}} \\
%     &+ \mathbb{E}\left(\sum_{t=1}^{T}(1 - E_{m^{\star}, a^{\dagger}}^{(t)})\right) + \delta T \left(\gamma\min_{x\in\mathcal{D}} \langle x, \theta_{a^{\dagger}}\rangle +1\right)\\
% \end{align*}
% Hence, because $\Tilde{R}_{T}^{1,\text{exp}} = o(T)$ and $\mathbb{E}\left(\sum_{t=1}^{T}(1 - E_{m^{\star}, a^{\dagger}}^{(t)})\right) = o(T)$ we have that for $\delta \leq 1/T$, the expected number of pulls of arm $a^{\dagger}$ is of order $o(T)$. In addition, the cost for the attacker is bounded by: 
% \begin{align*}
% \mathbb{E}\left(\sum_{t=1}^{T} c_{t}\right) &= \mathbb{E}\left(\sum_{t=1}^{T} \mathds{1}_{\{a_{t}\neq a^{\dagger}\}} \big|\max(-1, \min(C_{t, a_{t}},0))\big| \right)\\
% &\leq  \mathbb{E}\left( \sum_{t=1}^{T} \mathds{1}_{\{a_{t}\neq a^{\dagger}\}}\right)
% \end{align*}
% For strategy $i=2$, we have:
% \begin{align*}
%     \mathbb{E}\left(\sum_{t=1}^{T} \tilde{r}^{2}_{t, a^{\dagger}} - \tilde{r}^{2}_{t,a_{t}} \right) &=
%     \sum_{t=1}^{T} \mathbb{E}\left(r_{t, a^{\dagger}}\mathds{1}_{a_t\neq a^\star}\right)\\
%  &\geq (T-\mathbb{E}(N_{a^\star}(T)))\min_{x\in\mathcal{D}} \langle x, \theta_{a^{\dagger}}\rangle 
% \end{align*}
% Then, as $\Tilde{R}_{T}^{2,\text{exp}}\in o(T)$ and $\mathbb{E}\left(\sum_{t=1}^{T}(1 - E_{m^{\star}, a^{\dagger}}^{(t)})\right)\in o(T)$, we deduce that $\mathbb{E}(N_{a^\star}(T)) = T-o(T)$. For this strategy the cost is bounded by: 
% \begin{align*}
% \mathbb{E}\left( \sum_{t=1}^{T} c_{t}\right) &\leq \mathbb{E}\left(\sum_{t=1}^{T} \mathds{1}_{\{a_{t}\neq a^{\dagger}\}} + |\eta_{a_{t}, t}|+ |\eta_{a_{t},t}'|\right) \\
% &\leq (1 + 2\sigma) \mathbb{E}\left( N_{a^{\dagger}}(T)\right)
% \end{align*}
% \todol{add regret (I have it but need to write) + add details on line 480 col 2}
% \end{proof}
Algorithm \expfour has a regret $R_{T}^{\text{exp}}$ bounded by $\sqrt{2TK\log(N)}$, thus the total number of pulls of arms not in $A^{\dagger}$ %different from $a^{\dagger}$ 
is bounded by $\sqrt{2TK\log(M)}/\gamma$. This result also implies that for adversarial algorithms like \expthree \cite{auer2002finite}, the same type of attacks could be used to fool $\mathfrak{A}$ into pulling arms in $A^{\dagger}$ because the MAB problem can be seen as a reduction of the contextual bandit problem with a unique context and one expert for each arm.

%consisting of the mean of each arm as coordinate and each experts selects only one arm.

% !TEX root = main.tex
\section{Contextual Bandit Algorithms}\label{app:algorithms}

In this appendix, we present the different bandit algorithms studied in this paper. All algorithms we consider except \expfour uses disjoint models for building estimate of the arm feature vectors $(\theta_{a})_{a\in\llbracket 1, K\rrbracket}$. Each algorithm (except \expfour) builds least squares estimates of the arm features.

\begin{algorithm}[h]
  \caption{Contextual \linucb}
  \label{alg:linucb}
\begin{algorithmic}
  \STATE {\bfseries Input:} regularization  $\lambda$, number of arms $K$, number of rounds $T$, bound on context norms: $L$, bound on norms $\theta_{a}$: $D$
  \STATE Initialize for every arm $a$, $\bar{V}_{a}^{-1}(t) = \frac1\lambda I_{d}$, $\hat{\theta}_{a}(t) = 0$ and $b_{a}(t) = 0$
  \FOR{$t=1,..., T$}
  \STATE Observe context $x_{t}$
  \STATE Compute $\beta_{a}(t) = \sigma\sqrt{d\log\left(\frac{1 +  N_{a}(t)L^{2}/\lambda}{\delta}\right)}$ with $N_{a}(t)$ the number of pulls of arm $a$
  \STATE Pull arm  $a_{t} =\argmaxB_a \langle \hat{\theta}_{a}(t),x_t\rangle + \beta_{a}(t)||x_{t}||_{\bar{V}_{a}^{-1}(t)}$
  \STATE Observe reward $r_{t}$ and update parameters $\hat{\theta}_{a}(t)$ and $\bar{V}_{a}^{-1}(t)$ such that:
  \begin{align*}
      \bar{V}_{a_{t}}(t+1) = \bar{V}_{a_{t}}(t) + x_{t}x_{t}^{\intercal},\quad b_{a_{t}}(t+1) = b_{a_{t}}(t) + r_{t}x_{t},\quad\theta_{a_{t}}(t+1) = \bar{V}_{a_{t}}^{-1}(t+1)b_{a_{t}}(t+1)
  \end{align*}
  \ENDFOR
\end{algorithmic}
\end{algorithm}

\begin{algorithm}[h]
  \caption{Linear Thompson Sampling with Gaussian prior}
  \label{alg:linTS}
\begin{algorithmic}
  \STATE {\bfseries Input:} regularization  $\lambda$, number of arms $K$, number of rounds $T$, variance $\upsilon$
  \STATE Initialize for every arm $a$, $\bar{V}_{a}^{-1}(t) = \lambda I_{d}$ and $\hat{\theta}_{a}(t) = 0$, $b_{a}(t) = 0$
  \FOR{$t=1,..., T$}
  \STATE Observe context $x_{t}$
  \STATE Draw $\tilde{\theta}_{a}\sim\mathcal{N}(\hat{\theta}_{a}(t), \upsilon^{2}\bar{V}_{a}^{-1}(t))$
  \STATE Pull arm $a_{t} = \argmaxB_{a\in \llbracket 1, K\rrbracket} \left\langle \tilde{\theta}_{a}, x_{t}\right\rangle$
  \STATE Observe reward $r_{t}$ and update parameters $\hat{\theta}_{a}(t)$ and $\bar{V}_{a}^{-1}(t)$
    \begin{align*}
      \bar{V}_{a_{t}}(t+1) = \bar{V}_{a_{t}}(t) + x_{t}x_{t}^{\intercal},\quad b_{a_{t}}(t+1) = b_{a_{t}}(t) + r_{t}x_{t},\quad\theta_{a_{t}}(t+1) = \bar{V}_{a_{t}}^{-1}(t+1)b_{a_{t}}(t+1)
  \end{align*}
  \ENDFOR
\end{algorithmic}
\end{algorithm}

\begin{algorithm}[h]
  \caption{\epsgreedy}
  \label{alg:egreedy}
\begin{algorithmic}
  \STATE {\bfseries Input:} regularization  $\lambda$, number of arms $K$, number of rounds $T$, exploration parameter $(\varepsilon)_{t}$
	\STATE Initialize, for all arms $a$, $\bar{V}_{a}^{-1}(t) = \lambda I_{d}$ and $\hat{\theta}_{a}(t) = 0$, $\varepsilon_{t} = 1$, $b_{a}(t) = 0$
  \FOR{$t=1,..., T$}
  \STATE Observe context $x_{t}$
  \STATE With probability $\varepsilon_{t}$, pull $a_{t} \sim \mathcal{U}\left(\llbracket 1,K\rrbracket\right)$, or pull $a_{t} = \argmaxB \langle \theta_{a}, x_{t}\rangle$ 
  \STATE Observe reward $r_{t}$ and update parameters $\hat{\theta}_{a}(t)$ and $\bar{V}_{a}^{-1}(t)$
    \begin{align*}
      &\bar{V}_{a_{t}}(t+1) = \bar{V}_{a_{t}}(t) + x_{t}x_{t}^{\intercal},\quad b_{a_{t}}(t+1) = b_{a_{t}}(t) + r_{t}x_{t},\\
      &\theta_{a_{t}}(t+1) = \bar{V}_{a_{t}}^{-1}(t+1)b_{a_{t}}(t+1)
  \end{align*}
  \ENDFOR
\end{algorithmic}
\end{algorithm}

\begin{algorithm}[h]
  \caption{\expfour}
  \label{alg:exp4}
\begin{algorithmic}
	\STATE {\bfseries Input:} number of arms $K$, experts: $(E_{m})_{m\in\llbracket 1, N\rrbracket}$, parameter $\eta$
  \STATE Set $Q_{1} = (1/N)_{j\in\llbracket 1, N\rrbracket}$
  \FOR{$t=1,..., T$}
  \STATE Observe context $x_{t}$ and probability recommendation $(E_{m}^{(t)})_{m\in\llbracket 1, N\rrbracket}$
  \STATE Pull arm $a_{t}\sim P_{t}$ where $P_{t,j} = \sum_{k=1}^{N} Q_{t,k}E_{j,k}^{(t)}$ 
  \STATE Observe reward $r_{t}$ and define for all arms $i$ $\hat{r}_{t,i} = 1 - \mathds{1}_{\{ a_{t}=i\}}( 1 - r_{t})/P_{t,i}$
  \STATE Define $\tilde{X}_{t,k} = \sum_{a} E_{k, a}^{(t)}\hat{r}_{t,a}$
  \STATE Update $Q_{t+1, j} = \exp(\eta Q_{t,i})/\sum_{j=1}^{N} \exp(\eta Q_{t,j})$ for all experts $i$
  \ENDFOR
\end{algorithmic}
\end{algorithm}

% !TEX root = main.tex
\section{Appendix: Semi-Online Attacks}

\cite{liu2019data} studies what they call the offline setting for adversarial attacks on stochastic bandits. They consider a setting where a bandit algorithm is successively updated with mini-batch\changebr{es} of fixed size $B$. \changebr{The attacker can tamper with some of the incoming mini-batches. % Each mini-batch that can be t\changebr{a}mpered with by the attacker. 
 More precisely, they can modify the context, the reward and even the arm that was pulled for any entry of the attacked mini-batches.}
% That is to say, that the attacker can change the context, arm chosen and the reward obtained for each entry in the mini-batch. 
\changebr{The main difference between this type of attacks and the online attacks we considered in the main paper is that we do not assume that we can attack from the start of the learning process: the bandit algorithm may have already converged by the time we attack}. 

We can still study the cumulative cost for the attacker to change the mini-batch in order to fool a bandit algorithm to pull a target arm $a^{\dagger}$ (\changee{here we take $A^{\dagger} = \{ a^{\dagger}\}$}). Contrarily to \cite{liu2019data}, we call this setting semi-online. We first study the impact of an attacker on \linucb where we show that, by modifying only $(K-1)d$ entries from the batch $\mathcal{B}$, the attacker can force \linucb to pull arm $a^{\dagger}$, $M'B - o(M'B)$ times with $M'$ the number of remaining batches updates. The cost of our attack is $\sqrt{MB}$ with $M$ the total number of batches.

\paragraph{Cost of an attack:} If presented with a mini-batch $\mathcal{B}$, with elements $(x_{t}, a_{t}, r_{t})$ composed of the context $x_{t}$ presented at time $t$, the action taken $a_{t}$ and the reward received $r_{t}$, the attacker modifies element $i$, namely  $(x^{i}_{t}, a^{i}_{t}, r^{i}_{t})$ into $(\tilde{x}^{i}_{t}, \tilde{a}^{i}_{t}, \tilde{r}^{i}_{t})$. The cost of doing so is $c^{i}_{t} = ||x^{i}_{t} - \tilde{x}^{i}_{t}||_{2} + \big|\tilde{r}^{i}_{t} - r^{i}_{t}\big| + \mathds{1}_{\{a^{i}_{t} \neq \tilde{a}^{i}_{t}\}}$ and the total cost for mini-batch $\mathcal{B}$ is defined as $c_{\mathcal{B}} = \sum_{i\in \mathcal{B}} c_{t}^{i}$. Finally, we consider the cumulative cost of the attack over $M$ different mini-batches $\mathcal{B}_{1}, \hdots, \mathcal{B}_{M}$, $\sum_{l=1}^{M} c_{\mathcal{B}_{l}}$. The interaction between the environment, the attacker and the learning algorithm is summarized in Alg.~\ref{alg:semi_online_setting}.  
\begin{algorithm}[h]
	\caption{Semi-Online Attack Setting.}
  \label{alg:semi_online_setting}
\begin{algorithmic}
  \STATE {\bfseries Input:} Bandit alg.~$\mathfrak{A}$, size of a mini-batch: $B$
  \STATE Set $t = 0$
  \WHILE{True}
  \STATE $\mathfrak{A}$ observe context $x_{t}$
  \STATE $\mathfrak{A}$ pulls arm $a_{t}$ and observes reward $r_{t}$
  \STATE Interaction $(x_{t}, a_{t}, r_{t})$ is saved in mini-batch $\mathcal{B}$
  \IF{$\big|\mathcal{B}\big| = B$}
  \STATE Attacker modifies mini-batch $\mathcal{B}$ into $\tilde{\mathcal{B}}$
  \STATE Update alg.~$\mathfrak{A}$ with poisoned mini-batch $\tilde{\mathcal{B}}$
  \ENDIF
  \ENDWHILE
\end{algorithmic}
\end{algorithm}


% If the attacker can also choose the arm that is pulled, it is possible to attack \linucb with attacks of cumulative norm $O(\sqrt T)$. In practice, it can happen if the attacker manages to infiltrate a server on which part of training data is stored. Modifying or injecting $(K-1) \times d$ lines in the training data with attacks of norm $O(\sqrt T)$ is enough to make the algorithm pull the arm $a^{\dagger}$ all the time. 

The attack presented here is based on the Ahlberg–Nilson–Varah bound \cite{varah1975lower}, which gives a control on the sup norm of a matrix with dominant diagonal elements. More precisely, when presented with a mini-batch $\mathcal{B}$, the attacker needs to modify the contexts and the rewards. We assume that the attacker knows the number of mini-batch updates $M$ and has access to a lower-bound on the reward of the target arm, $\nu$ as in Assumption~\ref{assumption2}. 

The attacker changes $(K-1)\times d$ rows of the first mini-batch to rewards of $0$ with a context $\delta_a e_i$ for each arm $a \neq a^{\dagger}$ with $(e_i)$ the canonical basis of $\mathbb{R}^{d}$. Moreover, $\delta_{a}$ is chosen such that: 
\begin{equation}
    \delta_{a}  > \max\left(\sqrt{\frac{2MBL^{2}d}{\nu} + dMB},  \sqrt{\frac{4\beta_{max}^2L^{2}d}{\nu^{2}} + dMB}\right)
    \label{eq:delta_botnets}
\end{equation}
with $\beta_{max} = \max_{t=0}^{MB} \beta_a(t)$ and $M$ the number of mini-batch updates.

\begin{prop}\label{prop:attacker_can_choose}
After the first attack, with probability $1-\delta$, \linucb always pulls arm $a^{\dagger}$, 
\end{prop}

\begin{proof}
After having poisoned the first mini-batch $\mathcal{B}$, the latter can be partitioned into two subsets, $\mathcal{B}_{c}$ (with non-perturbed rows) and $\mathcal{B}_{nc}$ (with the poisoned rows). The design matrix of arm $a\neq a^{\dagger}$ for every time $t$ after the poisoning is:
\begin{align}
    V_{t,a} = \lambda I_d +  \sum_{l=1, a_{l} = a}^{t} x_{l}x_{l}^{\intercal} + \delta_{a}^2 \sum_{i=1}^{d} e_i e_i^{\intercal}
\end{align}
For every time $t$, non diagonal elements of $V_{t,a} = (v_{i,j})_{i,j}$ are bounded by: 
\begin{align}\label{non_diagonal_element}
    \forall i, r_i &:= \sum_{j \neq i} v_{i,j}\leq \sum_{j \neq i} \sum_{l=1, a_{l} = a}^{t} \lVert x_l x_{l}^{\intercal}\rVert_\infty\leq dN_{a}(kB)
\end{align}
Whereas for all diagonal elements, $v_{i,i} \geq \delta_{a}^2 > r_i$. Thus $V_{t,a}$ is strictly diagonal dominant and by the Ahlberg–Nilson–Varah bound \cite{varah1975lower}:
\begin{align}\label{eq:bound_norm_design_matrix}
\lVert V_{t,a}^{-1} \rVert_\infty &\leq \frac{1}{\min_{i} \left(\lVert v_{i,i}\rVert - r_i\right)} \leq \frac{1}{\delta_{a}^{2} - dMB}
\end{align}
Then, for every arm $a \neq a^{\dagger}$ and any context $x\in \mathcal{D}$ and any time $t$ after the attack:
\begin{align*}
    \langle\hat{\theta}_{a,t}, x\rangle + \beta_{a}(t) ||x||_{V_{t,a}^{-1}} & \leq  \sum_{l=1, a_{l} = a}^{t} r_t (V_{t,a}^{-1} x_t)^{\intercal} x + \beta_{a}(t)||x||_{1}\sqrt{\lVert V_{t,a}^{-1} \rVert_\infty}\\
    & \leq  \lVert V_{t,a}^{-1} \rVert_\infty dN_{t}(a) \sup_{y\in\mathcal{D}} ||y||_{2}^{2} \\
    &+ \beta_{\text{max}}\sqrt{d}\sup_{y\in\mathcal{D}} ||y||_{2}\sqrt{\lVert V_{t,a}^{-1} \rVert_\infty} < \nu
\end{align*}

    % &= \left({A'}^{-1} \sum_{t=0}^{N(a)} r_t x_t\right)^{\intercal} x + \beta_{a}(t) \sqrt{x^{\intercal} A'^{-1} x}\\
    % & =  (\sum_{t=0}^{N(a)} r_t {A'}^{-1} x_t)^{\intercal} x + \beta_{a}(t) \sqrt{x^{\intercal} A'^{-1} x}\\
    
We have shown that for any arm $a \neq a^{\dagger}$ and any time step $t$ after the attack, the upper confidence bound computed by \linucb is upper-bounded bu $\nu$ the arm $a^{\dagger}$. Then, with probability $1-\delta$, the confidence set for arm $a^{\dagger}$ holds and, for all $x \in \mathcal{D}$, arm $a^{\dagger}$ is chosen by \linucb. The total cost of this attack is $d \sum_{a\neq a^{\dagger}} \delta_{a} L = O(\sqrt{MB})$
\end{proof}







\chapter{Equitable and Optimal Transport with Multiple Agents}


\chapter{Asymptotic convergence rates for averaging strategies}
The present paper extends the paper presented in Chapter~\ref{chap:ppn-kbest}
\chapter{Asymptotic convergence rates for averaging strategies}
\label{paper:foga}
\section{Introduction}
Finding the minimum of a function from a set of $\lambda$ points $(x_i)_{i\leq \lambda}$ and their images $(f(x_i))_{i\leq \lambda}$ is a standard task used for instance in hyper-parameter tuning \cite{bergstra2012random}, or control problems. While random search estimate of the optimum consists in returning $\arg\min f(x_i)_{i\leq \lambda}$, in this paper we focus on the similar strategy that consists in averaging the $\mu$ best samples, i.e. returning $\frac1\mu\sum_{i=1}^\mu x_{(i)}$ where $f(x_{(1)})\leq\ldots\leq f(x_{(\lambda)})$.

These kinds of strategies are used in many evolutionary algorithms such as CMA-ES. Although experiments show that these methods perform well, it is not still understood why taking the average of best points actually leads to a lower regret. In \cite{ppsnkbest}, it is proved in the case of quadratic functions that the regret is indeed lower for the averaging strategy than for pure random search. In this paper, we extend the result of this paper by proving convergence rates for a wide class of functions including three times continuously differentiable functions with unique optima.


\subsection{Related Work}

\subsubsection{Better than picking up the best}
Given a finite number of samples $\lambda$ equipped with their fitness values, we can simply pick up the best, or average the ``best ones''~\cite{beyerbenefitofsex,ppsnkbest}, or apply a surrogate model~\cite{sm1,sm2,sm3,AST,bach}. Overall, the best is quite robust, but the surrogate or the averaging usually provides  better convergence rates. Using surrogate modeling is fast when the dimension is moderate and the objective function is smooth (simple regret in $O(\lambda^{-m/d})$ for $\lambda$ points in dimension $d$ with $m$ times differentiability, leading to superlinear rates in evolutionary computation~\cite{AST}). In this paper, we are interested in the rates obtained by averaging the best samples for a wide class of functions. We extend the results of~\cite{ppsnkbest} which only hold for the sphere function.

\subsubsection{Weighted averaging}
Among the various forms of averaging,
it has been proposed to take into account the fact that the sampling is not uniform (evolutionary algorithms in continuous domains typically use Gaussian sampling) in \cite{sumo}: we here simplify the analysis by considering a uniform sampling in a ball, though we acknowledge that this introduces the constraint that the optimum is indeed in the ball. \cite{arnoldweights,anneweights} have proposed weights depending on the fitness value, though they acknowledge a moderate impact: we here consider equal weights for the $\mu$ best.

\subsubsection{Choosing the selection rate}
The choice of the selection rate $\mu/\lambda$ is quite debated in evolutionary computation: one can find $\mu=\lambda/7$ \cite{amorales}, $\mu=\lambda/2$ \cite{cmsa}, $\mu=0.27\lambda$ \cite{escompr}, $\mu=\lambda/4$ \cite{HAN}, $\mu=\min(d,\lambda/4)$ \cite{chooselambda,fournierAlgorithmica} and still others in \cite{beyerbenefitofsex,jeb}. In this paper, we focus on the selection rate when the number of samples $\lambda$ is very large in the case of parallel optimization. In this case, the selection ratio would tend to $0$. We carefully analyze this ratio and derive convergence rates using this selection ratio.

\subsubsection{Taking into account many basins} While averaging the best samples, the non-uniqueness of an optimum might lead to averaging points coming from different basins. Thus we consider at first the case of a unique optimum and hence a unique basin. Then we aim to tackle the case where there are possibly different basins. Island models~\cite{islands} have also been proposed for taking into account different basins. \cite{ppsnkbest} has proposed a tool for adapting $\mu$ depending on the (non) quasi-convexity. In the present work, we extend the methodology proposed in \cite{ppsnkbest}.

\subsection{Outline}
In the present paper, we first introduce, in Section~\ref{sec:assumptions}, the large class of functions we will study, and study some useful properties of these functions in Section~\ref{sec:teclemmas}. Then, in Section~\ref{sec:randomsearch}, we prove upper and lower convergence rates for random search for these functions. In Section~\ref{sec:mubestrate}, we extend \cite{ppsnkbest} by showing that asymptotically in the number of samples $\lambda$, the handled functions satisfy a better convergence rate than random search. We then extend our results on wider classes of functions in Section~\ref{sec:wider}. Finally we validate experimentally our theoretical findings and compare with other parallel optimization methods.

\section{Beyond quadratic functions}
\label{sec:assumptions}
In the present section, we present the assumptions to extend the results from \cite{ppsnkbest} to the non-quadratic case. We will denote $B(0,r)$ the closed ball centered at $0$ of radius $r$ in $\mathbb{R}^d$ endowed with its canonical Euclidean norm denoted by $\lVert\cdot\rVert$. We will also denote by $\overset{\circ}{B}(0,r)$ the corresponding \emph{open} ball. All other balls intervening in what follows will also follow that notation. For any subset $S\subset B(0,r)$, we will denote $U(S)$ the uniform law on $S$. 

Let $f:B(0,r)\to \mathbb{R}$ be a continuous function for which we would like to find an optimum point $x^*$. The existence of such an optimum point is guaranteed by continuity on a compact set.
%if B(0,r) denotes the open ball, it is not compact. If we want to use compactness, we have to work with \bar{B}(0,r) the closed ball. This also ensures that the sublevel set is closed for the topology of R^{d} (as a subset of the closed ball which is also closed for the topology of the closed ball)
For the sake of simplicity, we assume that $f(x^{\star}) = 0$. We define the $h$-level sets of $f$ as follows.
%h-level set or sublevel set of level/height h I prefer.
\begin{definition}
Let $f:B(0,r)\to \mathbb{R}$ be a continuous function. The closed sublevel set of $f$ of level $h$ is defined as:
\begin{align*}
    S_{h}:=\{x\in B(0,r)\mid f(x)\leq h\}.
\end{align*}
\end{definition}
We now describe the assumptions we will make on the function $f$ that we optimize. 
\begin{assump}
\label{ass:principal}
$f:B(0,r)\to \mathbb{R}$ is a continuous function and admits a unique optimum point $x^\star$ such that $\lVert x^{\star}\rVert<r$. Moreover we assume that $f$ can be written:
\begin{align*}
f(x)=\left(x-x^\star\right)^T\mathbf{H}\left(x-x^\star\right)+ \left(\left(x-x^\star\right)^T\mathbf{H}\left(x-x^\star\right)\right)^{\alpha/2} \varepsilon(x-x^\star)
\end{align*}
for some bounded function $\varepsilon$ (there exists $M>0$ such that for all $x$, $\lvert \varepsilon(x)\rvert \leq M$), $\mathbf{H}$ a symmetric positive definite matrix and $\alpha>2$ a real number. 

\end{assump}
Note that $H$ is uniquely defined by the previous relation.
In the following we will denote by $e_1(\mathbf{H})$ and $e_d(\mathbf{H})$ respectively the smallest and the largest eigenvalue of $\mathbf{H}$. As $\mathbf{H}$ is positive definite, we have $0<e_1(\mathbf{H})\leq e_d(\mathbf{H})$. We will also set $\lVert x\rVert_{\mathbf{H}}=\sqrt{x^T\mathbf{H}x}$, which is a norm (\emph{the $\mathbf{H}$-norm}) on $\mathbb{R}^{d}$ as $\mathbf{H}$ is symmetric positive definite. We then have $f(x)=\lVert x-x^{\star}\rVert_{\mathbf{H}}^2+ \lVert x-x^{\star}\rVert_{\mathbf{H}}^{\alpha} \varepsilon(x-x^\star)$ 
%here this should be ||x-x^{\star}||_H. Do you want to redefine ||.||_H to change this ?
\begin{rmq}[Why a unique optimum ?]The uniqueness of the optimum is an hypothesis required to avoid that chosen samples come from two or more wells for $f$. In this case the averaging strategy would lead to a mistaken point because points from the different wells would be averaged. {Nonetheless, multimodal functions can be tackled using our non-quasiconvexity trick (Section \ref{nonqc}).}\end{rmq}
	\begin{rmq}[Which functions $f$ satisfy Assumption~\ref{ass:principal}?] One may wonder if Assumption~\ref{ass:principal} is restrictive or not. We can remark that three times continuously differentiable functions satisfy the assumption with $\alpha=3$, as long as the {unique} optimum satisfies a strict second order stationary condition. {Also, we will see in Section \ref{invar} that results are immediately valid for strictly increasing transformations of any $f$ for which Assumption~\ref{ass:principal} holds, so that we indirectly include all piecewise linear functions as well as long as they have a unique optimum. } So the class of functions is very large, and in particular allows non symmetric functions to be treated, which might seem counter intuitive at first. 
\end{rmq}


The aim of this paper is to study a parallel optimization problem as follows. We sample $X_{1},\cdots,X_{\lambda}$ from the uniform distribution
on $B(0,r)$. Let $X_{(1)},\cdots,X_{(\lambda)}$ denote
the ordered random variables, where the order is given by the objective
function 
\[
f(X_{(1)})\leq\cdots\leq f(X_{(\lambda)}).
\]
We then introduce the $\mu$-best average 
\[
\overline{X}_{(\mu)}=\frac{1}{\mu}\sum_{i=1}^{\mu}X_{(i)}
\]

In the following of the paper, we will compare the standard random search algorithm (i.e. $\mu=1$) with the algorithm that consists in returning the average of the $\mu$ best points. To this end, we will study the expected simple regret for functions satisfying the assumption: \[\mathbb{E}\left[f(\overline{X}_{(\mu)})\right]\]


\section{Technical lemmas}
\label{sec:teclemmas}
In this section, we prove two technical lemmas on $f$ that will be useful to study the convergence of the algorithm. The first one shows that $f$ can be upper bounded and lower bounded by two spherical functions. 
\begin{lemma}
\label{lem:sandwich}
Under Assumption~\ref{ass:principal}, there exist two real numbers $0<l\leq L$, such that, for all $x\in B(0,r)$:
\begin{align}
\label{eq:lip-cond}
   l\lVert x-x^\star\rVert^2 \leq f(x)\leq L\lVert x-x^\star\rVert^2.
\end{align}
Moreover such $l$ and $L$ must satisfy  $0< l\leq e_1(\mathbf{H})\leq e_d(\mathbf{H})\leq L$.
\end{lemma}
\begin{proof}
As $\mathbf{H}$ is symmetric positive definite, we have the following
classical inequality for the $\mathbf{H}$-norm
\begin{equation}\label{eq:sym-mat-ineq}
e_{1}(\mathbf{H})\lVert x-x^{\star}\rVert^{2}\le\lVert x-x^{\star}\rVert_{\mathbf{H}}^{2}\le e_{d}(\mathbf{H})\lVert x-x^{\star}\rVert^{2}
\end{equation}
Now set for $x\in B(0,r)\setminus\{x^{\star}\}$
\[
\phi(x):=\frac{f(x)-f(x^{\star})}{\lVert x-x^{\star}\rVert^{2}}=\frac{\lVert x-x^{\star}\rVert_{\mathbf{H}}^{2}}{\lVert x-x^{\star}\rVert^{2}}(1+\lVert x-x^{\star}\rVert_{\mathbf{H}}^{\alpha-2}\varepsilon(x-x^{\star})).
\]
By the above inequalities, we have 
\[
e_{1}(\mathbf{H})^{(\alpha-2)/2}\lVert x-x^{\star}\rVert^{\alpha-2}\le\lVert x-x^{\star}\rVert_{\mathbf{H}}^{\alpha-2}\le e_{d}(\mathbf{H})^{(\alpha-2)/2}\lVert x-x^{\star}\rVert^{\alpha-2}.
\]
Thus, as $\alpha>2$, we obtain $\lVert x-x^{\star}\rVert_{\mathbf{H}}^{\alpha-2}\rightarrow_{x\rightarrow x^{\star}}0$.
By assumption, the function $\varepsilon$ is also bounded as $x\rightarrow x^{\star}$.
\\
We thus conclude that there exists $\delta>0$ such that, for all
$x\in\overset{\circ}{B}(x^{\star},\delta)$
\[
\frac{1}{2}e_{1}(\mathbf{H})\le\phi(x)\le2e_{d}(\mathbf{H}).
\]
Now notice that $B(0,r)\setminus\overset{\circ}{B}(x^{\star},\delta)$
is a closed subset of the compact set $B(0,r)$ hence it is also compact.
Moreover, by assumption $f$ is continuous on $B(0,r)$ and $f(x)>0=f(x^{\star})$
for all $x\neq x^{\star}.$ Hence $\phi$ is continuous and positive
on this compact set. Thus it attains its minimum and maximum on this
set and its minimum is positive. In particular, we can write, on this set, for
some $l_{0},L_{0}>0$
\[
l_{0}\le\phi(x)\le L_{0}.
\]
We now set $l=\min\{l_{0},\frac{1}{2}e_{1}(\mathbf{H})\}$. Note that
$l>0$ because $l_{0}>0$ and $e_{1}(\mathbf{H})>0$ (as $\mathbf{H}$
is positive definite). We also set $L=\max\{L_{0},2e_{1}(\mathbf{H})\}$
which is also positive. These are global bounds for $\phi$ which gives the first part of the result.\\
For the second part, let $\mathbf{u}_{1}$ be a normalized eigenvector
respectively associated to $e_{1}(\mathbf{H})$. Then 
\begin{align*}
\frac{f(x^{\star}+\epsilon\mathbf{u}_{1})}{\lVert\epsilon\mathbf{u}_{1}\rVert^{2}}=e_{1}(\mathbf{H})+\epsilon^{\alpha-2}\varepsilon(\epsilon\mathbf{u}_{1})
\end{align*}
Taking the limit as $\epsilon\to0$. we get that, if $l$ satisfies~\eqref{eq:lip-cond},
then $l\leq e_{1}(\mathbf{H})$. Similarly, we can prove that $L\geq e_{d}(\mathbf{H})$.\end{proof}
Secondly, we frame $S_h$ into two ellipsoids as $h\to 0$. This lemma is a consequence of the assumptions we make on $f$.
\begin{lemma}
\label{lemma:sandwich-set}
Under Assumption~\ref{ass:principal}, there exists $h_0\geq 0$ such that for $h\leq h_0$, we have $A_h\subset S_h\subset B_h$ where:
\begin{align*}
A_h:=\{x\mid  \lVert x-x^\star\rVert_{\mathbf{H}}\leq \phi_-(h)\}\\
B_h:=\{x\mid   \lVert x-x^\star\rVert_{\mathbf{H}}\leq \phi_+(h)\}
\end{align*}
with $\phi_-(h)$ and $\phi_+(h)$ two functions satisfying 
\begin{eqnarray*}\phi_-(h)&=&\sqrt{h}-\frac{M}{2}h^{(\alpha-1)/2}+o(h^{(\alpha-1)/2}) \\
\mbox{ and } \phi_+(h)&=&\sqrt{h}+\frac{m}{2}h^{(\alpha-1)/2}+o(h^{(\alpha-1)/2})\end{eqnarray*} when $h\to 0$ for some constants $m>0$ and $M>0$ which are respectively a (specific) lower and upper bound for $\varepsilon$.
\end{lemma}
\begin{proof}
By assumption $\lvert\varepsilon\rvert\leq M$, hence we have: 
\begin{align*}
\{x\in B(0,r) & \mid\lVert x-x^{\star}\rVert_{\mathbf{H}}^{2}+M\lVert x-x^{\star}\rVert_{\mathbf{H}}^{\alpha}\leq h\}\subset S_{h}
\end{align*}
Let $g\colon u\mapsto u^{2}+Mu^{\alpha}$. This is a continuous,
strictly increasing function on $[0,+\infty)$. By a classical consequence
of the intermediate value theorem, this implies that $g$ admits a
continuous, strictly increasing inverse function. Note that $g(0)=0$
hence $g^{-1}(0)=0$. Thus we can write $\{u\geq 0|u^{2}+Mu^{\alpha}\le h\}=[0,g^{-1}(h)]$.
We now denote $g^{-1}$ by $\phi_{-}$. As $\phi_{-}$ is non-decreasing,
we get
\begin{align*}
\{x\in B(0,r) & \mid\lVert x-x^{\star}\rVert_{\mathbf{H}}^{2}+M\lVert x-x^{\star}\rVert_{\mathbf{H}}^{\alpha}\leq h\}=A_{h}\cap B(0,r)
\end{align*}
Now observe that for $h$ sufficiently small
\[
\{x\in B(0,r)\mid\lVert x-x^{\star}\rVert_{\mathbf{H}}^{2}+M\lVert x-x^{\star}\rVert_{\mathbf{H}}^{\alpha}\leq h\}=A_{h}.
\]
Indeed, if $x\in A_{h}$, we have by the triangle inequality and~\eqref{eq:sym-mat-ineq}
\begin{align*}
\lVert x\rVert & \le\lVert x^{\star}\rVert+\lVert x-x^{\star}\rVert\\
 & \le\lVert x^{\star}\rVert+e_{1}(\mathbf{H})^{-1/2}\lVert x-x^{\star}\rVert_{\mathbf{H}}\\
 & \le\lVert x^{\star}\rVert+e_{1}(\mathbf{H})^{-1/2}\phi_{-}(h)
\end{align*}
Recall that by assumption $\lVert x^{\star}\rVert<r$ and let $\delta=r-\lVert x^{\star}\rVert$.
As $\phi_{-}(h)\rightarrow_{h\rightarrow0}0$, for $h$ sufficiently
small, we have $e_{1}(\mathbf{H})^{-1/2}\phi_{-}(h)\le\delta$ hence
$\lVert x\rVert\le r$ for $h$ sufficiently small, which gives the inclusion $A_h \subset S_h$.\\
For the asymptotics of $\phi_{-}$, as we have by definition $\phi_{-}(h)^{2}(1+M\phi_{-}(h)^{\alpha-2})=h$,
and as $\phi_{-}(h)\rightarrow_{h\rightarrow0}0$ we deduce that $\phi_{-}(h)\sim_{0}\sqrt{h}$.
Let us define $u(h)=\phi_{-}(h)-\sqrt{h}$. We have $u(h)\in o(\sqrt{h})$.
We then compute: 
\begin{align*}
(\sqrt{h}+u(h))^{2}+M(\sqrt{h}+u(h))^{\alpha}=h
\end{align*}
This gives
\begin{align*}
u(h)(u(h)+2\sqrt{h}) & =-Mh^{\alpha/2}(1+\frac{u(h)}{\sqrt{h}})^{\alpha}\\
u(h)(\frac{u(h)}{2\sqrt{h}}+1) & =-\frac{M}{2}h^{(\alpha-1)/2}(1+\frac{u(h)}{\sqrt{h}})^{\alpha}
\end{align*}
As $u(h)\in o(\sqrt{h})$ for $h\rightarrow0$, we obtain
\[
u(h)\sim-\frac{M}{2}h^{(\alpha-1)/2}.
\]
which concludes for $\phi_{-}$.

On the other side, we recall that $f(x)>0$ for all $x\neq x^{\star}$
as $x^{\star}$ is the unique minimum of $f$ on $B(0,r)$. Write
\[
0<\lVert x-x^{\star}\rVert_{\mathbf{H}}^{2}(1+\lVert x-x^{\star}\rVert_{\mathbf{H}}^{\alpha-2}\varepsilon(x-x^{\star})).
\]
Now observe that, as $\lVert x^{\star}\rVert<r$, we have for $x\in B(0,r)$,
by the triangle inequality, $\lVert x-x^{\star}\rVert<2r$. Hence,
by the classical inequality for the $\mathbf{H}$-norm~\eqref{eq:sym-mat-ineq}, we get
\begin{align*}
\varepsilon(x-x^{\star}) & >-\frac{1}{\lVert x-x^{\star}\rVert_{\mathbf{H}}^{\alpha-2}}\geq-\left(\sqrt{e_{d}(\mathbf{H})}2r\right)^{-(\alpha-2)}=:-m
\end{align*}
So we have: 
\begin{align*}
S_{h}\subset\{x\in B(0,r) & \mid\lVert x-x^{\star}\rVert_{\mathbf{H}}^{2}-m\lVert x-x^{\star}\rVert_{\mathbf{H}}^{\alpha}\leq h\}
\end{align*}
The function $g\colon u\mapsto u^{2}-mu^{\alpha}$ is differentiable.
A study of the derivative shows that $g$ is continuous, strictly
increasing on $[0,r_{0}]$ and continuous, strictly decreasing on
$[r_{0},+\infty[$ where $r_{0}=(\frac{2}{\alpha m})^{1/(\alpha-2)}$.
Hence $g_{|[0,r_{0}]}$ admits a continuous strictly increasing inverse
$\phi_{+}$ and $g_{|[r_{0},+\infty[}$ a continuous strictly decreasing
inverse $\Tilde{\phi}$. We thus write 
\[
\{u\ge0|u^{2}-mu^{\alpha}\le h\}=[0,\phi_{+}(h)]\cup[\Tilde{\phi}(h),+\infty).
\]
Hence 
\begin{align*}
 \{x\in B(0,r)\mid&\lVert x-x^{\star}\rVert_{\mathbf{H}}^{2}-m\lVert x-x^{\star}\rVert_{\mathbf{H}}^{\alpha}\leq h\}\\
&=\big(B_{h}\cap B(0,r)\big)\cup\big(B(0,r)\cap V_{h}\big)
\end{align*}

with $V_{h}=\{x\in\mathbb{R}^{d}|\ \lVert x-x^{\star}\rVert_{\mathbf{H}}>\tilde{\phi}(h)\}$.
We now show that for $h$ sufficiently small
\[
\{x\in B(0,r)\mid\lVert x-x^{\star}\rVert_{\mathbf{H}}^{2}-m\lVert x-x^{\star}\rVert_{\mathbf{H}}^{\alpha}\leq h\}=B_{h}.
\]
Indeed, note first that if $x\in B(0,r)$, we obtain by~\eqref{eq:sym-mat-ineq}
\[
\lVert x-x^{\star}\rVert_{\mathbf{H}}^{2}\le e_{d}(\mathbf{H})\lVert x-x^{\star}\rVert^{2}<4e_{d}(\mathbf{H})r^{2}.
\]
where we have used that, as $\lVert x\rVert<r$, the triangle inequality
gives $\lVert x-x^{\star}\rVert<2r$. Hence $B(0,r)\subset\{x\in\mathbb{R}^{d}|\ \lVert x-x^{\star}\rVert_{\mathbf{H}}^{2}<4e_{d}(\mathbf{H})r^{2}\}$.
We now show that $B(0,r)\subset\{x\in\mathbb{R}^{d}|\ \lVert x-x^{\star}\rVert_{\mathbf{H}}\le\Tilde{\phi}(h)\}$.
Indeed, at $h=0$, $0=\phi_{+}(0)<\Tilde{\phi}(0)$ are by definition,
the two roots of 
\[
u^{2}-mu^{\alpha}=0.
\]
Hence $\Tilde{\phi}(0)=\sqrt{e_{d}(\mathbf{H})2r}$. By continuity
of $\Tilde{\phi}(h)$ at $h=0$, we obtain that $B(0,r)\subset\{x\in\mathbb{R}^{d}|\ \lVert x-x^{\star}\rVert_{\mathbf{H}}\le\Tilde{\phi}(h)\}$
for $h$ sufficiently small. As $\phi_{+}(h)\le\Tilde{\phi}(h)$,
we thus obtain that, for $h$ sufficiently small, $V_{h}\cap B(0,r)=\emptyset$.
Next, the same line of reasoning as the one for $\phi_{-}$, using
that $\phi_{+}(h)\rightarrow_{h\rightarrow0}0$ and $\lVert x^{\star}\rVert<r$,
shows that $B_{h}\cap B(0,r)=B_{h}$ for $h$ sufficiently small.
\\
Hence, for $h$ small enough we have
\[
\{x\in B(0,r)\mid\lVert x-x^{\star}\rVert_{\mathbf{H}}^{2}-m\lVert x-x^{\star}\rVert_{\mathbf{H}}^{\alpha}\leq h\}=B_{h}.
\]
This gives $S_h \subset B_h$.\\
Finally, similarly to $\phi_{-}$, we can show that $\phi_{+}(h)=\sqrt{h}+\frac{m}{2}h^{(\alpha-1)/2}+o(h^{(\alpha-1)/2})$,
which concludes the proof of this lemma. 
\end{proof}



\section{Bounds for random search}
\label{sec:randomsearch}
In this section we provide upper bounds and lower bounds for the random search algorithm for functions satisfying Assumption~\ref{ass:principal}. These bounds will also be useful for analyzing the convergence of the $\mu$-best approach.
\subsection{Upper Bound}
First, we prove an upper bound for functions satisfying Assumption~\ref{ass:principal}.
% \begin{lemma}[Upper bound for random search algorithm]\label{lem:upper1best}
% Let $f$ be a function satisfying Assumption~\ref{ass:principal}. Let $\lambda$ be a positive integer. As $\lambda\to\infty$ we have:
% $$\mathbb{E}_{X_1,\cdots,X_\lambda\sim U(B(0,r))}\left[ f\left(X_{(1)}\right)\right] \leq C_1 \lambda^{-\frac2d} +o(\lambda^{-\frac2d}) $$
% with $C_1>0$ a constant independent of $\lambda$.
% \end{lemma}

\begin{lemma}[Upper bound for random search algorithm]\label{lem:upper1best}
Let $f$ be a function satisfying Assumption~\ref{ass:principal}. There exists a constant $C_0>0$ and an integer $\lambda_0\in \mathbb{N}$ such that for all integers $ \lambda\geq \lambda_0$:
$$\mathbb{E}_{X_1,\cdots,X_\lambda\sim U(B(0,r))}\left[ f\left(X_{(1)}\right)\right] \leq C_0 \lambda^{-\frac2d}\quad. $$
\end{lemma}
\begin{proof}
Let us first recall the following classical property about the expectation of a positive valued random variable:    
\begin{align*}
\mathbb{E}_{X_1,\dots,X_\lambda\sim U(B(0,r))}&\left[ f\left(X_{(1)}\right)\right]= \int_0^\infty \mathbb{P}\left[ f\left(X_{(1)}\right)\geq t\right] dt
\end{align*}
By independence of the samples we have:  
\begin{align*}
\int_0^\infty \mathbb{P}\left[ f\left(X_{(1)}\right)\geq t\right] dt= \int_0^\infty \mathbb{P}_{X\sim U(B(0,r))}\left[ f\left(X\right)\geq t\right]^\lambda dt
\end{align*}
Then thanks to Lemma~\ref{lem:sandwich}:
\begin{align*}
     \int_0^\infty \mathbb{P}_{X\sim U(B(0,r))}&\left[ f\left(X\right)\geq t\right]^\lambda dt\\
 &\leq \int_0^\infty \mathbb{P}_{X\sim U(B(0,r))}\left[ L\lVert X-x^\star\rVert^2 \geq t\right]^\lambda dt\\
  &= \int_0^{L\left(r+\lVert x^\star\rVert\right)^2} \mathbb{P}\left[ \lVert X-x^\star\rVert \geq \sqrt{\frac{t}{L}}\right]^\lambda dt
\end{align*}
where the second equality follows because $\lVert X-x^{\star} \rVert \leq r$ almost surely.
Then, by definition of the uniform law as well as the non-increasing character of $t\mapsto \mathbb{P}_{X\sim U(B(0,r))}\left[ \lVert X-x^\star\rVert \geq \sqrt{\frac{t}{L}}\right]$, we obtain
\begin{align*}
  &\int_0^{L\left(r+\lVert x^\star\rVert\right)^2} \mathbb{P}_{X\sim U(B(0,r))}\left[ \lVert X-x^\star\rVert \geq \sqrt{\frac{t}{L}}\right]^\lambda dt \\
&=\int_0^{L\left(r-\lVert x^\star\rVert\right)^2} \mathbb{P}_{X\sim U(B(0,r))}\left[ \lVert X-x^\star\rVert \geq \sqrt{\frac{t}{L}}\right]^\lambda dt \\
&+ \int_{L\left(r-\lVert x^\star\rVert\right)^2}^{L\left(r+\lVert x^\star\rVert\right)^2} \mathbb{P}_{X\sim U(B(0,r))}\left[ \lVert X-x^\star\rVert \geq \sqrt{\frac{t}{L}}\right]^\lambda dt\\
&\leq \int_0^{L\left(r-\lVert x^\star\rVert\right)^2} \left[ 1-\left(\sqrt{\frac{t}{Lr^2}}\right)^d\right]^\lambda dt\\
 &+L\left(\left(r+\lVert x^\star\rVert\right)^2-\left(r-\lVert x^\star\rVert\right)^2\right) \mathbb{P}\left[ \lVert X-x^\star\rVert \geq r-\lVert x^\star\rVert\right]^\lambda \\
&\leq \int_0^{Lr^2} \left[ 1-\left(\frac{t}{Lr^2}\right)^{\frac{d}{2}}\right]^\lambda dt+4Lr\lVert x^\star\rVert\mathbb{P}\left[ \lVert X-x^\star\rVert \geq r-\lVert x^\star\rVert\right]^\lambda\\
&= Lr^2\int_0^{1} \left[ 1-u^{\frac{d}{2}}\right]^\lambda du+4Lr\lVert x^\star\rVert\mathbb{P}\left[ \lVert X-x^\star\rVert \geq r-\lVert x^\star\rVert\right]^\lambda
\end{align*}
Note that  $\mathbb{P}\left[ \lVert X-x^\star\rVert < r-\lVert x^\star\rVert\right] < 1$. Thus the second term in the last equality satisfies $\mathbb{P}\left[ \lVert X-x^\star\rVert < r-\lVert x^\star\rVert\right]^\lambda
\in o(\lambda^{-2/d})$.
The first term has a closed form given in~\cite{ppsnkbest}:
\begin{align*}
    \int_0^{1} \left[ 1-u^{\frac{d}{2}}\right]^\lambda du=\frac{\Gamma(\frac{d+2}{d})\Gamma(\lambda+1)}{\Gamma(\lambda+1+2/d)}
\end{align*}
Finally thanks to the Stirling approximation, we conclude:
\begin{align*}
\mathbb{E}_{X_1,\dots,X_\lambda\sim U(B(0,r))}&\left[ f\left(X_{(1)}\right)\right]\leq C_1 \lambda^{-2/d}+o(\lambda^{-2/d})
\end{align*}
where $C_1>0$ is a constant independent from $\lambda$.
\end{proof}
This lemma proves that the strategy consisting in returning the best sample (i.e. random search) has an upper rate of convergence of order $\lambda^{-2/d}$, which depends on dimension of the space. It also worth noting this result is common in the literature~\cite{bach,bergstra}

\subsection{Lower Bound}
We now give a lower bound for the convergence of the random search algorithm. We also prove a conditional expectation bound that will be useful for the analysis of the $\mu$-best averaging approach.


\begin{lemma}[Lower bound for random search algorithm]\label{lem:lower1best}
Let $f$ be a function satisfying Assumption~\ref{ass:principal}. There exist a constant $C_1>0$ and $\lambda_1\in \mathbb{N}$ such that for all integers $\lambda\geq \lambda_1$, we have the following lower bound for random search:
\begin{align*}
\mathbb{E}_{X_1,\dots,X_\lambda\sim U(B(0,r))}&\left[ f\left(X_{(1)}\right)\right]\geq C_1 \lambda^{-2/d}\quad.
\end{align*}
Moreover, let $(\mu_\lambda)_{\lambda\in\mathbb{N}}$ be a sequence of integers such that $\forall\lambda\geq 2$, $1\leq \mu_\lambda \leq \lambda -1$ and $\mu_\lambda\to\infty$. Then, there exist a constant $C_2>0$ and $\lambda_2\in \mathbb{N}$ such that for all $h\in [0,\max f]$ and $\lambda\geq\lambda_2$, we have the following lower bound when the sampling is conditioned: 
\begin{align*}
   \mathbb{E}_{X_1,\dots,X_\lambda\sim U(B(0,r))}&\left[ f\left(X_{(1)}\right)\mid f(X_{(\mu_{\lambda}+1)})= h\right] \geq C_2 h\mu_\lambda^{-2/d}\quad.
\end{align*}
\end{lemma}
\begin{proof}
The proof is very similar to the previous one. Let us first show the unconditional inequality. We use the identity for the expectation of a positive random variable
\begin{align*}
\mathbb{E}&_{X_1,\dots,X_\lambda\sim  U(B(0,r))}\left[ f\left(X_{(1)}\right)\right] \\
&= \int_0^\infty \mathbb{P}_{X_1,\dots,X_\lambda\sim  U(B(0,r))}\left[ f\left(X_{(1)}\right)\geq t\right] dt
\end{align*}
Since the samples are independent, we have
\begin{align*}
 \int_0^\infty &\mathbb{P}_{X_1,\dots,X_\lambda\sim U(B(0,r))}\left[ f\left(X_{(1)}\right)\geq t \right] dt\\ 
 & = \int_0^\infty \mathbb{P}_{X\sim  U(B(0,r))}\left[ f\left(X\right)\geq t\right]^\lambda dt
 \end{align*}
 Using Lemma~\ref{lem:sandwich}, we get:
 \begin{align*}
 \int_0^\infty& \mathbb{P}_{X\sim U(B(0,r))}\left[ f\left(X\right)\geq t\right]^\lambda dt\\
& \geq \int_0^\infty \mathbb{P}_{X\sim  U(B(0,r))}\left[ l\lVert X-x^\star\rVert^2 \geq t\right]^\lambda dt\\
& \geq \int_0^{l(r-\lVert x^\star\rVert)^2} \mathbb{P}_{X\sim U(B(0,r))}\left[ l\lVert X-x^\star\rVert^2 \geq t\right]^\lambda dt\\
&=\int_0^{l(r-\lVert x^\star\rVert)^2} \left[ 1-\left(\sqrt{\frac{t}{lr^2}}\right)^d\right]^\lambda dt
\end{align*}
%I don't think Chasles' relation is the right term in English.
We can decompose the integral to obtain:
\begin{align*}
    &\int_0^{l(r-\lVert x^\star\rVert)^2} \left[ 1-\left(\sqrt{\frac{t}{lr^2}}\right)^d\right]^\lambda dt\\
    &=\int_0^{lr^2} \left[ 1-\left(\sqrt{\frac{t}{lr^2}}\right)^d\right]^\lambda - \int_{l(r-\lVert x^\star\rVert)^2}^{lr^2} \left[ 1-\left(\sqrt{\frac{t}{lr^2}}\right)^d\right]^\lambda dt\\
    &\geq lr^2\frac{\Gamma(\frac{d+2}{d})\Gamma(\lambda+1)}{\Gamma(\lambda+1+\frac2d)}- l(r^2-(r-\lVert x^\star\rVert)^2)\left[ 1-\left(\frac{r-\lVert x^\star\rVert}{r}\right)^{d}\right]^\lambda\\
    &\geq \frac12 lr^2 \Gamma(\frac{d+2}{d}) \lambda^{-2/d}\text{ for $\lambda$ sufficiently large.}
\end{align*}
where the last inequality follows by Stirling's approximation applied to the first term and because the second term is $o(\lambda^{-2/d})$ as in previous proof.\\
This concludes the proof of the first part of the lemma. Let us now treat the case of the conditional inequality. Using the same first identity as above we have
\begin{align*}
\mathbb{E}&_{X_1,\dots,X_\lambda\sim U(B(0,r))}\left[ f\left(X_{(1)}\right)\mid f(X_{(\mu_{\lambda}+1)}) = h\right] \\
&= \int_0^\infty \mathbb{P}_{X_1,\dots,X_\lambda\sim U(B(0,r))}\left[ f\left(X_{(1)}\right)\geq t \mid f(X_{({\mu_{\lambda}}+1)}) = h\right] dt
\end{align*}
% Note that sampling $\mu$ independent ordered variables on $B(0,r)$ while conditioning on the $\mu+1$ belonging to the $h$-level set of $f$ is equivalent to sampling $\mu$ independent \emph{unordered} variables on the $h-$level set of $f$ directly.
%
\begin{rmq}
\label{rk:samples}
Note that if we sample $\lambda$ independent variables $X_1 \ldots X_\lambda$ while conditioning on $f(X_{(\mu+1)})=h$ % belonging to the $h$-level set of $f$
and keep only the $\mu$-best variables $X_i$ such that $f(X_i)\le h$, this is exactly equivalent to sampling directly $X_1 \ldots X_\mu$ from the $h$-level set. This result was justified and used in~\cite{ppsnkbest} in their proofs.
\end{rmq}
%
Hence we obtain
\begin{align*}
 \int_0^\infty &\mathbb{P}_{X_1,\dots,X_\lambda\sim U(B(0,r))}\left[ f\left(X_{(1)}\right)\geq t \mid f(X_{(\mu_{\lambda}+1)})= h\right] dt\\ 
 & = \int_0^\infty \mathbb{P}_{X\sim U(S_h)}\left[ f\left(X\right)\geq t\right]^{\mu_{\lambda}} dt
 \end{align*}
 Using Lemma~\ref{lem:sandwich}, we get:
 \begin{align*}
 \int_0^\infty& \mathbb{P}_{X\sim U(S_h)}\left[ f\left(X\right)\geq t\right]^{\mu_{\lambda}} dt\\
& \geq \int_0^\infty \mathbb{P}_{X\sim U(S_h)}\left[ l\lVert X-x^\star\rVert^2 \geq t\right]^{\mu_{\lambda}} dt\\
& \geq \int_0^\infty \mathbb{P}_{X\sim U(B(x^\star,\sqrt{\frac{h}{l}}))}\left[ l\lVert X-x^\star\rVert^2 \geq t\right]^{\mu_{\lambda}} dt
\end{align*}
where the last inequality follows from the inclusion $S_h\subset B(x^\star,\sqrt{\frac{h}{l}})$, which is also a consequence of Lemma~\ref{lem:sandwich}. We then get
\begin{align*}
\int_0^\infty& \mathbb{P}_{X\sim U(B(x^\star,\sqrt{\frac{h}{l}}))}\left[ l\lVert X-x^\star\rVert^2 \geq t\right]^{\mu_{\lambda}} dt\\
& = \int_0^h \mathbb{P}_{X\sim U(B(x^\star,\sqrt{\frac{h}{l}}))}\left[ l\lVert X-x^\star\rVert^2 \geq t\right]^{\mu_{\lambda}} dt\\
& = \int_0^{h} \left[ 1-\left(\sqrt{\frac{t}{h}}\right)^d\right]^{\mu_{\lambda}} dt\\
& = h\frac{\Gamma(\frac{d+2}{d})\Gamma(\mu_\lambda+1)}{\Gamma(\mu_\lambda+1+2/d)}\\
& \geq \frac12 h \Gamma(\frac{d+2}{d})\mu_\lambda^{-2/d}\text{ for $\lambda$ sufficiently large.}
\end{align*}
%TODO: change by $\mu_\lambda$
\end{proof}
This lemma, along with Lemma~\ref{lem:upper1best}, proves that for any function satisfying Assumption~\ref{ass:principal}, its rate of convergence is exponentially dependent on the dimension and of order $\lambda^{-2/d}$ where $\lambda$ is the number of points sampled to estimate the optimum. 
\begin{rmq}[Convergence of the distance to the optimum]
It is worth noting that, thanks to Lemma~\ref{lem:sandwich}, the convergence rates are also valid for the square distance to the optimum $x^\star$. 
\end{rmq}



\section{Convergence rates for the $\mu$-best averaging approach}
\label{sec:mubestrate}
% \begin{lemma} 
% \label{lem:condit-upper} 
% Let $f$ be a function satisfying Assumption~\ref{ass:principal}. Let $h\in[0,\max f]$. Let $\lambda$ and $\mu$ be two integers such that $1\leq \mu \leq \lambda-1$. Then we have
% \[
% \mathbb{E}_{X_{1},...X_{\lambda}\sim U(B(0,r))}\left[f(\bar{X}_{(\mu)})\mid f(X_{(\mu+1)})=h\right]\le C\frac{h}{\mu}+O(h^{\alpha-1})
% \]
% as $h\rightarrow0$, for some $C>0$ independent from $h$, $\mu$ and $\lambda$. \end{lemma} 
In the next section we focus on the case where we average the $\mu$ best samples among the $\lambda$ samples.  We first prove a lemma when the sampling is conditional on the $(\mu+1)$-th value.
\begin{lemma} 
\label{lem:condit-upper} 
Let $f$ be a function satisfying Assumption~\ref{ass:principal}. There exists a constant $C_3>0$ such that for all $h\in[0,\max f]$ and $\lambda$ and $\mu$ two integers such that $1\leq \mu \leq \lambda-1$, we have the following conditional upper bound:
\[
\mathbb{E}_{X_{1},...X_{\lambda}\sim U(B(0,r))}\left[f(\bar{X}_{(\mu)})|f(X_{(\mu+1)})=h\right]\le C_3\left(\frac{h}{\mu}+h^{\alpha-1}\right).
\]


\end{lemma} 
\begin{proof}
We first decompose the expectation as follows.
\begin{align}
\mathbb{E}&_{X_{1},...X_{\lambda}\sim U(B(0,r))}  \left[f(\bar{X}_{(\mu)})|f(X_{(\mu+1)}))=h\right]\nonumber\\
 & =\mathbb{E}_{X_{1},...X_{\mu}\sim U(S_{h})}\left[f(\bar{X}_{\mu})\right]\nonumber\\
 & =\mathbb{E}_{X_{1},\cdots,X_{\mu}\sim U(S_{h})}\left[\lVert \bar{X}_{\mu}-x^\star\rVert_{\mathbf{H}}^2\right]\label{eq:first-term}\\
 &+\mathbb{E}_{X_{1},\cdots,X_{\mu}\sim U(S_{h})}\left[\lVert\bar{X}_{\mu}-x^\star\rVert^{\alpha}_{\mathbf{H}}\varepsilon(\bar{X}_{\mu}-x^\star)\right]\label{eq:snd-term}
\end{align}
where we have use the same argument as in Remark~\ref{rk:samples} in the first equality.
We will treat the terms~\eqref{eq:first-term} and~\eqref{eq:snd-term} independently. We first look at~\eqref{eq:first-term}. We have the following ``bias-variance'' decomposition.
\begin{align*}
\mathbb{E}_{X_{1},\cdots,X_{\mu}\sim U(S_{h})}\lVert\bar{X}_{\mu}-x^{\star}\rVert^{2}_{\mathbf{H}} 
=&(1-\frac{1}{\mu})\lVert\mathbb{E}_{X\sim U(S_{h})}X-x^{\star}\rVert^{2}_{\mathbf{H}}\\
&+\frac{1}{\mu}\mathbb{E}_{X\sim U(S_{h})}\lVert X-x^{\star}\rVert^{2}_{\mathbf{H}}
\end{align*}
We will use Lemma~\ref{lemma:sandwich-set}. We have $A_h\subset S_h\subset B_h$. Hence for the variance term
\[
\frac{1}{\mu}\mathbb{E}_{X\sim U(S_{h})}\lVert X-x^{\star}\rVert^{2}_{\mathbf{H}}\leq\frac{1}{\mu}\mathbb{E}_{X\sim U(S_{h})} \phi_+(h)^{2} \leq \frac{\phi_+(h)^{2}}{\mu}\sim_{0} \frac{h}{\mu}.
\]
where $\sim_0$ means ''is equivalent to $\dots$ when $h\to0$, in other words, $u(h)\sim_0 v(h)$ iff $\frac{u(h)}{v(h)}\to 0$ as $h\to0$. For the bias term, recall that \[\mathbb{E}_{X\sim U(S_{h})}\left[X-x^\star\right] = \frac{1}{\mathrm{vol}(S_h)}\int_{S_h} (x-x^\star) dx.\] We then have by inclusion of sets
\begin{align*}
    \mathrm{vol}(A_h)\leq \mathrm{vol}(S_h)\leq \mathrm{vol}(B_h)
\end{align*}
Note that the volume of the $d$-dimensional ellipsoid $B_h$ satisfies $\mathrm{vol}(B_h)=\phi_+(h)^{d}\frac{\omega_{d}}{\mathrm{det}(\mathbf{H})}$ with $\omega_{d}=\mathrm{vol}(B(0,1))$ and similarly for $A_h$.
From this we deduce by the squeeze theorem that 
\[\mathrm{vol}(S_h)\sim \frac{\omega_dh^{d/2}}{\mathrm{det}(\mathbf{H})}.\]We now decompose the integral 
\begin{align*}
\int_{S_{h}}(x-x^\star)dx & =\int_{A_{h}}(x-x^\star)dx+\int_{S_{h}\setminus A_{h}}(x-x^\star)dx\\
 & =\int_{S_{h}\setminus A_{h}}(x-x^\star)dx
\end{align*}
(because $A_{h}$ is an ellipsoid centered at $x^\star$ hence the integral of $x-x^\star$
over it is $0$). We then upper-bound using the triangle inequality for the $\mathbf{H}-$norm:
\begin{align*}
\lVert\int_{{S}_{h}\setminus A_{h}}(x-x^\star)dx\rVert_{\mathbf{H}} &\leq \int_{{S}_{h}\setminus A_{h}}\lVert x-x^\star\rVert_{\mathbf{H}} dx\\
& \le \phi_+(h)\mathrm{vol}({S}_{h}\setminus A_{h})\\
 & = \phi_+(h)(\mathrm{vol}({S}_{h})-\mathrm{vol}(A_{h}))\\
 &\leq \phi_+(h)(\mathrm{vol}(B_h)-\mathrm{vol}(A_h))\\
 & \sim d\frac{\omega_d}{\mathrm{det}(\mathbf{H})}\frac{m+M}{2}h^{d/2}h^{(\alpha-1)/2}
\end{align*}
For the last equivalent, we used a Taylor expansion for the volume of $A_h$ and $B_h$. We conclude that there exist $h_1>0$ and a constant $C>0$ not depending on $\lambda$ and $\mu$ such that for $h\leq h_1$,
\begin{align*}
\lVert\mathbb{E}_{X\sim U(S_{h})}\left[X\right]-x^\star\rVert^{2}_{\mathbf{H}} \leq C h^{\alpha-1}
\end{align*}
Since $h$ is upper bounded by $\max f$, the previous inequality can be extended to $h\in[0,\max f]$, with a possibly larger constant still not depending on $\lambda$ and $\mu$. Let us now upper bound the remainder term~\eqref{eq:snd-term}.
As $\varepsilon \leq M$ by assumption, we can write
\begin{align*}
\mathbb{E}_{X_{1},\cdots,X_{\mu}\sim U(S_{h})}&\left[\lVert\bar{X}_{\mu}-x^\star\rVert^{\alpha}_{\mathbf{H}}\varepsilon(\bar{X}_{\mu}-x^\star)\right]\\
&\le M\mathbb{E}_{X_{1},\cdots,X_{\mu}\sim U(S_{h})}\left[\lVert\bar{X}_{\mu}-x^\star\rVert^{\alpha}_{\mathbf{H}}\right]
\end{align*}

We have $X_{1},\cdots,X_{\mu}\in S_{h}\subset B_{h}$
hence by the convexity of $B_h$ (which is a ball for the $\mathbf{H}$-norm) we also have $\bar{X}_{\mu}\in B_{h}$ and
thus, for $h$ sufficiently small, we have:
\[
\lVert\bar{X}_{\mu}-x^\star\rVert_{\mathbf{H}}\le \phi_+(h).
\]
Note that $\phi_+(h)\sim_{0} \sqrt{h} $ thus, for $h$ sufficiently small, $ \lVert\bar{X}_{\mu}-x^\star\rVert_{\mathbf{H}}\le 1$
almost surely, hence, as $\alpha > 2$
\begin{align*}
\lVert\bar{X}_{\mu}-x^\star\rVert^{\alpha}_{\mathbf{H}}\leq \lVert\bar{X}_{\mu}-x^\star\rVert^{2}_{\mathbf{H}}
\end{align*}
almost surely. Since $h$ is upper bounded, we have the existence of a constant $C'>0$ not depending on $\lambda$ and $\mu$, such that for all $h\in [0,\max f]$,
\begin{align*}
\lVert\bar{X}_{\mu}-x^\star\rVert^{\alpha}_{\mathbf{H}}\leq C'\lVert\bar{X}_{\mu}-x^\star\rVert^{2}_{\mathbf{H}}
\end{align*}
Thus we can upper bound the remainder with the same bounds as the one for the main term (up to constants), for any $h\in[0,\max f]$.
We now group the ``main'' term and remainder term to get the existence of a constant $C_3>0$ not depending on $\lambda$ and $\mu$ such that for all $h\in [0,\max f]$,
\[
\mathbb{E}_{X_{1},...X_{\lambda}\sim U(B(0,r))}\left[f(\bar{X}_{(\mu)})|f(X_{(\mu+1)})=h\right]\le C_3\left(\frac{h}{\mu}+h^{\alpha-1}\right)\quad.
\]


\end{proof}



% \begin{theorem} 
% Under assumptions~\ref{ass:principal}
% and also with $1\le \mu<n\big(\frac{L(r-||x^{\star}||)^{2}}{r}\big)^{d}$
% we have 
% \begin{align*}
% \mathbb{E}_{X_{1},...X_{\lambda}\simB(0,r)} & \left[f(\bar{X}_{(\mu)})-f(x^{\star})\right]\\
%  & \leq C\bigg[\mu^{\alpha/d}n^{\alpha/d}+\mu^{2/d-1}n^{-2/d}+\mu^{2/d-1}a^{n}+\exp(-\frac{2\varepsilon^{2}}{n})\bigg]
% \end{align*}
% for some $C\ge0$ depending on $L,\mu,r,d,x^{\star}$, some $|a|<1$
% and $\varepsilon=n\big(\frac{L(r-||x^{\star}||)^{2}}{r}\big)^{d}-\mu$.
% With the choice $\mu=C'n^{\frac{\alpha-2}{\alpha+d-2}}$, we get a main
% rate of $n^{-\alpha/(\alpha+d-2)}$ which is \emph{strictly }better
% than the rate of the 1-best whenever $\alpha>2$, $d>2$ 
% \end{theorem} 

We are now set to prove our main result, which is an upper convergence rate for the $\mu$-best approach. This is the main result of the paper.
\begin{thm}\label{thm:principal}
Let $f$ be a function satisfying Assumption~\ref{ass:principal}.  Let $(\mu_\lambda)_{\lambda\in\mathbb{N}}$ be a sequence of integers such that $\forall\lambda\geq 2$, $1\leq \mu_\lambda \leq \lambda -1$ and $\mu_\lambda\to\infty$. Then, there exist two constants $C,C'>0$ and $\Tilde{\lambda}\in \mathbb{N}$ such that  for $\lambda\geq\Tilde{\lambda}$, we have the upper bound: 
\begin{align*}
 \mathbb{E}_{X_{1},\dots,X_{\lambda}\sim U(B(0,r))}\left[f(\bar{X}_{(\mu_\lambda)})\right] \leq C\frac{\mu_\lambda^{\frac{2(\alpha-1)}{d}}}{\lambda^{\frac{2(\alpha-1)}{d}}}+C'\frac{\mu_\lambda^{\frac{2}{d}-1}}{\lambda^{\frac{2}{d}}}\quad.
\end{align*}
In particular if $\mu_{\lambda}\sim C^{''}\lambda^{\frac{2(\alpha-2)}{d+2(\alpha-2)}}$
for some $C^{''}>0$, we obtain:
\begin{align*}
\mathbb{E}_{X_{1},\dots,X_{\lambda}\sim U(B(0,r))}\left[f(\bar{X}_{(\mu_{\lambda})})\right] & \le C^{'''}\lambda^{-\frac{2(\alpha-1)}{d+2(\alpha-2)}}
\end{align*}
for some $C'''>0$ independent of $\lambda$.\end{thm}

{We note that $\frac{\mu}{\lambda}\to 0$ as $\lambda\to 0$. This makes sense intuitively: we average points in a sublevel set, which makes sense only if, asymptotically in $\lambda$, this sublevel set shrinks to a neighborhood of the optimum.}

\begin{proof}
The random variable $f(X_{(\mu_\lambda+1)})$ takes its values in $[0,\max f]$ almost surely. As such, thanks to Lemma~\ref{lem:condit-upper}, there exists a constant $C_3>0$ such that for all $\lambda\geq 1$:
 \begin{align*}
      \mathbb{E}\left[f(\bar{X}_{(\mu_\lambda)})\right]&=\mathbb{E}\left[\mathbb{E}\left[f(\bar{X}_{(\mu_\lambda)})\mid f(X_{(\mu_\lambda+1)}) \right]\right]\\
        &\leq \mathbb{E}\left[C_3\left(\frac{1}{\mu_\lambda}f(X_{(\mu_\lambda+1)})+f(X_{(\mu_\lambda+1)})^{\alpha-1}\right)\right]\\
      &= C_3\left(\frac{1}{\mu_\lambda}\mathbb{E}\left[f(X_{(\mu_\lambda+1)})\right]+\mathbb{E}\left[f(X_{(\mu_\lambda+1)})^{\alpha-1}\right]\right)
 \end{align*}
 
Let us first bound $\mathbb{E}\left[f(X_{(\mu_\lambda+1)})\right]$. Thanks to Lemma~\ref{lem:lower1best}, there exist a constant $C_2>0$ and $\lambda_2\in \mathbb{N}$ such that:
\begin{align*}
  \mathbb{E} \left[f(X_{(\mu_\lambda+1)})\right]&\leq \frac{\mu_\lambda^{2/d}}{C_2}\mathbb{E}\left[\mathbb{E} \left[f(X_{(1)})\mid f(X_{(\mu_\lambda+1)})\right]\right]\\
  &=\frac{\mu_\lambda^{2/d}}{C_2}\mathbb{E} \left[f(X_{(1)})\right]
\end{align*}

Thanks to Lemma~\ref{lem:upper1best}, there exists a constant $C_0>0$ and an integer $\lambda_0\in \mathbb{N}$ such that for all integers $ \lambda\geq \lambda_0$:
$$\mathbb{E}_{X_1,\cdots,X_\lambda\sim U(B(0,r))}\left[ f\left(X_{(1)}\right)\right] \leq C_0 \lambda^{-\frac2d}\quad. $$

Then finally for $\lambda\geq\max(\lambda_0,\lambda_2)$
\begin{align*}
    \mathbb{E} \left[f(X_{(\mu_\lambda+1)})\right]\leq\frac{C_0}{C_2} \frac{\mu_\lambda^{2/d}}{\lambda^{2/d}}\quad.
\end{align*}

For the term $\mathbb{E}\left[f(X_{(\mu_{\lambda}+1)})^{\alpha-1}\right]$,
we write thanks to Lemma~\ref{lem:lower1best}
\[
\mathbb{E}\left[f(X_{(\mu_{\lambda}+1)})^{\alpha-1}\right]\leq\frac{\mu_{\lambda}^{2(\alpha-1)/d}}{C_{2}^{\alpha-1}}\mathbb{E}\left[\mathbb{E}\left[f(X_{(1)})\mid f(X_{(\mu_{\lambda}+1)})\right]^{\alpha-1}\right].
\]
Then, by Jensen's inequality for the conditional expectation, we get
\[
\mathbb{E}\left[f(X_{(\mu_{\lambda}+1)})^{\alpha-1}\right]\le\frac{\mu_{\lambda}^{2(\alpha-1)/d}}{C_{2}^{\alpha-1}}\mathbb{E}\left[f(X_{(1)})^{\alpha-1}\right].
\]
Similarly to Lemma~\ref{lem:upper1best}, by replacing $\lVert X-x^\star\rVert^2$ by $\lVert X-x^\star\rVert^{2(\alpha-1)}$, one can show $\mathbb{E}\left[f(X_{(1)})^{\alpha-1}\right]\le C'_{3}\lambda^{-2(\alpha-1)/d}$
for some $C'_{3}>0$ independent of $\lambda$. We thus get $\mathbb{E}\left[f(X_{(\mu_{\lambda}+1)})^{\alpha-1}\right]\le C\frac{\mu_{\lambda}^{2(\alpha-1)/d}}{\lambda^{2(\alpha-1)/d}}$
for some $C>0$ independent of $\lambda$, which, combined with the
above bound on $\mathbb{E}\left[f(X_{(\mu_{\lambda}+1)})\right]$,
concludes the proof of the main bound.\\ To conclude for the final bound, it suffices to notice that this choice of $\mu_{\lambda}$ ensures that the two terms in the upper bound are of the same order.\end{proof}

This theorem gives an asymptotic upper rate of convergence for the algorithm that consists in averaging the best samples to optimize a function with parallel evaluations. The proof of the optimality of the rate is left as further work. We also remark that the selection ratio depends on the dimension and goes to $0$ as $\lambda\to\infty$. It sounds natural since the level sets might be assymetric and then keeping a constant selection rate would give a biased estimate of the optimum (see Figure~\ref{fig:levelset}). However, the choice proposed for $\mu$ is the best one can make with regards to the upper bound we obtained. We make two important remarks about the theorem.
\begin{figure}
\centering
	\includegraphics[width=.4\textwidth]{sections/appendix/foga2021-kbest/samples/joli.png}
	\caption{\label{zoli}Assume that we consider a fixed ratio $\mu/\lambda$ and that $\lambda$ goes to $\infty$. The average of selected points, in an unweighted setting and with uniform sampling, converges to the center of the area corresponding to the ratio $\mu/\lambda$: we will not converge to the optimum if that optimum is not the middle of the sublevel. This explains why we need $\mu/\lambda\to 0$ as $\lambda \to \infty$: we do not want to stay at a fixed sublevel. }
	\label{fig:levelset}
\end{figure}
\begin{rmq}[Comparison with random search] The asymptotic
rate obtained for the $\mu$-best averaging approach is of order $\lambda^{-\frac{2(\alpha-1)}{d+2(\alpha-2)}}$,
which is strictly better than the $\lambda^{-2/d}$ rate obtained
with random search, as soon as $d>2$ (because $\alpha>2$) .
This theorem then proves our claim on a wide range of functions. 
\end{rmq}
% \begin{rmq}[Non-smooth or high-dimensional functions]
% Moreover,
% thanks to Lemma~\ref{lem:sandwich}, both rates hold also for the
% expectation of the quadratic distance to the optimum $x^{\star}$: interestingly, this result becomes true for any $\tilde f = g \circ f$ for $g$ increasing and $f$ verifying Assumption \ref{ass:principal}. This extension, classical in evolutionary computation, is valid for our setting (see Section \ref{invar}).
% Results based on surrogate models (e.g. \cite{bach}) are not applicable here. Even in the differentiable case, our results is faster when the ratio smoothness/dimension is small.
% \end{rmq}


% \begin{rmq}
% The results we show are asymptotic in $\lambda$, and do not quantify the domain of validity of the bounds, hence the number of samples $\lambda$ might be exponential in function of the dimension for the bound to be valid non asymptotically. 
% \end{rmq}

\begin{rmq}[Comparison with~\cite{ppsnkbest}] ~\cite{ppsnkbest} obtained a rate of order $\lambda^{-1}$ for the sphere function. This rate is better than the one described in Theorem~\ref{thm:principal}. This comes from the bias term in Lemma~\ref{lem:condit-upper}. Indeed for the sphere function, sublevel sets are symmetric, hence the bias term equals $0$, which is not the case in general for functions satisfying Assumption~\ref{ass:principal}. In this paper we are able to deal with potentially non symmetric functions. One can remark, that if the sublevel sets are symmetric the bias term vanishes and we recover the rate of~\cite{ppsnkbest}.
\end{rmq}
% \begin{proof}
% In what follows, we will use many times a decomposition of the form
% \[
% h(Y)=h(Y)\bigg[\mathbb{I}_{g(Y)\le a}+\mathbb{I}_{g(Y)>a}\bigg]
% \]
% All the $h(Y)\mathbb{I}_{g(Y)>a}$ will be treated using the Hoeffding
% inequality and the fact that $f$ is bounded in essentially the same
% way. As such, we leave this for the end of the proof and we focus
% on the $\mathbb{I}_{g(Y)\leq a}$ term. \\
% We first decompose 
% \[
% f\left(\bar{X}_{(\mu)}\right)=f\left(\bar{X}_{(\mu)}\right)\left[\mathbb{I}_{f\left(X_{(\mu+1)}\right)\leq h_{0}}+\mathbb{I}_{f\left(X_{(\mu+1)}\right)>h_{0}}\right]
% \]
% where $h_{0}>0$ is sufficiently small so that for $h\leq h_{0}$,
% the conclusion of lemma~\ref{lem:condit-upper} holds. We first concentrate
% on the expectation of the first term. We use the notation $\mathbb{J}=\mathbb{I}_{f\left(X_{(\mu+1)}\right)\leq h_{0}}$.
% Note that this is a random variable generated by $f(X_{(\mu+1)})$.
% By lemma~\ref{lem:condit-upper}, we have 
% \begin{align*}
% \mathbb{E}_{X_{1},...X_{\lambda}\sim U(B(0,r))} & \left[f\left(\bar{X}_{(\mu)}\right)\mathbb{I}_{f\left(X_{(\mu+1)}\right)\leq h_{0}}\right]\\
%  & =\mathbb{E}\left[\mathbb{E}\left[f\left(\bar{X}_{(\mu)}\right)\mathbb{I}_{f\left(X_{(\mu+1)}\right)\leq h_{0}}|f\left(X_{(\mu+1)}\right)\right]\right]\\
%  & \leq\mathbb{E}\left[\mathbb{I}_{f\left(X_{(\mu+1)}\right)\leq h_{0}}[C\frac{f(X_{(\mu+1)})}{\mu}+O(f(X_{(\mu+1)})^{\alpha-1})]\right]\\
%  & \leq C'\mathbb{E}\left[\frac{f(X_{(\mu+1)})}{\mu}+f(X_{(\mu+1)})^{\alpha-1}\right]
% \end{align*}
% for some constant $C'>0$. For the first term above, we do another
% decomposition to use lemma 4
% \[
% f(X_{(\mu+1)})=f(X_{(\mu+1)})\bigg\{\mathbb{I}{}_{f\left(X_{(\mu+1)}\right)\leq L(r-||x^{\star}||)^{2}}+\mathbb{I}{}_{f\left(X_{(k+1)}\right)>L(r-||x^{\star}||)^{2}}\bigg\}
% \]
% As previously, we leave aside for now the second term. By lemma 4
% and then lemma 3 and using a Stirling approximation for the ratio
% of $\Gamma$ factors, we obtain
% \begin{align*}
% \mathbb{E}\left[f(X_{(\mu+1)})\mathbb{I}{}_{f\left(X_{(\mu+1)}\right)\leq L(r-||x^{\star}||)^{2}}\right] & \le C_{2}^{-1}\mu^{2/d}\mathbb{E}\left[f(X_{(1)})-f(x^{\star})\right]\\
%  & \le C_{1}C_{2}^{-1}\mu^{2/d}\bigg[\lambda^{-\frac{2}{d}}+o(\lambda^{-\frac{2}{d}})\bigg]
% \end{align*}
% We now study the term $f(X_{(\mu+1)})^{\alpha-1}$. We decompose again
% and focus on $\mathbb{E}f(X_{(\mu+1)})^{\alpha-1}\mathbb{I}{}_{f\left(X_{(\mu+1)}\right)\leq L(r-||x^{\star}||)^{2}}$.
% By lemma 4, we get
% \[
% \mathbb{E}f(X_{(\mu+1)})^{\alpha-1}\mathbb{I}{}_{f\left(X_{(\mu+1)}\right)\leq L(r-||x^{\star}||)^{2}}\le\mu^{2/d}C_{2}^{-1}\mathbb{E}\biggg[\mathbb{E}_{X_{1},...X_{\lambda}\sim }\left[f(X_{(1)})|f(X_{(\mu+1)})=h\right]\biggg]^{\alpha-1}
% \]
% We now use Jensen's inequality (for conditional expectation) on the
% convex function $u\mapsto u^{\alpha-1}$ to get
% \[
% \mathbb{E}f(X_{(\mu+1)})^{\alpha-1}\mathbb{I}{}_{f\left(X_{(k+1)}\right)\leq L(r-||x^{\star}||)^{2}}\le\mu^{\alpha/d}C_{2}^{-(\alpha-1)}\mathbb{E}_{X_{1,\cdots,}X_{\lambda}\sim U(B(0,r))}[f(X_{(1)})^{\alpha-1}]
% \]
% We compute the last quantity. By a proof similar to the one in lemma
% 3, we obtain, with $u_{\alpha}=\left(L\left(r-||x^{\star}||\right)^{2\alpha-2}-L\left(r+||x^{\star}||\right)^{2\alpha-2}\right)$
% \[
% \mathbb{E}_{X_{1},...X_{\lambda}\sim U(B(0,r)) }\left[ f(X_{(1)})^{\alpha-1}\right]=(\alpha-1)Lr^{2}\frac{2}{d}\frac{\Gamma(\frac{2}{d}(\alpha-1))\Gamma(\lambda+1)}{\Gamma(\lambda+1+2(\alpha-1)/d)}+u_{\alpha}\mathbb{P}_{X\sim U(B(0,r))}\left[||X-x^{\star}||\geq r-||x^{\star}||\right]^{\lambda}
% \]
% It remains to bound 
% \[
% \mathbb{E}_{X_{1},...X_{\lambda}\sim U(B(0,r))}\left[f\left(\bar{X}_{(\mu)}\right)\mathbb{I}_{f\left(X_{(\mu+1)}\right)>h_{0}}\right]
% \]
% \[
% \mathbb{E}\left[f(X_{(\mu+1)})\mathbb{I}{}_{f\left(X_{(\mu+1)}\right)>L(r-||x^{\star}||)^{2}}\right]
% \]
% \[
% \mathbb{E}f(X_{(\mu+1)})^{\alpha-1}\mathbb{I}{}_{f\left(X_{(\mu+1)}\right)>L(r-||x^{\star}||)^{2}}
% \]
% Recall that $f$ is upper-bounded (for instance because it is continuous
% on a compact set). Each quantity above can then be upper-bounded by
% a constant times $\mathbb{P}(f(X_{(\mu+1)})>a)$ for some $a>0$.
% We now note that 
% \[
% \mathbb{P}_{X_{1},\cdots,X_{\lambda}\sim U(B(0,r))}\left(f(X_{(\mu+1)})>a\right)=\mathbb{P}_{U\sim B(n,(a/r)^{d})}(U\le\mu)
% \]
% where $\mathcal{B}(n,p)$ means a binomial variable of parameters
% $(n,p)$ for some $a>0$. Let now $\delta=\lambda(\frac{a}{r})^{d}-\mu$.
% Assume that $\delta>0$. Then Hoeffding's inequality gives
% \[
% \mathbb{P}_{U\sim\mathcal{B}(n,(a/r)^{d})}(U\le\mu)\le\exp(-\frac{2\delta^{2}}{\lambda})
% \]
% By taking $b=\min a_{i}$ for the $a_{i}=a$ appearing above, and
% corresponding $\delta$ we thus obtain Note that theHence we obtain
% \[
% \mathbb{E}_{X_{1},...X_{\lambda}\sim U(B(0,r))}\left[f\left(\bar{X}_{(\mu)}\right)\right]\le C(\mu^{\alpha/d}\lambda^{\alpha/d}+\mu^{2/d-1}\lambda^{-2/d}+\mu^{2/d-1}o(\lambda^{-2/d})+\exp(-\frac{2\delta^{2}}{\lambda}))
% \]
% \end{proof}

\section{Handling wider classes of functions}
\label{sec:wider}
The results we proved are valid for functions satisfying Assumption~\ref{ass:principal}. In particular, the functions are supposed to be regular and have a unique optimum point. In this section, we propose to extend our results to wider classes of functions.

\subsection{Invariance by Composition with Non-Decreasing Functions}\label{invar}

{Mathematical results are typically proved under some smoothness assumptions: however, algorithms enjoying some invariance to monotonic transformations of the objective functions do converge on wider spaces of functions as well \cite{monoto}.}
Since the method is based on comparison between the samples, the rank is invariant when the function $f$ is composed with a strictly increasing function $g$. Let $f$ be a function satisfying Assumption~\ref{ass:principal} and $g$ be a strictly increasing function. Consider $h=g\circ f$. Then $h$ admits a unique minimum $x^\star$ coinciding with the one of $f$. As such, the expectation  $\mathbb{E}_{X_{1},...X_{\lambda}\sim U(B(0,r))}\left[\lVert X_{(\mu)}-x^\star\rVert^2\right]$ satisfies the same rates than Theorem~\ref{thm:principal}.
% Moreover,  let $(\mu_\lambda)_{\lambda\in\mathbb{N}}$ be a sequence of integers such that $\forall\lambda\geq 2$, $1\leq \mu_\lambda \leq \lambda -1$ and $\mu_\lambda\to\infty$. Then, there exist two constants $C,C'>0$ and $\Tilde{\lambda}\in \mathbb{N}$ such that  for $\lambda\geq\Tilde{\lambda}$, we have the upper bound: 
% \begin{align*}
%  \mathbb{E}_{X_{1},...X_{\lambda}\sim U(B(0,r))}\left[\lVert X_{(\mu)}-x^\star\rVert^2\right] \leq C\frac{\mu_\lambda^{\frac{2(\alpha-1)}{d}}}{\lambda^{\frac{2(\alpha-1)}{d}}}+C'\frac{\mu_\lambda^{\frac{2}{d}-1}}{\lambda^{\frac{2}{d}}}\quad.
% \end{align*}
This an immediate consequence of Lemma~\ref{lem:sandwich}. In particular, using the square distance criteria, the rate are preserved even for potentially non regular functions. 
{For example, our theorem can be adapted to convex piecewise-linear functions, compositions of quadratic functions with non-differentiable increasing functions, and many others.}
Results based on surrogate models  are not applicable here. 


\subsection{Beyond Unique Optima: the Convex Hull trick, Revisited}\label{nonqc}

One of the drawbacks of averaging strategies is that they do not work when there are two basins of optima. For instance, if the two best points $x_{(1)}$ and $x_{(2)}$ have objective values close to those of two distinct optima $x^{\star},y^{\star}$ respectively then averaging $x_{(1)}$ and $x_{(2)}$ may result in a point whose objective value is close to neither. However, in the presence of quasi-convexity this can be countered. It thus makes sense to take into account the possible obstructions to the quasi-convexity of the function and try to counter these, while still maintaining the same basic algorithm as in the case of a unique optimum. \cite{ppsnkbest} proposed to take into account contradictions to quasi-convexity by restricting the number $\mu$ of points used in the averaging. Based on their ideas, we propose the following heuristic.

Let us fix the number of initially selected  points equal to $\mu_{\max}$. Let $x_{(1)},\dots,x_{(\mu_{\max})}$ be these points ranked from best to worst. Define $S_i=(x_{(1)},\dots,x_{(i)})$ and $C_i$ the interior of the convex hull of $S_i$. Assume that there is no tie in fitness values, that is no $i\neq j$ such that $f(x_{i})=f(x_{j})$. Given $\mu_{\max}$, choose $\mu$ maximal such that
\begin{equation}\forall i\leq \mu, x_{(i)} \not \in C_i.\label{oldeq}\end{equation}

One can remark that
$x_{(\mu)}\in C_\mu\Rightarrow f\mbox{ is not quasi-convex on }C_\mu$. However, this may not detect all cases in which $f$ is not quasi-convex on $C_\mu$. More generally,
\begin{equation}\exists j > \mu-1,\ x_{(j)}\in C_\mu \Rightarrow f \mbox{ is not quasi-convex on }C_\mu.\label{neweq}\end{equation} 
If such a $j$ is not $\mu$, Eq. \eqref{oldeq} does not detect the non-quasiconvexity: therefore, \eqref{neweq} detects more non-quasiconvexities than Eq. \eqref{oldeq}.

Therefore we choose $\mu$ maximal such that for all $i<\mu, j>i$, $x_{(j)}\not\in C_i$. This heuristic leads to a choice of average which is "consistent" with the existence of multiple basins.

\begin{figure*}[!h]
    \centering
    \includegraphics[width=0.3\textwidth]{sections/appendix/foga2021-kbest/plots/sphere_3.pdf}~ \includegraphics[width=0.3\textwidth]{sections/appendix/foga2021-kbest/plots/sphere_6.pdf}~    \includegraphics[width=0.3\textwidth]{sections/appendix/foga2021-kbest/plots/sphere_9.pdf}\\ Sphere function\\
    \includegraphics[width=0.3\textwidth]{sections/appendix/foga2021-kbest/plots/rastrigin_3.pdf}~ \includegraphics[width=0.3\textwidth]{sections/appendix/foga2021-kbest/plots/rastrigin_6.pdf}~     \includegraphics[width=0.3\textwidth]{sections/appendix/foga2021-kbest/plots/rastrigin_9.pdf}\\ Rastrigin function\\
    
        \includegraphics[width=0.3\textwidth]{sections/appendix/foga2021-kbest/plots/perturbed_sphere_3.pdf}~  \includegraphics[width=0.3\textwidth]{sections/appendix/foga2021-kbest/plots/perturbed_sphere_6.pdf}~     \includegraphics[width=0.3\textwidth]{sections/appendix/foga2021-kbest/plots/perturbed_sphere_9.pdf}\\ Perturbed sphere function\\
    \caption{Average regret $f(\bar{X}_{(\mu)})-f(x^\star)$ in logarithmic scale in function of the selection ratio $\mu /\lambda$ for different values of $\lambda\in\{5000,10000,20000,50000\}$. The experiments are run on Sphere, Rastrigin and Perturbed Sphere function for different dimensions  $d\in \{3,6,9\}$. All results are averaged over $30$ independent runs. We observe, consistently with our theoretical results and intuition, that (i) the optimal $r=\frac{\mu}{\lambda}$ decreases as $d$ increases (ii) we need a smaller $r$ when the function is multimodal (Rastrigin) (iii) we need a smaller $r$ in case of dissymmetry at the optimum (perturbed sphere).}
    \label{fig:examples}
\end{figure*}
\section{Experiments}
\label{sec:xps}
We divide the experimental section in two parts. In a first part, we focus on validating theoretical findings, then we compare with existing optimization methods.

\subsection{Validation of Theoretical Findings}

In this section, we will assume that $r=1$ and that the optimum $x^*$ will be sampled uniformly in the ball of radius $0.9$. We compare results on the following functions:
\begin{enumerate}
    \item Sphere function: 
    \begin{align*}
        f(x)=\sum_{i=1}^d (x_i-x_i^\star)^2
    \end{align*}
    \item Rastrigin function: 
    \begin{align*}
        f(x)=\sum_{i=1}^d (x_i-x_i^\star)^2 + 1-\cos{\left(2\pi (x_i-x_i^\star) \right)}
    \end{align*}
    \item Perturbed sphere function:
       \begin{align*}
        f(x)=\sum_{i=1}^d (x_i-x_i^\star)^2 +\left(\sum_{i=1}^d g(x_i-x_i^\star) \right)^3
    \end{align*}
    with $g(x) = x$ if $x>0$ and $-2x$ otherwise. This function has highly non symmetric sublevel sets, but still satisfies Assumption~\ref{ass:principal}.
\end{enumerate}

We plotted in Figure~\ref{fig:examples} the regret $f(\bar{X}_{(\mu)})-f(x^\star)$ as a function of $\mu/\lambda$ for different dimensions $d$ and number of samples $\lambda$. The experiments are averaged over $30$ runs. We remark for instance on the Rastrigin function that for the $\mu$-best averaging approach to be better than random search, we need a very large number of samples as the dimension increases. Overall, these plots validate our theoretical findings that averaging a few best points leads to a better regret than only taking the best one.


\subsection{Comparison with Other Methods}
\begin{figure*}
    \centering
    \includegraphics{sections/appendix/foga2021-kbest/samples/fight_all.png}
    %\includegraphics{samples/xpresults_all.png}
    \caption{Experimental results: row A and col B presents the frequency (over all 144 test cases) at which A outperforms B in terms of average loss. Then rows are sorted per average winning rate and we keep the 6 best ones. Zero is a naive method just choosing zero: we see that, consistently with \cite{icmldoe}, many methods are worse than that when the dimension is huge compared to the budget.}
    \label{figxp}
\end{figure*}
In this section, we compare averaging strategies with other standard strategies, using the Nevergrad library~\cite{nevergrad}. 
Figure \ref{figxp} presents experimental results based on Nevergrad. Instead of the uniform sampling used in the theoretical results and the previous experimental validation, we use Gaussian sampling in this set of experiments. Following the notation from \cite{ppsnkbest}, we consider distinct averaging prefixes:
\begin{itemize}
    \item \texttt{AvgXX} = method \texttt{XX}, plus averaging of the $\mu=\lambda/(1.1^d)$ best points in dimension $d$.
    \item \texttt{HAvgXX} = = method  \texttt{XX}, plus averaging of the $\lambda/(1.1^d)$ best points, restricted by {the convex hull trick (Section \ref{nonqc}).}
\end{itemize}
{Many other methods are included: we refer to \cite{nevergrad} for more information.}
Recently, \cite{icmldoe,ppsnrescaling} pointed out that when the optimum is randomly drawn from a standard normal distribution, we should use rescaling methods for focusing closer to the center in high dimensional setting. Several such methods have been proposed:
\begin{itemize}
\item \texttt{QOXX} = method \texttt{XX}, plus quasi-opposite sampling \cite{quasiopposite}, i.e. each time we draw $x$ with ${\mathcal{N}}$, we also use $-rx$ where $r$ is uniformly independently drawn in $(0,1)$.
\item \texttt{XXPlusMiddlePoint} = method \texttt{XX}, except that there is one point forced at the center of the domain. 
\item \texttt{MetaRecentering} \cite{icmldoe}:  rescaling $\sigma = (1+\log(n))/(4\log(d))$, i.e. we randomly draw with $\sigma \times {\mathcal{N}(0,I_d)}$ instead of ${\mathcal{N}(0,I_d)}$.
\item \texttt{MetaTuneRecentering} \cite{ppsnrescaling}: rescaling $\sigma = \sqrt{\log(\lambda) / d}$, i.e. we randomly draw with $\sigma \times {\mathcal{N}(0,I_d)}$ instead of ${\mathcal{N}(0,I_d)}$.
\end{itemize}

\paragraph{Experimental setup.}

We measure the simple regret and compare methods by average frequency of win against other methods. For each test case, we randomly draw the optimum as ${\mathcal {N}(0,I_d)}$ (multivariate standard Gaussian), with
different budgets $\lambda$ in $\{30, 100, 300,$ $ 1000,3000, 10000, 100000\}$ and
dimensions $d$ in $\{3,10,30,100,300,$ $1000, 3000\}$. Due to their time of evaluation, we did not run the cases with both $d=3000$ and $\lambda = 100000$. We evaluated on 3 different functions: the sphere function, the Griewank function, and the Highly Multimodal function. Previous results~\cite{bousquet} from the literature have already shown that replacing random sampling by scrambled Hammersley sampling (i.e. modern low discrepancy sequences compatible with high dimension) leads to better results.



\paragraph{Analysis of results.}

Analyzing the table results from Figure \ref{figxp}, we observe that
\begin{itemize}
\item Averaging performs well overall: \texttt{AvgXX} is better than \texttt{XX};
\item The quasi-convex trick from Section \ref{nonqc} does work: \texttt{HAvgXX} is better than \texttt{AvgXX};
\item The rescaling strategy from \cite{ppsnrescaling} outperforms the ones in \cite{icmldoe} 
 (\texttt{MetaTuneRecentering} better than \texttt{MetaRecentering} or than \texttt{PlusMiddlePoint}) which are already better than standard quasi-random sampling. Quasi-Opposite sampling is also competitive.
 \end{itemize}
 We also include various methods present in the platform, including those which are based on Cauchy or Hammersley without scrambling (Hammersley in the name without ``Scr'' prefix), or sophisticated uses of convex hulls for estimating the location of the optimum (HCH in the name).


\section{Conclusion}
We proved that averaging $\mu>1$ points rather than picking up the best works even for non quadratic functions, in the sense that the convergence rate is better than the one obtained just by picking up the best point. We also proved faster rates than methods based on meta-models (such as \cite{bach}) unless the objective function is very smooth and low dimensional. {We also show that our results cover a wider family of functions (Section \ref{invar}).}
{We also propose a rule for choosing $\mu$, depending on $\lambda$ and the dimension. This shows that the optimal $\mu/\lambda$ ratio decreases to $0$ as the dimension goes to infinity, which is confirmed by Fig. \ref{fig:examples}. We also note, by comparing with \cite{ppsnkbest}, that the optimal ratio should be smaller (Fig. \ref{zoli}), which is confirmed by our experiments on the perturbed sphere (Fig. \ref{fig:examples}).
We also propose a method for adapting this $\mu$, by automatically detecting non-quasi-convexity and reducing it: and prove that it detects more non-quasiconvexities than the method proposed in \cite{ppsnkbest}. Finally, we validate the approach on a reproducible open-sourced platform (Fig. \ref{figxp}).}

\subsection*{Further Work}
Using density-dependent weights as in \cite{sumo} should allow us to get rid of the constraint $||x^*||<r$ using a Gaussian sampling instead of a uniform sampling. Better rates might be obtained with rank-dependent weights as in \cite{arnoldweights}. We also leave as further work the proof of the optimality of the rate for this strategy. Moreover, we also believe better rates can be obtained for smoother functions, and leave this study for further work.
{The case of noisy objective functions \cite{arnoldbeyer} is critical. The study is harder, and good evolutionary algorithms use large populations, making the overall algorithm closer to a small number of one-shot optimization algorithms: actually, some fast algorithms use mainly learning \cite{astete,clop,noisymesh}. Population control\cite{beyerhellwignoise} is successful and its last stage looks exactly like a one-shot optimization method.}
\end{appendices}


  %\include{Sources/AppendixResume}



\bookmarksetup{startatroot}
\backmatter


\bibliography{biblio}
\bibliographystyle{plainnat}

\end{document}
